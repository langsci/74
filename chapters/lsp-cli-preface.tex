\chapter*{Preface} 


This book presents the first edited compilation of selected lemmas of a Chakali lexical database which  I developed over the last 9 years, together with Chakali consultants, while being affiliated to the 
Norwegian University of Science and Technology (NTNU), 
Trondheim, Norway (2007-2011, 2012-2016),  to the 
Institute of African Studies, University of Ghana, Legon, Ghana (2012), and to the 
University of Leuven, Belgium  (2016-2017).   
In 2009 the first version  was printed out  and given to consultants to corroborate its content. Another version was distributed in 2011 in the community schools of \isi{Katua}, \isi{Motigu}, \isi{Ducie},  and \isi{Gurumbele}  as part of  an informal indigenous literacy awareness campaign. 

The content of this book is based on some parts of my unpublished doctoral  thesis  \citep{brin11} and recent publications. While the dissertation's  appendix was expanded to make up the dictionary and the  reversal index offered in the second and third parts of this book, the grammatical outline has been condensed and improved to make up the phonology and grammar sections presented in the fourth part.  Although the grammar is written with an academic audience in mind, an audience interested in  \ili{Grusi} linguistic topics, it does not presuppose any knowledge of any particular linguistic theory.  It should neither be compared to comprehensive grammars,  as many aspects are not thoroughly covered,  nor to  pedagogical grammars, as it does not propose any prescriptive standards or exercises. Therefore the grammar lies beyond the scope of a typical dictionary grammar. To  publish the data  while time and funds were still available and Chakali is still relatively vibrant was felt most  imperative.

For those who are sceptical about the time and energy spent on gathering and writing down linguistic  knowledge for an non-literate community, my stand is that if comes a  time where a significant minority of the Chakali-speaking community becomes literate, the language might have already changed considerably.  So the material may contribute to its study or revival.  Furthermore,   I constantly receive strong recognition of the value of our work by Chakali people who migrated and long for things and situations of the past, and by the local authorities who can at last see that their language  receives attention.

\newpage 
Making a dictionary is a never-ending task, but the consultants and myself are  proud to present this book, the first on the Chakali language. Being a work  in  progress, there is much left to do in order to reach a substantial dictionary and grammar of the language.  Nevertheless, it is my hope that there will be  future work on Chakali lexicography  and that it will be carried out mainly  by those who speak the language. 

\begin{flushright}
 Jonathan A. Brindle\\ 
 Leuven, Belgium\\
March 2017\\

 \end{flushright}


\thispagestyle{plain}

% A lexicon is a necessary objective in documenting an endangered and 
% undocumented 
% language.
