\chapter*{Preface} 



This book presents the first edited compilation of selected lemmas of a Chakali 
lexical database which  I developed over the last 7 years, together with Chakali 
consultants, while being affiliated to the Norwegian University of Science and 
Technology (NTNU), Trondheim, Norway, 
 and   to the Institute of African Studies, University of Ghana, Legon, 
Ghana.   In 2009 a draft version  was printed out  and given to consultants 
to corroborate its content. Another version was later distributed in the four 
community 
schools of Katua, Motigu, Ducie,  and Gurumbele  as part of  an informal 
indigenous literacy awareness campaign. 

The content of this book is based on some
parts of my unpublished doctoral  thesis entitled   {\it Aspects of the Chakali 
language} \citep{brin11}. While its appendix was expanded to make up the 
dictionary and the  reversal index offered in the first part of this book, the 
grammatical outline has been condensed to make up the phonology and grammar 
sections presented in the second part.  Although the grammar is written with an 
academic audience in mind, an audience interested in  Grusi linguistic topics, 
it does not presuppose any knowledge of any particular linguistic theory.  It 
should neither be compared to comprehensive grammars,  as many aspects are not 
thoroughly covered,  nor to  pedagogical grammars, as it does not propose any 
prescriptive standards or exercises. Therefore the grammar lies beyond the scope 
of a typical dictionary grammar: to  publish the material available while time 
and funds are still available and Chakali is still relatively vibrant was felt  
imperative.

For those who are sceptical about the time and energy spent on gathering and 
writing down linguistic  knowledge for an illiterate community, my stand is that 
 if comes a  time where a significant minority of the Chakali community 
becomes literate, the language might have already disappeared.  So the material 
may contribute to its study or revival.  Furthermore,   I constantly receive 
strong recognition of the value of our work by Chakali people who migrated and 
long for things and situations of the past, and by the local authorities who can 
at last see that their language  receives attention.


Making a dictionary is a never-ending task, but the consultants and myself are  
proud to present this book, the first on the Chakali language. There is much 
left to do in order to reach a 
  substantial dictionary of the language.  Nevertheless, it is my hope that 
there will be  future work on Chakali lexicography  and that it will be carried 
out mainly  by those who speak the language. 

\begin{flushright}
 Jonathan A. Brindle\\ 
 Trondheim, Norway\\
February 2015\\

 \end{flushright}


\thispagestyle{plain}

% A lexicon is a necessary objective in documenting an endangered and 
% undocumented 
% language.
