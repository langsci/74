\chapter{Grammar Outline}
\label{sec:gramsketch}


\section{Introduction}
\label{sec:Introduction}

This chapter  provides a broad outline of the grammar and introduces those 
aspects needed to understand the formations of words and sentences found in 
the 
dictionary. Further, it acts as a preliminary grammar of the language, which is 
and will always be essential for future description and analysis since it sets 
forth claims to be confirmed, rejected, challenged,  or improved.   First, the 
common clause structure, the 
main elements of syntax and clause coordination and subordination are presented. 
Then, the nominal syntax and morphology are introduced,  followed by the verbal 
syntax and morphology. Finally, elements of grammatical pragmatics and language 
usage phenomena are examined.  The work is descriptive and employs theory 
grounded in traditional grammar but  influenced by  recent work in linguistic 
typology. When necessary, the relevant theoretical assumptions are introduced 
and the relevant literature provided. Recall that the full list of glossing tags
is available on page \pageref{sec-ABB}.



\section{Clause}
\label{sec:GRM-nom}

A  clause which can stand as a complete utterance is an independent clause. When 
a grammatically correct clause cannot stand on its own, it is dependent on  a 
main clause.  Three sorts of speech act are covered: a statement, a question, or 
a command. The former is encoded in a declarative clause (Section 
\ref{sec:GRM-decl-clause}), and the latter two are encoded in interrogative 
clauses (Section \ref{sec:GRM-interr-clause}) and imperative clauses  (Section 
\ref{sec:GRM-imper-clause}) respectively.  Constructions are treated as 
clause-types; constructions are persistent formal and semantic frames which are 
conventionalized and display both compositional and non-compositional 
characteristics. In this section  the components of the common and frequent 
independent  clauses and the major constructions encountered are presented.  In 
Section \ref{GRM-clause-coord-subord},  clause coordination and subordination 
are introduced. Section \ref{sec:GRM-adjuncts} covers the adjunct constituents
responsible for modifying a main predicate and the function of the postposition.



\subsection{Declarative clause}
\label{sec:GRM-decl-clause}

Statements are expressed by a series of declarative clause types. The structure
of most common clauses consists of  a simple predicate, one or two arguments and
an optional adjunct. This structure is represented in
(\ref{ex:GRM-clause-frame})


\ea\label{ex:GRM-clause-frame}
 {\sc s|a}  $+$ {\sc p} $\pm$ {\sc o} $\pm$ {\sc adj} 
\z

where the symbol $+$  requires for the presence of the element preceding
and following it,  whereas  $\pm$ means that the term following it may be
optional.
The
symbol {\sc s} stands for the subject of an intransitive clause,  {\sc a} 
for the subject of a transitive clause, {\sc p}  for the predicate,  {\sc
o}  for the object of a transitive clause, and finally {\sc adj}
stands for an adjunct to a clause. The main instantiations  of this common
structure are shown in  (\ref{ex:GRM-cl-fr-inst}).\footnote{The way I represent
the components of a clause is inspired by \citet[31]{Bonv88}.}


\ea\label{ex:GRM-cl-fr-inst}

\ea\label{ex:GRM-cl-fr-inst-s-p}
 {\sc s}  $+$ {\sc p} 
\ex\label{ex:GRM-cl-fr-inst-s-p-o}
 {\sc a}  $+$ {\sc p} $+$ {\sc o}
\ex\label{ex:GRM-cl-fr-inst-s-p-adj}
 {\sc s}  $+$ {\sc p}  $+$ {\sc adj} 
 \ex\label{ex:GRM-cl-fr-inst-s-p-o-adj}
 {\sc a}  $+$ {\sc p} $+$ {\sc o} $+$ {\sc adj} 

\z
\z


The predicate ({\sc p})  is represented by a verbal syntactic constituent ({\it
v}) whereas  the arguments ({\sc s, a, o}) are represented by nominal syntactic
constituents   ({\it n}).  The adjunct constituent  ({\sc adj}) may consist of 
words or phrases referring to time, location, manner of action, etc.  (see
Section \ref{sec:GRM-adjuncts} on adjunct types).  An
argument may be seen as core or peripheral.  The core
argument of an intransitive clause is realized in the subject position ({\sc
s}), which
precedes the predicate. The core arguments of a transitive clause are realized
in the subject ({\sc a}) and object ({\sc o}), the former preceding and the
latter following the predicate in their canonical positions. These
characteristics are illustrated in 
(\ref{ex:GRM-core-S-A-O}).




%{ {\sc s}  $+$ {\sc p} }\\

%{ {\sc a}  $+$ {\sc p}  $+$  {\sc o}}\\




%  \begin{minipage}[h]{12cm}

 \ea
\begin{multicols}{2}

\ea\label{ex:GRM-core-S-A-O}{
\glll àfɪ́á díjōō\\
 {\sc s}  {\sc p}\\
{\it n} {\it v}\\
\glt `Afia ate.'
}

\ex\label{ex:GRM-core-S-O}{
\glll àfɪ́á díjōō kɪ̀ŋkáŋ̀\\
 {\sc s}  {\sc p} {\sc adj} \\
{\it n} {\it v}  {\it qual}  \\
\glt `Afia ate a lot.'
}


\ex\label{ex:GRM-core-A-O}{
\glll àfɪ́á dí sɪ̀ɪ̀máá rā  \\
 {\sc a}  {\sc p}  {\sc o} {}\\
{\it n} {\it v} {\it n} {\sc foc}\\
\glt `Afia ate food.'
}

\ex\label{ex:GRM-core-A-O}{
\glll àfɪ́á dí sɪ̀ɪ̀máá  kɪ̀ŋkáŋ nà \\
 {\sc a}  {\sc p}  {\sc o}  {\sc adj}  {}\\
{\it n} {\it v} {\it n}  {\it qual} {\sc foc}  \\
\glt `Afia ate food a lot.'
}

\z 
\end{multicols}
 \z
%  \end{minipage}
% \vspace*{15pt}



% Nominal
% and verbal syntactic constituents are discussed in
% Section \ref{sec:GRM-nom} and \ref{sec:GRM-verbals} respectively, whereas
% adjuncts are presented in section
% \ref{sec:GRM-adverbs}.

The grammatical relations are primarily determined by
constituent order. Thus, the subject and object functions are not
morphologically
marked,  except that the subject pronouns in {\sc s} and {\sc a} positions  can 
have  strong or  weak forms (see Section \ref{sec:GRM-personal-pronouns}). This
is extraneous to the marking of grammatical functions but pertinent to the
emphasis put on  an  event's participant. A peripheral argument  consists of a
constituent foreign to the core predication, that is, an argument which is not
part of the core participant(s) typically attributed to a predicate.  


As
peripheral argument,  an  adjunct  ({\sc adj}) may be realized by a single word
or a phrase. Reference to space, manner, and time are the
typical  denotations of peripheral arguments.  Adjuncts will be discussed in 
brief here, but details are offered in Section 
\ref{sec:GRM-adjuncts}.  Adjuncts are optional to the main
predication and can be added to both intransitive and transitive clauses, as 
shown in  (\ref{ex:vp26.12}), and (\ref{ex:GRM-core-S-O}) and 
(\ref{ex:GRM-core-A-O}) above (see Sections \ref{sec:SPA-blc}, 
\ref{sec:SPA-postp},  and \ref{sec:GRM-obl-phrase} for  discussions on the 
postposition).

\ea

\ea\label{ex:vp26.12}{\rm Manner expression  in intransitive clause}\\
\gll ʊ̀ ɲʊ̃́ã́ làɣá nɪ̀   \\
      {\psg} drink {\ideo} {\postp}    \\
\glt  `He drank quickly.' 
% 
\ex\label{ex:vp26.13.}{\rm  Manner expression in transitive clause}\\
\gll ʊ̀ ɲʊ̃́ã́ à nɪ́ɪ́  làɣálàɣá nɪ̀ \\
      {\psg} drink {\art} water  {\ideo} {\postp}     \\
\glt  `He drank the water quickly.' 

\z 
 \z



A variation of the prototype  clause in (\ref{ex:GRM-clause-frame}) is a
clause containing an additional core argument.  \citet[116]{Dixo10b} calls  a
clause which contains an
additional core argument, that is,  an extended argument (i.e. {\sc e}), an
{\it extended} (intransitive or transitive) clause. The
difference between an adjunct and an additional core argument is not a clear-cut
one;   still,   the locative phrase in (\ref{ex:GRM-add-arg-e}) is treated as
  an additional core argument of the predicate {\it bile} `put'. In section
\ref{sec:GRM-obl-phrase}, I call a clause constituent whose semantics is
characterized by an  affected or effected object, although realized in a
postpositional phrase, an oblique object phrase. Thus, the extended argument
in (\ref{ex:GRM-add-arg-e}) should be treated as an oblique object. 


\ea\label{ex:GRM-add-arg-e}{{\sc a} $+$ {\sc p}  $+$  {\sc o} $+$   {\sc e}}\\
\glll ŋmɛ́ŋtɛ́l {sìì    à    bìlè}  {ʊ̀  kùó}  {tìwìzéŋ nʊ̀ã̀  nɪ̄}\\
{\sc a} {\sc p}   {\sc o}   {\sc e}\\
spider    {raise.up   {\conn}    put}    {{3\sg.\poss}   farm } 
{road.large     {\reln}  {\postp}}\\

\glt  `Spider went establish his farm by a main road.' (LB 003)

\z

For the remaining, a ditransitive clause consists of a transitive clause with an
additional core argument.  In Chakali, the verb {\it tɪɛ} `give', a predicate 
that conceptually implies both a Recipient (R)  and a Theme (T) and is typically
associated with ditransitive clauses, restricts its (right-) adjacent argument
in object position as  beneficiary of the situation. The thing transfered
(i.e. Theme) can never follow the verb if the beneficiary of the transfer
(Recipient) is realized. This is shown in (\ref{ex:GRM-arg-e-ditrans}).

\ea\label{ex:GRM-arg-e-ditrans}

 \ea\label{ex:GRM-arg-e-ditrans-ben-the-1}
\glll kàlá tɪ́ɛ́ àfɪ́á {à lɔ́ɔ́lɪ̀} \\
{\sc a} {\sc p} {\sc o}$_{R}$ {\sc e}$_{T}$\\
K. give A.  {{\art} car}\\

\glt  `Kala gave Afia the car.' 

 \ex\label{ex:GRM-arg-e-ditrans-ben-the-2}
\glll  kàlá tɪ́ɛ́ ʊ̄  {à lɔ́ɔ́lɪ̀} \\
{\sc a} {\sc p} {\sc o}$_{R}$ {\sc e}$_{T}$\\
K. give {\sc 3.sg}  {{\art} car}\\

  `Kala gave her  the car.' 

 \ex\label{ex:GRM-arg-e-ditrans-the-ben-1}
 *Kala tɪɛ a lɔɔlɪ Afia 
 \ex\label{ex:GRM-arg-e-ditrans-the-ben-2}
*Kala tɪɛ ʊ Afia 

\z 
 \z

The assumption is that the verb {\it tɪɛ} `give'  is transitive and its
extended argument is always the tranfered entity (i.e.
Theme) in a ditransitive clause. This is supported by the extensive use of the 
{\it manipulative serial verb construction} (see Section
\ref{sec:GRM-multi-verb-clause}), used as an alternative strategy,  in order to
express transfer of
possession  and information.



\ea\label{ex:GRM-m-svc-give}
\glll  kàlá kpá  {à lɔ́ɔ́rɪ̀ / ʊ̄} tɪ̀ɛ̀ áfɪ́á  \\
{\sc a} {\sc p} {\sc o}$_{T}$  {\sc p}  {\sc o}$_{R}$ \\
K. take  {{\art} car / 3\sg} give A.\\

\glt  `Kala gave  the car/it to Afia.' ({\it lit.} Kala take the car/it give
Afia.)
\z

The extended argument in sentence (\ref{ex:GRM-arg-e-ditrans-ben-the-1})  and
(\ref{ex:GRM-arg-e-ditrans-ben-the-2})  above  is the \is{Theme}Theme argument 
of the verb
{\it kpa} `take'   in a \is{serial verb construction}serial verb construction   
in
(\ref{ex:GRM-m-svc-give}).   Ditransitive clauses are
very rare in the text corpus despite their grammaticality.  Multi-verb
clauses, which are discussed in Section \ref{sec:GRM-multi-verb-clause},
may offer  better strategies to arrange arguments and predicates than
ditransitive clauses. The following subsections present various clause types and
constructions which are based on the declarative clause structure introduced
above.  





\subsubsection{Identificational clause}
\label{sec:GRM-ident-cl}


An identificational clause can express generic and ordinary categorizations, or
assert the identity  of two expressions. Generic categorization involves the 
classification of a subset to a set (e.g. Farmers are hard-working),
whereas an ordinary categorization holds between a specific entity and a generic
set  (e.g.  Wusa is a farmer). The clause can assert the identity of the
referents of two specific entities, a clause also known as equative or identity
(e.g. Wusa is the farmer). The examples in
(\ref{ex:GRM-ident-cl}) illustrate the
distinctions. 

\ea\label{ex:GRM-ident-cl}

\ea\label{ex:GRM-ident-gen-cat}{\rm Generic categorization}\\
\gll
 bɔ̀là jáá kɔ̀sásēl lē\\
 elephant {\ident}  bush.animal {\foc}\\
\glt `The/An elephant is a bush animal.'

\ex\label{ex:GRM-ident-ord-cat}{\rm Ordinary categorization}\\
\gll
wʊ̀sá jáá pápátá rá\\
W. {\ident} farmer {\foc}\\
\glt `Wusa is a farmer.'

\ex\label{ex:GRM-ident-tk-id}{\rm Identity}\\
\gll
wʊ̀sá jáá à tɔ́ɔ̀tɪ̀ɪ̀ná\\
W. {\ident} {\art} landlord\\
\glt `Wusa is the landlord.'

\gll
wʊ̀sá jáá  à báàl tɪ̀ŋ ká sáŋɛ̃̄ɛ̃̄ kéŋ̀ \\
W. {\ident} {\art} man {\art} {\egr} sit.{\pfv} {\dxm}\\
\glt `Wusa is the man sitting like this.'

\gll
à báàl tɪ̀ŋ kà sáŋɛ̃̄ɛ̃̄ kéŋ̀  jáá wʊ̀sá  \\
 {\art} man {\art} {\egr} sit.{\pfv} {\dxm} {\ident}   W. \\
\glt `The man sitting like this is Wusa.'


%\ex\label{ex:GRM-ident-}{\it }

\z 
 \z

The verb {\it jaa}  ({\it gl.} {\ident}) always  occurs between two nominal  
expressions,  and, as shown in the last two examples in 
(\ref{ex:GRM-ident-tk-id}),  their order  does not matter, except for the 
generic categorization where the order is always [{\it hyponym} {\it jaa} {\it 
hypernym}].  So,  the sentences {\it papata ra jaa  wʊ̀sá} `farmer {\sc foc} 
is Wusa' and 
{\it 
a 
tɔɔtɪɪna  jaa  wʊ̀sá} `landlord {\sc foc} is Wusa'   are  as acceptable as in 
the 
order given in (\ref{ex:GRM-ident-ord-cat}) and the first example in 
(\ref{ex:GRM-ident-tk-id}).    


%nin na 
%re keng

\subsubsection{Existential clause} 
\label{sec:GRM-loc-cl}

One type of existential clause is the basic  locative construction, which is
described in Section \ref{sec:SPA-blc}. Its
two main characteristics are the obligatory presence of the postposition {\it 
nɪ},  which signals that the phrase contains the conceptual ground, and the
presence of a locative predicate or the general existential predicate {\it 
dʊ̀à}. An example is provided in (\ref{ex:GRM-loc-cl}).

\ea\label{ex:GRM-loc-cl}{\rm Basic Locative construction}\\
\gll à báál dʊ́ɔ́ à dɪ̀à nɪ̄\\
{\art}  man be.at {\art} house {\postp}\\
 \glt  `The man is at/in the house.'
\z

The existential predicate {\it dʊa} is glossed `be at', but it is not the case
that it is only used in spatial description. For instance,  adhering to a
religion may be expressed using the existential predicate {\it dʊa} and the
postposition {\it nɪ}, e.g.  {\it ʊ dʊa jarɪɪ nɪ} `he/she is a Muslim', even
though no space reference is involved in such an utterance. 

An existential clause is also used in order to express that something is at
hand, accessible or obtainable. The clause in (\ref{ex:GRM-avail-cl}) is called
here 
the availability construction. It slightly differs from the
locative
construction in (\ref{ex:GRM-no-avail-cl}) because of  the absence of the
postposition
{\it nɪ}.

\ea\label{ex:GRM-avail-vs-loc}

\ea\label{ex:GRM-avail-cl}{\rm Availability construction}\\
\gll à mòlèbíí dʊ́á dé\\
{\art}  money be.at {\dem} \\
\glt  `There is money (available).'

\ex\label{ex:GRM-no-avail-cl}
à mòlèbíí dʊ̄ā dé nɪ̀\\
`The money is there.'

\z 
 \z


Another use is the attribution of a property ascribed to a participant. The
example in (\ref{ex:GRM-loc-propascr}) reads literally `a sickness is at Wojo', 
i.e. a person named Wojo is sick.  In addition to the clause presented in
(\ref{ex:GRM-loc-propascr}), ascribed property may also be conveyed in a
possessive clause (see Section \ref{sec:GRM-poss-cl}). 


\ea\label{ex:GRM-loc-propascr}
\gll gàràgá dʊ́á wòjò nɪ̄\\
sickness be.at W. {\postp}\\
\glt  `Wojo is sick.'
\z

 The
verb {\it dʊa} has an allolexe (i.e. a combinatorial variant) used only in the 
negative.
Consider (\ref{ex:GRM-allolexe}).

\ea\label{ex:GRM-allolexe}

\ea\label{ex:GRM-allolexe-pos}
\gll ʊ̀  dʊ́á dɪ̀à nɪ̄ \\
{\sc 3.sg} be.at house {\postp}\\
\glt  `She is in the house.'

\ex\label{ex:GRM-allolexe-neg}
\gll ʊ̀  wáá tùwò dɪ̀à nɪ̀ \\
{\sc 3.sg} {\neg} {\neg}.be.at house {\postp}\\
\glt  `She is not in the house.'


\ex\label{ex:GRM-allolexe-pos-out}
 \textasteriskcentered ʊ  tuwo dɪa nɪ
\ex\label{ex:GRM-allolexe-neg-out}
 \textasteriskcentered ʊ  wa dʊa dɪa nɪ

\z 
 \z



\subsubsection{Possessive clause}
\label{sec:GRM-poss-cl}

A possessive clause expresses a relation between  a
possessor and a possessed.  Generally,  the  {\it
have-}construction  is used to convey a possessive relation. It consists of
the verb {\it kpaga} `have',  and two nominal expressions acting as subject and
object; the former being the possessor (\psor) of the relation, while  the
latter being  the possessed
(\psed).

\ea\label{ex:GRM-poss-have}
\glll kàlá kpágá nã̀ɔ̃̀ rā\\
K. have cow {\foc}\\
  {\psor} {}   {\psed} {} \\
\glt  `Kala has a cow'
\z

Example (\ref{ex:GRM-poss-have}) says that an animate alienable possession
relates  Kala (possessor) and a cow (possessed).  Since the  {\it
have-}construction does not encode animacy or alienability features,   staple
food can `have' lumps, i.e. {\it kapala kpaga bie}, and someone can `have' a
senior brother, i.e. {\it ʊ kpaga bɪɛrɪ}.  Abstract possession may also be
conveyed using the {\it have-}construction. In (\ref{ex:GRM-poss-have-abst}),
  shame, hunger,  thirst, and sickness are conceived as the possessors, the
possessed being the person experiencing these feelings. 



\ea\label{ex:GRM-poss-have-abst}

 \ea\label{ex:GRM-poss-have-abst-1}
\gll hɪ̃̀ɪ̃̀sá kpàgà   à   hã́ã̀ŋ    kɪ̀ŋkáŋ̀   \\
shame       have    {\art}   woman    much\\
\glt `The woman was ashamed ...' (CB 034)
\ex\label{ex:GRM-poss-have-abst-2}
\gll lʊ̀sá kpágáń̩ nà\\
hunger have.{1.\sg} {\foc}\\
\glt `I am hungry.'
\ex\label{ex:GRM-poss-have-abst-3}
nɪ́ɪ́ɲɔ̀ksá kpágán̩ nà \\
`I am thirsty.'
\ex\label{ex:GRM-poss-have-abst-4}
gàràgá kpágán̩ nà \\
`I am sick.'

\z 
 \z

Some characteristics ascribed to animate entitites are expressed by  the
word {\it tɪɪna} `owner' following the possessed.  However it
is an existential clause (\ref{ex:GRM-poss-owner-exist}), rather than the  {\it
have-}construction, which carries the
possessive phrase {\sc psed}{\it -tɪɪna}, as (\ref{ex:GRM-poss-owner})
illustrates.



\ea\label{ex:GRM-poss-owner}

 \ea\label{ex:GRM-poss-owner-exist}
\glll ʊ̀ jáá sísɪ́ámà-tɪ́ɪ́ná\\
{3\sg} {\ident} seriousness-owner\\
  {\psor} {}   {\psed} \\
\glt `He is serious'

 \ex\label{ex:GRM-poss-owner-have}
\gll ʊ̀ kpágá sísɪ́ámà rá\\
{3\sg} have {seriousness} {\foc}\\
\glt `He is serious'

\z 
 \z

Another way to express possession is by using a non-verb clause which
  exclusively identifies   the possessor. For instance, a speaker may utter 
{\it 
mɪ́n nà} `it is
mine' in order to say that a
certain thing belongs to him or her. This utterance consists solely of the third
singular strong pronoun followed by
the focus particle (see Section \ref{sec:GRM-pronouns} on pronouns).


\subsubsection{Non-verb clause}
\label{sec:GRM-noverb}

As its name suggests, a non-verb clause is a clause without verbal elements. 
Its
main function is to identify or assert the (non-) existence of 
something.  The examples in (\ref{ex:GRM-noverb}) assert the (non-) existence
of a
referent with a single nominal expression, followed by the focus particle in
the affirmative and the negative particle in the negative (see Section
\ref{sec:GRM-foc-neg} on focus and negation). 


\begin{multicols}{2}
\ea\label{ex:GRM-noverb}
 \ea\label{ex:GRM-noverb-aff-1}
\gll fʊ́n ná\\
knife {\foc}\\
\glt `It is a shaving knife.'
 \ex\label{ex:GRM-noverb-aff-poss}
\gll ǹ̩ fʊ́n ná\\
{\sc 1.sg.poss} knife {\foc}\\
 \glt `It is my shaving knife.'
 \ex\label{ex:GRM-noverb-neg-1}
\gll fʊ́n lɛ̀ɪ́\\
knife {\neg}\\
 \glt `It is not a shaving knife.'
 \ex\label{ex:GRM-noverb-neg-poss}
\gll ǹ̩  fʊ́n lɛ̀ɪ́\\
{\sc 1.sg.poss} knife {\neg}\\
 \glt `It is not my shaving knife.'

\z 
 \z
\end{multicols}

Correspondingly the manner deictics {\it keŋ} and {\it  nɪŋ} are also found in 
non-verb
clause. For instance, {\it kéŋ né} means `That is it!', but the same string is
more often heard as {\it kéŋ nȅȅ} `Is that so/it?',  i.e.
constructed  as a polar question (see Section \ref{sec:GRM-interr-polar} on 
polar questions, and Section \ref{sec:GRM-adv-pro}  on  {\it keŋ} 
and
{\it  nɪŋ}).


%see dakubu p22 The sentence is a two-argument topic-comment proposition with no
%verb.

\subsubsection{Multi-verb clause}
\label{sec:GRM-multi-verb-clause}



A multi-verb clause is a clause containing more than one verb. The main type of
multi-verb clause is the serial verb construction (SVC), the definition of which
is still subject to contention. Let us start by stating that the SVC in Chakali
has the following properties: (i) a SVC is a sequence of verbs which act
together as a single predicate, (ii) each verb in the series could make up a
predicate on its own, (iii)  no connectives  surface (coordination or
subordination), (iv)  tense, aspect, mood, and/or polarity are marked only once,
(v)  a verb involved in a SVC may be formally shortened,  (vi)  transitivity is
common to the series, so arguments are shared (one argument obligatorily), (vii)
the verbs in the series are not necessarily contiguous, and  (viii) the grammar
does not limit the number of verbs. These characteristics are not uncommon for 
SVCs in West-Africa \citep{Amek05a}. In this section, the SVC in Chakali is
identified using representative examples. 


Even though the construction has more than one
verb, it describes a single event and does not contain  markers of
subordination or coordination. The first sequence of verbs in
(\ref{ex:GRM-mvc-svc}) illustrates the phenomenon.



\ea\label{ex:GRM-mvc-svc}
\glll à     kɪ̀rɪ̀nsá      m̩̀      màsɪ̀   kpʊ́    àká    dʊ̀gʊ̀nɪ̀ tá\\
{\art}  tsetse.fly.{\pl}   {1.\sg}       beat   kill   {\conn} 
chase         let.free\\
 {} {} {}  {} [{\it v} {\it v}]  {} [{\it v} {\it v}]\\
\glt `I beat and killed the tsetse flies, and drove them away.' (CB 023)
\z

Together,  the verbs {\it masɪ} `beat' and  {\it kpʊ} `kill'  in 
(\ref{ex:GRM-mvc-svc})  constitute a single event.  The same can be said about 
the verbs {\it dʊgʊnɪ} `chase' and {\it ta} `let free' in the second clause
following the connective.   If the clause following the connective   {\it aka}
lacks a subject,  the subject of the preceding clause shares its reference in
the two clauses   (see Section \ref{GRM-clause-coord-ka-aka} on the connective 
{\it aka}). What we have in (\ref{ex:GRM-mvc-svc}) is one SVC separated from
another multi-verb clause by the connective {\it aka},  and the three verbs 
{\it 
masɪ},  {\it kpʊ} and {\it dʊgʊnɪ}  share the reference of the  nominal {\it a
kɪrɪnsa} `the tsetse flies' as their Theme argument and {\it m̩̀} as their Agent
argument, i.e. {\sc o} and {\sc s} respectively. The role of  the verb {\it ta}
in the sentence depicted in  (\ref{ex:GRM-mvc-svc}) is discussed at the end of
this section.

Tense/aspect (\ref{ex:GRM-svc-tense}), mood (\ref{ex:GRM-svc-aspect}), and/or
polarity value (\ref{ex:GRM-svc-negation}) are marked only once, usually with
preverb particles. This means that they are not repeated for each verb of which
a predicate is composed. The preverb particles are discussed in Section
\ref{sec:GRM-precerv}.


\ea\label{ex:GRM-svc-preverb}
 \ea\label{ex:GRM-svc-tense}{
\gll  ǹ̩ tʃɪ́ kàá màsɪ̀   kpʊ́   à     kɪ̀rɪ̀nsá rá\\
{1.\sg} {\cras} {\fut.\prog}   beat   kill    {\art} tsetse.fly.{\pl} {\sc foc} 
\\ 
\glt `I will be beating and killing the tsetse flies tomorrow.'
}
 \ex\label{ex:GRM-svc-aspect}{
\gll  ǹ̩  há màsɪ̀   kpʊ́   à     kɪ̀rɪ̀nsá rá\\
  {1.\sg}  {\mod}   beat   kill    {\art} tsetse.fly.{\pl} {\sc foc} \\ 
 \glt `I am still beating  and killing the tsetse flies.'
 }
 \ex\label{ex:GRM-svc-negation}{
\gll  ǹ̩   wà másɪ́   kpʊ́   à     kɪ̀rɪ̀nsá\\
  {1.\sg}  {\neg}   beat   kill    {\art} tsetse.fly.{\pl}\\
\glt `I did not beat and kill the tsetse flies.'
 }

\z 
 \z
%wàá will not

SVCs must share at least one core  argument. The example 
(\ref{ex:GRM-arg-sh-objsubj}) is an instance of argument sharing: the two verbs
in the construction share the (referent of the) noun {\it foto} `picture' and
are not contiguous. The
transitive verb {\it tawa} `pierce' takes  {\it foto} as its object, whereas 
{\it laga} takes  {\it foto} as its subject. A representation of object-subject
sharing (or switch sharing) appears under the free translation in
(\ref{ex:GRM-arg-sh-objsubj}).
%\footnote{The label {\it object-subject sharing}
%is borrowed from \citet[20]{Osam03}.} 

\ea\label{ex:GRM-arg-sh-objsubj}{\rm Object-subject sharing}\\
\glll  hɛ̀mbɪ́ɪ́ táwá fótò làgà dáá nɪ́\\
nail pierce picture hang wood  {\postp}\\
{} {\it v} {}  {\it v} \\
 \glt `A picture hangs from a nail on a wooden pole.'

{\it foto} $<x>$\\
{\it tawa} $<${\sc subj}$ =  y$ ,  {\sc obj}$=x$  $> $\\
{\it laga} $<${\sc subj}$ = x$ , {\sc obl} $= z $  $ >$\\
\z




Subject-subject and object-object sharing are more common than object-subject
sharing. In example
(\ref{ex:GRM-mvc-svc}), which is repeated below, the nominal expression {\it a 
 kɪrɪnsa} is the shared object of three verbs, i.e. {\it masɪ}, {\it kpʊ} and 
{\it dʊgʊnɪ}, whereas the pronoun {\it m̩} is the shared subject for the same 
three
verbs. However, only {\it masɪ} and {\it kpʊ}  make up the SVC. 

\begin{exe}


\exp{ex:GRM-mvc-svc}{\rm Subject-subject and Object-object sharing}\\
\gll à     kɪ̀rɪ̀nsá      m̩̀      màsɪ̀   kpʊ́    àká    dʊ̀gʊ̀nɪ̀ tá\\
{\art}  tsetse.fly.{\pl}   {1.\sg}       beat   kill   {\conn} 
chase         let.free\\
\glt `I beat and killed the tsetse flies, and drove them away.'

{\it m̩} $<x>$\\
{\it kɪrɪnsa} $<y>$\\
{\it masɪ} $<${\sc subj}$ =  x$ ,  {\sc obj}$=y$  $> $\\
%{\it  kpʊ} $<${\sc subj}$ = x$ , {\sc obj} $= y $  $ >$\\
{\it dʊgʊnɪ} $<${\sc subj}$ = x$ , {\sc obj} $= y $  $ >$\\
 
\end{exe}

SVCs often involve two verbs, but there can be three or more verbs involved. 
Examples of three-verb and four-verb sequences are given in
(\ref{ex:GRM-mvc-3-4}). Each of the verbs involved can otherwise act alone as
main
predicate. Notice that the free translations provided do not accommodate well
the idea that
the two examples in (\ref{ex:GRM-mvc-3-4}) are conceived as single event.
In Section \ref{GRM-clause-coord-subord},  it will be shown that connectives
are usually present  when one wishes to distinguish events.


\ea\label{ex:GRM-mvc-3-4}

\ea{
\glll ʊ̀ síí kààlɪ̄ nà\\
{3\sg} rise go see\\
{}   {\it v}$_{1}$  {\it v}$_{2}$  {\it v}$_{3}$ \\
\glt {\it lit.}  `She stood, went, and saw'
}

\ex{
\glll ʊ̀ brá tùù tʃɔ́ kààlɪ̀\\
{3\sg} turn go.down run go\\
{} {\it v}$_{1}$  {\it v}$_{2}$  {\it v}$_{3}$  {\it v}$_{4}$\\
\glt `She return down and ran away' (from a tree top or hill)
}

\z 
 \z



A  manipulative serial verb construction \cite[378]{Amek06} is a SVC
which  expresses a transfer of possession (e.g. give, bring, put)  or  
information (e.g. tell). It consists of the verb {\it kpa} `take' and another
verb following it. The example in (\ref{ex:GRM-m-svc-give}), repeated below,
illustrates a transfer of possession. 

\begin{exe}

\exp{ex:GRM-m-svc-give}{\rm Manipulative serial verb construction}\\
%{\it Manipulative serial verb construction}\\
\glll  kàlá kpá  {à lɔ́ɔ́lɪ̀ / ʊ̄} tɪ̀ɛ̀ áfɪ́á  \\
K. take  {{\art} car / 3\sg} give A.\\
{} {\it v} {}  {\it v} {} \\
\glt  `Kala gave the car/it to Afia' ({\it lit.} Kala take the car give Afia.)
 
\end{exe}


Frequent co-locations of the type presented in (\ref{ex:GRM-m-svc-give}) are 
{\it kpa wa}, {\it lit.}  take come,  `bring',  {\it kpa kaalɪ}, {\it lit.} 
take 
go,
`send', {\it kpa pɛ}, {\it lit.} take add,  `add', {\it kpa ta}, {\it lit.} take
let free, `remove', {\it kpa bile}, {\it lit.} take put,  `put (on)'  and {\it 
kpa
dʊ}, {\it lit.} take put,  `put (in)'. The two verbs may or may not be
contiguous;  usually the Theme argument of the  verb {\it kpa} `take'  is found
between the two verbs.






Finally, some multi-verb clauses are not  SVCs.  There are a few verbs
which
bear a
relation to the main predication and  contribute  aspects of the phase of
execution or scope of an event.\footnote{These verbs are similar 
to what \citet[108]{Bonv88}
calls {\it auxiliant}.} For instance, a {\it
terminative}  construction describes an event coming to an end or reaching a
termination, and  a {\it relinquishment} construction describes an event whose
result is the release or abandonment of someone or something.  The verbs {\it 
peti}
`finish' and {\it ta} `abandon' in (\ref{ex:GRM-mvc-pha-3.1}) and
(\ref{ex:GRM-mvc-pha-help}), together with a non-stative predication, determine
each construction. 



\ea\label{ex:GRM-mvc-phase}

\ea\label{ex:GRM-mvc-pha-3.1} {\it Terminative construction} \\
\glll làɣálàɣá hán nɪ̀ ǹ̩ kʊ̀tɪ̀ à ʔã́ã́ pétí\\
{\advt} {\dem} {\postp} {1.\sg} {skin} {\art} bushbuck  finish\\
{} {} {} {}  {\it v} {} {} {\it v} \\
\glt `I  just finished skinning the bushbuck.'

\ex\label{ex:GRM-mvc-pha-40.3}
\gll  m̩̀ pétì à tʊ́má rá \\
{1.\sg} finish {\art} work {\foc}\\
\glt `I have finished the work.'


\ex\label{ex:GRM-mvc-pha-help}{\rm Relinquishment construction}\\
\glll  kpá ǹ̩ néŋ tà \\
take {1.\sg} hand let.free\\
 {\it v} {}  {} {\it v} \\
\glt `Leave my hand' (Let me go!)

\ex\label{ex:GRM-mvc-pha-relish} 
\gll  à bʊ̃́ʊ̃́ŋ tá ʊ̀ʊ̀ bìè rē \\
{\art} goat abandon {3.\sg.\poss} child {\foc}\\
\glt `The goat abandoned its kids.'

\z 
 \z

The examples  in (\ref{ex:GRM-mvc-pha-3.1}) and (\ref{ex:GRM-mvc-pha-help}),
which may be called  {\it phasal  constructions},\footnote{The analysis of the
progressive and prospective in Ewe and Dangme in \citet{Amek08} influences the
way I approach and name the phenomenon.}  are treated as multi-verb clauses
since the predication is expressed with more than one verb. Yet, they are not
SVCs because the second verb in each example only specifies aspects of the
process
of the event  and does not contribute to the main predication as verb sequences
in SVCs do. Nonetheless, these verbs can function otherwise as main predicate,
as shown in (\ref{ex:GRM-mvc-pha-40.3}) and (\ref{ex:GRM-mvc-pha-relish}).
Similarly, the verb {\it baga} `attempt to no avail'  conveys
nonachievement, e.g. {\it ʊ buure kisie baɣa}, {\it lit.} he look.for knife 
fail,
`he did not find the knife',  and the verb {\it na} `see' conveys confirmation 
or
verification, e.g. {\it sʊɔrɛ dɪsa na}, {\it lit.} smell soup see, `smell the
soup'. Going back to example (\ref{ex:GRM-mvc-svc}) above, the verb {\it ta}
contributes to a {\it relinquishment} multi-verb construction, similar to
(\ref{ex:GRM-mvc-pha-help}) above, and not to a SVC. 
 
% 

\subsubsection{Basic locative construction}
\label{sec:SPA-blc}


The \is{basic locative
construction} basic locative
construction  of a language is  the prototypical  and predominant
construction used to locate a figure with respect to a ground 
\citep[15]{Levi06}. In Chakali it resembles the construction given in
(\ref{ex:PSPV4}), although some sentences appear with the \is{focus 
particle}focus particle
following the postposition {\it nɪ}. The
focus particle is a pragmatic marker which identifies for the hearer the topical
subject (i.e. may be distinct from the grammatical subject) and does not convey
locative meaning (Section \ref{sec:GRM-focus}). The focus particle will be 
ignored in the discussion. 
The second line  in (\ref{ex:PSPV4}) associates parts of the sentence with a
`conceptual level'. On
that line one can find notions such as {\it figure} and {\it ground},  and 
\textsc{trm}, which stands for \is{topological
relation marker} topological
relation marker. These are the linguistic expressions which convey  the 
spatial
relationships in Chakali. The third line makes a correspondence between the
utterance-level and the grammatical relation-level. The figure {\it a gar} `the
 cloth'  functions as subject and the phrase {\it a tabul ɲuu nɪ}
`on the table' functions as oblique object  of the main
predicate. The last line is a free translation which captures  the
general meaning of the situation. It is accompanied by a reference to the
illustration which the first line describes.



\begin{exe}
\ex\label{ex:PSPV4}
\glll {[à gár]} {[ságá]} {[à téébùl ɲúù nɪ̀]}\\
\textit{figure} \textsc{trm} {\textit{ground}+\textsc{trm}}\\
 \textsc{subj}   \textsc{pred} \textsc{obl}\\
\glt `The cloth is on the table.' (PSPV 4)
\end{exe}


In (\ref{ex:PSPV4}), the spatial relation is expressed via three topological
relation markers, that is,  the main predicate {\it saga} `be on' or `sit', the
postposition  {\it nɪ} and the relational nominal predicate {\it ɲuu} `top of'.
The main predicate  {\it saga}  denotes a stative event which  localizes the
figure with respect to the ground.  The postposition  {\it nɪ} has no other
function than to signal that the oblique object is a locative phrase. The
relational nominal predicate {\it ɲuu} designates the search domain and depends
on the reference entity of the ground (i.e. {\it teebul}). The latter 
two topological
relation markers are discussed in more detail in Section 
\ref{sec:SPA-postp} and  \ref{sec:SPA-relnoun}.



\subsubsection{Comparative construction}
\label{sec:GRM-compar-ct}

A comparative construction has the semantic function of assigning a graded
position on a predicative scale to two (possibly complex) objects.
The comparative construction of inaquality can be expressed with the
transitive predicate {\it kaalɪ} `exceed, surpass', whose  two arguments are
the objects compared.\footnote{\citet{Brin05} presents a lexical-functional
grammar (LFG)  account of the comparative construction in Gã, a language also
exhibiting an  `exceed'-  or `surpass'-comparative.}  One of the arguments 
represents the
standard
against which the other is
measured and found to be unequal.  The nominal expression in subject position is
the {\it comparee}, i.e. the objective of comparison, whereas the
one in object position is the {\it standard}, i.e. the object that
serves as yardstick for comparison \citep{Stas08}. The gradable
predicative scale is verbal and is normally adjacent to  the comparee, but may
be repeated adjacent to the standard. Given that both the scale and the
transitive predicate {\it kaalɪ} are verbs, a comparative construction is  a
 type of multi-verb clause.  If the predicative scale is absent, as in
(\ref{ex:GRM-comp-tr-sca-abs}),  one
may still interpret the construction as a comparative one, in which case both
the
context
and the meaning of  the nominals involved would provide the property on which
the
comparison  is made. These characteristics are illustrated in
(\ref{ex:GRM-comp-tr}).


\ea\label{ex:GRM-comp-tr}{\rm Comparative transitive construction}\\
\ea\label{ex:GRM-comp-tr-sca-pres}{
\glll wʊ̀sáá zɪ́ŋá kààlɪ̀ áfɪ́á\\
     W. grow surpass A.\\
[{\it n}]$_{comparee}$  [{\it v}]$_{scale}$ {\it v} [{\it n}]$_{standard}$ \\
\glt `Wusa is taller than Afia.'
}

\ex\label{ex:GRM-comp-tr-sca-abs}{
\glll wʊ̀sá bàtʃɔ́lɪ́ káálɪ́ kàlá bàtʃɔ́lɪ́\\
W.  running surpass K. running\\
[{\it n}  {\it n}]  {\it v} [{\it n} {\it n}]  \\
\glt `Wusa's running is better/faster than Kala's running.'
}

\z 
 \z

Another way to compose a comparative construction of inequality is with the
identificational clause bounded with a postpositional phrase.  It is referred
to as a
comparative intransitive construction since the standard is not encoded in the
grammatical object of a transitive verb. Instead, the predicative scale is
embedded in a nominalized property following the identificational verb {\it jaa}
(see Section \ref{sec:classifier} on classifiers).


\ea\label{ex:GRM}{\rm Comparative intransitive construction}\\

\glll wʊ̀sá jáá nɪ́hɪ̃̀ɛ̃̂ àfɪ̀á nɪ́\\
W.  {\ident} old A. {\postp}\\
[{\it n}]$_{comparee}$   {\it v} [{\it v}]$_{scale}$  [{\it n}]$_{standard}$  
{} \\
\glt `Wusa is older than Afia.'
\z

The same  two strategies are used to
express a superlative degree: surpassing or being superior to all others is
explicitly expressed by a phrase containing the pronoun {\it ba} `they, them'.
This
is shown in (\ref{ex:GRM-super}).


\ea\label{ex:GRM-super}{\rm Superlative construction}\\

\ea
\glll wʊ̀sá zɪ́ŋá kāālɪ́ bá\\
W. grow surpass {\sc 3.pl}\\
{} {\it v} {\it v} {} \\
\glt  `Wusa is the tallest.'
\ex 
\gll wʊ̀sá jáá nɪ́hɪ̃̀ɛ̃̂ bà nɪ́\\
W. {\ident} old {\sc 3.pl} {\postp}\\
\glt  `Wusa is the oldest.'

\z 
 \z

A comparison of equality (i.e. X is same as Y) consists of a subject
phrase containing both objects to be  compared joined by the  connective {\it 
(a)nɪ} followed by the scale, the verb {\it maase} `equal, enough, ever' and the
reciprocal word {\it dɔŋa} `each other'  (see Section 
\ref{sec:GRM-recipro-reflex} on reciprocity
 and reflexivity). This is shown in (\ref{ex:GRM-comp-equal}).

\ea\label{ex:GRM-comp-equal}{\rm  Comparison of equality construction}\\

\gll wʊ̀sá nɪ́ àfɪ̄ā bɪ̀nsá máásé dɔ́ŋá rā \\
W. {\conn} A.  year equal {\recp} {\foc}\\
%[{\it n}]_{comparee}   {\it v} [{\it v}_{scale}]   [{\it n}]_{standard} {} \\
\glt `Wusa is as old as Afia.'
\z
%old come from where

Finally,  the verb {\it bɔ} in (\ref{ex:GRM-comp-verb}) is a comparative
transitive verb which can be translated with the English comparative adjective
and prepositon `better than'.


\ea\label{ex:GRM-comp-verb}
\glll zàáŋ tʊ́má bɔ́ dɪ̀àrè tɪ̀ŋ tʊ̄mā \\
today work better.than yesterday {\art} work\\
{}  {}  {\it v} {} {}   {}\\
\glt `Today's work is better than yesterday's work'
\z


\subsubsection{Modal clause}
\label{sec:GRM-compar-ct}

A modal clause is a clause expressing  ability, obligation, possibility, etc. An
ability-possibility construction is a clause containing the word {\it kɪŋ} 
immediately preceding the main verb(s).  The construction conveys either
the
physical or mental
ability of something or someone, or    probability or possibility under some
circumstances. The construction is more frequent in the negative, but affirming
an ability or possibility is also possible in the positive using the
construction. The word {\it kɪŋ} is glossed  {\abl} to refer to
`ability'.\footnote{The word {\it kɪŋ} has a  nominal homophone meaning 
`thing' (and a classifier derived from the noun, see Section
\ref{sec:classifier}).  Its distribution would suggest that it is a kind of 
preverb (Section \ref{sec:GRM-precerv}), although it seems premature to
categorize it.}

\ea
\label{ex:}
{\upshape Ability-Possibility construction}\\

\ea
\label{ex:GRM-modal-12.2}
\gll ʊ̀ wà kɪ́ŋ wàà\\
{3\sg} {\neg} {\abl} come\\
\glt  `He is not able to come.'

\ex  
\gll ɪ̀ kàá kɪ̀ŋ kààlʊ̄ʊ́\\
 {2.\sg} {\fut} {\abl} go.{\foc} \\
\glt  `You may go.'

\ex\label{ex:GRM-modal-13.1}
\gll ǹ̩ kàá kɪ̀ŋ wàʊ̀ tʃȉȁ\\
 {1.\sg}  {\fut} {\abl} come.{\foc} tomorrow\\
\glt `May I come tomorrow?'

\z
\z



The phrase {\it a bɔnɪɛ̃ nɪ} `perhaps' is an adjunct phrase used when the
occurence of a situation  or an achievement  is in doubt. The word {\it bɔnɪɛ̃}
is not
used in any other context in the text corpus or lexicon. Thus, the dubitative
modality construction is a construction marked by the presence of the phrase
{\it a bɔnɪɛ̃ nɪ} clause-initially.



\ea
\label{ex:GRM-modal}{\rm Dubitative construction}\\

\ea
\label{ex:GRM-modal-45.5}
\gll {à bɔ́nɪ̃́ɛ̃́ nɪ́}  dʊ́ɔ́ŋ kàá wàʊ̄\\
{\dub} rain {\fut} come.{\foc}\\
\glt  `Perhaps it is going to rain.'


\ex
\label{ex:GRM-modal-45.3}
\gll  {à bɔ́nɪ̃́ɛ̃́ nɪ́}   ʊ̀ dɪ̀ wááwáʊ́\\
{\dub}  {3\sg} {\hest} come.{\pfv.\foc}\\
\glt `Perhaps he came yesterday.'

\z
\z



In some contexts, a speaker may prefer  to use a cognitive verb in a phrase
like {\it n̩ lisie} `I think (...)'  or
the phrase {\it a kʊ̃ʊ̃ n̩ na}, {\it lit.} it tires me {\sc foc},  `I wonder
(...)' as alternative to the dubitative
construction. 



\subsection{Interrogative clause}
\label{sec:GRM-interr-clause}

An interrogative clause consists either of a clause (i) with an initial
interrogative word/phrase, or (ii) with the absence of an initial interrogative
word but the presence of an extra-low tone at the end of the clause. The former
is called a `content' question and the latter a `polar' question. 

\subsubsection{Content question}
\label{sec:GRM-interr-content}

A content question contains an interrogative word/phrase whose typical position
is clause-initial. In (\ref{ex:GRM-inter-content}), {\it bàáŋ} `what' 
replaces
the complement of the verb {\it jaa}, whereas {\it (a)àŋ́} `who'  replaces
the subject constituent of the clause. The inventory of interrogative
words/phrases can
be found in Section \ref{sec:GRM-interg-pro}.

\ea\label{ex:GRM-inter-content}

\ea\label{ex:GRM-inter-content-what}
\gll bááŋ ʊ̀ kàà jáà\\
{\q} {3\sg} {\ipfv} do \\
 \glt  `What is he doing?' 
\ex\label{ex:GRM-inter-content-who}
\gll àáŋ káá wáá báŋ̄\\
{\q}  {\ipfv} come here\\
\glt  `Who is coming here?'

\z 
 \z

When an interrogative word/phrase cannot be located clause-initially,  it is
found at the canonical position of the constituent replaced. There are
rare cases, but in (\ref{ex:GRM-inter-content-who-rev-bear-in}), which is
equivalent to  (\ref{ex:GRM-inter-content-who-rev-bear-ex}),  the question
word {\it aŋ} `who' appears in the object position following the transitive verb
{\it maŋa} `beat' and is slightly lengthened. 


\ea\label{ex:GRM-inter-content-who-rev-bear}

\begin{multicols}{2}
 \ea\label{ex:GRM-inter-content-who-rev-bear-in}
\gll  zɪ̀ɛ́n máŋá àŋ́ŋ (?)\\
 Z. beat {\q} \\
 \glt  `Zien beat who?'

 \ex\label{ex:GRM-inter-content-who-rev-bear-ex}
\gll  àŋ́ zɪ̀ɛ̀n mȁŋȁ\\
{\q} Z. beat \\
 \glt  `Who did Zien beat?'

\z 
\end{multicols}
 \z





\subsubsection{Polar question}
\label{sec:GRM-interr-polar}

A polar question is characterized by an interrogative intonation, consisting of
an extra-low tone at the end of the utterance. Additionally, lengthening  on the
penultimate vocalic segment takes place. The properties differentiating an
assertive clause from a polar question are illustrated in
(\ref{ex:GRM-inter-polar}). The
extra-low tone is represented with a double grave accent (i.e.  ̏). 

\ea\label{ex:GRM-inter-polar}{\rm Assertion vs. question}\\

\begin{multicols}{2}
\ea
\gll ʊ̀ wááʊ̀\\
{3.\sg} come.{\ipfv .\foc}\\
\glt  `He is coming.'
\ex 
\gll ʊ̀ wááʊ̏ʊ̏ \\
{3.\sg} come.{\ipfv .\foc}\\
\glt `Is he coming?'%22.1.1

\z 
\end{multicols}
 \z

Possibly common to all Ghanaian languages, the  agreeing response
to a  negative polar interrogative
  takes into account the logical negation, as 
(\ref{ex:GRM-inter-polar-neg-rep}) illustrates. 



\ea\label{ex:GRM-inter-polar-neg-rep}

\begin{multicols}{2}
\ea\label{ex:GRM-inter-polar-neg-rep-S}{\rm Speaker}\\
\gll  ɪ̀ wàà kāālɪ̏ɪ̏\\
{2\sg} {\neg} go.{\q}\\
\glt `Aren't you going?'

\ex\label{ex:GRM-inter-polar-neg-rep-A}{\rm Addressee}\\
\gll ɛ̃̀ɛ̃́ɛ̃̀\\
yes\\
\glt `No' ({\it lit.} Yes, I am not going)


\z 
\end{multicols}
 \z

A negative polar interrogative in English usually asks about the
positive proposition, i.e. with `Aren't you going?' the speaker presupposes 
  that the addressee is going,   while in Chakali it questions the
negative proposition, i.e. with {\it  ɪ̀ wàá káálɪ̏ɪ̏} the speaker's belief
is that the addressee is not going. That is probably why we get `yes' in Chakali
and `no' in English for a corresponding negative polar interrogative.



\subsection{Imperative clause}
\label{sec:GRM-imper-clause}

An imperative clause consists of a clause  expressing direct commands, 
requests, and prohibitions. It can be an exclusively addressee-oriented clause 
or  can include the speaker as well. This distinction, i.e. 
exclusive-inclusive, is rendered in  (\ref{ex:GRM-imperative-exc-inc}). In 
(\ref{ex:GRM-imperative-exc}) the  speaker excludes herself  from the 
performers of the action, while in  (\ref{ex:GRM-imperative-inc}) she includes 
herself among the performers.

\ea\label{ex:GRM-imperative-exc-inc}

\ea\label{ex:GRM-imperative-exc}{\rm Exclusive}\\
\gll fùùrì à díŋ dʊ̀sɪ̀\\
blow {\art} fire quench\\
\glt `Blow on this flame (to extinguish it).'

\ex\label{ex:GRM-imperative-inc}{\rm Inclusive}\\
\gll tɪ̀ɛ̀ jà mùŋ làɣàmɛ̀ kààlɪ̀ tɔ́ʊ́tɪ́ɪ́ná  pé \\
give {1\pl} all gather go landlord end\\
\glt `Let's all go to the landlord  together.'

\z
\z

\par
[redo]
\par


\begin{multicols}{2}
 

\ea\label{ex:GRM-imper-exc-var}
\ea\label{ex:GRM-imper-exc-var-sg} (dɪ̀)  wàà\\
Come
\ex\label{ex:GRM-imper}  dɪ̀ wáá\\
Come
\ex\label{ex:GRM-imper}  máá wàà\\
Come
\ex\label{ex:GRM-imper-exc-var-out}  dɪ́ máá wāā\\
Come
\z
\z

\end{multicols}



When an order is given directly to the addressee, as in 
(\ref{ex:GRM-imper-exc-var}), the clause may be 
introduced with the
 particle {\it dɪ}. Some consultants believe that omitting the
particle may
be perceived as rude. In addressing a command to a group the
second person plural subject pronoun  usually appears in its canonical subject
position, but it may be absent if the speaker believes that the context allows a
single interpretation.\footnote{ If A asks `What does he want?', B may reply
{\it dɪ́  má dɪ́ wāā} `That you ({\it pl.}) should be coming'. In this 
case the first {\it dɪ}
heads a  clause which introduces indirect speech and the second is an 
imperfective particle,  covered in Section \ref{sec:GRM-ipfv-part}. }

Example (\ref{ex:GRM-hortative-vp11.3}) expresses a wish of the speaker and no
addressees are called for. Such a meaning is sometimes associated with 
\is{optative} optative
mood. Similarly but not  identically,  an utterance like the one in
(\ref{ex:GRM-hortativ-vp11.4})  assumes one or more addressees, yet the desired
state of affairs is not in the control of anyone in particular but of everyone. 
 As in (\ref{ex:GRM-imperative-inc}), the  strategy in both cases is
to use the verb {\it tɪɛ} `give'.  
% 
% , and
%  occurs exclusively  with the first person (\ref{ex:vp11.3}) or, implicitly or
% explicitly, the
% third person (\ref{ex:vp11.4}), singular and  plural. 

 

\ea\label{ex:GRM-hortative}

 
\ea\label{ex:GRM-hortative-vp11.3}{\rm Optative}\\
\gll tɪ̀ɛ̀ m̩̀ mɪ̀bʊ̀à bírgì \\
       give {\sc 1.sg.poss} life delay \\
\glt  `Let me live long!' 

\ex\label{ex:GRM-hortativ-vp11.4}{\rm Hortative}\\
\gll tɪ̀ɛ̀ à gʊ̀à píílé \\
     give {\art} dance start     \\
\glt  `Let the dance begin!'

\z 
 \z


A prohibitive clause consists of  a negated proposition conveying an imperative
(or hortative) mood. It is marked by 
the negative particle {\it tɪ}/{\it te} `not'   ({\it gl.} {\sc neg.imp})
occurring
clause-initially.


\ea\label{GRM-neg-imp-vp15.10.}
\gll té káálíí, dʊ́ɔ́ŋ kàà wáʊ̀ \\
     {\neg.\imp} go rain ?{\ipfv}? come.{\foc}   \\
\glt  `Don't go; it's going to rain.' 
\z
 
The prohibitive also involves a high front vowel   suffixed to its verb. The
quality of the vowel, i.e. {\it -ɪ}/{\it -i}, is determined by the quality of 
the
verbal stem.

\begin{multicols}{2}
\ea\label{ex:GRM-neg-imperative}
\ea gó\\
`Move in a circle around'
\ex té   góìí\\
`Don't move in a circle around'
\ex  kpʊ́\\
`Kill'
\ex  tɪ́    kpʊ́ɪ̀ɪ́\\
`Don't kill'
\z 

 \z
 \end{multicols}

In addition, a distinction within the prohibitive can be made
between a prohibition or advice for a future situation  (\ref{ex:GRM-neg-fut}), 
and  for an on-going situation (\ref{ex:GRM-neg-pres}). 


\begin{multicols}{2}
 
 

\ea\label{ex:GRM-neg-fut-pres}

 \ea\label{ex:GRM-neg-fut}
\gll  kʊ̀ɔ̀rɪ̀ sɪ̀ɪ̀mã́ã̀\\
  make food\\
 \glt `Make food.'
 
 
 \ex\label{ex:GRM-neg-pres}
\gll  té kʊ́ɔ́rɪ́ sɪ̀ɪ̀mã́ã̀\\ 
{\sc neg}  make food\\
 \glt `Do not start making food'
 \vfill
 \columnbreak
 
  \ex\label{ex:GRM-neg-pres}
  \gll  tɪ́ɪ́ kʊ̄ɔ̄rɪ̄ɪ̄\\
 {\sc neg.imp}  make\\
 \glt `Do not be make (food)' (hearer in the process of making)
   \ex\label{ex:GRM-neg-pres}
  \gll  tíí kʊ̄ɔ̄rɪ̀  sɪ̀ɪ̀mã́ã̀\\
 {\sc neg.imp}  make  food\\
 \glt `Do not be make food' (hearer in the process of making)
\z
\z

\end{multicols}




\subsection{Clause coordination and subordination}
\label{GRM-clause-coord-subord}
%embedded clause, subordinate clause) cannot stand alone as a sentence

A relation between two clauses is signaled with or without a morpheme,  and 
various  structures and morphemes  are used to relate clauses.  Two
relations are discussed below: coordination and subordination. 

\subsubsection{Coordination}
\label{GRM-clause-coord}


%conjunct ka, aka, a
The distribution of four clausal connectives which are used in coordinating
clauses is presented: these are {\it a}, {\it ka}, {\it aka} and {\it 
dɪ}.\footnote{See \citet[143--149]{Mcgi99} for an account of similar clausal
connectives in Pasaale.}  


\paragraph{Connective {\it a}}
\label{GRM-clause-coord-a}

The connective {\it a} `and'  introduces a clause without overt subject.  When 
it
is used between two clauses, the subject of the first clause must cross-refer to
the covert subject of the second clause  (and subsequent clauses).
It links a sequence of closely related events carried out by the same agent,
and the events are encoded in  verb
phrases denoting temporally distinct events. The example in
(\ref{ex:GRM-coor-vp8.1}) is  an illustration of four consecutive clauses
introduced by the connective  {\it a}.   This phenomenon is often referred to as
`clause chaining'. 

\ea\label{ex:GRM-coor-vp8.1}
\gll dɪ̀àrɪ̀ tɪ̀ŋ ǹ̩ dɪ̀ káálɪ́ bɛ̀lɛ̀ɛ̀ rá, \textbf{à} \underline{jàwà 
nàmɪ̃̀ɛ̃́},
\textbf{à} \underline{kpá wàà dɪ̀á}, \textbf{à}  \underline{wà tɪ̀ɛ̀ ǹ̩ 
hã́ã̀ŋ},
\textbf{à} 
\underline{ŋmá tɪ̀ɛ̀ ǹ̩ hã́ã̀ŋ} dɪ́ ʊ́ʊ́ tɔ́ŋà. ʊ̀ tɔ̀ŋà jà dí \\ 
{\advt} {\art} {\sc 1.sg} {\hest} {go} {G.}  {\foc} {\conn}  {buy.meat} 
{\conn} {take.come.home} {\conn} {come.give.my.wife}  {\conn} {say.give.my.wife}
{\comp}  {\sc 3.sg}   {cook}    {\sc 3.sg}   {cook}  {\sc 1.pl} {eat} \\
\glt  `Yesterday I went to Gurumbele,  bought some meat, brought it
home to my wife, told her to cook it. She cooked and we ate.'
\z


\paragraph{Connectives {\it ka} and {\it aka}}
\label{GRM-clause-coord-ka-aka}

Generalizing from the examples available, for both the connectives {\it ka} and 
{\it aka}, either (i) the subject of the clause preceding the connective is 
inferred in the second clause, i.e. as for  the connective  {\it a} above, or 
(ii) a different subject surfaces in the second clause. Each case is shown in 
(\ref{GRM-clause-conn-ka-1-subj}) and (\ref{GRM-clause-conn-(a)ka-2-subj}) 
respectively.   


\ea\label{GRM-clause-conn-ka-1-subj} 
\gll    [ŋmɛ́ŋtɛ́l   láá nʊ̃̀ã̀  nɪ́]    \textbf{ká}  [ŋmá dɪ́    ʊ́ʊ́  
wá  
ɲʊ̃̀ã̀ nɪ́ɪ́] \\
spider collect mouth {\postp}  {\conn} say {\comp}  
{\sc 3.sg}  come   drink water\\
\glt  `(Monkey went to spider's farm to greet him.)  \textbf{Spider} accepted
(the
greetings) and (Spider) asked him (Monkey) to come and drink water.'  (LB 011)
 \z



\ea\label{GRM-clause-conn-(a)ka-2-subj} 

\ea\label{GRM-clause-conn-ka-2-subj} 
\gll  [dɪ̀      \underline{ɪ̀}       wáà          párà]  \textbf{ká} 
[\underline{kìrìmá}     wà    dʊ́mɪ́ɪ́]\\
{\conn}    {\sc 2.sg}    {\ingr}     {farm ({\it v})}     {\conn} 
tstse.fly.{\pl}        {\ingr}  bite.{\sc 2.sg} \\
\glt  `When  \textbf{you} are doing the weeding and  \textbf{tsetse flies} bite
you (...)' (CB 003) 

\ex\label{GRM-clause-conn-aka-2-subj} 
\gll  [dɪ́   \underline{námùŋ}    tɪ́  bɪ́ wàà   jɪ́rà kɪ̀ŋkùrùgíé
ŋmɛ́ŋtɛ́l sɔ́ŋ] \textbf{àká}  [\underline{ɪ̀}    jɪ̀rà   kéŋ̀]  \\
{\comp}   anyone              {\neg}   {\itr} {\ingr}  
   call         enumeration          
eight    name      {\conn}   {\sc 2.sg}   call     {\adv} \\
\glt  `(The monkey said:  ``They said) that \textbf{anyone} should not say the
number eight and \textbf{you} have called the number eight''.' (LB
017) 

\z 
 \z


Secondly, the connectives  {\it ka} and {\it aka} `and'  may encode a 
`logical' or `natural'
sequence of events.   For instance, in (\ref{GRM-clause-conn-ka-1-subj}), 
someone traveling (or coming from the road) expects to be offered
water to drink
after the greetings are exchanged. 



Thirdly, the  connectives  {\it ka} and {\it aka} `and' 
 also suggest a causal relation between interdependent clauses. In 
(\ref{GRM-clause-conn-a}), it is the counting of the mounds which caused Spider
to be confused, which can be seen as an unexpected outcome.  

% % % , but the left-hand conjunct in the {\it % (a)ka}-construction in 
% % (\ref{GRM-clause-conn-a})  behaves somehow  like a
% % % subordinate clause.
% % 
% % % Once again, the fact that the subordinate causal
% % % clause is a VP adjunct, thus a syntactic constituent of the matrix sentence,
% % % explains its
% % % behaviour concerning focus marking structures. The impossibility of focusing
% % %the
% % % second segment of a Justification construction through negation reinforces its
% % % syntactic
% % % peripheral status.


\ea\label{GRM-clause-conn-a} 
\gll   ʊ́ʊ́wà  ŋmɛ́ŋtɛ́l   já          kùrò 
àkà    bùtì \\
{\sc 3.sg.emph}   spider do count     
{\conn}  confuse\\
\glt  `(Because) he himself (Spider) did count and he became confused'
(LB  007) 
 \z


Nevertheless the connectives  {\it ka} and {\it aka}  can 
introduce a clause denoting an event
which is not necessarily related to the event of the previous clause. It looks
as if  the connectives in (\ref{GRM-clause-conn-ka-then}) is used to
integrate an unrelated event to  the overall situation.   

\ea\label{GRM-clause-conn-ka-then} 
\gll [{nànsá sú bárá múŋ̀.}] ká [dʊ̃́ʊ̃́ tɪ̀ŋ ŋmá dɪ́ kɪ̀ndɪ́gɪ́ɪ́ 
dʊ́ɔ́ à dɪ̀ā nɪ́]\\
{meat fill place all} {\conn} python {\art} say {\comp} something is  {\art}
house {\postp}\\
\glt `Meat was all over the place. Then,  Python said: ``there is something in
the room''.' (PYTH??)
 \z

\ea\label{GRM-clause-conn-ka-transition} 
\gll  [à     bìpɔ̀lɪ́ɪ́     sìì           tʃɪ́ŋá]     àká   [ŋmá,  
ámɪɛ̃̀ɛ̃̀   ɪ̀               ɲɪ́ná  {...} \\
{\art} young.man      raise   stand {\conn} said,  {\adv}   {\sc 2.sg.poss} 
father  {...}\\
\glt `The young man stood up and said:  ``So, when your father (...)''.' (CB
010)
 \z

Notice that the `standing' and `saying' events in
(\ref{GRM-clause-conn-ka-transition}) are strictly transitional, but this is not
the case in (\ref{GRM-clause-conn-ka-then}). The connective {\it ka} in
(\ref{GRM-clause-conn-ka-then}) opens a sentence which marks a shift from a
scene description (i.e.  `there was meat all over the place') to a character's
intervention (i.e. `Python speaking').  

Perceived event integration  seems to be what predicts the choice between 
{\it ka} and {\it aka}, but no firm conclusions can be drawn. 

\ea\label{GRM-ev-int-1} 

\ea\label{GRM-ev-int-1-ka} 
\gll Kàlá káálɪ́ jàwá ká jàwà múrò rō\\
K. go market {\sc conn} bought rice {\sc foc}\\
\glt  `Kala went to the market and bougth rice.'

\ex\label{GRM-ev-int-1-aka} 
\gll  Kàlá káálɪ́ jàwá àká pɪ̀ɛ̀sɪ̀ bùlèŋà tíísà\\
K. go market {\sc conn} ask B. station\\
\glt  `Kala went to the market and asked for the Bulenga station.' 

\z 
 \z


The cause-consequence relation in (\ref{GRM-ev-int-1-ka}) may be seen as 
`tighter' than the relation betwen the clauses in (\ref{GRM-ev-int-1-aka}). 
Buying items is a stronger effect of going to the market than 
looking for a location; market is where buying items happens. The examples in 
(\ref{GRM-ev-int-1}) suggest that {\it aka} connects `less-intergrated' 
clauses. 
 
\ea\label{GRM-ev-int-2} 
 
\ea\label{GRM-ev-int-2-ka-1} 
\gll ʊ̀ zʊ́ʊ́ dɪ̀á ká dí sɪ̀ɪ̀máá rā\\
 {\sc 3.sg} enter house {\sc conn} eat food  {\sc foc}\\
\glt `She entered the house and ate the food.' (expected)

\ex\label{GRM-ev-int-2-aka} 
\gll  ʊ̀ zʊ́ʊ́ dɪ̀á àká vrà sɪ̀ɪ̀máá rā\\
{\sc 3.sg} enter house {\sc conn} knock food  {\sc foc}\\
\glt `She entered the house and knock the food.' (unexpected)

\ex\label{GRM-ev-int-2-ka} 
\gll ʊ̀ zʊ́ʊ́ dɪ̀á ká vrà sɪ̀ɪ̀máá rā\\
{\sc 3.sg} enter house {\sc conn} knock food  {\sc foc}\\
\glt `She entered the house and knock the food.' (knowledge of intention)

\z 
 \z

Commenting on each hypothetical situation in which (\ref{GRM-ev-int-2}) may be 
uttered, one consultant agreed 
that in (\ref{GRM-ev-int-2-ka}) the intention of the subject's referent are 
known and confirmed in the second clause, which is not the case in   
(\ref{GRM-ev-int-2-aka}). The  events expressed in the second cluase in  
(\ref{GRM-ev-int-2-ka-1}) and  (\ref{GRM-ev-int-2-ka}) are perceived as more 
predictable given the first clause (and world knowledge) than the event 
expressed in the second clause  in (\ref{GRM-ev-int-2-aka}).\footnote{The 
connectives {\it aŋ}  and {\it ka} in Pasaale \citep{mcgi99} offers a good 
baseline for comparison.}  




\paragraph{Connective dɪ}
\label{GRM-clause-coord-di}
The clausal connective {\it dɪ} `and' or `while'  is homophonous with a
complementizer particle (Section \ref{GRM-clause-comp-di}), a connective used in
conditional constructions (Section \ref{GRM-clause-subord}),   and a preverb
particle signaling the imperfective aspect (Section \ref{sec:GRM-ipfv-part}). It
connects two clauses which encode different events, yet these events must be
interpreted as occurring simultaneously.  A clause introduced by the connective
{\it dɪ} has no overt subject, instead the subject is inferred, as it has the
same referent as the subject of the preceding clause. Two examples are
provided in (\ref{GRM-clause-conn-di}). 

% Some examples of the clausal connective
% {\it dɪ} in the corpus may be argued to convey intention or purpose, e.g.
% (\ref{GRM-clause-conn-di-3}). 

\ea\label{GRM-clause-conn-di}

\ea\label{GRM-clause-conn-di-vp22.4.9.}
\gll líé ʊ̀ kààlɪ̀ dɪ̀ wá \\
     {\q} {\sc 3.sg} go {\conn} come  \\
\glt  ` Where is he coming from?' ({\it lit.} where he left and  come)

\ex\label{GRM-clause-conn-di-vp47.2.9.}
\gll kpá sɪ̀ɪ̀má háŋ̀ dɪ̀ káálɪ̀  \\
    take   food {\dem}  {\conn} go \\
\glt  ` Take this food away! ({\it lit.} take this food and go)

%  \ex\label{GRM-clause-conn-di-3}
% \gll ʊ dʊa ja gantal nɪ dɪ wa\\
%  {\sc 3.sg} be.at {\sc 1.pl.poss} back {\postp} {\sc conn} come\\
% `She is following us' ({\it lit.}  she is at our back and come)

\z 
 \z



\subsubsection{Subordination}
\label{GRM-clause-subord}

The morpheme {\it tɪŋ} is mainly used as a  determiner in noun phrases  (see
Section \ref{sec:GRM-np-def}).  However, there are instances where the discourse
following {\it tɪŋ} must be treated as subordinated and related to the noun
phrase of which  {\it tɪŋ} is part. One may argue that the morpheme {\it tɪŋ} 
can
function  as a relativizer. Consider (\ref{GRM-clause-subord-rel}).

\ea\label{GRM-clause-subord-rel}
\gll  kúrò [píé tɪ́ŋ]$_{NP}$ \underline{ʊ̀ kà tɔ́ à kùò nɪ́ kéŋ̀}
tɪ̀ɛ̀ʊ́\\
   count   yam.mound.{\sc pl}  {\art}   {{\sc
3.sg}-{\egr}-cover-{\art}-farm-{\postp}-{\adv}}         give.{\sc 3.sg}\\
\glt  `(Spider$_{x}$ asked Buffalo to) count  for him$_{x}$ the yam mounds which
he$_{x}$ covered at the farm.' (LB 006)
 \z

In (\ref{GRM-clause-subord-rel}), the phrase {\it ʊ̀ kà tɔ́ à kùò nɪ́
kéŋ̀} is (i) in apposition to the noun phrase {\it píé tɪ́ŋ}, and (ii)
in a subordination relation with the noun phrase.\footnote{Examples 
LB 004, LB 012, CB 019 and  CB 026 in the appendix display the same sort of
subordination.}


In a conditional construction like the one in
(\ref{GRM-clause-subord-if-vp46.11}), the subordinate
clause is headed by the particle {\it dɪ},  whereas the main clause follows
the subordinate clause. Proverbs are typically  conditional
constructions.  An example is given in (\ref{GRM-clause-subord-proverb}).


\ea\label{GRM-clause-subord-di}

\ea\label{GRM-clause-subord-if-vp46.11}

\gll dɪ̀ ǹ̩ fɪ̀ tú kààlɪ̀ dē, bà kàá  tùgúǹ nō \\
      {\sc conn}  {\sc 1.sg} {\mod} {go.down} go {\adv} {\sc 3.pl.h+} {\fut}
beat.{\sc 1.sg} {\foc} \\
\glt  `If I was to go down there, they will beat me.' 

\ex\label{GRM-clause-subord-proverb}
\gll dɪ̀ ɪ̀ zíŋ wā zɪ̀ŋà,  ɪ̀ wàá kɪ̀ŋ gáálɪ́ díŋ nɪ̄ \\
  {\sc conn} {\sc 2.sg}  tail {\ingr} long  {\sc 2.sg} {\neg} {\abl} be.over
fire
{\postp}\\
`If you have a long tail, you cannot cross fire.'


 \z 
 \z
 

Adverbial expressions are used as connectives in similar clause-relating
functions. The subordinate clause of a concessive construction is introduced by
the expression  {\it anɪ amuŋ}, {\it lit.} and-all, `despite',  `in spite
of', `although' or `even though'. A subordinate clause  which conveys a
consequence or a justification of the proposition in the main clause  is
introduced by the expressions {\it aɲuunɪ} or {\it awɪɛ}, {\it lit.} the-head-on
 and  the-matter,  respectively,  `therefore' or
`because'. Examples are shown in (\ref{GRM-clause-conces-consec}).

\ea\label{GRM-clause-conces-consec}

\ea\label{GRM-clause-conces}

\gll ʊ̀ wááwáʊ́ {ànɪ́ ámùŋ} dɪ́ ʊ̀ wɪ́ɪ́ʊ̀ \\
       {\sc 3.sg} come.{\pfv .\foc} {\conn} {\comp}  {\sc 3.sg} sick.{\foc}\\
\glt  `He came in spite of his illness.' 


\ex\label{GRM-clause-consec-1}

\gll ǹ̩ wà kpágá sákɪ̀r, {àɲúúnɪ̀} ǹ̩ dɪ̀ válà nã̀ã̀sá \\
{\sc 1.sg} {\neg} have bicycle {\conn} {\sc 1.sg}   {\ipfv} walk
leg.{\pl}\\
\glt `I don't have a bicycle, therefore I am
walking.'


 \z 
 \z
 

\paragraph{Complementizer dɪ}
\label{GRM-clause-comp-di}


Example (\ref{GRM-clause-comp-inds}) shows that the complementizer {\it dɪ}
introduces indirect speech. 
\begin{exe}
 \ex\label{GRM-clause-comp-inds}
 \gll    kùórù   bìnɪ̀hã́ã̀ŋ          ŋmá   dɪ́     ɛ̃̀ɛ̃́ɛ̃́ɛ̃̀  \\
 chief   young.girl   say   {\comp}    yes\\
 \glt  `The chief's daughter answered yes.'  (CB 011) 
 \z


Direct speech is usually introduced by a speech
verb only, e.g. {\it ŋma (tɪɛ)} `say (give)',   {\it tʃagalɪ} `teach, show,
indicate', {\it hẽsi} `announce', etc.  This is shown in 
(\ref{GRM-clause-comp-ds}) with {\it hẽsi}.

\begin{exe}
 \ex\label{GRM-clause-comp-ds}
 \gll tɔ́ʊ́tɪ́ɪ́ná ŋmá dɪ́ bá hẽ́sí \underline{má ká pàrà kùó}\\
 landlord say  {\comp} {\sc 3.pl.g}b  announce {{\sc 2.pl} {\egr} farm farm}\\
 \glt  `The landowner says that they announced:  ``You go and work at the
farm''.' 
 \z

In (\ref{GRM-clause-comp-2-int}),  the complementizer {\it dɪ} introduces a
clause which conveys the intention of the event in the main clause. In a literal
sense, the husband {\it lala} `open'  the wife {\it in order to} have her {\it 
sii} `raise up'. In (\ref{GRM-clause-comp-2-pur}) it is shown that a purpose (or
an intention) can be encoded when {\it dɪ} introduces the goal. In the latter 
case however consultants say that the complementizer {\it dɪ} is optional.



\ea\label{GRM-clause-comp-2}

 \ea\label{GRM-clause-comp-2-int}
\gll  tʃʊ̀ɔ̀sá   pɪ́sɪ̀,    ʊ̀   báàl    tɪ̀ŋ té    lálá    à hã́ã̀ŋ   
 dɪ̀  ʊ́         síí        dùò    nɪ̀.\\
  morning scatter          {\sc 3.sg.poss}   husband  {\art} 
early  wake.up 
{\art}  wife         {\comp}      {\sc 3.sg}         raise.up    asleep  
{\postp} \\
 \glt  `Early in the morning her husband woke up the wife from sleep.' ({\it
lit.} that she should/must stand up)  (CB 030)

 \ex\label{GRM-clause-comp-2-pur}
 \gll ʊ̀ káálɪ́ (dɪ́) ʊ́ʊ́ ká ɲʊ̃̀ã̀ nɪ̄ɪ̄   \\
{\sc 3.sg} go   {\comp}  {\sc 3.sg}   {\egr} drink  water  \\
 \glt  `He went to have a drink of water.' 


  \z 
 \z






 \paragraph{Clause apposition}
 \label{GRM-dep-comp-clause}
% 
In (\ref{GRM-clause-appo}) it is shown that a desire can be
encoded by two clauses in apposition. 
% 
 \begin{exe}
 \ex\label{GRM-clause-appo}
 \gll jà búúrè nɪ̄ɪ̄ rā já ɲʊ̃́ã̀\\
{\sc 2.pl} want water {\foc} {\sc 2.pl} drink\\
 \glt  `We want some water to drink.' 
 \z


 
 
\subsection{Adjunct adverbials and postposition}
\label{sec:GRM-adjuncts}


The notion `adverbial' is used in the sense of  `modifying main predicate', 
that 
is,  a manner, a place, or a time in which a state of affair is carried out. An 
adjunct adverbial is thus an expression, word or phrase, which is not an 
argument of  the main predicate, modifies the main predicate, and is positioned 
at the periphery 
in 
an adjunct constituent  ({\sc adj}), repeated in  (\ref{ex:GRM-clause-frame}). 

\begin{exe}
\exp{ex:GRM-clause-frame}
 {\sc adj}  $\pm$ {\sc s|a}  $+$ {\sc p} $\pm$ {\sc o} $\pm$ {\sc adj} 
\end{exe}



Adjuncts are usually found following the core constituent(s), but may also be
found at the beginning of a clause. As
shown in (\ref{ex:GRM-pre-adj}), 
reference to time may be found at the beginning of a clause before 
the 
subject.

%to time (\advt), location (\dem) or manner (\dxm).

\ea\label{ex:GRM-pre-adj}
{{\sc adj} $+$ {\sc s}  $+$ {\sc p}  $+$  {\sc o}}\\
\glll  {[tʃʊ̀ɔ̀sá  pɪ̀sɪ̀]}   {à    bìpɔ̀lɪ́ɪ̀}  kpá {ʊ̀ páŕ}\\ 
 {\sc adj}  {\sc s}  {\sc p} {\sc o}\\
{morning   scatter}   {{\art} young.man} take {{3\sg.\poss} hoe}\\
  \glt  `The following day the young man took his hoe along...' (CB 005)
\z

 In Section \ref{sec:GRM-compar-ct}, the dubitative construction was 
identified with the phrase  {\it à bɔ́nɪ̃́ɛ̃́ nɪ́}  `perhaps'  opening the 
clause. There are other constructions in which temporal, locative, manner, or 
tense-aspect-mood meaning is signaled by the presence of an adjunct adverbial  
initially that introduces new information.  

\ea\label{ex:GRM-phra-adv}

\ea\label{ex:GRM-phra-adv-time}{\rm Temporal}\\
\gll [tàmá fìníì] ʊ̀ fɪ̀ sʊ́wá\\
few little {\sc 3.sg} {\mod} die\\
\glt `A little longer and she would  have died.'


\ex\label{ex:GRM-phra-adv-evi}{\rm Evidential}\\
\gll [wɪ́dɪ́ɪ́ŋ ná] dɪ́ ʊ̀ náʊ́ rā\\
truth {\foc} {\comp} {\sc 3.sg} see.{\sc 3.sg} {\foc} \\
\glt  `It is certain that he saw him.

\z 
 \z



In (\ref{ex:GRM-phra-adv-time}), the phrase {\it tama finii} is not inherently 
temporal, but must be interpreted as such in the given context. In 
(\ref{ex:GRM-phra-adv-evi}) the verbless clause {\it wɪdɪɪŋ na}  can be seen as 
adding an illocutionary force; it additionally signifes that the speaker has 
evidence and/or wish to convince the hearer about the proposition. 

% 
% % They are glossed {\sc dem} (i.e. adverb locative), {\sc dxm}  (i.e.
% % adverb manner) and {\sc advt}  (i.e. adverb time) respectively. Examples are
% % provided below.

\subsubsection{Temporal  adjunct}
\label{sec:GRM-manner-adv}

A temporal nominal adjunct  ({\it gl.} {\sc advt}) is an expression which 
typically indicates when  an event occurs. In Section 
\ref{sec:GRM-preverb-three-int-tense}, the three-interval tense system is 
introduced.  It was shown that the temporal nominal {\it  dɪare} `yesterday' and 
{\it tʃɪa} `tomorrow'  have preverbs counterpart.  The  temporal nominal  {\it 
zaaŋ} (or {\it zalaŋ}) expresses `today',  and   {\it tɔmʊsʊ} can express either 
 `the day before yesterday' or  `the day after tomorrow',   yet neither {\it 
zaaŋ} and {\it tɔmʊsʊ}   have a corresponding preverb. Thus {\it  dɪare} 
`yesterday',  {\it tʃɪa} `tomorrow',   and  {\it zaaŋ} `today', which  typically 
function as adjunct  and can be disjunctively connected by the nominal 
connective {\it anɪ}  are  treated as nominals.
%connected with what connectives?

\ea\label{ex:GRM-adj-temp-adv}

\ea\label{ex:GRM-adj-temp-adv-thatday}

\gll àwʊ̀zʊ́ʊ́rɪ̀ ǹ̩ wà tùwò nɪ̄  \\
{\advt} {\sc 1.sg} {\neg} {be.at} {\postp}\\
\glt `That day I wasn't there.'


\ex\label{ex:GRM-adj-temp-adv-LB5}

\gll àwʊ̀zʊ́ʊ́rɪ̀  dɪ́gɪ́ɪ́           kɔ̀sánã́ɔ̃́   válá    (...)\\
{\advt}  one           buffalo    walked (...) \\
\glt `One day a buffalo walked (by and greeted the spider).' (LB 005)

\ex\label{ex:GRM-adj-temp-adv-CB17}
\gll [dénɪ̀],         [sáŋà      dɪ́gɪ́ɪ́]         à   
hã́ã̀ŋ já   pàà    à     báál         zōmō  (...) \\
{\advt}    time      one           {\art}  wife     {\hab}   take.{\pl} 
{\art} husband insult.{\pl} (...)\\
\glt `[During their life, it happened] on one occasion that the woman
did insult  the man (...)' .  (CB 017)

\ex\label{ex:GRM-adj-temp-adv-everyday}
\gll  ǹ̩ já kààlɪ̀ ʊ̀ pé rè [{tʃɔ̀pɪ̀sɪ̀} bɪ́ɪ́-mùŋ]\\
 {\sc 1.sg}    {\hab} go {\sc 3.sg} end {\foc} day.break {\itr}-all\\
\glt `I visit him every day.'

\ex\label{ex:GRM-adj-temp-adv-nownow}
\gll [làɣálàɣá háǹ nɪ̄] ǹ̩ kʊ̀tɪ̀ à ʔã́ã́ pétí\\
{\advt} {\dem} {\postp} {1.\sg} {skin} {\art} bushbuck  finish\\
\glt `I  just finished skinning the bushbuck.'


\z 
 \z


Some expressions tagged as temporal nominal are treated as complex, though
opaque, expressions. For instance,  {\it awʊzʊʊrɪ} is translated into  `that
day' in (\ref{ex:GRM-adj-temp-adv}), but the forms {\it wʊ̀sá} `sun' and
{\it zʊʊ} `enter'  are perceptible. The phrase {\it laɣalaɣa han nɪ} in
(\ref{ex:GRM-adj-temp-adv-nownow}) literally
means `now.now this on' ({\advt} {\dem} {\postp}), but `only a moment
ago'  is a better translation.  Similarly, {\it denɪ} is analyzed as a
temporal nominal, but usually functions as a connective. It is made from  the
spatial demonstrative {\it de} and the potsposition {\it nɪ}, and is translated 
to
English as `thereupon', `after that', `at that point', or simply `then'. It is
mainly used at the beginning of a sentence to signal a transition  between the
preceding  and the following situations;
(\ref{ex:GRM-adj-temp-thereupon}) suggests a transition of the resultative type.
The appendix contains other examples of {\it denɪ}, e.g.  CB (008, 017, 019)
and 
LB (006, 016).


\ea\label{ex:GRM-adj-temp-thereupon}
\gll dénɪ̄      rè,           ʊ̀ʊ̀               hã́ã́ŋ   tɪ̀ŋ 
ŋmá   dɪ́     ààí (...) \\
 {\advt}   {\foc}     {\sc 3.sg.poss}  wife     {\art}  say   {\comp} 
no    (...)\\
\glt [The man said: `Don't cry, if you tell your father that I drove the tsetse
flies away,  weeded the farm and took you as a wife, I will also tell your
father you are freeing yourself in bed.']  `Then, the wife said: ``No, (I won't
say
anything to my father'.)'' (CB 036)
\z




\subsubsection{Manner adjunct}
\label{sec:GRM-manner-adv}

A manner  expression describes the way the event denoted by
the verb(s) is carried out. The examples in (\ref{ex:GRM-adj-mann}) illustrate
the meaning and distribution of  manner expressions.


\ea\label{ex:GRM-adj-mann}


\ea\label{ex:GRM-adj-mann-carefully}
\gll dɪ̀ sã́ã́ bʊ̃̀ɛ̃̀ɪ̃̀bʊ̃̀ɛ̃̀ɪ̃̀ \\
{\comp} drive {carefully}\\
\glt `Drive carefully.'

\ex\label{ex:GRM-adj-mann-slowly}
\gll dɪ̀ ŋmà bʊ̃̀ɛ̃̀ɪ̃̀bʊ̃̀ɛ̃̀ɪ̃̀\\
{\comp} talk {slowly}\\
\glt `Talk slowly.'

\ex\label{ex:GRM-adj-mann-lighly}
\gll ʊ̀ tʃɔ́jɛ̄ kààlɪ̀ félfél\\
 {\sc 3.sg} run.{\pfv} go {lightly} \\
\glt `She ran away lightly (manner of movement, as a light weight
entity).'

\ex\label{ex:GRM-adj-mann-silently}
\gll  ǹ̩ kàà wáá dɪ̀ à   hã́ã́ŋ     sáŋà   tʃérím \\
{\sc 1.sg} {\sc ipfv} come {\sc comp} {\sc art} woman sit quietly\\
\glt `When I was coming, the woman sat quietly' 

\z 
 \z

It is common for an \is{ideophone}ideophone to function as a manner expression 
 (Section \ref{sec:GRM-onoma}). One could argue that  all the manner 
expressions in
(\ref{ex:GRM-adj-mann}) are ideophones, i.e. they display reduplicated forms
and {\it tʃerim} is one of a few words which ends with a bilabial nasal. The
examples in (\ref{ex:GRM-adj-mann-ideo-adv}) show the repetition of two
expressions; one is an \is{ideophone}ideophone, i.e. {\it kaŋkalaŋ} `crawl of a 
snake', and the
other  a  reduplicated  manner expression,  i.e. {\it 
laɣalaɣa} 
`quickly' from {\it laɣa} 
`now'. The
repetition of {\it kaŋkalaŋ} and {\it  laɣalaɣa} conveys that the motion was
({\it kpa} `taken'  and) occurring with great speed.

%inceptive meaning of kpa


\ea\label{ex:GRM-adj-mann-ideo-adv}


\ea\label{ex:GRM-adj-mann-ideo}
\gll à bààŋ kpá {kàŋkàlàŋ kàŋkàlàŋ kàŋkàlàŋ}\\
{\conn} just take crawl.rapidly\\
\glt `(She was after the python) but (he) started to crawl away like a shot.' 
(REF?)

\ex\label{ex:GRM-adj-mann-adv}
\gll  kà bààŋ kpá làɣàlàɣà làɣàlàɣà\\
{\conn} just take {\dxm} {\dxm}\\
\glt `(She) started to (walk) quickly.'


\z 
 \z


The manner adverb {\it kɪŋkaŋ} `abundantly',  which is composed of the 
classifier
{\it kɪn} and the verb {\it kana} `abundant',  typically quantifies or 
intensifies
the
event and always comes after the word encoding the event.  Notice in
(\ref{ex:GRM-adj-mann-alot-v})  and (\ref{ex:GRM-adj-mann-alot-n})   that {\it 
kɪŋkaŋ} follows a verb and a nominalized verb respectively. However, in
(\ref{ex:GRM-adj-mann-alot-quant}), {\it kɪŋkaŋ} does not function as a
manner adverb but as a quantifier.



\ea\label{ex:GRM-adj-mann-alot}

\ea\label{ex:GRM-adj-mann-alot-v}
\gll gbɪ̃̀ã́           ɪ̀       jáárɪ́jɛ́        kɪ́ŋkāŋ       nà
(...)\\ 
monkey   you     unable.{\pfv}     {\dxm} {\foc} (...)\\
\glt `Monkey, you are so incompetent, (...).' (LB 016)

\ex\label{ex:GRM-adj-mann-alot-n}

\gll dúó tʃʊ̄ɔ̄ɪ̀ kɪ́ŋkāŋ wà wíré\\
asleep lie.{\nmlz} {\dxm} {\neg} good\\
\glt `Sleeping too much is not good.'

\ex\label{ex:GRM-adj-mann-alot-quant}
\gll kùórù   kùò    tɪ́ŋ   kà   kpágá kìrìnsá  kɪ́ŋkāŋ, dé rē jà 
kààlɪ̀\\
 chief farm {\art} {\rel} have tsetse.fly.{\sc pl}  {\quant} {\sc dem} {\sc 
foc} {\sc 1.pl} go\\
\glt   `The chief's farm that has  many tsetse flies, there we went' 


\z 
 \z



\subsubsection{Postposition {\it nɪ} and (non-) locative adjunct}
\label{sec:SPA-postp}

First, the postposition {\it nɪ} signals  that the constituent in which it  
appears  is locative.  In fact, in the majority of the  utterances in the 
 corpus,\footnote{The {\it Chakali Location and Position Corpus} 
\citep{JAB-space-C-10} contains the audio material of the elicitation 
procedure described in \citet{Bowe93, Amek99, Meir01a, Meir01b}.}  the  
postposition {\it nɪ} is present irrespective of the
locative verb involved or whether or not a relational noun occurs. Only a  few
exceptions can be found,   and they are
systematically accounted for by two factors: (i) some verbs  do not co-occur 
with {\it nɪ}, e.g. {\it tɔ} `cover', {\it kpaga} `have' and {\it su} `fill',  
and (ii) some situations are described using an intransitive clause, e.g.   
{\it 
à bónsó tʃíégìàō}    `the cup is broken'  (TRPS 26).  
\citet[370]{Amek06} present the Ewe verb {\it le}, glossed  `be at',  which is 
used in the majority of the sentences denoting the situations of the TRPS.   
The 
translation of  Ewe {\it le} to Chakali would then be equivalent to {\it dʊa 
{\rm NP} nɪ}.\footnote{The Ewe verb {\it le} may also function as predicator of 
qualities \citep[373]{Amek06}. In Chakali,  it was shown  in Sections 
\ref{sec:GRM-ident-cl} and \ref{sec:classifier} that   {\it jaa} predicates 
over 
qualities,  not  {\it dʊa}.}



Second, the postposition {\it nɪ} identifies an oblique object phrase, and
 conveys that the oblique object phrase contains the ground object with which a 
figure related. The complement precedes the  postposition. The examples  
displayed in ({\ref{ex:postp-corres}) show that the complement of  {\it nɪ} is 
a 
noun phrase (see {\it RelP} in Section \ref{sec:SPA-relnoun}). Since there are 
no prepositions in the language, the abbreviation PP in 
({\ref{ex:postp-corres}}) unambiguously stands for Postposition Phrase.

\ea\label{ex:postp-corres}
 \ea {[}{[}{[}a dɪa{]}$_{NP}$ ɲuu{]}$_{RelP}$ nɪ{]}$_{PP}$  {\rm `on the roof 
of the 
house'} \\
 \ex {[}{[}a dɪa{]}$_{NP}$ nɪ{]}$_{PP}$ {\rm  `in/at the house'} \\
 \ex  {[}{[}baŋ{]}$_{NP}$ nɪ{]}$_{PP}$ {\rm  `here'}\\
 \ex {[}{[}de{]}$_{NP}$ nɪ{]}$_{PP}$  {\rm `there'}\\
 \ex {[}{[}ʊ{]}$_{NP}$ nɪ{]}$_{PP}$  {\rm `at/on/in him/her/it'}\\
\z
\z

Nevertheless, the postposition does not inform the hearer on any of the 
elementary topological spatial notions; none of the concepts of proximity, 
contiguity, or containment is encoded in  the postposition {\it nɪ}. It never 
selects particular figure-ground configurations but must be present for all of 
them. 

Besides the description of static configurations, the postposition {\it nɪ}  is 
used frequently in adverbial/connective expressions: {\it à bɔ́nɪ̃́ɛ̃́ nɪ́} 
`maybe, perhaps', {\it à ɲúú nī} `therefore', {\it búŋbúŋ ní} `at 
first', etc. These expressions do not have a purely locative function, but are 
rather used as clausal adjuncts,  or to introduce logical conclusion (see 
Sections   \ref{GRM-clause-subord} and \ref{sec:GRM-adjuncts}). 



Interestingly, if the postposition  occurs between the focus particle 
(Sections \ref{sec:GRM-foc-neg} and \ref{sec:GRM-focus}) 
and
the preceding nominal, one would expect  the focus particle to surface in its
default form, i.e. {\it ra},  since the required adjacency is no longer 
satisfied
(Section  \ref{sec:focus-forms}). 


\ea\label{ex:GRM-focus-form}

\ea\label{ex:GRM-foc-form-X}
\TExt{\Txt{$\alpha$}\COn{rr}\ \  & \Txt{\it nɪ} & \ \  \TXT{\foc}}
 
\ex\label{ex:GRM-foc-form-1}
\gll  à máŋkɪ́sɪ̀ ɲúú nɪ̄ rò/rè \\
       {\art} {match} {\reln} {\postp} {\foc}\\
\glt `on the top of the matchbox'

\ex\label{ex:GRM-foc-form-2}
\gll  à  pùl nɪ́ rō/rē \\
       {\sc art} {river} {\sc postp}  {\sc foc}\\
\glt `on/at the river'

\z 
 \z

However, on several occasions, the postposition becomes `transparent' and 
vowel-harmony can still operate (i.e. though not the place 
assimilation of consonant). The
phenomenon is shown in (\ref{ex:GRM-focus-form}).\footnote{A more extreme case
is found in example (\ref{ex:GRM-obl-obj-no-spa-foc}).}



\subsubsection{Oblique object phrase}
\label{sec:GRM-obl-phrase}

The oblique object phrase is an element of a clause whose  semantics is
characterized by an  affected or effected object, although realized by a
postpositional phrase.  In Section \ref{sec:SPA-postp},  it is claimed that the
postposition {\it nɪ} (i) identifies an oblique object phrase, (ii) conveys that
the oblique object phrase contains the ground object in localization, and
(iii) follows its complement. 

While localization is
the main function of  {\it nɪ}, the postposition can also be found when there
is no  reference to space. For instance, in Section \ref{sec:GRM-manner-adv}, I
discuss the connective {\it denɪ} (i.e. {\dem}+{\postp}), whose role in
discourse is to signal a temporal transition, not a spatial one.  The
examples in (\ref{ex:GRM-obl-obj-no-spa}) illustrate some of the non-spatial
uses
of the oblique object phrase headed by {\it nɪ}.


\ea\label{ex:GRM-obl-obj-no-spa}

\ea
\gll ʊ̀ ɲʊ̃́ã́  [làɣálàɣá nɪ̄]  \\
      {\psg} drink {\dxm} {\postp}    \\
\glt  `He drinks quickly.' 

\ex
\gll bàáŋ ɪ̀ fɪ́ kàà sʊ́ɔ́gɪ̀ [tʃʊ̀ɔ̀sá tɪ́n nɪ̄]\\
 {\q} {\sc 2.pl} {\pst} {\egr} crush  morning {\art} {\postp}    \\
\glt  `What were you crushing this morning?' 

\ex\label{ex:GRM-obl-obj-no-spa-foc}
\gll à kùórù ŋmá dɪ́ ʊ̀ bááŋ káá sīī [ǹ̩ nɪ́] rē\\
{\art} chief say {\comp} {\sc 3sg.poss} temper {\egr} raise {\sc 1.sg} {\postp}
{\foc}\\
\glt  `The chief told me that he was very angry with me.' 


\z 
 \z

 


%\begin{exe}
% \ex\label{GRM-clause-appo}
% \gll    \\
% \\
% \glt  `' 
%  \z

% 
% \ex\label{ex:vp19.2.}
% \gll anɪ a muŋ dɪ ʊ dia boloo , n̩ ja ka lɪ ʊ pe re tʃʊɔpɪsɪ bɪ muŋ \\
%         \\
% \glt  ` Although his room is far away, I visit him every day.  ' 




\section{Nominals}
\label{sec:GRM-nom}


The term `nominal'  identifies  a formal and functional  syntactic level and
lexemic level. At the syntactic level, a noun phrase is a nominal  which can
either function as core or peripheral argument.  Its composition may
vary from a single pronoun to a noun with modifier or series of
modifiers. At the lexeme level, two categories of lexemes are assumed:
nominal and verbal. These two types correspond roughly to the semantic division
{\it entity} and {\it event}, but do not correspond to the syntactic categories
{\it noun} and {\it verb}. That is because lexemes are assumed to not be
specified for syntactic category. The diversity  of forms and functions of
nominals is presented below. 


\subsection{Noun phrases}
\label{sec:GRM-verb-phrases}

A noun phrase (NP)  consist of a nominal head, and optionally, its dependent(s).
In this section,  the internal components of noun phrases and the roles these
components have within the noun phrase are described. First,   indefinite and
definite noun phrases are considered. Then, the elements which can be found in
the noun phrase are introduced. 

\subsubsection{Indefinite noun phrase}
\label{sec:GRM-np-indef}

Indefinite noun phrases are used when ``the speaker invites the addressee to
construe a referent [which conforms with] the properties specified in the term''
\citep[184]{Dik97}.  In Chakali, a noun standing alone can  constitute a noun
phrase. Such a 
noun phrase can be interpreted as indefinite, i.e. the noun phrase is a
non-referring expression,  or   generic, i.e. the noun phrase 
denotes  a kind or class of entity  as opposed to an individual.  In rare cases,
a
definite noun phrase can be interpreted from a single noun  (i.e. lacking  an
  article). Each interpretation is obviously dependent on the context of
the utterance in which the noun occurs.

\ea\label{GRM-np-type-indef}{\rm  N = NP}\\

 
\ea\label{GRM-np-indef-1}
\gll  kàlá jáwá   [pɪ́ɛ́ŋ]$_{NP}$ ná\\
     Kala buy mat {\sc foc}\\
\glt  `Kala bought a MAT' 


\ex\label{GRM-np-indef-2}
\gll  [dʒɛ̀tɪ̀]$_{NP}$ kɪ̀m-bɔ́n  ná \\
  {lion.{\sc sg}} {\sc clf}-dangerous.{\sc sg} {\sc foc} \\
\glt  `A lion is DANGEROUS'



\z 
 \z

In (\ref{GRM-np-type-indef}),  the noun phrase {\it  pɪɛŋ}  describes any mat
and
is interpreted as a novelty in the hearer's knowledge of Kala, while
{\it dʒɛtɪ} describes the entire class of lions. 

Noun phrases  containing the
numeral {\it dɪ́gɪ́ɪ́} `one'  may be translated as
English `a certain'.    The
expression {\it badɪgɪɪ} can be translated as `one of them', `someone' or
`anyone'  (e.g. {\it ʊ wa ja badɪgɪɪ}, {\it lit.} he-not-be-one.of.them, `he is
an
illegitimate child'). 

%Other strategies to introduce a specific object or set of
%objects that is believed to be new to the addressee is the 


\subsubsection{Definite noun phrase}
\label{sec:GRM-np-def}

Definite noun phrases are employed when ``the speaker invites the addressee to
identify a referent which he (the speaker) presumes is available to the
addressee'' \citep[184]{Dik97}. A definite noun phrase may consist of  a single
pronoun in subject position, as
shown in (\ref{GRM-np-type-pro}).

\begin{exe}
 \ex\label{GRM-np-type-pro}{\rm pro = NP}\\
\gll  [ʊ̀]$_{NP}$  sʊ́wáʊ́ \\
      {\sc 3.sg} die\\
\glt  `She died'
\z


A possessive noun phrase is always definite. A possessive pronoun followed by a
noun is analyzed as a succession of a noun phrase and a noun. Thus,  the noun
phrase in (\ref{GRM-np-type-pro-n})  is analyzed as a sequence of the noun
phrase {\it ʊ} and the  noun  {\it mãã}. 


\begin{exe}

 \ex\label{GRM-np-type-pro-n}{\rm pro + N = NP}\\
\gll   [ʊ̀ mã̀ã̀]$_{NP}$ ŋmá dɪ́ ʔői̋ \\
      {\sc 3.sg.poss} mother say  {\sc comp} {\sc interj}\\
\glt  `Her mother said, ``Oi!''.'
\z


The treatment  NP+N for possessive noun phrase is   motivated  by the
possibility of  recursion of  an attributive possession relation. The expression
{\it lɔɔlɪgbɛrbɪɪ}$_{N}$  `car 
key' is the head in the three possessive noun phrases   {\it karedʒa
lɔɔlɪgbɛrbɪɪ}
`Kareja's car key', {\it karedʒa ɲɪna lɔɔlɪgbɛrbɪɪ} `Kareja's father's car
 key' and {\it karedʒa ɲɪna bɪɛrɪ lɔɔlɪgbɛrbɪɪ}
`Kareja's father's senior brother's car  key'.  Notice that in these examples
  the nominal head consists of the right-most element in the noun phrase. The
compound noun  {\it  lɔɔlɪgbɛrbɪɪ}  `car-key', and correspondingly,
the head of the left daughter of the  possessive noun phrase is the right-most
element, e.g.   [[karedʒa ɲɪna \underline{bɪɛrɪ}]$_{NP}$
[lɔɔlɪgbɛrbɪɪ]$_{N}$]$_{NP}$,
{\it lit.} `the key of the brother of the father of
Kareja'. The syntactic tree in (\ref{ex:GRM-poss-def-tree}) illustrates the
structure of this definite noun phrase.

%add lɔɔlɪgbɛrbɪɪ to dict

\ea\label{ex:GRM-poss-def-tree}
\Tree[-1]{  &&&&&\it q{NP} \Bq{dlll}\Bq{drrr} &&&&\\
&&\it q{NP} \Bq{dll}\Bq{drr} &&&&&& \it q{N}\Bq{dl}\Bq{dr} &\\
\it q{karedʒa} &&&& \it q{NP} \Bq{dl}\Bq{dr} &&& \it q{lɔɔlɪ}  &&\it 
q{gbɛrbɪɪ}\\
&&& \it q{ɲɪna} && \it q{bɪɛrɪ} &&&&}
\z





\paragraph{Articles a and tɪŋ}
\label{sec:GRM-np-def-articles}
%This treatment predicts that the head of a possessive noun phrase 

There are two articles in Chakali; one which encodes specificity and the other
definiteness. The first one is the 
article {\it a} ({\it gl.} {\sc art1}) and the other is  {\it tɪŋ}  ({\it
gl.} {\sc art2}).  

The  article {\it a} is translated with the English article
`the'.\footnote{A pre-nominal article is not found in Tampulma or Pasaale. The
fact that
Waali and Dagaare use an identical  article suggests that the definite article
{\it a}
is a contact-induced innovation.} It must precede the head noun and cannot
co-occur with the possessive pronoun.  In the context of
(\ref{GRM-np-type-det1}), the speaker assumes that the hearer is informed about
Kala's interest in buying a mat. 


\begin{exe}
 \ex\label{GRM-np-type-det1}{\rm  a + N = NP}\\
\gll  kàlá jáwá  [à pɪ́ɛ́ŋ]$_{NP}$ ná\\
     Kala buy {\sc art1}  mat {\sc foc}\\
\glt  `Kala bought the MAT' 
\z


The type of mat,  its color or the location where Kala bought the mat and so on
are not necessarily shared pieces of information between the speaker and hearer
in (\ref{GRM-np-type-det1}).  The only information the speaker believes they
have in common is Kala's interest in purchasing a mat. The article {\it a} is 
treated as a functional word which makes the noun phrase specific but not
necessarily
definite.  When a noun phrase is  specific, the speaker should have a particular
referent in mind whereas the addressee may or may not share this knowledge.


The article {\it tɪŋ}  ({\it gl.} {\sc art2}) can also be seen to correspond to
English `the',  but a preferable paraphrase would be `as referred previously' or
 `this (one)'.  The article {\it  tɪŋ} appears when the speaker knows that the
hearer will be able to identify the referent of the noun phrase. In that sense,
the referent is familiar.\footnote{In the giveness hierarchy of
\citet[278]{Gund93}, the status {\it familiar} is reached when ``the addressee
is able to uniquely identify the intended referent because he already has a
representation of it in memory.''}   When {\it tɪŋ} follows a noun, the referent
must either have been mentioned previously or the speaker and addressee have an
identifiable referent in mind. Thus, compared to the examples
(\ref{GRM-np-type-indef}) and (\ref{GRM-np-type-det1}) above, a proper
interpretation of example (\ref{GRM-np-type-det2}) requires that both the
speaker and addressee have a particular mat in mind. In terms of word order, the
article  {\it a}  initiates the noun phrase and  the article {\it tɪŋ}  belongs
near the end of the noun phrase. 
 

\begin{exe}
 \ex\label{GRM-np-type-det2}{\rm   (a +) N + tɪŋ = NP}\\
\gll  kàlá jáwá  [à pɪ́ɛ́ŋ  tɪ́ŋ]$_{NP}$ nā\\
     Kala buy {\sc art1}  mat {\sc art2} {\sc foc}\\
\glt  `Kala bought the MAT'
\z

Consider the slight meaning difference between
(\ref{GRM-np-type-det2-a}) and (\ref{GRM-np-type-det2-b}).


\ea\label{GRM-np-type-det2-ab}
% \vspace{-12pt}
 
  \ea\label{GRM-np-type-det2-a}
\gll ɲɪ̀nɪ̃̀ɛ̃́ ɪ̀ ɲɪ́ná kà dʊ́\\
    {\sc q} {\sc 2.sg.poss} father {\sc  egr} be  \\
\glt  `How is your father?'

  \ex\label{GRM-np-type-det2-b}
\gll ɲɪ̀nɪ̃̀ɛ̃́ ɪ̀ ɲɪ́ná tɪ́ŋ kà dʊ́ \\
   {\sc q} {\sc 2.sg.poss} father {\sc art2} {\sc  egr} be \\
\glt  `How is your father?'
  
 
\z 
 \z


Both sentences may be translated with `How is your father?'. However, whereas 
the sentence (\ref{GRM-np-type-det2-a}) can request  a general description
of the father (i.e. skin color, size, general health, etc.), the sentence
in (\ref{GRM-np-type-det2-b}) asks for a particular aspect of the
father's condition which both the speaker and the addressee are aware of, for
instance the father's sickness. As sketched above, the article {\it tɪŋ}  in
(\ref{GRM-np-type-det2-b}) establishes that a particular disposition of the
father is known  by both the speaker and the addressee,  and the speaker
asks, with the question word {\it ɲɪnɪ̃ɛ̃} `how',   for details. 

The two  articles {\it a} and {\it tɪŋ}  are not in complementary distribution.
The article {\it tɪŋ} may occur following the head of a possessive noun phrase,
although it is not attested  following a weak pronouns. When the articles {\it 
a} and {\it tɪŋ} co-occur,  language consultants could omit
the preposed {\it a}  without affecting the interpretation of the
 proposition. 

Notice that any of the terms {\it article}, {\it determiner} and
{\it demonstrative} could have been used to identify {\it a} and {\it tɪŋ}.
Similar forms/functions are labelled differently in the
literature; for instance \citet[47]{Bodo97} calls the prenominal {\it a} in
Dagaare an article (`the')  and  {\it nyɛ} a demonstrative (`this').  Yet, an
analysis of the paradigms in (\ref{GRM-np-det-dem}),  in which Central Dagaare
is
included for illustration,\footnote{Central Dagaare as it is spoken in Nadowli,
Sombo, Dafɪama, etc. Thanks to John Gaanaa for providing the examples. } has
never been offered. It is shown that a sequence of two demonstratives (in
\citeauthor{Bodo97}'s term)  can occur in postnominal postition.


% \begin{minipage}[h]{12cm}
\ea\label{GRM-np-det-dem}{\rm Corresponding  noun phrases in Chakali and
Central  Dagaare} 

\begin{multicols}{2}

 \ea\label{GRM-np-det-dem-dag}{\rm Central Dagaare}\\
à bíé   `the child'\\
à bíé ŋà   `the child this'\\
à bíé  ŋánɛ́ɛ́ ŋà  `the child this this'\\
 \textasteriskcentered  a bie  ŋa ŋanɛɛ  

 \ex\label{GRM-np-det-dem-cli}{\rm Chakali}  (no frame, see p133 for frame)\\
à bìé   `the child'\\
à bìè tɪ́ŋ̀   `the child this'\\
à bìè háŋ̀ tɪ̀ŋ  `the child this this'\\
 \textasteriskcentered a bie  tɪŋ haŋ  

 
\z
\end{multicols}
 \z




The question raised by paradigm  (\ref{GRM-np-det-dem}) is  whether one
should call {\it tɪŋ} a demonstrative (and not an article) based  on the English
glosses  supplied by  \citet[47]{Bodo97}, or show that Chakali and Central
Dagaare stack determiner-like function words in an unusual way and that this is
a problem for the general description of noun phrases.\footnote{The
demonstrative {\it haŋ} `this' is presented in Section \ref {sec:GRM-demons}.}
The latter position is
chosen and a description of  {\it a} and {\it tɪŋ} is provided above. The
noun phrase {\it a bie haŋ tɪŋ} means `this child of which we (both speaker and
hearer)   have a familiar/common representation'.  The discourse implications
 of  {\it a} and  {\it tɪŋ}  need 
further study. 

Now that the indefinite and definite noun phrases have been presented, the
subsequent sections introduce the elements which can compose  either  indefinite
or  definite noun  phrases.


% The position taken here is
% that only article and demonstrative exist in the language; as opposed to the
% latter,  the former lacks deictic power and cannot function as noun
% phrase on its own.  


\subsection{Nouns}
\label{sec:GRM-noun}

In this section the elements admitted in the
schematic representation (\ref{sec:GRM-noun-strcut}) are discussed.

\ea\label{sec:GRM-noun-strcut}
[[ {\sc lexeme}]$_{stem}$ - [{\sc noun class}]]$_{n}$
\z

A stem may have 
nominal or verbal lexeme status. The latter has either a state (i.e. stative) or
a process (i.e. active) meaning.  A stem can be either atomic or complex and a
noun class suffix may be overt or covert.  In a
 process which turns a lexeme into a noun-word,  the noun class provides the
syntactic category {\it noun}. 




\subsubsection{Noun classes}
\label{sec:GRM-noun-classes}

The accepted view is that ``the Gurunsi languages, and indeed all Gur languages,
had historically a system of nominal classification which was reflected in
agreement. The third person pronominal forms and other parts of speech were at a
certain time a reflection of the nominal classification''  \citep{Nade89}.
 Similar affirmations are present in \citet{Mane69b, Waa71, Nade82, Nade98,
Tcha07}.  In this section and in Section
\ref{sec:GRM-gender}, it is suggested that
an eroded form of this `reflection' is still observable in Chakali.
\citet{Brin07c} claims that in Chakali inflectional class
(i.e. noun class) and agreement class (i.e. gender) should be distinguished and
analyzed as separate phenomena at a synchronic level.

 The identification of noun classes is based on non-syntagmatic evidence; noun
class is a type of inflectional  affix, independent of agreement
phenomena, where the values of number
and class are exposed. In Chakali, as in all  other Southwestern Grusi  
languages,\footnote{\citet[136]{Nade98} state that ``[i]n
Vagla most traces
of this [noun-class system where paired singular/plural noun affixes correlate
with concording pronouns and other items] system have been lost. The
morphological declensions of nominal pluralization have not yielded to a clear
analysis''.  Even though the authors do not attempt to allot nouns into classes,
Marjorie Crouch's field notes (1963, Ghana Institute for Linguistics, Literacy
and Bible Translation (GILLBT)) present seven classes. Nominal classifications
are proposed for other SWG languages (number of classes for each language in
parenthesis): Sisaala of Funsi in \citet{Rowl66} (2), Sisaala-Pasaale in
\citet{Mcgi99} (5) and Isaalo in \citet{Mora06} (4).  The number of classes is 
of
course determined by the linguist's analysis.\label{foot:noun-class}}  the
values are exposed by
suffixes: number refers to either singular or plural, and class can be regarded
as phonological and/or semantic features encoded in the lexemes for the
selection
of the proper pair of singular and plural suffixes. This will be considered in
Section \ref{sec:GRM-sem-ass-crit}. 



 \begin{table}[!h]
 \caption{The five most frequent noun classes \label{tab:GRM-synop-nc}}
   \centering
   \begin{Itabular}{lccccc}

 \lsptoprule
             &  {\sc cl.1} & {\sc cl.2}  & {\sc cl.3} & {\sc cl.4} & {\sc cl.5} 
 \\  [1ex] \midrule
{\sc sing} & -V&  \O&  \O& -V  & \O \\
{\sc plur} & -sV& -sV & -V & -V  & -nV\\ 
 \lspbottomrule
   \end{Itabular}
 \end{table}



One method used to identify the noun classes of a language appears in
\citet[23]{Rowl66}. The author writes that ``[t]he nouns in Sissala may be
assigned to groups on the basis of the suffixes for singular and plural''. 
 According to this definition, there are nine noun 
classes, of which four are rare.   A synopsis is displayed in Table 
\ref{tab:GRM-synop-nc}, and each
of them is discussed below. 



\paragraph{Class 1}
\label{sec:class1}

Class 1 allows a variety of stems:  CV, CVC, CVVCV,  and CVCV are possible.
It gathers the nouns whose singular is formed by a single vowel
suffix {\it -V} and plural by a
light syllable {\it -sV}.


\begin{table}[h]

\caption{Class 1 \label{tab:freq-noun-class-1}}
\centering
%\subfloat[][{\sc class 1}]{
 \begin{Itabular}{lllll}
  \lsptoprule
{\sc class} & Stem    & {\sc sg.} &   {\sc pl.} & Gloss \\ [1ex] 
\midrule
{\sc cl.1}  &   va   &  váà   &  vá{\T ꜜ}sá  & dog \\ 

%{\sc cl.1}  &  hɛn   &  hɛ̀ná   &  hɛ̀nsá  & bowl \\
%{\sc cl.1}  &  da   &  dáá   &  dààsá & tree\\

{\sc cl.1}  &  pɛn   &  pɛ̀ná   &  pɛ̀nsá  & moon\\
{\sc cl.1}  &  gun   &  gùnó   &  gùnsó  & cotton \\
{\sc cl.1}  &  tʃuom   & tʃùòmó  & tʃùònsó   & togo hare\\
{\sc cl.1}  &  bi   &  bìé   &  bìsé  & child\\
{\sc cl.1}  &  gbiegi   &gbìègíè   &  gbìègísé  & type of hawk  \\

  \lspbottomrule
 \end{Itabular} 
 %}

\end{table} 

 The quality of the vowels of the singular and plural is
determined by
the quality of the stem vowel and the harmony rules in operation. The rules were
stated in Section \ref{sec:vowel-harmony} and correspond to the noun class
realization rules given in 
(\ref{ex:GRM-Hrules}).


\ea\label{ex:GRM-Hrules}

\ea\label{ex:mod-front-suffix}
-(C)V$_{nc}$  $>$ [ $\beta${\sc ro},  {\sc +atr}, {\sc -hi}]  / [ $\beta${\sc 
ro}, 
{\sc +atr}] C* \_  \\

A noun class suffix vowel becomes {\sc +atr} if preceded by a {\sc +atr}
stem vowel, and shares the same value for the
feature {\sc ro}  as the one specified on the preceding (stem) vowel. A noun
class suffix is always {\sc -hi}.

 \ex\label{ex:low-suffix}
-(C)V$_{nc}$  $>$ {\sc +lo}  / {\sc -atr} C* \_ \\

A noun class suffix vowel becomes {\it +lo} if the preceding stem vowel is 
either
{\it ɪ}, {\it ɛ}, {\it ɔ}, {\it ʊ} or {\it a}.\\

\z 
 \z


 



 
 \paragraph{Class 2}
\label{sec:class2}
 
Table \ref{tab:freq-noun-class-2} displays  nouns assigned to
class 2. Typically, this class consists of nouns whose stems are CVV or CVCV.
While the singular form  displays no overt
suffix,  {\it -sV} is suffixed onto the stem to form the plural.  

\begin{table}[h]


\caption{Class 2 \label{tab:freq-noun-class-2}}
\centering
%\subfloat[][{\sc class 2}]{
 \begin{Itabular}{lllll}
  \lsptoprule
{\sc class} & Stem    & {\sc sg.} &   {\sc pl.} & Gloss \\ [1ex] 
\midrule

%{\sc cl.2}  &nãã &  nã̀ã̀    &    nã̀ã̀sá & leg \\
{\sc cl.2}  &  daa   &  dáá   &  dààsá & tree\\
%{\sc cl.2} &  tii     &  tìì   &  tìsè  & akee tree  \\
{\sc cl.2}  &  bɔla    &    bɔ̀là   &  bɔ̀làsá  &  elephant\\
%{\sc cl.2}  &  bʊɔ    &  bʊ̀ɔ́   &  bʊ̀ɔ́sá  &  hole\\
%{\sc cl.2}  &  joŋ   &  jòŋ́   &  jósó  & slave  \\
%{\sc cl.2} &  ziŋ  &  zíN   &  zísé  & tail \\
{\sc cl.2} &  kuoru    &  kùórù   &  kùórùsó  & chief \\
{\sc cl.2} &tomo &tòmó & tòmòsó& type of tree\\

%{\sc cl.2} &  ŋmɛŋ   &  ŋmɛ̀ŋ   &  ŋmɛ̀sà  & rope \\
%{\sc cl.2}  &   tuto    &tútò   &  tùdùsó  & mortar  \\
{\sc cl.2} &  bele  &    bèlè  &  bèlèsé &type of wild dog  \\
{\sc cl.2} & tii   &  tíì    &  tíìsè & type of tree \\
%put as type of tree
  \lspbottomrule
 \end{Itabular} 
% }

\end{table}



 The rules in  (\ref{ex:GRM-Hrules}) capture the majority of the
singular/plural pairs of class 1 and 2. However, it is insufficient in some
cases, that is, there are cases which raise uncertainty in the allotment of
the pairs into one class or the other. Consider the examples in
Table \ref{tab:uncer-noun-class}.


\begin{table}[h]
\caption{Uncertain class 1 or 2 \label{tab:uncer-noun-class}}
\centering
%\subfloat[][{\sc class 3}]{
 \begin{Itabular}{lllll}
  \lsptoprule
 {\sc sg.} &   {\sc pl.} & Gloss \\ [1ex] 
\midrule
dʊ̃́ʊ̃̀  & dʊ̃́s{\T ꜜ}á  &   python\\
kìrìmá & kɪ̀rɪ̀nsá & tsetse fly\\
kɔ̀wɪ̀ɛ́   & kɔ̀wɪ̀sá   & soap\\
lɛ́hɛ́ɛ́  & lɛ̀hɛ̀sá&  cheek\\
%hõõ  & hõsa  & grasshopper\\
  \lspbottomrule
 \end{Itabular} 


\end{table}


 Two questions are raised by looking at the data in table
\ref{tab:uncer-noun-class}: (i) What is the stem of these nouns?  (ii) Is
there a good reason to favor final vowel deletion instead of insertion, e.g.
 /kɪrɪma/ vs. /kɪrɪm/?
Addressing  the first question, consider the first pair of words of table
\ref{tab:uncer-noun-class}, i.e. {\it dʊ̃ʊ̃}  and {\it dʊ̃sa}. On the one hand, 
if
 {\it dʊ̃} is treated as   the stem and  the word for `python' is assigned to
class
1,   the refutation of the rule in   (\ref{ex:GRM-Hrules}) must be explained,
i.e.
vowel suffixes are always {\sc -hi}.  On the other hand, if  the stem
is  {\it dʊ̃ʊ̃},  a deletion rule which reduces the length of the 
vowel, i.e. {\it /dʊ̃ʊ̃-sa/}  $\rightarrow${\it [dʊ̃́s{\T ꜜ}á]},  must be 
stated.
Such a decision  would
assign
the word for `python' to class 2.  The decision taken here is to respect the
rule in
(\ref{ex:GRM-Hrules}), which is empirically supported, and assume an {\it ad
hoc} deletion rule. The deletion rule may be explained by general prosody,
something which is not considered here. The word pairs in table
\ref{tab:uncer-noun-class} are assigned the following classes: `python' is in
class 2 and the last stem vowel is deleted in the plural, `tsetse fly' is in
class 1 and its stem is /kirim/, `soap' is in  class 1 and its stem is /kɔwɪ/,
and finally  `cheek' is in class 2 and the last stem vowel is
deleted in the plural.



 \paragraph{Class 3}
\label{sec:class3}

Nouns in class 3 generally have a sonorant coda consonant, i.e. {\it l}, {\it 
n}, 
{\it r}, etc. Class 3 contains nouns whose singular forms have no overt
suffix and plural forms  which have a single vowel as suffix. As for class 1 and
2, the
plural vowel suffix of class 3 is determined by the harmony rule given in
(\ref{ex:GRM-Hrules}).



\begin{table}[h]
\caption{Class 3 \label{tab:freq-noun-class-3}}
\centering
%\subfloat[][{\sc class 3}]{
 \begin{Itabular}{lllll}
  \lsptoprule
{\sc class} & Stem    & {\sc sg.} &   {\sc pl.} & Gloss \\ [1ex] 
\midrule

{\sc cl.3}  &  nɔn    &  nɔ́ŋ   &  nɔ́nã́  &  fruit\\
%{\sc cl.3}  &  ɲiŋ    &  ɲíŋ   &  ɲíŋá  &  tooth \\
%{\sc cl.3}  &  par    &  pár   &  párá  &  hoe\\
%{\sc cl.3  &  kUr    &  k\'Ur    &  k\'Ur\'U  & bench  \\
{\sc cl.3}  &  hããn    & hã́ã̀ŋ   &  hã́ã̀nà  & woman \\
{\sc cl.3}  &  gɔŋ    &  gɔ́ŋ     &  gɔ́ŋá  & river  \\
%{\sc cl.3}  &  hamoŋ    &  hàmõ̀ŋ     &  hàmõ̀nà  & child  \\
{\sc cl.3}  &  nar    &  nár   &  nárá  &  person\\
%{\sc cl.3  &  bUɔ ŋ   &  bʊʊ̀ŋ    &  b\'Uná  & goat \\
%{\sc cl.3}  &  tɔn    &  tɔ́ŋ    &  tɔ́ná  & skin/book  \\
%{\sc cl.3}  &  sɔŋ    &  sɔ́ŋ    &  sɔna  &  name\\
{\sc cl.3}  &  ʔol     &  ʔól    &  ʔóló  & type of mouse \\
%{\sc cl.3}  &  ʔul     &  ʔúl    &  ʔúló  &  navel\\
{\sc cl.3}  & butet    &   bùtérː  &  bùtété & turtle \\
{\sc cl.3}  &   sel  &   sélː  & sélé  & animal \\
  \lspbottomrule
 \end{Itabular} 
 %}
 

\end{table}
 
 
 
 \paragraph{Class 4}
\label{sec:class4}

The major characteristic of class 4 is that all the stems have a final
syllable consisting of  {\sc [+hi, -ro]} vowel(s) (see table
\ref{tab:freq-noun-class-4}) .  Class 4 is analyzed in the following way: in
both  the singular and the plural, a  vowel is added to the stem, i.e. V]\# $>$
V]-V\#. The suffix vowel of the singular is always an exact copy of the stem
vowel.  If the stem vowel is {\sc [+atr, +hi]} the plural suffix vowel is {\it 
-e},
 and if the stem vowel is  {\sc  [-atr, +hi]}, the  plural suffix vowel  {\it 
-a}.
This low vowel is then raised due to the height of the stem vowel. In normal
speech, one can perceive either  {\it -a} or {\it -ɛ} in that position. Given 
the
rules in (\ref{ex:GRM-Hrules}),  class 4 is certainly the most problematic in
terms of uniformity. However,  class 4 is productive. A similar noun class was
found in both Tampulma, Vagla,  and  Dɛg. 


 
 \begin{table}[h]
\caption{Class 4 \label{tab:freq-noun-class-4}}
\centering

 \begin{Itabular}{lllll}
  \lsptoprule

{\sc class} & Stem    & {\sc sg.} &   {\sc pl.} & Gloss \\ [1ex] 
\midrule

{\sc cl.4}  &  begi   &  bégíí    &  bégíé  & heart \\
{\sc cl.4}  &  si   &  síí    &  síé  & eye\\
{\sc cl.4}  &fili &fílíí&fílíé&bearing tray\\
{\sc cl.4}  &  bɪ   &  bɪ́ɪ́    &  bɪ́á  & stone \\
{\sc cl.4}  &  wɪ   &  wɪ́ɪ́    &  wɪ́ɛ́  & matter, thing  \\
{\sc cl.4}  &  wɪlɪ   & wɪ́lɪ́ɪ́   &  wɪ́lɪ́ɛ́  & star \\
  \lspbottomrule
 \end{Itabular}
\end{table} 



Class 4 also includes nominalized verbal lexemes.  In Section
\ref{sec:GRM-verb-act-stem},  it is shown that one way to make  a noun from a
verbal lexeme is
to suffix a  high-front vowel to the verbal stem. For instance,  the lexeme  
{\it zɪn} may be translated into English `drive', `ride' or `climb'. In the 
word  
{\it kɪ́nzɪ̀nɪ́ɪ́} `horse', {\it lit.} thing-riding, the suffix  -[{\sc +hi,
-ro}]  is added to the verbal lexeme {\it zɪn} making it nominal.
Consequently,
the plural of {\it kɪ́nzɪ̀nɪ́ɪ́} `horse'  is {\it kɪ́nzɪ̀nɪ́ɛ́}. The sequences 
{\it -ie} and {\it -ɪɛ} of class 4  often coalesce and may be  perceived as 
{\it -ee}
and {\it -ɛɛ}
respectively. 
 
 
 \paragraph{Class 5}
\label{sec:class5}


 The monosyllabic stems of class 5  nouns can either be CVV or CVC. Class 5
consists of nouns which  form their singular with no overt suffix and form their
plural with the suffix {\it -nV}. The quality of the consonant is determined by
the stem and the place assimilation rules introduced in Section
\ref{sec:focus-forms}, some of which are repeated in  (\ref{GRM-cl-5}). The
vowel of the plural suffix is determined by the stem vowel and the rules in 
(\ref{ex:GRM-Hrules}). 
 


\ea\label{GRM-cl-5}
{\it Class 5 suffix -/nV/ surfaces -[lV] if the  coda consonant of the stem is
[l].}\\
{\sc -/[nasal]V/}$_{nc}$  $>$  {\sc -/[lateral]V/}$_{nc}$  /  {\sc [lateral]} 
\_\\

\z


 
 \begin{table}[h]
 \caption{Class 5 \label{tab:freq-noun-class-5}}
\centering
 \begin{Itabular}{lllll}
  \lsptoprule
{\sc class} & Stem    & {\sc sg.} &   {\sc pl.} & Gloss \\ [1ex] 
\midrule

{\sc cl.5}  &  zɪn    &  zɪ̀ŋ́    &  zɪ́nná  &  type of bat \\
{\sc cl.5}  &hʊ̃n&hʊ̃̀ŋ́&hʊ̃́nná& farmer or hunter gear\\
{\sc cl.5}  &  kuo    &  kùó   &  kùónò  & farm \\
{\sc cl.5}  &  ɲuu    &  ɲúù   &  ɲúúnò  & head  \\
%{\sc cl.5}  &  sũũ    &  sũ̀ũ̀   &  sũ̀ũ̀nó  &  guinea fowl \\
{\sc cl.5}  &  vii    & víí   &  vííné &   type of cooking pot\\
{\sc cl.5}  &din&díŋ & dínné & fire \\
{\sc cl.5}  &pel &pél & péllé & burial specialist\\

  \lspbottomrule
 \end{Itabular}
\end{table} 
 

 \paragraph{Nasals in noun classes}
\label{sec:gene-sum}


 
Apart from the singular of class 4,  much of the same vocalic morpho-phonology
is found in all classes. This was reduced to the two rules in
(\ref{ex:GRM-Hrules}). Furthermore, in all the noun classes, the nasal
consonants surface differently depending on the phonological context. The rules
in  (\ref{ex:GRM-nrules}) predict the observed outputs and are derived from the
nasal assimilation rules in Section \ref{sec:internal-sandhi-nasal-place}.
 
\ea\label{ex:GRM-nrules}\textit{Possible outputs of  nasals}\\

\ea\label{ex:GRM-Nrules}
 C[{\sc +nasal}]\   $>$ ŋ / \_ \# \\
 /hããn/  $>$     [hã́ã̀ŋ]  `female' {\sc cl.3.sg}   


\ex\label{ex:GRM-Msrules}
 /m/ $>$ n / \_  C [{\sc -labial, -velar}] \\
 /tʃuom/   $>$ [tʃùònsó]   `togo hares'  {\sc cl.1.pl}  

\ex\label{ex:GRM-NGsrules}
 /ŋ/ $>$ n / \_  C [{\sc -labial, -velar}] \\
/kɔlʊ̃ŋ/ $>$  [kɔ̀lʊ̀nsá]  `wells'   {\sc cl.2.pl} 


\z 
 \z

The rule in  (\ref{ex:GRM-Nrules})  says that  any nasal consonant occurring
word finally becomes [ŋ]. The rule in (\ref{ex:GRM-Msrules}) changes a bilabial
nasal into an alveolar when it precedes a non-labial and non-velar consonantal
segment. The rule in (\ref{ex:GRM-NGsrules}) changes a velar nasal into an
alveolar in the same environment.

 
 \paragraph{Generalization and summary}
\label{sec:gene-sum}

While the method proposed suggests that one should look for pairs of forms, the
present classification treats phonologically empty suffixes as `exponents'. What
counts as a noun class is the paradigm determined by the  inflectional
pattern of the lexeme. The five  most frequent pairs were presented in tables
\ref{tab:freq-noun-class-1} to \ref{tab:freq-noun-class-5} and the exponents are
gathered in  Table \ref{tab:GRM-nc-exponent}.\footnote{The percentage is based
on a list
of 978 singular/plural pairs  (lexicon 02/10/10 version). The five classes in
Table \ref{tab:GRM-nc-exponent} make up 88\% of the nouns which are assigned a
class in the lexicon.} 
%see all numbering/percentage as it was changed

 \begin{table}[!h]
 \caption{The five most frequent noun classes   \label{tab:GRM-nc-exponent}}
   \centering
   \begin{Itabular}{lccccc}
 \lsptoprule
             &  {\sc cl.1} & {\sc cl.2}  & {\sc cl.3} & {\sc cl.4} & {\sc cl.5} 
 \\  [1ex] \midrule
{\sc sing} & -V&  \O&  \O& -V  & \O \\
{\sc plur} & -sV& -sV & -V & -V  & -nV\\ \midrule
                &      8\%&     32\%  &     23\% &   17\%   & 8\%\\
 \lspbottomrule
   \end{Itabular}
 \end{table}



In practice the most productive and regular patterns are those recognized as
noun classes. However, some words do not fit perfectly into the patterns
described
above but are not totally alien to genetically related languages and the
reconstructions of Proto-Grusi in \citet{Mane69a, Mane69b} and
Proto-Grusi-Kirma-Tyurama  in \citet{Mane82}.   In fact, there are more
possibilities
and surfaces forms when the  classes  ({\it sg./pl.})  {\it \O/\O},  {\it 
\O/ta},
{\it \O/ma} and {\it ŋ/sV} are included in the classification. Examples are 
given  in table
\ref{tab:GRM-less-pro-nc}.  
 
\begin{table}[h]
\caption{Noun classes 6, 7, 8, and 9 \label{tab:GRM-less-pro-nc}}
\centering
  \begin{Itabular}{lllll}
  \lsptoprule
{\sc class} & Stem    & {\sc sg.} &   {\sc pl.} & Gloss \\ [1ex] 
\midrule

{\sc cl.6}  & dʒɪɛnsa & dʒɪ́ɛ̀nsá & dʒɪ́ɛ̀nsá & twin\\
{\sc cl.6}  &kapʊsɪɛ &  kápʊ̀sɪ́ɛ́ & kápʊ̀sɪ́ɛ́ & kola nut\\
{\sc cl.6}  & kpibii & kpìbíí & kpìbíí & louse\\[0.2ex] \midrule

{\sc cl.7}  & kuo & kúó &kùòtó  &roan antelope\\
{\sc cl.7}  &kie  &kìé & kìété & half of a bird \\
{\sc cl.7}  &fɔʊ̃ &fɔ́ʊ̃̀& fɔ̀tá & baboon \\[0.2ex] \midrule
%{\sc cl.7}  &  taa    &  tàá    &  tàátá  & language  \\
{\sc cl.8}  & naal &náàl&nááləmà&grand-father\\
{\sc cl.8}  &ɲɪna &ɲɪ́nà&ɲɪ́námà &father\\
{\sc cl.8}  &  hɪɛŋ & hɪ́ɛ́ŋ &hɪ́ɛ́mbá &relative\\[0.2ex] \midrule

 {\sc cl.9}  &  jo   &jóŋ̀ & jósò  & slave  \\
{\sc cl.9}  & zi &zíŋ̀ &zísè &tail\\
{\sc cl.9}  & ŋmɛ&ŋmɛ́ŋ̀&ŋmɛ́sà&rope\\

  \lspbottomrule
 \end{Itabular} 

\end{table} 


 The nouns in class 6 do not formally differentiate singular and plural.   Those
in class 7 mark their plural with the suffix {\it -tV} and  class 8 with the 
suffix {\it -mV}.  The singular exponent of class 7 and 8 is covert. Finally,
the nouns of class 9 have a suffix {\it -ŋ} in the
singular and {\it -sV} in the plural. In Table \ref{tab:l-leasttive-class},  the
percentage of occurence of the less productive noun classes 6, 7, 8
and 9 is given.
 
  
 \begin{table}[h]
   \caption{Less productive  noun classes 
\label{tab:l-leasttive-class}}
   \centering
   \begin{Itabular}{lcccc}

  
 \lsptoprule
&  {\sc cl.6}   & {\sc cl.7}     &  {\sc cl.8}    & {\sc cl.9}    \\
[1ex] \midrule
    {\sc sing}   & \O     & \O      & \O & -N \\
{\sc plur}   & \O & -tV  & -mV & -sV\\ \midrule
                &      7\%&            1.8\% &   0.9\%   & 0.8\%\\
 \lspbottomrule
     
   \end{Itabular}
 \end{table}

In addition, there are pairs which can only imperfectly be reduced  to the nine
classes presented until now. However, the problem lies in the stem and not in
the inflectional patterns. For example the color terms ({\it sg./pl.}) {\it 
pʊ̀mmá}/{\it pʊ̀lʊ̀nsá} `white' and {\it búmmó}/{\it bùlùnsó} `black'  do
not have comparable pairs and do not fit the noun classes described above. One
would expect *{\it pʊmmasa} to be the plural form for  `white' (also *tɪɪnama 
for
{\it tɪ̀ɪ̀ná}/{\it tʊ́mà} `owner'). Other examples are the  pairs {\it 
tɪ́ɛ̀}/{\it tɛ́sà} `foetus' and {\it túò}/{\it tósó} `bow'. Also here, one 
expects  the
last vowel to delete in each of the plural forms instead of the penultimate one.
Moreover,  inconsistent class assignment across speakers, across villages, and
surprisingly different forms (predominantly in the plural) from the same speaker
on different elicitation sessions do arise, but the latter case rarely occurs. 
 

%also tʊ̀ɔ́nɪ́ã̀ tʊ̀ɔ́nsà  type of genet
%also kùòlíè  kùòlúsò  type of tree 
%apocope?
%  hɔ̃́ʊ̃̀
%  type of grasshopper
% \cl 2
%  hɔ̃́sà


 
\paragraph{Semantic assigment criteria}
\label{sec:GRM-sem-ass-crit}

Several authors have presented their views on  the semantic classification of
nominals.   The general idea is that there must be an underlying system which
can explain, first, why some words display identical number morphology, and
second, how these words are related in `meaning'. \citet[23]{Tcha07} shows that
Tem organizes its nominals on the basis of semantic values such as humanness,
size, and countability. \citet[41]{Awed07} argues that nominal groupings  in
Kasem should take into consideration phonological and semantic characteristics,
in addition to other more cultural factors.  Similarly, \citet{Assi07}
argues at length on the shortcoming of traditional semantic rules and argues for
abandoning them. 

The semantic value of the noun class suffixes has proven difficult to
establish. It is possible that there are analogies in class assignment based on
semantic criteria but it is more likely that synchronically (i) the phonological
shape of the stem triggers the suffix type, and that (ii) some classes can be
identified as residues of former semantic assignment. Let us comment on each
point: 


\begin{enumerate}
\item[(i)]

Most class 3 nouns have  a sonorant consonant in the coda position,
the stems of  class 4 nouns must have their last vowel specified for  [{\sc -hi,
-ro}] and a typical class 2 noun is either   CVV or CVCV.  These are some of
the characteristics  described for the noun classes. It seems that the
phonological
shape of the stem plays a role in class assignment and that there is no
productive class
where most of its  members are assigned to a particular semantic domain.  Using
four features of the animacy hierarchy
of  \citet{Comr89}, i.e.  human $[${\sc
hum}$]$, animal (exclude human) or other-animate
and insects $[${\sc anim}$]$, concrete inanimate $[${\sc conc}$]$ and abstract
(inanimate) $[${\sc abst}$]$,  \citet{Brin08} shows that the noun
classes  do not encode any of these distinctions. Such
distinctions may have
been  expected given the nominal classification of other Gur languages. For
instance in Dagaare, an Oti-Volta language and linguistic `neighbor' of
Chakali, \citet[124]{Bodo94} presents the Class 2 (V/ba) as ``unique in that it
is the only class that has exclusively [+human] nouns in it''. From a
diachronic point of view, this suggests that Chakali has dropped all animacy
distinctions in the noun class system while preserving one distinction in
agreement (see Section \ref{sec:GRM-gender}).



 \item[(ii)]
Geographically and genetically, languages related to Chakali had noun class
systems whose classifications were based, at least partially, on semantic
criteria. To my knowledge, the most conservative system today  within Grusi is
Tem (see {\it identification sémantique} in \citet{Tcha07}). When and how the
speakers of Chakali  classified nouns based on semantic criteria is impossible
to know,  but traces can be detected in  the   {\it less productive noun
classes}, that is class 6, 7, 8, and 9.
Some members of class 6 consist of
nouns with mass or abstract  denotations, i.e. rice,  louse, struggle, profit,
etc.  Class 7 also contains mass and abstract nouns, i.e. oil,  honey, water, 
and
 taboo, but also bush animals such as bushbuck, waterbuck, baboon, roan
antelope and hartebeest. Class 7 represents 1.8\% of the noun sample
(see Table \ref{tab:l-leasttive-class}) and  mass/abstract nouns and bush
animals
represent each  
30\% of class 7 membership. Class 8 is likely to be the class where kinship and 
human
classification terms were assigned, 
as
mother, father, and `owner of' are among
remnant
members of that class.  Finally, a  common trait of class 9 may be
`elongated things', since words referring to  rope, arm, tail, and ladder are
members. Yet, only eight nouns are
assigned to class 9. Despite the arbitrary nature of the semantic
assignment of class 9,  \citet[94]{Mane75} maintains that there are Oti-Volta
languages which show relics of  the Proto Oti-Volta class {\it *ŋu- *u},
which is  itself a remnant of Proto-Gur class 3   according to
\citet[11]{Mieh07}, and that this class contains ``les noms du bâton, du pilon,
du balai, de la corde, de la peau et du chemin''.  Although these nouns seem
to
 denote `elongated  things',   Manessy claims that they cannot contribute to
an hypothesis. Generally, howewer, the fact that members of classes 6, 7, 8, and
9 are
similarly clustered in other languages suggests that these classes are remnants
of a more productive semantic assignment system. %six nouns
Beside semantic
domains, the simple empirical fact that homonyms are found with
different suffixes excludes a purely phonologically-based class assignment.
There is no way a speaker can correctly pluralize the stems {\it tii} `type of
tree' and {\it tii} `type of ants' based entirely on their (segmental)
phonological shape.\footnote{I put segmental in parenthesis since  homonyms {\it
with the same tonal melody} belonging to two different
classes have not yet been  found. The pair {\it pól} ({\sc cl.5}) `river' and
{\it pól}
({\sc cl.3}) 
`vein' may be treated as one example, but their meanings point to a
common etymology. Nevertheless, \citet{Bonv88}, \citet{Awed07} and 
\citet{Tcha07}
provide data to support a similar claim.} 
 

\end{enumerate}

A combination of both (i) and (ii) seems consistent with the data
observed. Chakali speakers seem to acquire the noun classes as French or Dutch
speakers acquire
the grammatical gender of inanimate entities which lack natural gender.

%mostly semantically arbitrary
%a healthy mix of lexical storage and 
%rule-based semantic and phonologica prediction

Finally, class assignment in complex stem nouns indicates  that the 
denotation of a word plays no role in determining its noun
class (see Section \ref{sec:GRM-com-stem-noun}). 
The class of a complex stem noun is always determined by the rightmost stem.
Given that compounding is highly productive,  this purely  formal
process suggests that semantic criteria in
noun class assignment are inoperative.

\paragraph{Tone patterns of noun classes}
\label{sec:GRM-tone-p}

Some tonal melodies are identified. One of them is the  general tendency for
the singular and  plural words in a pair to display the same tonal melody. For
instance, a HL melody may be associated with both the singular and the plural,
e.g.  {\it zíŋ̀}/{\it zísè} `tail' ({\sc cl.9}) 
and   {\it lʊ́l̀}/{\it lʊ́là} `biological relation'  ({\sc cl.3}). These cases 
are tonally regular. 
Another common pattern is when a singular noun displays a H melody, but the
plural a LH melody, e.g.  {\it dáá}/{\it dààsá} `tree' ({\sc cl.2}). While 
it
 seems that  the
plural suffix -{\it sV}  depresses a preceding H,  it does not do so in class 9
nouns.
The majority of class 4 nouns in the data available are high tone irrespective
of the number of moras and they are all tonally regular. Some cases involving
singular CVC words with moraic
coda exhibit the deletion of a low tone;  {\it zɪ̀ŋ́}/{\it zɪ́nná} `bat' ({\sc 
cl.5}),   {\it gèŕ}/{\it gété} `lizard' ({\sc cl.3}) and
{\it sàĺ}/{\it sállá} `flat roof' ({\sc cl.3})  have a LH tonal melody in 
the
singular but  H in the plural. The downstep rule (Section
\ref{sec:tone-intonation})  predicts that a high tone preceded by a low tone is
perceived as lower than a preceding high
tone, e.g. {\it váà} {\it HL},  {\it vá{\T ꜜ}sá} {\it HLH}  `dog' 
({\sc cl.1}).  
In
spite of some variations,  it seems that there are recurrent
melodies. Representative examples are presented in
Table \ref{tab:GRM-tm-nc-1-5}.
 

 \begin{table}[htb!]
   \caption{Tonal melodies in noun classes 1--5
\label{tab:GRM-tm-nc-1-5}}
   \centering
   \begin{Itabular}{lp{1cm}lp{1cm}ll}

 \lsptoprule
{\sc class}    &  Tone  melody {\it sg.}  &   Singular   &  Tone  melody  {\it
pl.} &   Plural & Gloss
\\ [1ex]

\midrule

{\sc cl.1}   &   HL   & váà & HLH &   vá{\T ꜜ}sá  & dog\\
  &     LH &  gùnó&  LH& gùnsó  & cotton\\
& HL & tʃíníè & HL &  tʃínísè  & type of climber\\
  &    L &  dɪ̀gɪ̀nà  & LH & dɪ̀gɪ̀nsá&  ear\\[0.2ex] \midrule


{\sc cl.2}  &  H  &   síé &  LH   &
sìèsé&  face\\
&  L   &  bɔ̀là   &    LH&   bɔ̀làsá  &
elephant\\
&  LH &  tòmó  &  LH &  tòmòsó  &  type of
tree\\
&   LH   & sòntògó &     LH &   sòntògòsó & base \\
&  HL &  júò  &  HLH   &  júòsó  &
quarrel \\[0.2ex] \midrule


{\sc cl.3} &   H &   hóg  &   H   &   hógó  &   bone\\
&      H   &  sɔ́nná  &  H  &  sɔ́nnəsá
& lover\\
    &  LH &  gèŕ  &  HH  &  gété  
& lizard\\
    &  LH &  pààtʃák  &  LH   &  pààtʃàgá 
& leaf\\[0.2ex] \midrule


{\sc cl.4} &  H &  síí  &  H& síé  &  type of dance\\
    &  H &  bégíí    &  H & bégíé
& heart\\
& H & tʃɪ̃́ɪ̃́  & H & tʃɪ̃́ã́ & dawadawa seed\\
& LH& sòkìé & LH & sòkìété & type of tree \\[0.2ex] \midrule




{\sc cl.5}&H &  víí  &  H &  vííné
& cooking 
pot\\
        &  L &  bɔ̀g  &  L &  bɔ̀ɣənà   &
type of tree\\
        &  HL &  bámpɛ̀g  &  HL&   bámpɛ̀gənà  &
half of nut\\
        &  LH &  kùó  &  LHL &  kùónò  &  farm\\
        &  LH &   zɪ̀ŋ́  &  H&   zɪ́nná  &  bat\\
        &  L &  tʃàl   &  LH&   tʃàllá  &  blood\\
    &  LH &  sàĺ  &  H &  sállá    &  flat
roof\\

  
 \lspbottomrule
   \end{Itabular}
 \end{table}





\paragraph{Noun class reconstruction}
\label{sec:GRM-noun-class-recons}
The numerical labeling of the noun classes in Table \ref{tab:GRM-nc-exponent}
and \ref{tab:l-leasttive-class} is arbitrary. Given  the state of the
documentation on nominal classifications in other SWG languages, and the fact
that  almost all singular suffixes
have disappeared in today's SWG languages, a reconstruction  is practically
impossible. Nonetheless,  some 
observations  on similarities between the noun class
system in Chakali and other SWG noun class systems can be put forward. The
information sources are my field notes on neighboring languages, the
reconstruction of the
noun class suffixes of Grusi in \citet{Mane69a, Mane69b},  and the 
reconstruction
of noun classes in Gur in \citet{Mieh07}; the latter being for the most part an
up-date and synthesis of Manessy's work \citep{Mane69a, Mane69b, Mane75, Mane79,
Mane82, Mane99}. Needless to say, the following statements are first
impressions.

%need name of pasaale girl

Field notes on neighboring languages, supported with unpublished
material produced
by GILLBT's staff,\footnote{\label{ft:GRM-naden-donate}In 2008, Tony Naden gave
me  a copy of his ongoing
Vagla and Dɛg lexicons. I am also 
indebted to: Kofi Mensa (New
Longoro) for Dɛg, Modesta
Kanjiti  (Bole) for Vagla and Dɛg, Pastor Alex Kippo (Tuossa) for Vagla and 
Yusseh Jamani (Bowina) for Tampulma.}  provided relevant information on
the
(dis-)similarities of Chakali with other SWG languages. As in all SWG languages,
a typical Vagla noun class is characterized by  suffixation.
The most frequent plural markers in Vagla are {\it -zi}, {\it -nɪ} and
{\it -ri}. The pattern found in Chakali  class 4 is similar to the one found in
Vagla, i.e.
({\it sg.}/{\it pl.}) {\it bàmpírí}/{\it bàmpíré} `chest',  {\it 
hūbí}/{\it hūbé}  `bee' and   {\it gíngímí}/{\it gíngímé} `hill'.  In 
Dɛg,   the most
frequent plural markers are mid vowel suffixes, often rounded,  and
the {\it -rV}, {\it -nV} and  {\it -lV} suffixes, with which the vowel 
harmonizes
in
roundness and {\sc atr} with the stem vowel. Both Vagla and Deg display 
miscellaneous classes which are characterized by  a simple difference
in vowel quality between the last vowel of the singular and the plural, e.g. Dɛg
{\it dala}/{\it dale} `cooking place'. Attested
alternations  ({\it sg.}/{\it pl.}) in Vagla are {\it -i}/{\it -e},  {\it 
-i}/{\it -a},  {\it -a}/{\it -i}, 
{\it -u}/{\it -a},  {\it -o}/{\it -i} and  {\it -e}/{\it -i},   and in Dɛg {\it 
-a}/{\it -e}, {\it -e}/{\it -a}, {\it -i}/{\it -e}, {\it -o}/{\it -i} and  {\it 
-i}/{\it -a}.\footnote{These singular/plural pairings are extracted from the 
Vagla and
Deg lexicons (fn. \ref{ft:GRM-naden-donate}) and are not exhaustive.}  The noun
classes of Tampulma and Pasaale
correspond more to those of Chakali. Tampulma has at least
the
following class suffix pairs ({\it sg.}/{\it pl.}): {\it \O}/{\it -V}, {\it 
-i}/{\it -e}, {\it \O}/{\it -nV},  {\it  \O}/{\it -sV}, {\it  -V}/{\it -sV},  
{\it -hV}/{\it -sV} and  
{\it \O}/{\it -tV}.
Tampulma displays similar harmony rules to those found in Chakali. Apart from
the singular suffix {\it -hV}, all the noun class suffixes in Tampulma are
manifested in Chakali.  Correspondingly, Pasaale reveals  pairs and
harmony rules similar to those of Chakali and Tampulma.\footnote{As mentioned in
footnote
\ref{foot:noun-class}, the number of noun classes is determined by the
linguist's analysis.  \citet[5--12]{Mcgi99} is a good example of 
the consequence of analyzing noun classes differently. For instance,  
 \citet[7]{Mcgi99} postulate a subclass  ({\it sg.}/{\it pl.})  {\it -l/-lA} 
for 
word pairs like {\it baal/baala} `man', {\it gul/gulo} `group', 
{\it miibol/miibolo} `nostril' and  {\it mɔl/mɔlɔ} `stalk'. If these words were 
part
of the
Chakali data, they would have been alloted to class 2 ({\it -\O/-V}), that is,
I would have treated the /l/ as a coda consonant of the stem instead of a noun
class
suffix consonant. In addition, whereas I derive the quality of the vowel
entirely from harmony rules,  \citeauthor{Mcgi99} assume archiphonemes, like A
and E, which surface depending on harmony rules.}

It is important to keep in mind that the analysis in \citet{Mane69a, Mane69b} is
based on a very
limited set of SWG data,  most of the data being extracted from
\citet{Bend65}. He
stresses often the tentative nature of his claims and  sets forth more than one
 hypothesis on several occasions. The Chakali plural suffix of class 8 {\it -mV}
may be treated as a descendant of the Proto-Grusi Class {\it *B$_{1}$A}
\citep[32]{Mane69b}, class 9 {\it -ŋ} as a descendant of the Proto-Grusi Class
{\it *NE}   \citep[37, 41]{Mane69b}, class 1 {\it -V} as a descendant of the
Proto-Grusi
Class {\it *K$_{1}$A}  \citep[39]{Mane69b}, classes 1, 2, and 9 {\it -sV} as 
descendants
of the Proto-Grusi Class  {\it *SE}  \citep[39]{Mane69b} and class 7 {\it -tV} 
as
a descendant of the Proto-Grusi Class {\it *TE/O}  \citep[43]{Mane69b}. The 
vowel
suffixes of class 1 and 4 may also descend from the Proto-Grusi Class {\it *YA}
\citep[34]{Mane69b}. 

In consulting \citet[7--22]{Mieh07}, Chakali's  most frequent plural suffix  
{\it -sV}, found in class 1, 2, and 9, would seem to correspond to Proto-Gur 
Class 13
*{\it -sɪ}, the plural suffix of class 5 {\it -nV} to  Proto-Gur Class 2a *{\it 
-n.ba}
or Proto-Gur Class 10 *{\it -ni}, class 7 {\it -tV} to Proto-Gur Class 21 *{\it 
-tʊ} and class 8  {\it -mV}  to Proto-Gur Class 2 *{\it -ba}. The singular 
suffix
{\it -ŋ} would correspond to Proto-Gur Class 3 *{\it -ŋʊ}.


Needless to say, these observations  deserve further investigation. Even though
there is  literature to support the reconstruction of the Gur classes, little
can be done in the SWG
area unless descriptions of  nominal classifications in the languages  Winyé,
Vagla, Tampulma, Phuie,  Deg, Siti,  and the dialects of Sisaala  are made 
available. A synthesis of these
descriptions could be
compared to  `better-documented' nominal classfications of Grusi languages 
such as Kasem (Northern Grusi, \citet{Awed79, Bonv88, Awed03}),  Lyélé (Northern
Grusi, \citet{Delp79}),  Lama  (Eastern Grusi, \citet{Arit87, Ours89}), Kabiyé
(Eastern Grusi, \citet{Tcha07}),  Chala   (Eastern Grusi, \citet{Klei00}) and 
Tem
(Eastern Grusi, \citet{Tcha72, Tcha07}), to evaluate the Proto-Grusi noun class
suffixes of \citet{Mane69b} and Proto-Gur of \citet{Mieh07}, and to reconstruct
the nominal classifications of SWG  languages.


\subsubsection{Atomic stem nouns}
\label{sec:GRM-sim-bas-noun}

The notion of stem in the present context refers to the host of a noun class
suffix or the  host of a nominalizer, i.e. the element which conveys the lexical
meaning and  to which affixes attach. A stem can be either irreducible or
reducible morphologically: they are referred to as atomic  and complex stem
respectively.  Complex stems are presented in  Section
\ref{sec:GRM-com-stem-noun}.   An atomic stem is always a  nominal or a verbal
lexeme.  A verbal lexeme may either be of the type `state' or `process'. Three
types of nominalization formation (i.e. nominalizers) are attested: suffixation,
prefixation, and reduplication.  

%A few prefixes are attested. Some are treated as classifiers (see Section
%\ref{sec:}). Other formal means for nominalization are suffixation and
%reduplication. 

\paragraph{Nominal  stem}
\label{sec:GRM-nom-stem}
A nominal stem is a predicate denoting a class of entities.   Nouns composed by
the combination of  a nominal stem and a noun class affix are the most common. A
nominal stem can be juxtaposed with various noun class affixes, yielding forms
with
different meanings. For instance, the lexeme {\it baal} is associated with the
general meaning `man'. In a context where the lexeme is used in the singular,
{\it baal} can mean either `a man' or `a husband'. Given the same context but
used in the plural, the lexeme {\it baal} is disambiguated by the 
plural suffix it takes;  {\it baala} `men'  ({\sc cl.3}) and  {\it baalsa}
`husbands'  ({\sc cl.2}). Another
example is the lexeme {\it natɔʊ} `shoe'. The word {\it natɔwa} refers to `a 
pair
of shoes' whereas {\it natɔʊsa} refers to `pairs of shoes' or `unsorted shoes'.
Evidence from other Grusi languages suggests that the situation where    lexemes
are found in different noun classes was certainly a   more common
phenomenon than it is today \cite[126--128]{Bonv88}. This may coincide
with semantically richer
noun class suffixes. In addition, for many noun classes the singular forms are
not overtly marked and the plural forms are by and large less frequent. This
situation makes it difficult to provide the necessary evidence which would
demonstrate that nominal stems are found together with different noun classes.  

Nominal stems exist in opposition to the verbal ones. To classify a stem in such
a dichotomy, the linguistic test carried out consists of placing the stem in
several core predicative positions, i.e. positions where an
argument must appear. If the stem is perceived as grammatical in the given
context by 
language consultants, it cannot be nominal. For instance, in French the word
{\it bille}  `marble' cannot take a nominal argument in a non-genitive
predication, e.g. *Marie billait/a billé  `Mary marbled/has
marbled'.\footnote{\label{ft:GRM-fre-eng-deri}French and English are not 
appropriate languages to use
for the point I want to make  because of their preference  for the categorial
derivation n$>$v. I believe that
Chakali cannot as easily derive a verb from a noun.}


The examples in
(\ref{GRM-nom-or-verb}) illustrate a simple classification procedure. It uses a
frame where the predicate is in the 
perfective aspect and  the same 
predicate, as opposed to the argument,  is in focus.

\ea\label{GRM-nom-or-verb}
 
\ea /di/   {\rm`eat'}  $\rightarrow$ {\it ʊ̀  díjōó}  {\rm |{\sc 3.sg} 
eat.{\sc pfv.foc}| 
`he
ate'}

  \ex /kpeg/  {\rm `hard'}  $\rightarrow$ {\it ʊ̀ kpégéó} {\rm  |{\sc 3.sg} 
hard.{\sc
pfv.foc}| `he is strong'}

   \ex /sɪama/  {\rm`red'}  $\rightarrow$ *{\it ʊ sɪamao}, {\rm but}  {\it ʊ̀ 
sɪ́árēó} 
 {\rm |{\sc
3.sg} red.{\sc
pfv.foc}|  `it is
red'} 
   \ex /bi/  {\rm `child'} $\rightarrow$  *{\it ʊ bio} 
 % \ex //
 

\z 
 \z


The test displayed in (\ref{GRM-nom-or-verb}) shows that  {\it di} and  {\it 
kpeg} are verbal,  whereas {\it sɪama} and {\it bi} are not. In
Section \ref{}, it will be shown that color terms change forms
depending on whether
they occur  in a nominal or verbal context (TO DO).  


\paragraph{Verbal process stem}
\label{sec:GRM-verb-act-stem}

%Somewhere should be included a section on derivative affixes (and/or other
%types of nominalizer) 

Verbal process stems denote non-stative events. Table \ref{tab:GRM-nom-process} 
displays  two types of nominalization formation -- suffixation and 
reduplication -- involving verbal process stems,  `agent of X' and `action of 
X', where X replaces the meaning of the verbal process stem. 

\begin{table}[htb!]

\centering
\caption{Examples of nominalization of verbal process stem
\label{tab:GRM-nom-process}}
 \begin{Itabular}{llll}
 \lsptoprule
Sem. value & Verb. process stem & Nmlz & Form\\
 \midrule

Agent of X &  gʊ̀ɔ̀ `dance' &  -/r/ & gʊ́ɔ́r `dancer'\\
Agent of X &  kpʊ́  `kill' &   -/r/  & kpʊ́ʊ́râ  `killer'\\
Agent of X &   búól   `sing' &  reduplication &   bùòlbúóló  `singer'\\
Agent of X &   sùmmè `help' &  reduplication &   súsúmmá 
`helper'\\[1ex]\midrule

Action of X  &    gʊ̀ɔ̀ `dance' &  -/[{\sc +hi, -bk}]/ & gʊ́ɔ́ɪ́ɪ́ `dancing'\\
Action of X &  kpʊ́  `kill' &  -/[{\sc +hi, -bk}]/  & kpʊ́ɪ̀ɪ́  `killing'\\
Action of X  &  búól   `sing'  &-/[{\sc +hi, -bk}]/  & búólíí    
`singing'  \\
Action of X  &   sùmmè `help'   &-/[{\sc +hi, -bk}]/  &  súmmíí  `helping'
\\
\lspbottomrule
 
 \end{Itabular} 

\end{table} 


% The first column describes in prose the meaning of each nominalization,
%the second column provides the stem, the third column provides  the
%nominalization formations and the fourth provides the  translation.


 In Table \ref{tab:GRM-nom-process}, the column entitled semantic value (i.e.
Sem. value)  identifies the meaning of the verbal nominalization. In such a
context, `agent of X'  refers to the instigator or doer of the state of affairs
denoted by  the predicate X and
the nominalization is generally accomplished by the suffix -/{\it r(a)}/.  
However,  there are some expressions with the equivalent
agentive denotation which
do
not suffix    -/{\it r}/  to the  predicate, e.g.  {\it ʔɔra} `to sew' vs.
{\it ʔɔta} `sewer' and {\it maŋa} `to beat' vs.  {\it kɪŋmaŋana} `drummer'.  The
singular forms  are given in the fourth column: 
the plural of agent nominals  of
this type, i.e. nominalized by the suffix -/{\it r}/, is made by  a single vowel
suffix  ({\sc cl.3}) whose surface
form  depends on harmony rules.\footnote{One language consultant
had a problem retrieving the plural of some agent nouns. He often repeated the
singular entry
for the plural. I interpret this as  either a situation where agent nouns do
not show differences in the singular and plural ({\sc cl. 6}), or different {\it
sg.}/{\it pl.}
forms exist but he could not retrieve them. The pair {\it kpʊra}/{\it kpʊrəsa}
`killer(s)'   is unusual.  The word {\it  sãsaar} means
`woodcarver' and not `car driver'  even though {\it sãã} can mean both `carve'
and `drive vehicle'.  People usually use  {\it lɔ́ɔ́lɪ̀sã́ã́r}, or the English
word {\it dərávɛ̀}, 
 which is common all over Ghana, to refer to a `car driver'. } Another verbal
nominalization process conveying `agent of X' is reduplication. The evidence
suggests that  only the first syllable  is reduplicated.

The second nominalization process is  interpreted as `action of X' or
`process of X' and
consists of the suffixation of a  high front vowel to the verbal
 stem.\footnote{The nominalization `the process X' is often not distinguishable 
from
`the result of a process X'.  Does `dancing'  refer to `the process of dance', 
`the result of the process of dance' or both?} The surface form of the vowel
depends
on the quality of the stem vowel and {\sc atr}-harmony  (see
rule \ref{RULE-atr} in Section \ref{sec:vowel-harmony}). 
Consider example (\ref{ex:vp36.1.}).


\ea\label{ex:vp36.1.}
\glll ʊ̀ píílì wáɪ́ɪ́ rá \\
ʊ pile wa-ɪ-ɪ ra \\
        {\sc 3.sg} start come-{\sc nmlz}-{\sc cl.4} {\sc foc}\\
\glt  `He begins coming' 
\z

 The final vowels in the
words referring to `the process of X' are analyzed as a sequence of two
vowels: first a nominalizer suffix (i.e. {\sc nmlz})  on the verbal stem,  and
second,  a noun class suffix.  Such nominalized verbal stems are alloted to
noun class 4;  their singular suffix is a copy of the {\sc nmlz} vowel,
and their  plural suffix is the low vowel {\it a}, raised to a mid height, e.g.
{\it pɛrɪɪ}/{\it pɛrɪɛ} `weaving(s)'  ($<$ {\it pɛra} `weave', see class 4
in Section \ref{sec:class4}).

%The infinite forms are always prefixed by the low vowels. The transcription
%shows harmony in some instances but it is a topic to further investigate. 


\paragraph{Verbal state stem}
\label{sec:GRM-verb-state-stem}

Verbal state stems  denote static events. They generally
function  as verbs, but they can take the role of attributive modifiers in
noun phrases, referred to as  `qualifiers' in Section \ref{sec:GRM-qualifier}.
In that role, their semantic value is similar to the value of adjectives in
English: they denote a property  assigned to a referent.  To function as  a
qualifier, some verbal state predicates must be nominalized. As with verbal 
process stems,  verbal state stems are found in nouns which
have
been nominalized by suffixation of a  high front vowel, i.e. `the state of X'.
For instance, the verbal state predicate {\it kpeg} has a general meaning which 
can be
translated into English as `hard' and `strong'. The expression {\it kpégíí} 
in
{\it a tebul kpegii dʊa de} `The hard table is there' functions as qualifier in
the noun phrase {\it a tebul kpegii} {\it lit.} `the table hard'. 


\ea\label{exːGRM-v-sta-p-hard}{\rm Verbal state stem {\it kpeg}  `hard' in
complex stem nouns}\\
 
 \ea\label{exːGRM-v-sta-p-hard-head}
{\it ɲúú{\T ꜜ}kpég} $<$ {\sc head-hard} {\rm `stubborness'} 
%pl. ɲuukpegse

 \ex\label{exːGRM-v-sta-p-hard-arm}
{\it nékpég} $<$ {\sc arm-hard} {\rm `stingy'} 

 \ex\label{exːGRM-v-sta-p-hard-tree}
{\it dààkpég} $<$ {\sc wood-hard} {\rm `strong wood'}
 
\z 
 \z
 

 Examples  are provided in
(\ref{exːGRM-v-sta-p-hard}) using  {\it kpeg} again for
the sake of illustration.  Notice that only (\ref{exːGRM-v-sta-p-hard-tree}) has
a transparent and compositional meaning. Verbal state stems are 
usually found in complex stem nouns (Section \ref{sec:GRM-com-stem-noun}). 





%In a nominal context, the form in occurs in a  modifier position, i.e. ,  and
%the form in is found in compound formation, i.e. .



% Nevertheless there are  verbal state predicates which cannot be
% nominalized. For instance, the verbs {\it dʊa} and  {\it jaa} cannot
%list verb state predicate
%locative; dua
%identificational; ja


 


\subsubsection{Complex stem nouns}
\label{sec:GRM-com-stem-noun}


A  complex stem noun, as opposed to an atomic one,  is formed by the
combination of two stems (XY). Either X or Y in a  XY-complex stem noun may be 
atomic or complex.  Nominal stems ({\sc ns}), verbal state stems ({\sc ss}) and
verbal process stems ({\sc ps}), together with a single noun class  suffix 
(and/or other
types of nominalizer) are
the linguistic elements which take part in the
formation of complex stem nouns. 


\ea\label{exːGRM-cplx-stm}
 
 %\ex\label{exːGRM-cplx-stm-SS-AS} {\it deŋlii}  `straight'  (deŋ|lii
%= `single'|`thither' 
 % $>$  {\sc ss} + {\sc ps} + {\sc cl.4}.sg)

  \ea\label{exːGRM-cplx-stm-NS-NS-1}%
{\it nébíí} {\rm  `finger'}\\  %
ne-bi-i  $>$  {\sc arm-seed} \\%
 {\sc ns} + {\sc ns} + {\sc cl.3.sg}
 

  \ex\label{exːGRM-cplx-stm-NS-NS-2}
 {\it pàtʃɪ́gɪ́búmmò} {\rm  `secretive'}\\ %
patʃɪgɪ-bummo  $>$  {\sc stomach-black}  \\  %
 {\sc ns} + {\sc ns} (+ {\sc cl.1.sg})

 \ex\label{exːGRM-cplx-stm-NS-SS}
 {\it ŋmɛ́ŋhʊ̀lɪ́ɪ̀} {\rm   `dried okro'} \\ %
ŋmɛŋ-hʊl-ɪ-ɪ $>$   {\sc okro-dry}  \\  %
 {\sc ns} + {\sc ss} + {\sc nmlz}+ {\sc cl.4.sg}
 
 \ex\label{exːGRM-cplx-stm-PS-PS}
 {\it jàwàdír̄} {\rm  `business person'} \\ %
jawa-di-r  $>$ {\sc buy-eat-agent} \\  %
 {\sc ps} + {\sc ps} + {\sc nmlz} (+ {\sc cl.3.sg})

 
\z 
 \z



In (\ref{exːGRM-cplx-stm-NS-NS-1}) and (\ref{exːGRM-cplx-stm-NS-NS-2}),  all 
stems are nominal. In (\ref{exːGRM-cplx-stm-NS-SS}),  the verbal state stem {\it 
hʊl} `dry'  follows a nominal stem,  and  in  (\ref{exːGRM-cplx-stm-PS-PS}) both 
stems are of the type verbal process.  In these stem appositions, it is the noun 
class suffix of the rightmost stem which appears. Further, stems are lexemes, as 
opposed to nouns or verbs.  This is readily apparent in  
(\ref{exːGRM-cplx-stm-NS-NS-1}) and (\ref{exːGRM-cplx-stm-NS-NS-2}), in which 
the leftmost stems {\it ne} and {\it patʃɪgɪ} would appear as {\it néŋ̀} and 
{\it pàtʃɪ́gɪ́ɪ́} if they were full-fledged nouns. Thus, although complex stem 
nouns contain more than one stem, there is only  one noun class associated with 
the noun and it is always the noun class associated with the rightmost stem.  
This was mentioned in Section \ref{sec:GRM-sem-ass-crit} to support the claim  
that semantic criteria in noun class assignment may be  nonexistant. 


If  stems are treated as lexemes, there is still a problem in accounting for the
 `reduced' form of  some lexemes when they occur in stem appositions. That
is, the first stem  of a complex stem noun is often reduced to a single
syllable in the case of a polysyllabic lexeme, or a monosyllabic lexeme of the
type CVV is reduced to CV. For
example,  {\it lúhò}  and  {\it lúhòsó} are respectively the singular and
plural forms for 
`funeral' ({\sc cl.2}).  The expectation is that
when the lexeme takes part in position X of a XY complex stem noun,
it should exhibit its lexemic form, i.e.   {\it lúhò}. Yet, the word for `last
funeral'  is {\it lúsɪ́nnà}, {\it lit.} funeral-drink,  and not {\it 
*luhosɪnna}.  Not all  lexemes get reduced in that particular 
environment, nevertheless, it is  more common (and visible)  for polysyllabic 
lexemes or
monosyllabic ones built on a heavy syllable. Moreover, some lexemes are more
frequent in that environment than others.


The relation between the stems in a complex stem noun is asymmetric.  The 
relation is defined in terms of what the referents of the stems and the complex 
noun as a whole have to do with each other.  As in a syntactic relation between 
a head and a modifier, one of the stems modifies while the other stem is 
modified. The semantic relations between the stems  are of two types: 
`completive' modification and  `qualitative' modification. These distinctions 
are discussed in  Sections \ref{sec:GRM-comp-completive} and 
\ref{sec:GRM-comp-quality} below.


\paragraph{Completive modification}
\label{sec:GRM-comp-completive}

A completive modification in a complex stem noun XY can translate as `Y of X' of 
which Y is the head. For instance {\it sììpʊ́ŋ}   `eyelash', {\it lit.} 
eye-hair, is a kind of hair and not a kind of eye. And {\it ʔɪ̀lnʊ̃̀ã̀} 
`nipple', {\it lit.} breast-mouth, is most likely seen as a kind of orifice than 
as  a kind of breast.  In both cases, the noun class is suffixed to the 
rightmost stem, incidentally to the head of the morphological construction, i.e. 
{\it sììpʊ́ŋ}/{\it sììpʊ́ná} {\sc (cl.3)} and {\it ʔɪ̀lnʊ̃̀ã̀}/{\it 
ʔɪ̀lnʊ̃̀ã̀sá} {\sc (cl.2)}. As mentioned earlier,  either X or Y  in a complex 
noun XY can be complex. The word {\it nèpɪ́ɛ̀lpàtʃɪ́gɪ́ɪ́} `palm of the hand' 
is an example of two completive modifications. It consists of a complex stem 
{\it nepɪɛl} `hand', which is composed of  {\it ne} `arm' and {\it pɪɛl} `flat', 
and the atomic stem {\it patʃɪgɪ} `stomach', yielding in turn  `flat of hand' 
and then `inside of flat of hand'. 


\paragraph{Qualitative modification}
\label{sec:GRM-comp-quality}

A qualitative modification in a complex stem noun is the same as the  syntactic
modification  noun-modifier. The difference lies in the formal status of the
elements: when the
relation is held at a syntactic level the elements are words, whereas at the
morphological level they are lexemes. As mentioned earlier,  either X or Y  in a
complex noun XY can be complex. For instance, the word {\it nebiwie} consists of
the combination of {\it ne}  `arm' ({\sc cl.9}) and {\it bi} `seed'    ({\sc
cl.4}), then the combination of {\it nebi} `finger' and {\it wi} `small'. The 
noun class  of {\it wi} `small'  is {\sc cl.1}, so the singular and plural
forms for the word `little finger' are {\it nèbìwìé} and  {\it nèbìwìsé}
respectively. The first relation involved is a completive modification, i.e.
`seed  of arm', while the second is a qualitative one, i.e. `small seed  of
arm' or `small finger'.  A qualitative modification in a complex noun XY can
translate as `X has
the property Y'  of which X is the head. Therefore, unlike many languages,  it
is not necessarily the head of the morphological construction which determines
the type of inflection.


\begin{table}[htb!]

\centering
\caption[Distinction between completive and qualitative
modification]{Distinction between completive and qualitative
modification using /daa/ `tree' or `wood'.  Abbreviations: {\sc h}= head,  {\sc
m}=
modifier, {\sc ns}= nominal stem, {\sc ss}= verbal state stems,  {\sc ps}=
verbal process stem, \label{tab:GRM-complet-and-qualit}}
\begin{Itabular}{llllll}
\lsptoprule
&  \multicolumn{3}{c}{Structure}    &    Stems &  Word\\  \cline{2-4}
    & Lex. type &  Function & Semantic& &\\[1ex]\midrule
%\multirow{4}{5mm}{\begin{sideways}\parbox{15mm}{Completive}\end{sideways}}
\multirow{4}{5mm}{\begin{sideways}\parbox{20mm}{Completive}\end{sideways}}

& {\sc ns-ns} & {\sc m-h}&{\sc whole-part}&/daa/-/luto/&   dààlútó \\
& &&&`tree'-`root'&`root of tree' \\[1ex]

&{\sc ns-ss}&{\sc m-h}&{\sc whole-part}&/daa/-/pɛtɪ/ &dààpɛ́tɪ́ \\
&&&&`tree'-`end'& `bark'\\[1ex]

&{\sc ns-ns}&{\sc m-h}&{\sc whole-part}&/kpõŋkpõŋ/-/daa/&kpõ̀ŋkpõ̀ŋdāā\\
&&&&`cassava'-`wood' &`cassava plant'\\[1ex] \midrule


 \multirow{4}{5mm}{\begin{sideways}\parbox{20mm}{Qualitative}\end{sideways}}

&{\sc ns-ns}&{\sc h-m}&{\sc thing-charac}&/daa/-/sɔta/& dààsɔ̀tá \\
&&&&`tree'-`thorn'& `type of tree'\\[1ex]

&{\sc ns-ns}&{\sc h-m}&{\sc thing-charac}&/ɲin/-/daa/& ɲíndáá\\
&&&&`tooth'-`wood'&  `horn'\\[1ex]

&{\sc ps-ns}&{\sc h-m}&{\sc purpose-thing}&/tʃaasa/-/daa/&tʃáásádàà \\
&&&&`comb'-`wood'&`wooden comb'\\[1ex]

\lspbottomrule
\end{Itabular} 
\end{table} 


The examples in Table \ref{tab:GRM-complet-and-qualit} illustrate the
distinction between the completive and qualitative modification. The form {\it 
daa}  conveys either the meaning `tree' or `wood'. Both meanings may function as
head or as modifier.  If the head stem follows its modifier, it is a completive
modification, and vice-versa for the qualitative modification. A semantic
relation between the stems may  be a whole-part relation, a characteristic added
to define an entity or a purpose  associated with an entity. 

So far,   XY-complex stem nouns were assumed to be  endocentric compounds whose
head is X in qualitative modification and the head is Y in completive
modification.  However, a word such as {\it patʃɪgɪbummo} `secretive', {\it 
lit.}
stomach-black, suggests that some XY-complex stem nouns may  either lack a head
or have more than one head. These possibilities are not ignored, but in this
particular case the complex stem noun may be seen as involving  the abstract
senses of {\it patʃɪgɪɪ} and {\it bummo}, that is  `essence' and 
`subtle/restrained' respectively, making {\it patʃɪgɪbummo} a qualitative
modification  which can be formulated literally as `subtle/restrained essence',
i.e.   a property applicable to humans. Thus, the stem {\it patʃɪgɪɪ} is treated
as the head, and {\it bummo} as the stem functioning as the qualitative 
modifier.
Another example is {\it dààdùgó}. This word consists of the stems {\it daa}
`tree' and  {\it dugo} `infest'   and refers to a type   of insect. Unlike the
analyzed expressions displayed in  Table \ref{tab:GRM-complet-and-qualit}  none
of the stems can be treated as the head of the expression and the meaning of the
whole noun cannot be transparently  predicted from its constituent parts. This
leads me to provisionally consider the expression {\it dààdùgó}  as an
exocentric compound, i.e. a complex stem noun without a head.


 
\paragraph{Compound or circumlocution}
\label{sec:GRM-comp-vs-circum}

For a few expressions,  it is hard to tell whether they are compounds, i.e. the
results of  morphological operations, or circumlocutions, i.e.  the results of
syntactic operations \cite[165]{Alla01}. Clear cases of circumlocution
nevertheless exist. For instance,  the word {\it kpatakpalɪ} `type of hyena'  is
treated by one language consultant as {\it kpa ta kpa lɪɪ} {\it lit.}
 `take let.free take leave'.\footnote{Yet {\it kpatakpari} is the word for
`hunting trap' in Gonja \citep{Rytz66}.}  Another example is {\it 
sʊ́wàkándíkùró} `parasitic plant'. This expression refers to a type of
parasitic plant lacking a  root which grows upon and survives from the
nutrients provided by its  hosts. The word-level expression originates from the
sentence  {\it sʊwa ka n̩ di kuoro}, {\it lit.}  die-and-I-eat-chief, `Die so
that I can become the chief'. It is common to find names of individuals being
constructed in this way: the oldest woman in Ducie is known as {\it n̩wabɪpɛ}, 
{\it lit.}  {\it n̩ wa bɪ pɛ}  `I-not-again-add'. Since two successive husbands 
died early,  she used to say that she will never marry again. For that reason
people call her {\it ǹ̩wábɪ̀pɛ̄}.  



\subsubsection{Derivational morphology}
\label{sec:GRM-der-morph}

A derivational morpheme is an affix which combines with a stem to form a word.
The meaning it carries combines with the meaning of the stem.     By definition,
a derivational morpheme is a bound affix, and  thus 
cannot exist on its own as a word. This property keeps apart complex
stem nouns and derived nouns. Yet, the distinction between a bound affix and
a lexeme is not obvious, mainly because some bound affixes were probably lexemes
at a previous stage, or still are today. 
% \paragraph{}
% \label{sec:GRM-der-}


\paragraph{Maturity and sex of animate entities}
\label{sec:GRM-der-matur}


The specification of the maturity
and sex of an animate entity is accomplished in the following
way: male, female, young, and adult are organized in morphemes encoding
one or two distinctions. These morphemes are suffixed to the rightmost stem.
To distinguish between male and female, the morphemes ({\it sg.}/{\it pl.}) 
{\it wal/wala} `male'  and {\it nɪɪ/nɪɪta} `female'  are used as
(\ref{exːGRM-sex-ent}) illustrates.


\ea\label{exːGRM-sex-ent}
 
 \ea\label{exːGRM-sex-en} {\it bɔ̀là-wál-\O} /  {\it bɔ̀là-wál-á} \\
elephant-male-sg / elephant-male-pl ({\sc cl.3})
 \ex\label{exːGRM-sex-en}  {\it bɔ̀là-nɪ́ɪ́-\O}  / {\it bɔ̀là-nɪ̀ɪ̀-tá}\\
elephant-female-sg / elephant-female-pl ({\sc cl.7})
 
 
\z 
 \z


The language employs two strategies to express the distinction between  the
adult animal and its young, which is  called here 
`maturity'.  The first
is to simply add the morpheme {\it -bi} `child'  to the head,
e.g. {\it bɔlabie/bɔlabise} `young elephant(s)'. In the second strategy 
both the sex and maturity distinctions are conveyed by the morpheme.  This is
shown in Table \ref{GRM-maturity-sex}. 

%componential-

\begin{table}[htb!]

\caption{Morphemes encoding maturity and sex of animate entities}
\centering
 \begin{Itabular}{lcc}
\lsptoprule
&\textsc{male}&\textsc{female}\\
\midrule
\textsc{young}& -w(a|e)lee & -lor\\
\textsc{adult} & -wal & -nɪɪ\\
\lspbottomrule

  
 \end{Itabular} 

\label{GRM-maturity-sex}
\end{table} 

Some examples are more opaque than others. For instance, the onset consonant of
the morpheme {\it wal/wala} `male' may surface as a bilabial plosive,  e.g. 
{\it bʊ̃́ʊ̃̀mbāl} `male goat'.  One can also observe a difference in form 
between
the word {\it pìèsíí} `sheep',   {\it pèmbál}  `male sheep' and  {\it 
pènɪ̀ɪ́} `female sheep'. The words displayed in the first three rows of table
\ref{tab:GRM-matur-sex-ex} show the least transparent derivations.  The
annotation of tone is a first impression and the symbol - indicates that the
consultants could not associate a word for the intended meaning. 


%\ex\label{exːGRM-sex-en}  {\it bɔlɛbie}  /  {\it bɔlɛbise}
 %\ex\label{exːGRM-sex-en}  {\it bɔlɛbiwal}  /  {\it bɔlɛbiwala}
%. a case of qualitative
%modification


\begin{table}[htb!]

\caption{Maturity and sex of animals}
\centering
 \begin{Itabular}{llllll}
\lsptoprule
Animal & Generic & \multicolumn{2}{c}{Adult} 
& \multicolumn{2}{c}{Young}  \\ \cline{3-4} \cline{5-6}
 && \multicolumn{1}{c}{Male} &  \multicolumn{1}{c}{Female} & 
\multicolumn{1}{c}{Male} &   \multicolumn{1}{c}{Female}\\
\midrule


fowl        &zál̀& zím{\T ꜜ}bál    & zápúò&
zímbéléé  & zápúwìé\\

sheep      &píésíí &pèmbál    & pènɪ̀ɪ́&
pémbéléè&pélòŕ\\

goat     &bʊ̃́ʊ̃̀ŋ & bʊ̃́ʊ̃́mbál   & bʊ̃̀nɪ̀ɪ́ &  bʊ̃̀mbéléè & 
bʊ̃̀ʊ̃̀lòŕ\\

pouched rat    &sàpùhĩ́ẽ̀ & sàpúwál    & sàpúnɪ̀ɪ̀&
 sàpúwáléè& sàpúlòr\\

antilope      &ʔã́ã́ &ʔã̀ã̀wál  
&ʔã́ã́nɪ́ɪ́&ʔã̀ã̀wéléè&ʔã̀ã̀lòr\\

dog      &váà  &váwàl    &vánɪ̀ɪ̀&
váwáléè&válòŕ\\

cat        &dìébìé &díèbəwál   
&díèbənɪ̀ɪ̀&díèbəwáléè&díèbəlòr\\

cow       &nã̀ɔ̃́ &nɔ̃̀wál  
&nɔ̃̀nɪ̀ɪ́&nɔ̃̀wáléè&nɔ̃̀lòŕ\\

elephant    &bɔ̀là &bɔ̀lwàl    & bɔ̀lənɪ̀ɪ́&  -  &bɔ̀llòŕ\\

guinea fowl    &sũ̀ṹ&sũ̀wál    &sũ̀nɪ̀ɪ́& - & -\\
bush mouse    &ʔól&ʔólwál    &  ʔólnɪ̀ɪ́ & - & - \\
%house mouse&dàgbòŋó&dagboŋowal  &dagboŋonɪɪ& - & - \\
lizard     &gèŕ &géwál    &génɪ̀ɪ́&  -& - \\

\lspbottomrule
 \end{Itabular}
\label{tab:GRM-matur-sex-ex}
\end{table} 



\paragraph{Inhabitant of ...}
\label{sec:GRM-inhabitant-of}

To express `I am from X',  where `be from X' refers  to the place where
someone is born and/or the place where someone lives, the verb {\it lɪ̀ɪ̀} is
used, e.g. {\it sɔɣla n̩ lɪɪ} `I am from Sogola'.  

Expressions with the meaning `Inhabitant of X'  can be  noun words referring to
this same idea, that is  `being from X'. The examples in table
\ref{tab:inhabitant-of} show that the meaning is captured in suffixes {\it 
-(l)ɪɪ/(l)ɛɛ/la} which display vowel qualities in the singular and plural
similar to those found in noun class 4. 


\begin{table}[htb!]

\caption{Inhabitant of ... \label{tab:inhabitant-of}}
\centering
 \begin{Itabular}{llllll}
\lsptoprule
Location & sg. & pl. & Location & sg. & pl.  \\[1ex] \midrule
Chakali   & tʃàkálɪ́ɪ́   & tʃàkálɛ́ɛ́    &
Katua  &kàtʊ́ɔ́lɪ́ɪ́   &kàtʊ́ɔ́lɛ́ɛ́ \\
Motigu   &mòtígíí  &mòtígíé    &%
Tiisa    &  tíísàlí  &tíísàlá\\
Ducie   &dùsélíí  &dùséléé&%
Chasia  &tʃàsɪ́lɪ́ɪ́    & tʃàsɪ́lɛ́ɛ́  \\
  Bulinga  & búléŋíí & búléŋéé&%
Wa    &  wáálɪ́ɪ́  &wáálà\\
Gurumbele&grʊ̀mbɛ̀lɪ́lɪ́ɪ́& grʊ̀mbɛ̀lɪ́lɛ́ɛ́&%
Tuosa  &  tʊ̀ɔ̀sálɪ́ɪ́  & tʊ̀ɔ̀sálá\\
\lspbottomrule

 \end{Itabular}
\end{table} 
 

%({\it mũ̀sóró} `type of food ingredient') , 
%({\it dárà} `type of game')



\paragraph{Category switch}
\label{sec:GRM-der-cat-switch}

The
phenomenon called `category switch' refers to a derivational process whereby two
words with  related meanings and composed of the same segments change category
based entirely on their tonal melody. Examples are provided in 
(\ref{exːGRM-der-cat-switch}).


\begin{exe}
 \ex\label{exːGRM-der-cat-switch}
tʊ̀mà  ({\it v}) `work'  $\leftrightarrow$ tʊ́má ({\it n}) `work'\\
gʊ̀à ({\it v})  `dance' $\leftrightarrow$ gʊ̀á ({\it n}) `dance'\\
jɔ̀wà ({\it v}) `buy'   $\leftrightarrow$ jɔ̀wá  ({\it n})  `market'\\
mʊ̀mà ({\it v}) `laugh'  $\leftrightarrow$ mʊ̀má ({\it n})  `laughter'\\
gòrò ({\it v}) `circle'  $\leftrightarrow$ góró ({\it n})  `bent'
\z


\paragraph{Agent- and event-denoting nominalizations}
\label{sec:GRM-der-agent}

Agent- and event-denoting nominalizations were discussed in
Section \ref{sec:GRM-verb-act-stem} in connection with the licensing of verbal
stems in atomic noun formation. Apart from their roles in complex stem nouns, it
was shown that both verbal state and verbal process stems undergo these two
nominalizations processes in order to function as atomic nouns.
The two
processes are summarized in (\ref{exːGRM-der-agent}) and 
(\ref{exːGRM-der-action}). Notice in (\ref{exːGRM-der-agent}) that the two
agent-denoting nominalizations can occur on a single stem. Also, the
noun class does not seem to be  determined by the suffix
-[r].  While one consultant prefers that all agent nouns end as ({\it
sg.}/{\it pl.}) {\it -r/-rV}, another consultant varies between {\it -r/-rV} 
({\sc cl.3}) and  {\it -ra/-rəsV} ({\sc cl.1}).  In addition, there is another
agent-denoting word formation which simply adds the word {\it kʊɔrɪ} `make' to
the noun denoting the product, e.g. {\it nã̀ã̀tɔ̀ʊ̀kʊ́ɔ́rá} / {\it 
nã̀ã̀tɔ̀ʊ̀kʊ́ɔ́rəsá} ({\sc cl.1}) `shoemaker(s)' $<$ {\it nããtɔʊ} ({\it n})
`shoe' + {\it kʊɔrɪ} ({\it v}) `make'.


\ea\label{exːGRM-der-agent}{\rm Agent nominalization}\\

\ea\label{exːGRM-der-agent-suffix}
{\it A verb stem takes the suffix -[r]  to express agent-denoting
nominalization.} \\
{\it  sʊ̃̀ã̀sʊ́ɔ́r} / {\it sʊ̃̀ã̀sʊ́ɔ́rá} ({\sc cl.3}) `weaver(s)' \\
 $<$  {\it sʊ̃̀ã̀} ({\it v}) `weave'\\
{\it  lúlíbùmmùjár} / {\it lúlíbùmmùjárá} ({\sc cl.3})  `healer(s)' 
\\
$<$ {\it lulibummo} ({\it n}) `medicine' + {\it ja} ({\it v}) `do'


 \ex\label{exːGRM-der-agent-redup}
{\it A verb stem gets partially reduplicated   to express
agent-denoting nominalization.} \\
{\it  sùsùmmà} / {\it sùsùmmə̀sá} ({\sc cl.3})  `helper(s)' \\
$<$ {\it sùmmè} ({\it v}) `help' \\
{\it  sã́sáár} / {\it sã́sáárá} ({\sc cl.3})  `carver(s)' \\
$<$ {\it sã̀ã̀} ({\it v}) `carve' 


\z 
 \z


\begin{exe}
 \ex\label{exːGRM-der-action}{\rm Event nominalization}\\
{\it A verb stem takes the suffix -/[{\sc +hi, -bk}]/   to express
event-denoting
nominalization.} \\
{\it lʊ́lɪ́ɪ́ } / {\it  lʊ́lɪ́ɛ́} ({\sc cl.4}) `giving birth'  \\
 $<$ {\it  lʊla} ({\it v}) `give birth'\\
{\it kpégíí} / {\it   kpégíé}  ({\sc cl.4})  `hard'  or
`strong' \\
 $<$ {\it  kpeg} ({\it v}) `hard' or `strong'\\

\z

\subsubsection{Proper nouns}
\label{sec:GRM-prop-noun}

% 
% Are these the names these people were given by their families? they look like
% nick-named if they are not actually titles.  The whole paragraph sounds
%strange. %  Either get more ethnographic information on naming or cut some of
%this out.

%Proper nouns  are  usually characterized by their  inability  to inflect for
%number and to co-occur with articles and modifiers. 

As a rule,   proper nouns
have  unique referents:  they  name people, places, spirits, and so on.  So in
the
area where Chakali is spoken, there is only one river named {\it gorogoro}, only
one hill named {\it dɔ̀lbɪ́ɪ́}, one
village
named {\it mòtīgú},  only one shrine named {\it dàbàŋtʊ́lʊ́gʊ́}, etc.  
Nevertheless more than one person can have the same
name, and the same applies to a lesser extent to villages. For instance,
{\it sɔɣla} may refer to the Chakali village situated between Tuosa and Motigu, 
or to
a Vagla village situated at the junction of the Bole-Wa and Damongo-Wa road. To
identify the former, one must say {\it tʃakalsɔɣla}. 


A  Chakali person may bear two or three names: his/her father's name, the name
of his/her grandfather or great-grandfather, and his own (common) name. In the
case of the (great-) grandfather's name, it is a feature of the newborn or an
external sign which suggests the child's name.  The common name may be changed
in
the course of one 's life. Today, regardless of whether a  person is Muslim or
not,
common names are mainly from Arabic, Hausa, and Gonja origins, probably due to
the
Islamization of Chakali (see Section \ref{sec:SOC-why-cli-still}).


Common names among the  elders (over 50 years old) consist of the name of a
non-Chakali village,  together with {\it nàà} `chief'. In Tuosa, Ducie, and
Gurumbele, one finds one or more Kpersi Naa, Mangwe Naa, Jayiri Naa, Wa Naa, 
Sing Naa,  Busa Naa, etc. The next generation (below 50 years old) tend to
have either `Muslim' names or `English-title' names. Common Muslim names are
Idrissu, Fuseini, Mohamedu, Ahmed, Mohadini, etc.  Typical 
`English-title' names are {\it spɛ́ntà} `inspector',  {\it dɔ́ktà} `doctor', 
{\it títʃà} `teacher', etc. Apart from `teacher',  which can identify actual
teachers in communities in which schools are present, none of the individuals
are actual teachers, doctors or inspectors. The same can be said about the older
generation, none of them are/were chief of those places. The villages are not
Chakali villages and these individuals have no connections with the villages
used in their names. It seems that these common names are trendy  nicknames
peers  assign to each other. One consultant claims that the elders can be
ranked in terms of power and influence according to their nicknames. 

In Chakali society, one may have two additional names, a drumming name and a
Sigu name. A drumming name is used in drummed messages sent to other villages
about weddings or deaths,  while a {\it  sígù} name is a name one receives
when initiated to the shrine {\it  dààbàŋtólúgú}. 

Because of their pragmatic function,  proper nouns  are rarely observed in a
plural form, but some contexts may allow this. In
(\ref{ex:GRM-propn-noun-plur}), the proper name {\it Gbolo} takes the plural
marker {\it -sV}.\footnote{The context of (\ref{ex:GRM-propn-noun-plur}) makes
sense when one understands that the name `Gbolo' has a particular meaning.  It
is understood that when a couple has a  fertility problem,  it is common to
travel to the community of Mankuma and to consult their shrine. If the woman
gets pregnant after the visit, they must return to Mankuma to appease the
shrine. Subsequently, the child must be named `Gbolo' and automatically acquires
 the Red Patas monkey as  totem.}


  \begin{exe}
   \ex\label{ex:GRM-propn-noun-plur}
\gll  gbòlò-só bá-ŋmɛ̀nàá ká dʊ̀à dùsèè ní\\
gbolo.({\sc g.}b)-{\sc pl}  {\sc g.}b-{\sc q} {\sc  egr} exist Ducie {\sc
postp}\\
\glt  `How many Gbolos are there in Ducie?' 
    \z

Finally, circumlocution is a common process found in names of people and dogs 
 (e.g. the example of {\it n̩wabɪpɛ}, 
{\it lit.}  {\it n̩ wa bɪ pɛ}  `I-not-again-add', was given in Section
\ref{sec:GRM-comp-vs-circum}).   A few examples of dog names are given in
(\ref{ex:GRM-propn-dog-name}).


  \ea\label{ex:GRM-propn-dog-name}{\rm Examples of dog names}\\
 
% \ex\label{ex:GRM-propn-dog-name-} 

 \ea\label{ex:GRM-propn-dog-name-1} 
{\it jàsáŋábʊ́ɛ̃̀ɪ̀} `Let's keep peace'\\ 
ja-saŋa-bʊɛ̃ɪ   $>$  we-sit-slowly  
 \ex\label{ex:GRM-propn-dog-name-2} 
{\it ǹ̩nʊ̀ã́wàjàhóò}   `I will not open my mouth again'  \\ 
 n̩-nʊã-wa-ja-hoo  $>$ my-mouth-not-do-hoo 
 \ex\label{ex:GRM-propn-dog-name-3} 
{\it kùósòzɪ́má}   `God knows' \\  
kuoso-zɪma $>$ god-know

\z 
 \z



% In many folktales, animals are the characters and their acts evoke
% human beings' behaviors. There is a tendency in folktales to switch between
% using the animal name as common noun and using it as proper noun. (example)


%1. born in village but moved from there
%2. not necesarily chief but from royal family
%3. gan naa= `more than a chief' grand father of John
%4. pure nick name

\subsubsection{Loan nouns}
\label{sec:GRM-borr-noun}


 A loan noun,  or more generally a loanword, can be defined as  ``a word that at
some point came into a language by transfer from another language''
\citep[58]{Hasp08}.  When a word is found in both Chakali and in another
language, many loan scenarios are conceivable. However,  for some semantic
domains such as   bicycle or car parts, school material and so on, the past and
present sociolinguistic situations  suggest that Chakali is
the recipient language and Waali, English, Hausa, Akan, and Dagaare are the
donor
languages.  Loan scenarios differ and are harder to establish when other SWG
languages are involved. It is often unfeasible to demonstrate whether the same
form/meaning in two languages was inherited from a common ancestor, or  
borrowed by one and subsequently passed on to other SGW languages. Moreover, it
may be unwise to assume that in all cases Chakali is  the recipient language,
especially for loan words in domains which  were in the past fundamental in
Chakali lifestyle,  but to a lesser degree for neighboring ethnic groups. 
Thus, Chakali as a donor language can be evaluated in a wider Grusi-Oti Volta
genesis, or  at a micro-level where the influence of Chakali on Bulengi is
established.

It is unlikely that Chakali borrowed from English through contact. And Ghanaian
English, in Wa town and Chakali communities,  is not an effective mode of
communication, at least in social spheres where Chakali men and women 
interact (see discussion in Section \ref{sec:SOC-unesco}).  Nonetheless, the
situation is different for school children  who are
exposed to Ghanaian English on a regular basis. I believe that Ghanaian English
spoken by native speakers of Waali, Dagaare, or Chakali  is the only potential
variety of English which can function as a donor language. Examples of words 
ultimately  from English origin are: {\it ʔàbəlù} `blue', {\it ʔásɪ̀bɪ́tɪ̀} 
`hospital',    {\it dɔ́ktà} `doctor', {\it bàlúù} `balloon', {\it 
bátərbɪ́ɪ́} 
`battery (-stone)', {\it bɛ́lɛ́ntɪ̀} `belt',  {\it təráádʒà} `trouser',  
{\it 
détì} `date',    {\it mɪ́ntɪ̀} `minute',   {\it dʒánsè} `type of dance',  
{\it 
kàpɛ́ntà} `carpenter', {\it kɔ́lpɔ̀tɛ̀} `coal pot', {\it kɔ́tà} `quarter',  
{\it 
lɔ́ɔ́lɪ̀} `lorry (any four-wheel vehicle)',   {\it sákə̀r} `bicycle',  {\it 
pɛ̀n} 
`pen', {\it sùkúù} `school',   {\it tʃítʃà} `teacher' and many more.  
There 
is a recurrent falling tonal melody (i.e. HL) among the loan nouns of  
ultimately English origins. Many of them,  if not all, can be found in other 
languages of the area \citep{sisa75, daku07}. 

People are aware of the linguistic fragility of Chakali; some of the
language
consultants confirm that many people do not bother making  the effort to use
Chakali and that many
prefer to use Waali expressions.  The knowledge and interest of our
language consultants in their language made  it possible  to reduce the number
of
Waali
expressions in the dictionary in Part/chapter \ref{sec:cli-eng-dic}.  Despite 
that,
when a word is found both in
Waali and Chakali, it is not automatically classified as borrowed from Waali,
yet it is only suspected to be non-Chakali.  Examples such
{\it dʒɪ̀ɛ̀rá} `sieve', {\it dʒùmbùrò} `type of medicine', {\it gbàgbá}
`duck', {\it kákádùrò} (Akan)  `ginger', {\it kàpálà} `fufu', {\it 
káṹ}  `mixture
of sodium carbonate' , {\it nāsárá}  (Hausa) `white man', and  {\it 
sɛ̀nsɛ́nná}
`prostitute' are some of the Waali/Chakali nouns  found in transcribed texts, or
by chance. 

The weekdays are from Arabic (probably via Hausa or Oti-Volta languages).
Vagla and
Tumulung Sisaala,  but  not Dɛg, use similar expressions \citep[60]{Nade96}: 
{\it atanɛa} `Monday', {\it atalata} `Tuesday', {\it alarba} `Wednesday',   
{\it alamussa} `Thursday',  {\it ardʒɛmaa} `Friday' , {\it asɛbɛtɛ} `Saturday'  
 
 and
{\it allahadi} `Sunday'.  The expressions for the lunar months seem to be
borrowed from Waali, but Dagbani and Mamprusi have similar expressions. In
these
Oti-Volta languages, some of the names  correspond to important festivals, i.e.
1, 3, 7, 9, 10, and 12 below. In Chakali, only {\it dʒɪ́mbɛ̄ntʊ̀} is celebrated
and  is considered the first month.\footnote{Dagbani {\it buɣum} and Waali 
{\it  dʒɪmbɛntɪ} are both treated as first month by the speakers of these
languages.} The lunar months are: {\it dʒɪ́mbɛ̄ntʊ̀} `first month
(1)', {\it sífə̀rà}  `second month (2)', {\it dùmbá} `third month (3)', 
{\it dùmbáfúlánààn} `fourth month (4)',  {\it dùmbákókórìkó} `fifth 
month
(5)', {\it kpínítʃùmààŋkùná} `sixth month (6)', {\it kpínítʃù} 
`seventh
month (7)', {\it ʔàndʒèlìndʒé} `eighth month (8)', {\it sʊ́ŋkàrɛ̀} `ninth
month (9)', {\it tʃíŋsùŋù} `tenth month (10)', {\it dùŋúmààŋkùnà}
`eleventh month (11)' and {\it dùŋú} `twelth month (12)'.  It was understood
that
these terms and concepts are not known by the majority.\footnote{Thanks to Tony
Naden, who helped me clarify some issues relating to this topic.} 




\subsubsection{Relational Nouns}
\label{sec:SPA-relnoun}

%                                               Some relational nouns are de-
% rived from body parts, e.g., ..., but not all, e.g.,
% -baamban??

Many  languages present formal identity between body parts terms and expressions
used to designate elements of space. The widely accepted view is that
diachronically  spatial relational nouns (sometimes called spatial nominals
\citep[895]{Hell07} or adpositions \citep[137]{Hein97}) are ``the result of
functional split'' and that ``they are derived from nouns denoting body parts or
locative concepts through syntactic reanalysis'' \citep[256]{Hein84}. In Ewe
(Kwa) for instance, it is claimed that  the prepositions have evolved from verbs
and postpositions from nouns \citep[367--369]{Amek06}. 

% 
% In the theory of grammaticalization's terms
% \citep[137]{Hein97} 
% In  \citet{Amek06}
% it is claimed that i
% 
% What we call a relational noun has received several other labels in the
% literature; spatial nominal \citep[895]{Hell07}, postposition
% \citep[256]{Hein84}. 


Chakali relational nouns are formally identical to body part nouns although 
not all body part nouns have a relational noun counterpart. For instance,
whereas {\it 
ɲuu} can have both  a spatial meaning, i.e. `on top of X', and  a body part one,
i.e. `head',  the body part terms {\it bembii} `heart', {\it hog} `bone'  or 
{\it 
fʊ̃ʊ̃} `lower back', among others,  cannot convey  spatial meanings. Table
\ref{tab:relvsbody} displays the body parts found in the data which
convey spatial meaning.\footnote{The body part term {\it gantal}
`back' is from
the Ducie lect and corresponds to {\it habʊa} in the Motigu, Gurumbele,
Katua, Tiisa, and Tuosa lects.}



\begin{table}[h!]
\caption[Spatial nominal relations and body part nouns]{Spatial nominal
relations and body part nouns: similar forms and different, but related,
meanings\label{tab:relvsbody}}
\centering
\begin{small}
 \begin{Qtabular}{lllll}
\lsptoprule
Projection& Spatial relation & PoS: {\it reln}  & Body parts &
 PoS: {\it n}\\   \midrule
Intrinsic & &&&\\

& \textsc{top}  & {  ɲuu(x,y)} & head & {  ɲuu(x)}\\
&\textsc{containment} &  { patʃɪgɪɪ(x,y)}  & stomach & {
patʃɪgɪɪ(x)}\\

& \textsc{side} &  { logun(x,y)}& rib & { logun(x)}\\
& \textsc{mouth} &  { nʊã(x,y)}& mouth & { nʊã(x)}\\
& \textsc{base/under} &  { muŋ(x,y)}& arse & { muŋ(x)}\\
& \textsc{middle} &  { bambaaŋ(x,y)}& chest box & { bambaaŋ(x)}\\
\cline{1-3}

Relative  & &&&\\
& \textsc{left} &  {neŋgal(x,y)}& left hand &  {neŋgal(x)}\\
& \textsc{right}  & {nendul(x,y)}  & right hand &{nendul(x)} \\
& \textsc{back} &  {gantal(x,y)}  & dorsum & {gantal(x)}\\
&    \textsc{front} & {sʊʊ(x,y)}  & front  & {sʊʊ(x)}\\

 \lspbottomrule
 \end{Qtabular}
\end{small}

\end{table} 







The contrast between an intrinsic and a relative frame of reference was brought
up in Section \ref{sec:SPA-exper1-sum} where we confirmed that both  an 
intrinsic and a relative exocentric and egocentric  frames of reference were
being used. An intrinsic frame of reference is  invariant no matter how the
spatial relation is viewed by the speaker or the addressee, whereas a
relative
frame of reference depends on
how a spatial relation is  viewed. 


% Sure that is right. Also suppose that you put the cutlass you used to cut the
% head on the head. Now you can say, 'a karintie saga a nyuu nyu ni'. In your
% given example, the kpulikpuli is on the head. Let's change it and say the
% kpulikpuli is under the head. Then we would have, 'a kpulikpuli dua a nyu mun
% ni'. I'm not sure how these will allow, could you give examples? gantal gantal
% 'behind the back'
% nua nua  'at the entrance of the
% mouth' pachigi pachigi 'inside the stomach' (Kasim)

How can we distinguish a relational
noun from a noun?  Above all,  the differentiation between relational
nouns and body part nouns cannot rely solely on surface syntax criteria,
precisely
because the configuration of a possessive noun phrase and a
relational noun phrase are identical. This is shown in
 (\ref{ex:SPA-reln-vs-bpn}). 

 
\ea\label{ex:SPA-reln-vs-bpn}
\ea\label{ex:SPA-bpn}{\rm Possessive attributive phrase}\\
 {[\textsc{n}$_{1}$-\textsc{n}$_{2}$]}$_{NP}$ where \textsc{n}$_{2}$=body 
part,   e.g. {\it baal  ɲuu} ``a man's head''

\ex\label{ex:SPA-reln}{\rm Spatial nominal  phrase}\\
 {[{\sc n}$_{1}$-{\sc n}$_{2}$]}$_{NP}$ where {\sc n}$_{2}$=spatial relation,   
e.g. {\it tebul  ɲuu} ``top of the table''
\z
\z

Even though the two corresponding nominal structures may cause ambiguities,
the
interpretation is generally disclosed by the meaning of the nominal preceding
the \textsc{n}$_{2}$ in  (\ref{ex:SPA-reln-vs-bpn}). The term  {\it  ɲuu}, 
for
instance, can only mean `top of' in a 
phrase in which it follows another nominal and refers to a projected
location of \textsc{n}$_{1}$'s referent. In (\ref{ex:SPA-bpn}), even
though {\it ɲuu}
 immediately follows a nominal,  it would not normally refer to the projected
location `on the top' but only to the man's head. Nevertheless, despite any
attempts to identify  structural characteristics which may contribute to the
disambiguation of a phrase involving a body part term,  
ambiguities may still arise.


%a kpulikpuli dua a statue nyuu nyuu ni






Another aspect of body part terms is their different function in  morphological
and syntactic structure. While a relational noun is a syntactic word,  body
part terms may also function as morphemes in compound nouns to express a
specific
part-whole relationship or a conventionalized metaphor \citep[141]{Hein97}. 
Whereas the distinction may be formally and semantically hard to distinguish,
the number of body
part terms which can be the stem in a compound noun is larger than those
functioning as relational nouns. Some examples are shown in table
\ref{tab:SPA-bpt-compound}.

%fix the etymology

\begin{table}[h]
\caption{Body part terms in compound nouns\label{tab:SPA-bpt-compound}}
\centering
%\begin{small}
 \begin{Qtabular}{lllp{1.6in}}
\lsptoprule
Body part term & Compound noun  & Morph. gloss & Gloss \\ \midrule


eye &    tɔ́ʊ́-sìì  & village-eye & village's center\\
 & kpã̀ã̀n-síí & yam-eye & yam stem\\
& nɪ̀ɪ̀-síí & water-eye & deepest area of a  river\\
 &  nã̀ã̀-síí & leg-eye & ankle bump\\

 mouth 	&   gɔ́ŋə-nʊ̀ã́ & river-mouth & river bank \\
		&   ʔɪ̀l-nʊ̀ã́  & breast-mouth& nipple \\
                &   dɪ́à-nʊ́ã̀& house-mouth& door \\

leg &   gɔ́n-nã́ã́   & river-leg & split of a river  \\ 

&   dáá-nã̀ã̀  & tree-leg & branch\\




 head &   kùósò-ɲúù  & god-head & sky\\
             &   tìì-ɲúù  &land-head  ({\G etym})& west\\
%stomach&   \it tɔ̀ʊ́pàtʃɪ́gɪ́ɪ́  & village-stomach & inside the village\\

 arse &  tìì-múŋ &land-arse  ({\G etym}) & east \\

neck &  vìì-báɣəná & pot-neck & neck of a container\\

testicle &   mááfà-lúrò  &  gun-testicle &  gun powder container\\

penis &  mááfà-péŋ  & gun-penis &gun trigger\\

ear &   mááfà-dɪ́gɪ́ná   & gun-ear & flintlock frizzen \\

arm &  fálá-nèŋ & calabash-arm & calabash stem\\
 
navel & fálá-ʔúl & calabash-navel & calabash node\\

nose & píí-mɪ́ɪ́sà & yam mound-nose &  part of a yam mound\\
liver & tɔ́ʊ́-pʊ̀ɔ̀l & village-liver &  important community member\\
\lspbottomrule
 \end{Qtabular}
%\end{small}

\end{table} 

Ignoring for the moment the structure in which they are involved, there seem to
be two types
of spatial interpretation accessible with body part terms. And there also seems
to be a gray zone between the two.\footnote{This gray zone may receive a
diachronic interpretation.  In
\citet[1072]{Amek07c},  the postpositions in Sɛkpɛlé are seen as evolving
``from body part and environment terms''  and have a similar, but not
identical, 
function as those of Chakali relational nouns. For instance, Sɛkpɛlé's  
postpositions ``cannot be modified'' nor can they vary ``with respect to number
 marking''.  As we shall see below, the latter property is not applicable to
Chakali relational nouns. The former is undetermined as I do not know what
counts as a `modified postposition'.} The first interpretation is the literal
attribution of human
characteristics (i.e. anthropomorphic) in  reference to parts of object. In a
such a
case, a body part term refers to a part of an object in analogy to an animate
entity. For instance, a trigger of a gun (i.e. the lever that activates the
firing
mechanism) is  attributed to the penis to characterize its physical appearance.
The
second interpretation does not designate a fixed part of an object
but a location projected from a part of an object.  In such a case it designates
a spatial environment in contact with or detached from an object
\citep[44]{Hein97}. To make the
distinction clear,  in the sentence `a label is glued on the neck of the bottle'
the body part term {\it neck} designates a breakable part of the bottle, whereas
in the sentence `John is standing at the back of the car' the body part term
{\it back} does not designate any part of the car but a relative spatial
location, the area behind the car. 


 In \citet[44]{Hein97}, the variety of denotations of body part terms is
accounted for in a diachronic perspective. The  claim is sketched in
(\ref{ex:Heine-stage}).


\ea\label{ex:Heine-stage}{\rm From body part to spatial concept: A four-stage
scenario \citep[44]{Hein97}}\\
Stage 1: a region of the human body\\
Stage 2: a region of an (inanimate) object\\
Stage 3: a region in contact with the object\\ 
Stage 4:  a region detached from the object\\
\z


However, synchronically each of these stages is observable: A  Chakali 
relational noun is more easily interpreted as  a region in contact with or
detached from an object, a body part term in a compound
noun designates
a part of an object, and a body part term used as
a full-fledged noun is associated with a part of the human body. 
Nevertheless, the examples provided in Table \ref{tab:SPA-bpt-compound} show
that the distinction is not a clear-cut one: Does the expression {\it tebul ɲuu}
designate a spatial environment  in contact with or detached from a table or  a
part of table? Both interpretations seem acceptable.

% For instance, {\it dáámúŋ}, literally `tree under', can  mean either a
% resting area or a location for the initiation rite (neither being obligatorily
% under a tree), and the sentence {\it ʊ dʊa daa muŋ nɪ} can only be interpreted
%as
% `it is under the tree', but never as `it is at the resting area' nor as  `it
%is
% at the location for the initiation rite'.
Relational nouns
are rarely found in their plural form: but on grammatical grounds, nothing
prevents them from being expressed in the plural. 
To describe a situation where  for every bench there is a calabash sitting on
it, 
the sentence in (\ref{ex:reln-plu-head}) is appropirate.


\ea\label{ex:reln-plu-head}
\gll falasa saga a koro ɲuuno nɪ \\
calabash.{\pl}  sit {\art} bench.{\pl}  {\reln .\pl} {\postp}\\
\glt `The calabashes sit on top of the benches.' 
\z

One may argue that the `top of a bench' is a spatial environment  in contact
with the bench, even a physical part of the bench, so pluralization may simply
suggest that the   `top of a bench' is a word refering to an entity,  and not a
locative phrase. Two pieces of evidence go against this view: first,  notice
that {\it koro} `bench'  in {\it koro ɲuuno} is plural.  Recall Section
\ref{sec:GRM-com-stem-noun} in which a noun class ({\it sg./pl.} marking) was
argued to  appear only at the end of a word. If  `top of a bench' was a word and
not a phrase, we would expect its plural form to be  *{\it korɲuuno}. Secondly, 
deciding whether or not the `top of' is indeed in contact with or detached from
the bench is not conclusive. To describe a situation where several balls are
under several tables, one may use the sentence in (\ref{ex:reln-plu-stomach}), 
in which case it cannot be argued that  under of the table is a physical part of
the table.\footnote{One may argue that it is indeed a part of the table,
identical to the interior space of a container.}

\ea\label{ex:reln-plu-stomach}
\gll a bɔlsa dʊa tebulso patʃɪgɪɛ nɪ  \\
{\art} ball.{\pl}  be.at  table.{\pl}  {\reln .\pl} {\postp}\\
\glt `The balls are under the tables.'
\z



% kopu suguli a tebul nyu- logun ni
% a haan dua u gantal ni 'the woman is at his back'
% The above sentences are right. But in the case of "a haana dua ba gantala ni"
%we  only pularise the subject (haana) but keep the location (gantal) singular
% because 'ba' has already indicated that they are many. So we have 'a haana dua
% ba gantal ni' NOT 'a haana dua ba gantala ni'. However, a balsa dua tebulSO
% pachigiE ni is right. Why because, we are referring to objects and not humans
%in  which case we cannot use 'ba' indicate it's a single object or many. For
% instance, we can have " a fota saga dasa nyuuni ni" (the baboons are on the
% trees) 'a zinni laga da nyu ni' (the bats are hanging on a tree).

Another aspect of relational nouns and oblique phrases in general is that
they are structurally very rigid, that is, one does not usually extract from
them nor does
one preposes the oblique object phrase or elements from the phrase. Although
rarely
topicalized, the
sentences in  (\ref{ex:reln-extract-in-1})  is acceptable. 

%other charac: conjunction

\ea\label{ex:reln-extract}
\begin{xlist}
\ea\label{ex:reln-extract-in-1}

\gll  a tebul  ɲuu nɪ, a fala saga  \\
 {\art } table {\reln} {\postp}  {\art } calabash sit\\
\glt `On top of the table, the calabash sits.'

 \ex\label{ex:reln-extract-in-2}
\gll  tebul lo, a fala saga ʊ   ɲuu nɪ  \\
 table {\foc} {\art} calabash sit {3.\sg.\poss} {\reln. top} {\postp}   \\
\glt `Table, the calabash sits on top of it.' ({\it lit.} `sits on its head')
 \ex\label{ex:reln-extract-out-3}
* tebul lo, a fala saga  ɲuu nɪ
 \ex\label{ex:reln-extract-out-1}
 * ʊ   ɲuu nɪ, a fala saga tebul
 \ex\label{ex:reln-extract-out-2}
 *   ɲuu nɪ, a fala saga tebul

 \z
\z
\z

The sentence in  (\ref{ex:reln-extract-in-2}) is acceptable but odd. It shows
that the nominal complement of the relational noun
{\it ɲuu} can be uttered at the beginning of the sentence while the possessive
pronoun {\it ʊ}  is located  in the complement slot of   the relational noun, 
functioning as anaphora. The sentence is ungrammatical if the
pronoun is absent {\it in situ} (\ref{ex:reln-extract-out-3}),  or if the
oblique object phrase is preposed but the nominal {\it tebul} stranded, whether
an anaphora referring to {\it tebul} is present  (\ref{ex:reln-extract-out-1}) 
or absent
(\ref{ex:reln-extract-out-2}). 



We now have  evidence for treating the relational nouns as
members of a closed class of lexical items whose function is to localize the
figure to a
search domain.  It is not only that body part terms acquire spatial meaning
following a noun referring to inanimate entities, but that, in diachrony, 
only a limited set of body part
terms has acquired that spatial meaning,  and, in synchrony, 
they form a subtype of nominal identified as relational noun. They are 
nouns since they can pluralize, but they acquire the status of functional words
since they constitute a formal class with limited membership where each of
the members expresses spatial meaning and requires a nominal complement.

\ea\label{ex:postp-struct}
 [[[a dɪa]$_{NP}$ ɲuu]$_{RelP}$ nɪ]$_{PP}$   `on the roof of the house' 
\z

In the structure  of the BLC oblique  phrase  in (\ref{ex:postp-struct}), the 
relational noun {\it ɲuu} is within the complement phrase of the postposition 
{\it 
nɪ}.  A relational noun phrase (RelP) consists of a head  and noun phrase 
complement.  We are now in a  better position to state that the complement 
phrase of the postposition is a (nominal) phrase which corresponds to the 
conceptual ground. 


To summarize, on a diachronical basis, it is believed that the function of 
relational nouns  as locative adpositions originates from their purely `entity'
meaning
through grammaticalization \cite[44, 83]{Hein84}. The form of Chakali body part
terms supports the claim.   On a synchronical basis, only  {\it patʃɪgɪɪ}
`stomach',  {\it logun } `rib',  {\it gantal} `dorso', {\it muŋ}   `arse', {\it 
nʊã} `mouth',  {\it sʊʊ} `front', {\it bambaaŋ} `chest box'  and {\it ɲuu} 
`head'
are relational nouns. Relational nouns are  nouns which lack the 
referential power of the default interpretation of  body part term  (i.e.
interpreted in  isolation), and which take a
complement which must obligatorily be filled by an entity capable of projecting
a spatial environment.

%\cite[137]{Hein97}
%inherent vs. relative spatial terms
%check with Linguistic Semantic book, examples...


\subsection{Pronouns and pro-forms}
\label{sec:GRM-pronouns}

 A pronoun is a type of pro-form.  The difference between pronouns and 
pro-forms depends on whether they can be 
anaphors of nominal arguments. In 
this section, the personal, impersonal,
demonstrative, and possessive pronouns are introduced, followed by the
expressions used to convey reciprocity and reflexivity.   In
Section \ref{sec:GRM-adv-pro},  the adverbial pro-forms are  introduced.


\subsubsection{Personal pronouns}
\label{sec:GRM-personal-pronouns}
Table \ref{tab:GRM-pers-pro} gives an overview of the personal pronoun forms 
in the language.


\begin{table}[h]
 \caption{Weak, Strong, and Emphatic forms of personal 
pronouns\label{tab:GRM-pers-pro}}
  \centering
  \begin{Itabular}{llll}
\lsptoprule 
Pronoun & Weak form ({\sc wk})   & Strong form ({\sc st})  & Emphatic form 
({\sc emph}) \\
Gram. function  &    {\sc s|a} and {\sc o}  &  {\sc s|a} & {\sc s|a} \\[1ex]
\midrule
{\sc 1.sg.} &  n̩ &   mɪ́ŋ & ń̩wà \\
{\sc 2.sg.}  &   ɪ& hɪ́ŋ & ɪ́ɪ́wà\\
{\sc  3.sg.}  &  ʊ&  wáá & ʊ́ʊ́wà\\
{\sc 1.pl.}  &   ja&  jáwáá & jáwà\\
{\sc 2.pl.} &    ma &   máwáá & máwà\\
{\sc  3.pl.g}a &  a  &   áwáá & áwà\\
{\sc 3.pl.g}b  &   ba&   báwáá & báwà\\
    
\lspbottomrule
  \end{Itabular}
\end{table}



The  personal pronouns presented in Table \ref{tab:GRM-pers-pro} do not encode 
a 
gender distinction in the singular. In the plural  however an animacy  
distinction is made between non-human and  human. They are glossed {\sc 
3.pl.g}a 
 and {\sc  3.pl.g}b  respectively (see Section \ref{sec:GRM-gender} for the 
role 
of gender in agreement).  The weak forms can surface either with a low or high 
tone; the former being conditioned by the irrealis mood (\ref{sec:}).  The 
strong and emphatic forms are attested with the melodies with which they are 
associated. 

While an emphatic pronoun can co-occur with a focus marker,  a strong pronoun 
cannot. Both emphatic and strong pronouns may be fronted but weak pronouns 
cannot. While both emphatic and strong pronouns may appear  at the front of the 
sentence, the canonical object position tells us that a strong pronoun is an 
argument while the   emphatic pronoun is a  co-referent, a sort of cleft 
pronoun. 
 
 


% %Tristan Purvis
% % STR
% % – strong pronouns (proposed gloss in lieu of
% % EMPH
% % atic for personal pronouns, reserving
% % EMPH
% % atic for other emphatic func
% % tions, following the analysis proposed in this paper;
% % weak pronouns are default) 


  \ea\label{ex:GRM-pro-WSE}
   
 \begin{multicols}{2}
  

   \ea\label{ex:GRM-pro-W}
\gll ʊ̀ dí kʊ̄ʊ̄ rā\\
     {\sc 3.sg} eat t.z. {\sc foc} \\
\glt  `She ate T. Z.' 
   
   \ex\label{ex:GRM-pro-S}
   \gll wáá dí kʊ̄ʊ̄ (*ra)\\
     {\sc 3.sg.st} eat t.z. \\
\glt  `SHE ate t. z.' 
   
 \ex\label{ex:GRM-pro-E}  
      \gll ʊ́ʊ́wà dí kʊ̄ʊ̄ rā\\
     {\sc 3.sg.emph} eat t.z.  {\sc foc} \\
\glt  `It is her who ate t. z.' 

 \ex\label{ex:GRM-pro-W-cleft}  
    * ʊ m̩ maŋʊʊ ra
     
     \ex\label{ex:GRM-pro-S-cleft}  
      \gll  wáá  m̩̀ màŋà\\
  {\sc 3.sg.st} {\sc 1.sg}  beat (*ra) \\
 \glt   `Him, I beat.'
  \ex\label{ex:GRM-pro-E-cleft}  
      \gll  ʊ́ʊ́wà  m̩̀ máŋʊ́ʊ́ rā\\
        {\sc 3.sg.emph}  {\sc 1.sg}  beat.{\sc 3.sg}   {\sc foc} \\
 \glt   `It is him who I beat.'

  \z
  
   \end{multicols}
  \z
  
  
  The first person singular pronoun is a syllabic nasal which assimilates its 
place feature from the following phonological material (see Section 
\ref{sec:ext-nasal-place}). The  distinction between weak ({\sc wk}) and strong 
({\sc st})  is relevant when pronouns function as subject. Their proper use is 
conditioned by the emphasis  placed on the participant(s) of the event or the 
event itself, and by the polarity of the clause in which they appear.  





  \ea\label{ex:GRM-weak-strong-arg}
  
   \ea\label{ex:}
\gll   mɪ́ŋ      jáwàà   kɪ̀nzɪ́nɪ́ɪ̀\\
     {\sc 1.sg.st}  buy  horse\\
\glt  `I bought a horse.' 
   
   
   \ex\label{ex:}
\gll    ǹ    jáwá kɪ̀nzɪ́nɪ́ɪ́ rà\\
     {\sc 1.sg.wk}  buy  horse  {\sc foc}  \\
\glt  `I bought a HORSE.'
   
   \ex\label{ex:}
\gll    ǹ̩    wà jáwá kɪ̀nzɪ́nɪ́ɪ̀\\
    {\sc 1.sg.wk} {\sc neg} buy  horse \\
\glt  `I did not buy a horse.' 
  
   \ex\label{ex:GRM-out-STR-FOC-buy}
    *mɪŋ      jawa   kɪnzɪnɪɪ ra
   \ex\label{ex:GRM-out-STR-NEG-buy}
     *mɪŋ    wa  jawa   kɪnzɪnɪɪ 

   
  \z 
 \z




  \ea\label{ex:GRM-weak-strong-verb}
   

   
   \ea\label{ex:}
\gll    ǹ̩   pétījó\\
    {\sc 1.sg.wk}  {terminate.{\sc pfv.foc}}  \\
\glt  `I finished.' 


   \ex\label{ex:}
\gll    mɪ́ŋ   pétījé\\
     {\sc  1.sg.st} {terminate.{\sc pfv}} \\
\glt  ` `I finished.'  

   \ex\label{ex:GRM-out-STR-FOC-finish}
  *mɪŋ   petijo

   \ex\label{ex:GRM-out-STR-NEG-finish}
  *mɪŋ   wa petije
         
   
  \z 
 \z

Thus, strong pronouns cannot co-occur in a sentence in which another
constituent is in focus, that is a nominal phrase  flanked by the focus
marker  or   a
verb ending with the assertive suffix vowel   {\sc -[+ro,  +hi]}    (see
examples 
(\ref{ex:GRM-out-STR-FOC-buy}) and (\ref{ex:GRM-out-STR-FOC-finish})). In
addition,  in
sentences 
where a negative operator occurs, strong pronouns are disallowed, as  
(\ref{ex:GRM-out-STR-NEG-buy}) and   (\ref{ex:GRM-out-STR-NEG-finish}) show.

  
  %clitizised pronoun in objetc position here. Bring a CVVV example
  
  
 \subsubsection{Impersonal pronouns}
 \label{sec:GRM-impers-pro}

An impersonal pronoun does  not refer to a particular person or thing. The form
{\it a} is treated as an impersonal pronoun  in some
particular context.

               
\begin{exe} 
\ex\label{ex:imps-pro-sing}
\gll à mááséjó kéŋ̀\\
     {\sc 3.sg.imps}  enough.{\sc pfv.foc} {\sc adv}\\
\glt `That's enough' or `That's it' or `Stop'
\z

Example (\ref{ex:imps-pro-sing}) is treated as a type of impersonal
construction. It is characterized by its  subject position being  occupied by 
the
pronoun {\it a}, which may be seen as referring to the situation,  but not to 
any
participant. The example in  (\ref{ex:imps-pro-sing}) may
be appropriate in contexts involving pouring  liquids or giving food on a plate,
or when people are quarrelling. In these hypothetical contexts, using the
personal
pronoun {\it ʊ} instead of  the  impersonal pronoun  {\it a} would be
unacceptable.  

The language does not have a passive construction as one finds in English.
Nonetheless,  an argument  can be demoted  by placing it in object position.
This is shown in (\ref{ex:GRM-vp23.3.}).

\ea\label{ex:GRM-vp23.3.}
\gll ká à nàmɪ̃̀ã́?  bà tíéú rò\\
      {\sc q} {\sc art} meat {\sc 3.pl.g}b  eat.{\sc pfv}.{\sc 3.sg.obj} {\sc
foc}\\
\glt  `Where is the meat? It has been eaten.'
\z


This type of impersonal
construction illustrated in  (\ref{ex:GRM-vp23.3.})  is characterized by the
personal pronoun {\it ba} ({\sc 3.pl.g}b)  in
subject position. In this context,  the subject is not a known agent and the 
pronoun {\it ba} does not refer to anyone/anything in 
particular. Therefore,  the pair {\it a}/{\it ba} is treated  as the singular 
and
plural impersonal pronouns, only when they occur in impersonal constructions, 
as shown above.






\subsubsection{Demonstrative pronouns}
\label{sec:GRM-demons-pro}


In the examples (\ref{ex:GRM-demons-pro-reply-1}) to
(\ref{ex:GRM-demons-pro-quest}),  the demonstrative pronouns  
function as noun phrases. All the examples below were accompanied with
 pointing gestures when uttered.



\ea\label{ex:GRM-demons-pro-reply-1}{\rm Replies to the question: Which cloth
has she chosen?}
 
 
  \ea\label{ex:GRM-demons-pro-reply-1sg} 
 \gll háǹ nā\\
   {\sc dem.sg} {\sc foc}\\
 \glt `It is this one' 
   
      \ex\label{ex:GRM-demons-pro-reply-1pl}
       \gll hámà rā\\
   {\sc dem.pl} {\sc foc}\\
 \glt `It is these ones' 
 
\z 
 \z



\ea\label{ex:GRM-demons-pro-quest}{\rm The speaker asks the addressee whether 
he had moved a certain object.} 

 \gll   ɪ̀ jáá háǹ nȁ\\
  {\sc 2.sg} do {\sc dem.sg}  {\sc foc}\\
 \glt `You did THIS?' 

\z


\ea\label{ex:GRM-demons-pro-quest}{\rm Explanation on how the fingers cooperate
when they scoop t. z. from a bowl.}

 \gll hámàā ká zɪ̀ pɛ́jɛ̀ɛ̀ à zɪ́ já wà tììsè  háŋ̀\\
 {\sc dem.pl} {\sc egr} then add.{\sc pfv} {\sc conn}  then do come support {\sc
dem.sg} \\
 
\glt `These (two fingers) are then added,
and then they come to support  this one.' 

\z



The expressions {\it háŋ̀} ({\it sg.}) and {\it hámà}
({\it pl.}) are employed for spatial deixis, specifically as proximal
demonstratives, corresponding to English `this' and `these' respectively. The
language does not offer another set for distal demonstratives.



\subsubsection{Interrogative words}
\label{sec:GRM-interg-pro}


Interrogative constructions are of two types:  {\it yes/no} interrogatives
and  {\it pro-form} interrogatives (see Section
\ref{sec:GRM-interr-clause}). The former
type, as the dichotomy suggests, requires  
a `yes' or a `no' answer.  A {\it pro-form} interrogative  uses  an
interrogative word which identifies the sort of information requested. In
Chakali,  some interrogative words may be treated as pronouns, while others may
be treated as the combination of a noun and a pronoun.  Table
\ref{tab:GRM-interg-pro} gives a list of interrogative words, together with an
approximate English translation,  the sort of information requested by each  and
a link to an illustrative example of {\it pro-form} interrogatives. The examples
are listed in (\ref{ex:GRM-interg-pro}). The question words are glossed as  {\sc
q}.


\begin{table}[htb!]

 \caption{Interrogative pronouns \label{tab:GRM-interg-pro}}
  \centering
  \begin{Itabular}{llll}
\lsptoprule 
Pronoun & Gloss  & Meaning requested & Example  \\[1ex] \midrule
bàáŋ́ & what &  non-animate entity, event & \ref{ex:vp1.11.a}\\
 àŋ́ & who & animate entity & \ref{ex:vp2.5}\\
 lìé & where & location & \ref{ex:vp9.25}\\
ɲɪ̀nɪ̃̀ɛ̃́ & why/how & condition, reason& \ref{ex:vp22.4.1}\\
(ba/a)wèŋ́  & which &  entity, event & \ref{ex:vp22.4.4}\\
 (ba/a)ŋmɛ̀nà & (how) much/many & entity, event & \ref{ex:vp22.4.10}\\
 sáŋ(a)-wèŋ́ & when & time & \ref{ex:vp22.4.15}\\
\lspbottomrule
  \end{Itabular}
 
\end{table}


  
  \ea\label{ex:GRM-interg-pro}
  
\ea\label{ex:vp1.11.a}
\gll bàáŋ́ ɪ̀ kàà jáà \\
{\sc q} {\sc 2.sg} {\sc egr} do\\
\glt `What are you doing?' 


\ex\label{ex:vp2.5}
\gll  àŋ́ ɪ̀ kà ná à tɔ́ʊ́ nɪ̄ \\
     {\sc q} {\sc 2.sg}  {\sc egr}  see {\sc art} village {\sc postp} \\
\glt  `Whom did you see in the village?' 

%check comp here
\ex\label{ex:vp9.25}
\gll lìé nī dɪ̀ tʃʊ̀ɔ̀lɪ́ɪ́ kà dʊ̀ɔ̀ \\
        {\sc q} {\sc postp} {\sc comp} sleeping.room   {\sc egr} exist \\
\glt  `Where is the room for sleeping?' 



\ex\label{ex:vp22.4.1}
\gll ɲɪ̀nɪ̃̀ɛ̃́ ɪ̀ já kà jááʊ́ \\
      {\sc q}  {\sc 2.sg} {\sc hab}   {\sc egr} do.{\sc 3.sg.obj} \\
\glt  `How do you do it?' 



\ex\label{ex:vp22.4.4}
\gll áwèŋ́ ɪ̀ kà kpàɣà \\
      {\sc q}   {\sc 2.sg}  {\sc  egr}    catch  \\
\glt  `Which one did you catch?' 


\ex\label{ex:vp22.4.10}
\gll àŋmɛ̀ná ɪ̀ kà kpàgàsɪ̀ \\
         {\sc q}    {\sc 2.sg}  {\sc  egr}  catch.{\sc pv}  \\
\glt  `How many of them did you catch? (non-human reference)' 

\ex\label{ex:vp22.4.15}
\gll {sáŋáwèŋ́} ɪ̀ kàà wáá \\
       {\sc q} {\sc 2.sg} {\sc  egr}    come \\
\glt  `When are you coming?' 
  
   
  \z 
 \z


When the question word {\it lie} `where' is  followed by the locative
postposition {\it nɪ},  a request for a particular location is interpreted. 
This question word can
also be
followed by the noun  {\it pe} `end' in which case it should be interpreted
as
`where-towards' or `where-by', e.g. {\it lie pe ɪ ka vala} `Where did he go
by?'.  Another form used to request information on a location is {\it ká(á)}.
This form is neither specific to Chakali nor to location {\it per se}:
the languages Waali and Dagaare use it for the same purpose and the
form is even used to request other types  of information. For instance, {\it 
káá tʊ́má} means `how is work?' in the three languages. It might be that
Chakali borrowed the form from Waali.  It was
employed consistently in an experiment I carried out, which is discussed  in
Section \ref{sec:SPA-exper1}. Example  (\ref{ex:GRM-vp23.3.}),  repeated below, 
illustrates the use of {\it ka(a)} as interrogative word.

\begin{exe}
 

\exp{ex:GRM-vp23.3.}
\gll ká à nàmɪ̃̀ã́?  bà tíéú rò\\
      {\sc q} {\sc art} meat {\sc 3.pl.b} chew.{\sc pfv}.{\sc 3.sg.obj} {\sc
foc}\\
\glt  `Where is the meat? It has been eaten.'

\end{exe}





When they stand alone as interrogative words, the
expressions {\it weŋ} and {\it ŋmɛna}, roughly
corresponding to English `which' and `how much/many', must be prefixed by either
{\it a-} or {\it ba-} reflecting a distinction between non-human and human 
entities respectively (see Section \ref{sec:GRM-gender}). The expression {\it 
saŋa weŋ} in (\ref{ex:vp22.4.15.}) is
literally translated as `time which'.     The question word {\it baaŋ} can be 
used together with {\it wɪɪ} to correspond to English `why', i.e.
{\it baaŋ wɪɪ ka wa ɪ dɪ wii}? `Why are you crying?'.  The expression {\it baaŋ
wɪɪ} is equivalent to English `what matter'. 





%\begin{exe}
%\ex\label{ex:vp23.1.}
%\glll  àná ká tūgùù?\\
 %aŋ ka tuga-u\\
    %{\sc qw}   {\sc  egr} beat-{\sc 3.sg.obj}\\
%\glt  `By whom is he being beaten.'
%\z

%The question words may be followed by the focus particle. This is shown in
%example (\ref{ex:vp23.1.}) with the question word {\it aŋ}  `who'.




\subsubsection{Possessive pronouns}
\label{secːGRM-poss-pro}

The possessive pronouns are displayed in Table \ref{tab:posspro}. 

\begin{table}[h!]
  \caption{Possessive pronouns \label{tab:posspro}}
  \centering
  \begin{Itabular}{ll}
\lsptoprule 
Pronoun       &  Form  \\
Gram. function    &   Possessive  \\[1ex] \midrule
{\sc 1.sg.poss}     & n̩(ː) \\
{\sc 2.sg.poss}      &   ɪ(ː)\\
{\sc  3.sg.poss}      &  ʊ(ː) \\
{\sc 1.pl.poss}      &   ja  \\
{\sc 2.pl.poss}       &    ma \\
{\sc  3.pl.a.poss}    &  a(ː)  \\
{\sc 3.pl.b.poss}      &  ba \\
    
\lspbottomrule
  \end{Itabular}
\end{table}

A possessive pronoun with a form C or V  tend to be lengthened,  although their 
length have no meaning. These pro-forms are used in the possessor slot ({\sc 
psor}), but never in
the possessed slot ({\sc psed}) of an attributive possessive relation. This
is shown in (\ref{ex:vp7.15}). 

\ea\label{ex:vp7.15}
\glll à kùórù ŋmá dɪ́ ʊ̀ʊ̀ hã́ã́ŋ tʃɔ́jɛ̄ʊ́ \\
{} {} {}  {} {\sc psor} {\sc psed}  {}\\
     {\sc art} chief say {\sc comp} {3.sg.poss} wife ran.{\sc pfv.foc}   \\
\glt  `The chief said that his wife ran away.' 
\z

The  weak personal pronouns and the possessive pronouns have the same forms, the
differences between the two being their respective syntactic positions and their
argument structures. The weak pronoun normally precedes a verb while the
possessive pronoun normally precedes a noun, and the weak pronoun is an
argument of a verbal predicate while the possessive pronoun can only be the
possessor in a possessive attributive construction. 
%tone differences


\ea\label{ex:vp7.15}
\glll                                                                    
{\it mɪ́n{\T ꜜ}ná}  \\
 mɪŋ na\\
{\sc 1.sg.st.} {\sc foc}\\

\glt  `It is MINE.' 
\z


Phrasal possessives, as in English `mine, yours, etc.', are expressed with the
strong personal pronoun  in a verbless identificational
construction. This is shown in (\ref{ex:vp7.15}).




\subsubsection{Reciprocity and reflexivity}
\label{sec:GRM-recipro-reflex}


Reflexive and reciprocal pronouns do not exist in Chakali.  Instead,
reciprocity and reflexivity  are
encoded in  the nominals {\it dɔŋa}   and {\it tɪntɪn}, which are glossed in 
the
texts as {\sc recp}  and {\sc refl} respectively.   Reciprocity is illustrated 
in
(\ref{ex:GRM-recipro}) and reflexivity in (\ref{ex:GRM-reflex}). 

%more like emphasis than reflexivity

% They are categorized nominals since {\it % dɔŋa}  is believed to 

  \ea\label{ex:GRM-recipro}
   
   
\ea\label{ex:vp24.1.}
\gll à nɪ̀báálá bálɪ̀ɛ̀ kpʊ́ dɔ́ŋá wā \\
     {\sc art} men two kill    {\sc recp}  {\sc foc} \\
\glt  `The two men killed EACH OTHER.' 

\ex\label{ex:vp24.2.}
\gll jà kàá kpʊ́ dɔ́ŋá wá \\
      {\sc 1.pl} {\sc fut} kill  {\sc recp}   {\sc foc} \\
\glt  `We will kill  EACH OTHER.' 

\ex\label{ex:vp24.3.}
\gll à hàmṍwísè káá júó dɔ́ŋá rā \\
     {\sc art} children {\sc  egr} fight {\sc recp}   \\
\glt  `The children are fighting against one another.' 
 
   
  \z 
 \z



 \ea\label{ex:GRM-reflex}
  
   
\ea\label{ex:vp25.1.}
\gll  à báál kpʊ̄ ʊ̀ tɪ̀ntɪ̀ŋ \\
      {\sc art} man kill  {\sc 3.sg.poss} {\sc refl.sg} \\
\glt  `The man killed himself.' 

\ex\label{ex:vp25.2.}
\gll jà kàá kpʊ̄ jà tɪ̀ntɪ̀nsá wá \\
     {\sc 1.pl}  {\sc fut} kill {\sc 1.pl.poss}  {\sc refl.pl} {\sc foc} \\
\glt  `We shall kill OURSELVES  .'

\ex\label{ex:vp25.4.}
\gll à bìé kpá kísìé dʊ̄ ʊ̀ʊ̀ tɪ̀ntɪ̀ŋ dáŋɪ́ɪ́\\
       {\sc art} child take knife put     {\sc 3.sg.poss}   {\sc
refl.sg} wound.{\sc nmlz}\\
\glt  `The child wounded himself with his knife.' 
  
   
  \z 
 \z




\subsection{Qualifiers}
\label{sec:GRM-qualifier}

% 
% % 
% % % In the naming data the great
% % % majority of the 1560 expressions referring to the 62 tiles have the form  
% % {\it % a-X} or {\it kɪn-X}. In Section \ref{sec:gramsketch} and 
% % \ref{sec:gramsketch}
% % % respectively  these prefixes were described as  (i)
% % %  affixing  on property- and state-denoting predicates, (ii) encoding a
% % %selection
% % % of
% % % semantic features of the referent to which the property modifies,  and 
% (iii) 
%  
% % %  as  purely grammatical prefixes, the former renders a property into a
% % % qualifier and the latter into a noun. Thus the words {\it % a-pʊmma} 
% `white' 
% %  % and  {\it kɪn-pʊmma} `white'  are syntactically qualifiers
% % % %and
% % % % nouns
% % % % respectively. 
% % % %  Also
% % % % recurrent in the naming data is the focus marker {\it ra}, introduced in
% % % %section
% % % % \ref{sec:gramsketch}, following the expressions {\it a-X} or {\it % 
% kɪn-X}. 
% % % The three frames are illustrated in (\ref{ex:most-frame}).
% % % % a- prefix
% % 
% % % 
% % 
% % \subsubsection{Visual perception terms}
% % \label{sec:GRM-visual-percep}

Since qualifiers display singular/plural pairs (as  do nouns) and verbs do not 
inflect for number (but see pluractional verb in  Section 
\ref{sec:GRM-PluralVerb}), qualifiers are treated  as  nominals. Examples are 
presented in (\ref{ex:GRM-qual}).


\ea\label{ex:GRM-qual}
 
  \ea\label{ex:GRM-qual-red}
sɪ̀àmá {\it sg.}, sɪ̀ànsá {\it pl.}   ({\sc cl.}1) `red'
  \ex\label{ex:GRM-qual-bad}
 bɔ́ŋ̀ {\it sg.}, bɔ́má {\it pl.}  ({\sc cl.}3)  `bad'
  \ex\label{ex:GRM-qual-real}
dɪ́ɪ́ŋ {\it sg.}, dɪ́ɪ́má {\it pl.} ({\sc cl.}3) `true, real' 

  
\z 
 \z


As shown in (\ref{ex:GRM-qual}), qualifiers agree in number with the head of the
noun phrase. 

\ea\label{ex:GRM-qual-agree}
 
  \ea\label{ex:GRM-qual-agree-sg}

\gll à  nɪ̀hã́ã́ŋ pɔ́lɪ̀ɪ̀\\
{\art} woman.{\sc cl.3.sg} fat.{\sc cl.4.sg}   \\
\glt `The fat woman' ({\it hã́pɔ́lɪ̄ɪ̀})


  \ex\label{ex:GRM-qual-agree-pl}

\gll a nɪhããna pɔlɛɛ\\
{\art} woman.{\sc cl.3.pl} fat.{\sc cl.4.pl}   \\
\glt `The fat women' ({\it hã́pɔ́lɪ̄ɛ̀})

  
\z 
 \z

The examples in (\ref{ex:GRM-qual}) differ from complex stem nouns (Section 
\ref{sec:GRM-com-stem-noun}) since they are phrases and not words: each is 
provided with a possible equivalent word.
Many qualifiers are assigned to noun class 4, the
reason being that qualifiers are often nominalized verbal stems, e.g. {\it 
hʊlɪɪ/hʊlɪɛ} ({\it qual}) `empty' $\leftarrow$ {\it hʊl}  ({\it v})  `dry'.  
Examples are
provided in  (\ref{ex:GRM-qual-cl4}).


\ea\label{ex:GRM-qual-cl4}
 
  \ea\label{ex:GRM-qual-cl4-call}
jɪra `call' $>$ jɪ́rɪ́ɪ́  {\it sg.},  jɪ́rɪ́ɛ́  {\it pl.} `calling'
\ex\label{ex:GRM-qual-cl4-give-birth}
lʊla `give birth' $>$ lʊ́lɪ́ɪ́ {\it sg.},   lʊ́lɪ́ɛ́ {\it pl.} `giving birth'
\ex\label{ex:GRM-qual-cl4-die}
sʊwa `die' $>$ sʊ́wɪ́ɪ́ {\it sg.},  sʊ́wɪ́ɛ́ {\it pl.} `corpse'

  
\z 
 \z



Nonetheless, the two categories, noun, and qualifier, are differentiated by the
following characteristics: (i)  a qualifier must be semantically verbal (i.e. 
denoting a state or an event),
a noun must not necessarily be, and (ii) while a qualifier
modifies a noun,  a  noun functions as  the
nominal argument of the qualifier. The asymmetry is reflected in
(\ref{ex:GRM-qual-hot}).

\ea\label{ex:GRM-qual-hot}{\rm  /nʊm/ `hot'}
 
  \ea\label{ex:GRM-qual-hot-cmp-stem}
  \glll nɪ̀ɪ̀nʊ́ŋ ná \\
 nɪɪ-nʊŋ na \\
     water-hot {\sc foc}\\
  \glt `It is HOT WATER.'

 \ex\label{ex:GRM-qual-hot-head}
  \glll  à nɪ́ɪ́ nʊ́mã́ʊ̃́\\
 a  nɪɪ nʊma-ʊ\\
      {\sc art} water hot-{\sc pfv.foc}\\
  \glt `The water is HOT.'

 \ex\label{ex:GRM-qual-hot-qual}
  \glll  à nɪ̀ɪ̀ nʊ́mɪ́ɪ́ dʊ́á dé nɪ̄\\
 [a nɪɪ nʊm-ɪ-ɪ]$_{NP}$ dʊa de\\
  {\sc art} water hot-{\nmlz}-{\sc cl.4} exist {\sc dem}\\
  \glt `The hot water is there.'
  
\z 
 \z

In (\ref{ex:GRM-qual-hot-cmp-stem}) the stem {\it nʊm} `hot' is part of the
complex stem noun {\it nɪ́ɪ́nʊ̀ŋ} `water-hot' (see Section
\ref{sec:GRM-com-stem-noun}).  In this morphological configuration, a
qualitative
modification is  established  between the stem {\it nʊm} and the stem {\it nɪɪ}.
In (\ref{ex:GRM-qual-hot-head}), {\it nʊm}  functions as a verbal predicate in
the
intransitive clause, and the definite noun phrase {\it a nɪɪ} `the water'
occupies
the argument position. In (\ref{ex:GRM-qual-hot-qual}) the stem {\it nʊm} is
nominalized and the singular of  noun class 4 is suffixed. The word {\it 
nʊ́mɪ́ɪ́} may be translated as  `the result of heat'. It is treated as a
qualifier since {\it nɪɪ} `water' is  (the head of) the argument of the
predicate, and {\it dʊa} is a predicate which needs   one core argument. Since 
{\it nʊm}  can neither function as main predicate nor as head noun of the
argument phrase, and since {\it nʊm}  is understood to be a property of the
entity
and not of the event, then {\it nʊm} in (\ref{ex:GRM-qual-hot-qual}) is viewed 
as
a qualifier.


Given the arguments put forward, one could analyze the qualifiers as adjectives.
Both are  seen  categorically as nominals  and semantically as properties or
states.  However, there are no lexemes in Chakali  which can be assigned
the category adjective, that is, no lexeme which, in all  linguistic contexts,
can be identified as categorically distinct from nouns and verbs.  For instance,
the lexeme `intelligent' in English is an adjective in all linguistic contexts
and can `never' function as a noun or as  a verb.\footnote{Although
`intelligent' is a noun in a construction like `the
intelligent find cryptic crosswords challenging' (S. Foldvik, p. c.),  it would
not be
controversial to say that `intelligent'  undergoes a sort of  zero-derivation, 
i.e.
adjective $>$ noun.  As mentioned in footnote \ref{ft:GRM-fre-eng-deri},
distinctions between categories in English and French (and other Indo-European
languages) are often not formally signaled. }  There are no such lexemes in
Chakali. Qualifiers are
either derived
linguistic entities or idiomatic
expressions. More than one procedure is attested to construct qualifiers. In
(\ref{ex:GRM-qual-types}),   some types of qualifiers are provided.

\ea\label{ex:GRM-qual-types}
 
 \ea\label{ex:GRM-qual-t0} àbúmmò `black'  
     \ex\label{ex:GRM-qual-t1} àpʊ́lápʊ́lá `pointed, sharp'
  \ex\label{ex:GRM-qual-t2}  wɪ̀ɛ́zímíí  `wise' 


 
\z 
 \z


The expression {\it bummo} in (\ref{ex:GRM-qual-t0}) is a nominal lexeme. When
it functions as a qualifier within a noun phrase,  the prefix vowel {\it a-} is
suffixed to the nominal stem. Notice that this prefix vowel also occurs on
numerals (see chapter \ref{sec:NUM}). The type of qualifier found in
(\ref{ex:GRM-qual-t1}) is often used to
describe perceived patterns, including color, texture, sound, manner of motion,
e.g. {\it gã́ã́nɪ̀gã́ã́nɪ̀} `cloud state',  {\it adʒìnèdʒìnè}
`yellowish-brown',  {\it tùfútùfú} `smooth and soft'. Reduplication
characterises the form of this type of qualifiers. When a reduplicated qualifier
occurs in attributive function, i.e. following the head noun, it takes the
prefix {\it a-} as well.\footnote{Although the prefix {\it a-} on qualifier 
tends
to disappear in
normal speech. The prefix {\it a-} is unacceptable in (\ref{ex:GRM-qual-t2}).}
The word in (\ref{ex:GRM-qual-t2}) is segmented as [[[{\sc
theme}-v]-{\sc nmlz}]-{\sc cl.4}]. The verbal stem {\it zɪm} `know'   sees  its
theme argument incorporated, i.e.  {\it wɪɛ-zɪm} `matters-know',  a structure
which is in turn nominalized by what I called in Section \ref{sec:GRM-der-agent}
event-nominalization.  The qualifiers in
(\ref{ex:GRM-qual-types}) are presented 
in corresponding syntactic positions in (\ref{ex:GRM-qual-types-bar}). 

%  The prefix {\it a-} is also found with verbal state
% qualifiers.


\ea\label{ex:GRM-qual-types-bar}
 

 \ea   {\it  [X àbúmmò]$_{NP}$ dʊ̀à dé}  `The black X  is
there'
  \ex   {\it  [X àpʊ́lápʊ́lá]$_{NP}$ dʊ̀à dé } `The
sharp  X is there'
\ex    {\it  [X wɪ̀ɛ́zímíí]$_{NP}$  dʊ̀à dé } `The wise X is
there'
  
\z 
 \z



There are  limitations
on the number of qualifiers allowed within a noun phrase. Noun phrases with 
more than three qualifiers are often rejected by language consultants in
elicitation sessions.  The
language simply employs other strategies to stack properties. In fact noun
phrases with two qualifiers are rarely found in the texts
collected. 

The language has phrasal expressions which correspond to  monomorphemic
adjectives in some other languages. These expressions have the characteristic of
being metaphorical; their lexemic denotations may be seen as secondary, and
phrasal  denotations as non-compositional. For instance, a speaker must say 
{\it ʊ kpaɣa bambii}, {\it lit.}`he has heart', if he/she wishes to express `he 
is
brave'. The word `brave' cannot be translated to {\it bambii}, since its primary
meaning is `heart',  but to {\it kpaɣa bambii}  `to be brave'. Another way of
expressing `brave'  is {\it bambii-tɪɪna}, {\it lit.} `owner of heart'. Other
examples  are {\it síí-nʊ̀mà-tɪ́ɪ́nà}, {\it lit.} `eye-hot-owner', `wild,
violent  person'   and {\it síí-tɪ̄ɪ̄nà}, {\it lit.} `eye-owner', `stingy,
greedy person'. These expressions are more frequently used as nouns in the
complement position of the identificational construction, such as in {\it ʊ̀
jáá sísɪ̀àmàtɪ̀ɪ̀ná}, {\it lit.} she is eye-red-owner, `she is serious'.
As mentioned in Section \ref{sec:GRM-idiom},  it is often hard to establish
whether an expression is idiomatic when only one of the its components
is used in a non-literal sense.


%\begin{table}[h]
 %\begin{tabular}{lp{4cm}}
%Color & (see Section )\\
%Value & good, bad\\
 %Age & new, old, years old\\
 %Human propensity& mental state, physical state, behavior\\
 %Physical property& sense, consistency, texture, temperature, 
%edibility, sustantiality configuration\\
%Quantity&\\
 %\end{tabular}
%\caption{(frawley
%p.463) \label{tab:GRM-mod}}
%\end{table}

% Examples of 
% physical properties encoding
% 
% Sense
% Consistency
% Temperature
% Edibility
% Substantiality
% Configuration
% 




\subsection{Quantifiers}
\label{sec:GRM-quantifier}

%complex quantifier
Quantifiers are expressions denoting quantities. They refer to the size of the
referent ensemble. The words {\it muŋ} `all',   {\it banɪɛ} `some' and {\it 
tama}
`few, some' constitute the  monomorphemic quantifiers. The  former can be
expanded with a  nominal prefix. For instance, in {\it ba-muŋ} `{\sc hum}-all'
and {\it wɪ-muŋ} `{\sc abst}-all',  the prefixes identify the semantic class of
the entities which the expressions quantify (see Section \ref{sec:classifier}). 
The form of the quantifier {\it banɪɛ} `some'  is  invariable: *{\it anɪɛ}, 
*{\it abanɪɛ} and *{\it babanɪɛ} are unacceptable words.  The same can be said 
for the
word {\it tama}
`few', which stays unchanged even  when it  modifies  nouns of different
semantic classes.  Another word treated as quantifier is {\it máŋá} `only' as
in {\it a nɪhããŋ maŋa kaa waa} `Only the woman is coming'.  



The 
expression {\it kɪ̀ŋkáŋ̀} `a lot, many', which is made out of the  
classifier  {\it kɪŋ-} and the quantitative verbal state lexeme {\it kaŋ}  
`abundant'   (Sections  \ref{sec:classifier} and 
\ref{sec:GRM-verb-stative-active}, respectively) should not be confused with 
the 
intensifiers,  to which
we turn to examine the difference.

The expression {\it kɪ̀ŋkáŋ̀} `a lot, many', which  is made out of the  
classifier  {\it kɪŋ-} plus the quantitative verbal state lexeme {\it kan}  
`abundant'   (Sections  \ref{sec:classifier} and 
\ref{sec:GRM-verb-stative-active}, respectively) should not be confused with the 
intensifiers  examined in Section \ref{sec:GRM-intensifier}.
The  lexeme {\it kan} `abundant'  is semantically verbal but turns into a 
quantifier when {\it kɪŋ-}  is prefixed to it.  Other evidence for its verbal 
status  is the utterance {\it à kánã́ʊ̃́} `they are many' compared to {\it à 
jáá tàmá} `they are few'.  In the former, {\it kan} is the main verb of an 
intransitive perfective clause, while in the latter, {\it tama} is the 
complement of the verb {\it jaa} in an identificational construction  (Section 
\ref{sec:GRM-ident-cl}). Apart from {\it  kɪŋkan}  `a lot, many',  other 
plurimorphemic (or complex) quantifiers are based on the suffixation the 
morpheme {\it  -lɛɪ} `not'. The expression {\it wɪ-muŋ-lɛɪ}, {\it lit.} {\sc 
abst}-all-not, as well as {\it kɪŋ-muŋ-lɛɪ},  {\it lit.} {\sc conc}-all-not,  
both correspond to the English word `nothing' (Section \ref{sec:classifier} on 
negation).

%may be the word maŋa `only'?
%The word {\it kɔta}`is a measure term from English quarter.

The meaning `a few' can be conveyed by  the word {\it aŋmɛna} `how
much/many', which was introduced in Section \ref{sec:GRM-interg-pro} as an
interrogative
word. Example  
(\ref{ex:GRM-quant-int-only}) suggest that the word {\it aŋmɛna} can also be
used in a non-interrogative way,  co-occurring here with {\it maŋa} `only',  in
which case it is interpreted  as `amount' or `a certain number'.


\begin{exe}
 \ex\label{ex:GRM-quant-int-only}
\gll àŋmɛ̀nà máŋá tʃájɛ̄ɛ́\\
   amount only remain.{\pfv}\\
\glt `Only a few are left.'
\z

Another way to express `(a) few'  is to duplicate the numeral {\it dɪgɪɪ} `one',
e.g.
{\it dɪgɪɪ-dɪgɪɪ ra} `There are just a few of them'.  The  examples in 
(\ref{ex:GRM-quant-mean}) show that the numeral {\it dɪgɪɪ} `one' can
participate in the denotations of both total and partial quantities. 

\ea\label{ex:GRM-quant-mean}

 \ea {\it mùŋ} `all' ({\it total collective})
 \ex  {\it  dɪ́gɪ́ɪ́ mùŋ} `each' ({\it total distributive})
 \ex  {\it  dɪ̄gɪ̄ɪ̄ dɪ́gɪ́ɪ́} `some, few' ({\it partial distributive})
 
\z 
 \z

The word {\it galɪŋga} `waist' or `middle'  can also carry quantification. In
(\ref{ex:GRM-most}),  {\it galɪŋga} is equivalent to {\it bàkánà} (< {\it 
bar-kaŋ},
{\it lit.} part-abound),  and means `most'.

\begin{exe}
 \ex\label{ex:GRM-most}
\gll   à kpã́ã́má  gálɪ̀ŋgà/bàkánà tʃájɛ̄ɛ́ à láʊ́ nɪ́\\
{\art} yam.{\pl} most remain.{\pfv} {\art} farm.hut {\postp}\\
\glt  `Most of the yams remain/are left in the farm hut.'
\z

The word {\it gba} `too' is treated as a quantifier and restricted to appear
after the subject, e.g. (\ref{ex:GRM-too-out-1})-(\ref{ex:GRM-too-out-4}). In
(\ref{ex:GRM-too-pos}), the speaker  considers himself/herself  as part of a
previously established set of individuals who beat their respective child. The
quantifier is additive such that  the denotation of the subject constituent is
added to this previously established set.  In (\ref{ex:GRM-too-neg}), it is
shown that negating the quantified expression results in an interpretation where
the speaker asserts that he/she is not a member of the set of individuals who
beat their child. Since generally there is only one `in focus' constituent in a
clause and that negation and focus cannot co-occur (see Sections
\ref{sec:GRM-foc-neg} and  \ref{sec:GRM-focus}), example (\ref{ex:GRM-too})
suggests that {\it gba} is not a focus particle.


\ea\label{ex:GRM-too}

 \ea\label{ex:GRM-too-pos}
\gll ŋ̩̀ gbà máŋá m̩̀ bìè rē \\
{\sc 1.sg} {\quant} beat {\sc 1.sg.poss}  child {\foc}\\
\glt  `I beat my child too.' ({\it lit.} I too/as well/also beat my
child)

 \ex\label{ex:GRM-too-neg}
\gll ŋ̩̀ gbà lɛ̀ɪ́ máŋá  m̩̀ bìé  \\
{\sc 1.sg} {\quant}  {\neg} beat {\sc 1.sg.poss}  child \\
\glt  `I do not beat my child.' ({\it lit.} It is not me who  also
beat my child)


 \ex\label{ex:GRM-too-out-1}   \textasteriskcentered  gba m̩  maŋa a bie re
\ex \textasteriskcentered  m̩ maŋa gba a bie re
 \ex \textasteriskcentered  m̩ maŋa  a bie gba re
 \ex\label{ex:GRM-too-out-4} \textasteriskcentered  m̩ maŋa  a bie  re gba

\z 
 \z

\subsection{Intensifiers}
\label{sec:GRM-intensifier}

Although semantically it is a predicate modifier, marks a degree,  and make a 
statement stronger, I identify an intensifier on compatibility ground. Only 
color   and temperature terms have been found to be selected 
by intensifiers.


\ea\label{ex:intens-ideo} 

\ea  ásɪ̀àmā tʃʊ̃́ɪ̃́tʃʊ̃́ɪ̃́   {\rm `very/pure  red'}
\label{ex:BCTmod-prop-red}

\ex ábúmmò jírítí    {\rm `very/pure black'}
\label{ex:BCTmod-prop-black} 

\ex  ápʊ̀mmá píópíó  {\rm `very/pure  white'}
\label{ex:BCTmod-prop-white}

\ex  sʊ́ɔ́nɪ̀ júlúllú  {\rm `very cold'}
\label{ex:BCTmod-prop-cold}

\ex   nʊ̀mà kpáŋkpáŋ  {\rm `very/pure hot'}
\label{ex:BCTmod-prop-hot}


\z
\z

The  intensifier ideophones  {\it tʃʊ̃ɪ̃tʃʊ̃ɪ̃},  {\it jiriti},  {\it piopio},  
{\it julullu},  and    {\it kpaŋkpaŋ} are translated into English as `very' (or  
`pure' in the case of color) in (\ref{ex:BCTmod-prop}). They are treated 
together as one kind of degree predicate modifier.  Note that no other 
properties have been found together with a (unique and) corresponding degree 
modifier. For instance, if one wishes to express `very X', where X refers to a 
color other than black, white, or red,   one has to employ the degree modifier 
{\it pááá} following the term, which is a common expression in many Ghanaian 
languages. 


 \subsection{Numerals}
\label{sec:GRM-numeral}


\subsubsection{Atomic and Complex Numerals}
\label{sec:NUM-bas-comp}


Following \citet[263]{Gree78b}, I assume that  the simplest lexicalisation of a 
number is called a numeral
atom, whereas a complex numeral is an expression in which  one can infer at
least one arithmetical function.  A numeral atom can stand alone or can
be combined
with another numeral, either atomic or complex, to form a complex numeral. Atoms
are treated as  those forms which are not decomposable morpho-syntactically at a
synchronic level. Table \ref{tab:numeralatoms} displays the twelve
atoms of the numeral system.

 \begin{table}[!h]
  \caption{Atomic numerals from 1 to 8, 10, 20, 100, and 1000
\label{tab:numeralatoms}}
   \centering
  \begin{tabular}{llll}
\lsptoprule
Chakali &     English &  Chakali &     English\\ \midrule
 dɪ́gɪ́máŋá & one &   àlʊ́pɛ̀   &seven \\       
álɪ̀ɛ̀ &two   &   ŋmɛ́ŋtɛ́l &eight  \\                 
átòrò &three &   fí &ten \\         
ànáásɛ̀ &four & màtʃéó  &twenty   \\                 
 àɲɔ̃́ &five  &  kɔ̀wá (pl.  kɔ̀sá)   & hundred(s)  \\        
  álòrò   &six &   tʊ́sʊ̀  (pl.  tʊ́sà) &thousand(s)    \\                
 
\lspbottomrule
\end{tabular}
\end{table}

The term for `one'  is expressed  as  {\it dɪ́gɪ́máŋá},  but  {\it dɪ́gɪ́ɪ́}
alone  can also
be used. In general, the meaning associated with the morpheme {\it máŋá} is
`only', e.g.  {\it bahɪɛ̃ maŋa n̩ na} {\it old man-only-I-saw} `I saw only an 
old
man'. 
 The number 8 is designated with  {\it ŋmɛ́ŋtɛ́l}, an expression which is also
used to refer to the generic term for  `spider'.  Whether they are homonyms,
or whether
 their
meanings enter into a polysemous/heterosemous relationship is unclear. Another
characteristic is that the higher
numerals 100 and 1000  have their own plural form. To say a few words about 
some of the possible origins of higher numerals, the genesis of most of SWG 
higher numerals involves diffusion from non-Grusi sources, rather than from  
common SWG descents. I believe that higher numerals in the linguistic area 
where Chakali is spoken have two origins: one is Oti-Volta and the other is 
Gonja. The  forms for 100 and 1000  in Vagla and Dɛg  are similar
to Gonja's forms with the same
denotation, i.e. Gonja {\it  kɪ̀làfá} `100' and  {\it  kíɡ͡bɪ́ŋ} `1000'.  
Similar
form-denotation can be found in other North Guang languages (e.g.
Krache, Kplang, Nawuri, Dwang, and Chumburung) and {\it lafa} is found in many
other Kwa languages, as well as  non-Kwa languages, e.g. Kabiye (Eastern
Grusi)  \citep{Chan09}. Borrowing is  supported by the claim that the Vaglas
and Dɛgas were where they are today before the arrival of the Gonjas
(\citealt[12-13]{Good54}; \citealt[516]{Ratt32a}), and the fact that they, but
mostly the Vaglas, are still in contact with the former conquerer, the Gonjas. 
Of all Western  Oti-Volta languages, the Tampulmas have had more contact with
Mampruli  than any other, whereas the Chakali and the Pasaale
with Waali, a language close to  Dagbani and Dagaare, all of them classified as
Western  Oti-Volta languages. Variations of Manessy's {\it oti-volta commun}
reconstructed forms {\it *KO / *KOB}  `hundred'  and {\it 
*TUS}  `thousand'  are found
distributed all over Northern
Ghana, cutting across genetic relationship.  It seems that the two high
numerals are areal features spread by Western  Oti-Volta languages,   and that
Chakali, Pasaale, and Tampulma speakers may  have borrowed them from languages 
with which they had the most contact, i.e.   Waali, Dagbani, Dagaare,  and 
Mampruli.

From the atoms,  the complex numerals are now examined. The arithmetical 
functions inferred are called operations. In Chakali three types of operation 
are found: addition, multiplication, and subtraction. An operation always has 
two arguments which are identified in \citet{Gree78b} as: 

\vspace{3ex}

\begin{tabular}{ll}
{ Augend:} & A value to which some other value is
added.\\
{ Addend:} & A value which is added to some other
value.\\
{ Multiplicand:} & A value to which some other
value multiplies.\\

{ Multiplier:} & A value which is multiplied to
some other value. \\

{ Subtrahend:}  & The number subtracted.\\
{ Minuend:}  & The number from which subtraction takes
place.\\
\end{tabular}
\vspace*{10pt}


The numeral {\it dɪ́ɡɪ́tʊ̀ʊ̀} expresses the number 9. It is the only
expression associated with subtraction.  The subtrahend is the expression {\it 
dɪɡɪɪ} `one'.   In {\it dɪ́ɡɪ́tʊ̀ʊ̀},  the last syllable   is analyzed
as the
operation. It may originate from the state predicate  {\it tùwó} which is
translated 
`not exist'  or `absent from' (Section \ref{sec:GRM-loc-cl}). Thus, assuming the
covert minuend 10, the numeral
expression receives the functional notation [1 {\sc absent from} 10], or 10
minus 1.  The number 9 may also be expressed as {\it sàndòsó}  (or
{\it sandʊsə} in Tuosa and Katua). This expression is
used by some individuals in Ducie, Tuosa, and Katua, all of them from the most
senior generation.  One language consultant
associates  {\it sàndòsó} with the language of women, but his claim is not
sustained by other language consultants. For the number 9, \citet[33]{Good54}
reports
{\it saanese}
from the village Katua and  \citet[117]{Ratt32b} puts {\it sandoso} as the form
 for 9 in Tampulma. 

 
A proper  treatment of  atomic versus  complex numerals   relies  on evidence as
to whether
a numeral is synchronically  decomposable. In  that spirit,  numbers from 
11 to 19 are expressed with  complex numerals:  one piece of evidence, which is
presented in Section \ref{sec:GRM-gender} and repeated in section
\ref{sec:NUM-npstruc}, comes from the gender agreement between the head of a
noun phrase and the
cardinal numeral functioning as modifier.  Table \ref{tab:numral11-19}
provides the  numerals from 11 to 19 with a common structure
[fi$_{10}$-d(ɪ)-X$_{1-9}$]. 




  \begin{table}[!h]
  \caption{Complex numerals from 11 to 19  \label{tab:numral11-19}}
   \centering

  \begin{tabular}{ll}
\lsptoprule
 Chakali & English   \\ \midrule
 fí dɪ̄ dɪ́ɡɪ́ɪ́ & eleven  \\
 fí dɪ̄ álìɛ̀ & twelve  \\
  fí dɪ̄ átòrò &  thirteen \\
fí dɪ̄ ànáásɛ̀ &  fourteen \\
 fí dɪ̄ àɲɔ̃́ &  fifteen \\
 fí dɪ̄ álòrò  & sixteen  \\
fí dɪ̄ àlʊ̀pɛ̀ &  seventeen \\
  fí dɪ̄ ŋmɛ́ŋtɛ́l  &  eighteen \\
  fí dɪ̄ dɪ́ɡɪ́tʊ̀ʊ̀ &  nineteen \\

\lspbottomrule
\end{tabular}
\end{table}

The criterion employed for  the distinction between augend and addend is that an
augend is serialized, that is, it is the expression which is constant in a
sub-progression. This expression is called the base. In the progression from 
eleven to nineteen shown  in  Table  \ref{tab:numral11-19},  the augend is {\it 
fi} and the addends are the expressions for one to nine. Given the
above definition of a base,  the expression 
{\it fi} is  the base in complex
numerals  from 11 to 19. The operator for
addition is {\it dɪ} and its vowel surfaces only when the
 following word starts with a consonant (i.e. {\it fídɪ̀ŋmɛ́ŋtɛ́l} `18', but
{\it 
fídànáásɛ̀} `14').




Table \ref{tab:numeral21-99} provides the sequences of  numeral atoms
forming the complex numerals referring to  numbers from 21 to 
99. 



  \begin{table}[!h]
  \caption{Complex numerals from 21 to 99  \label{tab:numeral21-99}}
  \centering

  \begin{tabular}{lll}
\lsptoprule 
Number  & Numeral & Meaning  \\ \midrule
    
21-29& atom {\it anɪ} atom &  20  + (1 through 9) \\ 
30  &  atom  {\it anɪ} atom  & 20  + 10\\   
31-39&  atom {\it anɪ} complex  & 20  + (11 through 19)     \\      
40 &  atom  atom & 20 $\times$    2 \\
41-49&   atom  atom  {\it anɪ} atom &  20 $\times$  2  + (1 through 9) \\     
50 &  atom  atom  {\it anɪ} atom & 20 $\times$ 2 + 10 \\ 
51-59 & atom  atom  {\it anɪ} complex &20 $\times$ 2  + (11 through 19)\\ 
60 & atom  atom & 20 $\times$ 3\\ 
61-69 & atom  atom {\it anɪ} atom  &20 $\times$ 3 + (1 through 9) \\
70 &  atom  atom  {\it anɪ} atom& 20 $\times$ 3 + 10\\ 
71-79 &atom  atom  {\it anɪ} complex  &20 $\times$ 3   + (11 through 19)\\ 
80 & atom  atom  & 20 $\times$ 4\\ 
81-89 & atom  atom {\it anɪ} atom&20 $\times$ 4 + (1 through 9)\\ 
90 &  atom  atom  {\it anɪ} atom&20 $\times$ 4 + 10 \\ 
91-99 & atom  atom  {\it anɪ} complex& 20 $\times$ 4   + (11 through 19)\\      
\lspbottomrule
\end{tabular}
\end{table}


% Mathematics The number by which another number is multiplied. In 8 X 32, the
%multiplier is 8.

Table \ref{tab:numeral21-99} shows us that either (i) an atom can follow another
atom without any intervening particle  or (ii) the particle {\it anɪ} can step 
in
between two atoms, or one atom and one complex numeral. Case (i) is understood
as a phrase which multiplies the numerical values of  two atoms. For
instance, 
{\it matʃeo atoro} [20
3] results in the product `sixty'.  All numeral phrases from 20 to 99 use {\it 
matʃeo} in their formation. 
In case (ii),  the particle {\it anɪ} is treated as an operator similar to the
semantics of  `and' in English numerals since it adds the value of each
argument, either atom or complex.  The same form is also found in noun phrases
expressing the union of two or more entities (see Section
\ref{sec:GRM-conjunc-nom}).
The vowels of {\it anɪ} are reduced when preceded and followed by
vowels.

The same criterion applies for the distinction between multiplier and
multiplicand: the latter  is identified on the basis of what Greenberg calls
`serialization'. So, a  base may be   a serialized multiplicand as well since it
is the constant term in the complex expressions involved in a sub-progression.
The expression  {\it matʃeo} `20' is therefore the base in complex numerals  
from 21
to 99. The composition of complex numerals is summarized in table
\ref{tab:threecompo}.


\begin{table}[h]
\caption{General structure of complex numerals  \label{tab:threecompo}}
  \centering

\begin{tabular}{lll}
\lsptoprule
  Argument & Meaning & Restriction\\
\midrule
 ($y$)   $x$   tuo  & subtraction  &$y={10}$\\
&& $x={1}$\\ 
   $x$ anɪ $y$ & addition  &$x>y$ \\
$x$ dɪ $y$  & addition &$x={10}$ \\
&& $y={1 \textrm{-}9}$\\ %left sister x is 10

$x y$ & multiplication &$x=20$  \\
&& $y={2,3,4}$\\ %right sister y smaller than left sister x
$x y$  & multiplication  &$x={100}$ \\
&& $y={2 \textrm{-}9}$\\ %left sister x is 10
$x y$  & multiplication  &$x={1000}$ \\
&& $y={2 \textrm{-}999, 1000}$\\ %left sister x is 10
\lspbottomrule
\end{tabular}
\end{table}


As earlier mentioned, in subtraction  the minuend $y$ is covert. The only case
of subtraction 
is the numeral {\it dɪ́ɡɪ́tʊ̀ʊ̀} `nine'.  Both
addition  and multiplication take two overt arguments  $x$ and
$y$. They
are presented in the first column  of Table \ref{tab:threecompo} with their
surface linear order. The operator for addition {\it dɪ} is used only  for the
sum of 10 and numbers between 1 and 9. The form {\it anɪ} is found in a variety
of structures, but it restricts the right sister $y$ to be lower than the left
sister $x$. In multiplication  the value of the argument $y$ 
depends on the
value of $x$. For the numerals designating  2000 and above, the argument $x$
must be
the atom {\it tʊsʊ} `thousand' and $y$  any atom or complex numeral between 2 
and
999. There are no terms to express  `million' in Chakali. One can hear 
individuals at the
market  using the English word `million' when referring to  currency. According
to my consultants,  the expression {\it  tʊsʊ tʊsʊ} [1000 $\cdot$ 1000]
`million' was
common, but became archaic even before the change of currency  in July 
2007. Examples of numerals are presented in (\ref{ex:diffstrings}).


   

%\begin{multicols}{2}


\ea\label{ex:diffstrings}
  \ea\label{ex:82}
\gll  matʃeo  anaasɛ anɪ  alɪɛ \\
      {twenty} {four} {and} {two}\\
\glt `82'

\ex\label{ex:121}
\gll  kɔwa  anɪ  matʃeo  anɪ   dɪɡɪmaŋa  \\
       {hundred}  {and}  {twenty}  {and}    {one}\\
\glt `121'

\ex\label{ex:232}
\gll  kɔsa  alɪɛ anɪ  matʃeo  anɪ fidalɪɛ\\
       {hundred}  {two}  {and}   {twenty}  {and}  {twelve}\\
\glt `232'

\ex\label{ex:395}
\gll kɔsa atoro anɪ matʃeo anaasɛ anɪ fidaɲɔ̃ \\
 {hundred} {three}  {and} {twenty}  {four} {and}  {fifteen}   \\
\glt `395'

\ex\label{ex:501}
\gll kɔsa  aɲɔ̃ anɪ  dɪgɪmaŋa\\
       {hundred} {five}  {and}  {one}  \\
\glt `501'

\ex\label{ex:1225}
\gll tʊsʊ  anɪ   kɔsa  alɪɛ   anɪ  matʃeo  anɪ  aɲɔ̃\\
       {thousand}  {and}  {hundred}  {two}  {and}  {twenty}  {and} {five}\\
\glt `1225'

\ex\label{ex:21231}
\gll tʊsʊ  matʃeo   anɪ  dɪgɪmaŋa    anɪ  kɔsa  alɪɛ  matʃeo  anɪ  fi  anɪ  
dɪgɪmaŋa    \\
       {thousand}   {twenty}   {and} {one}  {and}  {hundred}  {two}  {twenty} 
{and} {ten} {and} {one}\\
\glt `21231'

\ex\label{ex:692381}
\gll tʊsʊ kɔsa aloro anɪ matʃeo anaasɛ anɪ alɪɛ anɪ kɔwa atoro anɪ matʃeo anaasɛ
anɪ fi dɪ dɪgɪɪ \\
{thousand} {hundred} {six} {and} {twenty} {four} {and} {two} {and} {hundred}
{three} {and} {twenty} {four} {and} {ten} {and} {one}\\
\glt `682 391'

\z
\z
%\end{multicols}

In summary,  the 
numeral system of Chakali is decimal (base-10) and vigesimal (base-20) and the 
base-20  operates throughout the formation of 20 to 99. In
\citet{Comr08}, numeral systems similar to the one described here are called
\textit{hybrid vigesimal-decimal}. 


\subsubsection{Numerals as modifiers}
\label{sec:NUM-npstruc}

To a certain extent, Chakali offers a rigid word order within the noun
phrase (Section \ref{sec:GRM-foc-neg}). Table \ref{tab:npstruc}
offers an overview of the noun phrase structure, supported by the data in
(\ref{ex:npstrucall}). The numeral occurs following  the head and the
qualifier(s). It precedes the article, the demonstrative and the
quantifier.  The noun phrases in  (\ref{ex:npstrucall}) were
collected on a paradigm filling session.\footnote{The examples in
(\ref{ex:npstrucall}) were elicited in subject position of the sentence frame
{\it X ka waa ra} `X is/are going to Wa'.}
 


\begin{table}[!h]
 \caption{Linear order of elements in a noun phrase \label{tab:npstruc}}
  \centering
\small
  \begin{tabular}{lcccccccc}
    \lsptoprule

&\textsc{art/poss}&\textsc{head}&\textsc{qual$_\textrm{1}$}&\textsc{
qual$_\textrm{2}$} &\textsc {num} & 
\textsc{quant} &  \textsc{dem} &  \textsc{foc/neg} \\
\midrule
\ref{ex:all-w}& $\surd$ &$\surd$ &&&&$\surd$ &&\\

\ref{ex:all-ten-w}&$\surd$&$\surd$&&&$\surd$&$\surd$&&\\

\ref{ex:all-fat-ten-w}&$\surd$&$\surd$&$\surd$&&$\surd$&$\surd$&&\\

\ref{ex:all-fat-blind-two-w}
&$\surd$&$\surd$&$\surd$&$\surd$&$\surd$&$\surd$&&\\

\ref{ex:all-fat-ten-w-those}
&$\surd$&$\surd$&$\surd$&&$\surd$&$\surd$&$\surd$&\\

\ref{ex:all-fat-ten-w-n}
&$\surd$&$\surd$&$\surd$&&$\surd$&$\surd$&&$\surd$\\

\ref{ex:full-temp} &$\surd$&$\surd$&$\surd$&&$\surd$&$\surd$&$\surd$&$\surd$\\
\lspbottomrule
  \end{tabular}
\end{table}

%\begin{multicols}{2}
\ea
  \ea\label{ex:all-w}
\gll a nɪhããn-a muŋ\\
\textsc{art} {woman-\textsc{pl}} \textsc{quant}\\
\glt `All women'

 \ex\label{ex:all-ten-w}
\gll a nɪhããn-a fi muŋ\\
\textsc{art} {woman-\textsc{pl}} \textsc{num} \textsc{quant}\\
\glt `All ten women'


\ex\label{ex:all-fat-ten-w}
\gll a nɪhããn-a pɔlɛɛ fi muŋ\\
\textsc{art} {woman-\textsc{pl}} {\qual} \textsc{num} \textsc{quant}\\
\glt `All ten fat women'

\ex\label{ex:all-fat-blind-two-w}
\gll ʊ nɪhããn-a  pɔlɛɛ ɲʊlʊma  alɪɛ muŋ\\
\textsc{poss} {woman-\textsc{pl}} {\qual}  {\qual} \textsc{num} 
\textsc{quant}\\
\glt `Both his two fat blind wives'

\ex\label{ex:all-fat-ten-w-those}
\gll a nɪhããn-a pɔlɛɛ fi muŋ  hama\\
\textsc{art} {woman-\textsc{pl}} {\qual} \textsc{num} \textsc{quant}
\textsc{dem}\\
\glt `Those all ten fat women'


\ex\label{ex:all-fat-ten-w-n}
\gll a nɪhããn-a pɔlɛɛ fi muŋ  lɛɪ\\
\textsc{art} {woman-\textsc{pl}} {\qual} \textsc{num} \textsc{quant}
\textsc{neg}\\
\glt `Not all ten fat women'

 \ex\label{ex:full-temp}
\gll a nɪhããn-a pɔlɛɛ fi muŋ hama  lɛɪ\\
\textsc{art} {woman-\textsc{pl}} {\qual}  \textsc{num} \textsc{quant}
\textsc{dem}
\textsc{neg}\\
\glt `Not all those ten fat  women'

\z
\z

When they appear as noun modifiers  (or in predicative position),  a limited
number of numerals act as targets in gender agreement, i.e. only the forms 2-7. 
 This grammatical
phenomenon provides us with a  motivation to treat  the expressions for numbers
11-19 as complex numerals.
In Section \ref{sec:GRM-gender},  Chakali is analyzed as having two values for
the
feature gender (i.e. \textsc{g}{\it a} or \textsc{g}{\it b} in Section 
\ref{sec:GRM-personal-pronouns}). The assignment is based on the humanness 
property and
plurality of a referent.  Table \ref{tab:genders}(c) is repeated as Table 
\ref{tab:distagree} for convenience. 


\begin{table}[h]
\caption{Prefix forms on the numeral modifiers  from 2 
to 7\label{tab:distagree}}
\centering
 \begin{tabular}{lcc}
\lsptoprule
&\textsc{-hum}=\textsc{g}\textit{a}&\textsc{+hum}=\textsc{g}\textit{b}\\
\midrule
\textsc{sg}&a&a\\
\textsc{pl}&a&ba\\
\lspbottomrule
 \end{tabular} 


\end{table} 

The following examples display gender agreement between the numeral {\it 
a-naasɛ}
`four' and the nouns {\it bʊ̃́ʊ̃̀nà}  `goats' in (\ref{ex:NUM-domnumA.pl}), 
{\it
vííné} `cooking pots' in (\ref{ex:NUM-domnumH-.pl}), {\it tàátá} 
`languages' in (\ref{ex:NUM-domabst.pl}) and {\it bìsé} `children'  in
(\ref{ex:NUM-domnumH+.pl}). Again, the only numerals that agree in gender with
the noun they modify are {\it álìɛ̀} `two', {\it átòrò}  `three', {\it
ànáásɛ̀} `four', {\it àɲɔ̃́} `five', {\it álòrò}  `six',  and   {\it 
àlʊ̀pɛ̀}
 `seven' (see examples \ref{ex:NUM-ungramhum-} and \ref{ex:NUM-ungramhum+}).
 The data in (\ref{ex:NUM-domnumA.pl})-(\ref{ex:NUM-domnumH+.pl}) tells us
that, when they function as controllers of agreement, nouns denoting non-human
animate, concrete inanimate and abstract entities  trigger the prefix form 
[{\it a-}] on the modifying numeral whereas nouns denoting human entities 
trigger the
form [{\it ba-}]. 


  \ea\label{ex:NUM-domnum}{\rm Agreement Domain: Numeral + Noun}
\ea\label{ex:NUM-domnumA.pl}
\gll   ŋ̩  kpaga   bʊ̃ʊ̃-na {\bf a}-naasɛ  \\
       \textsc{1.sg}  {have}  {goat(\textsc{g}\textit{a})-\textsc{pl}} 
{\textsc{3pl.g}\textit{a}-four} \\
\glt `I have four goats.' \\


\ex\label{ex:NUM-domnumH-.pl}
\gll   ŋ̩  kpaga vii-ne  {\bf a}-naasɛ  \\
        \textsc{1.sg}  {have}  {pot(\textsc{g}\textit{a})-\textsc{pl}}  
{\textsc{3pl.g}\textit{a}-four} \\
\glt `I have four cooking pots.' \\


\ex\label{ex:NUM-domabst.pl}
\gll  ŋ̩ ŋma  taa-ta {\bf a}-naasɛ  \\
        \textsc{1.sg}  {speak}  {language(\textsc{g}\textit{a})-\textsc{pl}}  
{\textsc{3pl.g}\textit{a}-four} \\
\glt `I speak four languages.' \\


\ex\label{ex:NUM-domnumH+.pl}
\gll   ŋ̩  kpaga bi-se  {\bf ba}-naasɛ \\
        \textsc{1.sg}  {have}  {child(\textsc{g}\textit{b})-\textsc{pl}}  
{\textsc{3pl.g}\textit{b}-four} \\
\glt `I have four children.' \\

\ex\label{ex:NUM-ungramhum-}
\gll   ŋ̩  kpaga vii-ne   *aŋmɛŋtɛl/*adɪɡɪtʊʊ (ŋmɛŋtɛl/dɪɡɪtʊʊ)\\
        \textsc{1.sg}  {have}  {pot(\textsc{g}\textit{a})-\textsc{pl}}  
{} {eight/nine} \\
\glt `I have eight/nine cooking pots.' \\

\ex\label{ex:NUM-ungramhum+}
\gll   ŋ̩  kpaga bi-se   *baŋmɛŋtɛl/*badɪɡɪtʊʊ (ŋmɛŋtɛl/dɪɡɪtʊʊ)\\
     \textsc{1.sg}  {have}  {child(\textsc{g}\textit{b})-\textsc{pl}} 
{} {eight/nine} \\
\glt `I have eight/nine children.' \\

\ex\label{ex:NUM-domnumH+.sg}
\gll     ŋ̩  kpaga vii-ne fidanaasɛ \\
        \textsc{1.sg}  {have}  {pot(\textsc{g}\textit{a})-\textsc{pl}}  
{fourteen} \\
\glt `I have fourteen cooking pots.' \\


\ex\label{ex:NUM-domnumH+.sg.14}
\gll    ŋ̩  kpaga  bi-se *fidanaasɛ/fidɪ{\bf ba}naasɛ \\
       \textsc{1.sg}  {have}  {child(\textsc{g}\textit{b})-\textsc{pl}}  
{fourteen} \\
\glt `I have fourteen children.' \\
\z
\z

Recall that in Table \ref{tab:numral11-19} the numbers from 11 to 19 were all
presented with the form
{\it  fid(ɪ)X}    `Xteen'. Their treatment as complex numerals makes one crucial
prediction: since they   have a common structure
[fi$_{10}$-d(ɪ)-[X$_{1-9}$]$_{atom}$]$_{complex}$ and not [fid(ɪ)X]$_{atom}$,
 agreement   has
access to the atoms X$_{2-7}$ within {\it fid(ɪ)X}. This is
illustrated with the examples (\ref{ex:NUM-domnumH+.sg}) and
(\ref{ex:NUM-domnumH+.sg.14}) using the word {\it fidanaasɛ}
`fourteen'.
These two examples show that in cases where a controller is specified for
both \textsc{g}{\it b} and \textsc{pl}, it must trigger the form
[ba-] on X$_{2-7}$   within the expressions referring to the numbers 12-17.



\subsubsection{Enumeration}
\label{sec:NUM-enum}

Chakali has enumerative forms. These  are numerals
with a purely sequential order characteristic and are used when one wishes
to count without 
any referential source or  to count off items one by one.  The forms are {\it 
diekee} `one',
{\it ɲɛwãã} `two', {\it toroo} `three', {\it naasɛ} `four',  {\it ɲɔ̃ɔ̃} 
`five',
{\it loroo} `six', {\it lʊpɛɛ} `seven', {\it ŋmɛŋtɛl} `eight', etc. Basically, 
what
differentiate the numerals of Table \ref{tab:numeralatoms} from the list above
are (i) a specific enumerative use, (ii) the tendency to lenghten the last
vowel,\footnote{I also perceived lenghtening  in Waali, Dɛg and
Vagla for the
same
sequences.} (iii)  the numerals  expressing two, three, four, five,
six, and seven do not usually display the agreement prefix,  and
(iv) the forms for `one'
and `two' differ to a greater extent. The rest of the enumerative numerals
correspond almost entirely to those shown in table
\ref{tab:numeralatoms}.  In (\ref{ex:monkey}), an excerpt of a folk tale
displays the
enumerative use of numerals.

\begin{exe}
 \ex\label{ex:monkey}

\gll gbɪ̃ã piili diekee, ɲɛwã, toroo, naasɛ, ɲɔ̃, loro, lʊpɛ, anɪ haŋ
ŋ̩ ka saŋɛɛ nɪŋ, dɪgɪtʊʊ, fi\\
Monkey starts one two three four five six seven {\conn}  \textsc{dem}
\textsc{1.sg}
\textsc{egr} sit  \textsc{advm} nine ten\\

\glt `The monkey started to count: one, two, three, four, five, six, seven, the
one I'm sitting on, nine, ten.' (CB 013)
\end{exe}



\subsubsection{Distribution}
\label{sec:NUM-distri}

Reduplication has several functions in Chakali and example 
(\ref{ex:NUM-distri1}) shows that the meaning of
distribution is expressed by the reduplication of a numeral.

\begin{exe}
\ex\label{ex:NUM-distri1}
 \gll  nɪɪ-ta alɪɛ-lɪɛ  n̩  dɪ tɪɛba dɪgɪ-dɪgɪɪ\\
  {water-\textsc{pl}} {two-two}   \textsc{1.sg}   \textsc{hest}   {give.\sc
3.pl} {one-one}   \\
\glt  `Yesterday I gave two water bags to each individual.'\\
\end{exe}

%nɪ (human) is good instead of ba above

In (\ref{ex:NUM-distri1}) the phrase containing the thing distributed and
its quantity opens the utterance. The recipient of the giving event is covert
but is
understood here as being more than one individual. Both  forms express the
quantity of things distributed and the number of recipients per things
distributed. This is how the distributive reading is
encoded in the utterance. Compare (\ref{ex:NUM-distri2a-1010}) and
(\ref{ex:NUM-distri2b-1010}) with
(\ref{ex:NUM-distri2c-10}).

 
\begin{exe}
\ex\label{ex:NUM-distri2-10}
\begin{xlist}
 
\ex\label{ex:NUM-distri2a-1010}{\it }
\gll a kuoru  zʊʊ dɪa  muŋ  no  a laa kpããm-a fi-fi \\
  \textsc{art}  {chief}  {enter}  {house.\textsc{sg}}   {all}  \textsc{foc} 
\textsc{conn}  {collect}  {yam-\textsc{pl}}  {ten-ten}      \\
\glt  `From each house the chief takes 10 yams.'

\ex\label{ex:NUM-distri2b-1010}{\it }
\gll a  zaga  muŋ tɪɛ  a  kuoru ro  kpããm-a  fi-fi\\
  \textsc{art} {compound} {all} {give}  \textsc{art}  {chief}  \textsc{foc}
yam-\textsc{pl}  {ten-ten}      \\
\glt  `Each house gives 10 yams to the chief.'

\ex\label{ex:NUM-distri2c-10}{\it }
\gll a  zaga  muŋ tɪɛ  a   kuoru ro kpããm-a  fi \\
  \textsc{art} {compound} {all} {give}  \textsc{art}  {chief}  \textsc{foc}
yam-\textsc{pl}  ten     \\
\glt  `All the houses (the village) give 10 yams to the chief.'
\end{xlist}
\end{exe}


In (\ref{ex:NUM-distri2b-1010}) and (\ref{ex:NUM-distri2c-10}), the sources of
the giving event are kept constant. The reading in which
ten yams per house are being collected by the chief is accessible only
if the numeral {\it fi}  `ten' is reduplicated (i.e.  {\it fifi}).

\ea
\ea\label{ex:NUM-door-two-two}

 \gll  tɪɛ  a gar  nʊã zenii  a nãɔ̃-na  ja  zʊʊ  alɪɛ-lɪɛ\\
  {give}   \textsc{art}  {fence}  {mouth}   {big}  \textsc{art} 
{cow-\textsc{pl}}   {do} {enter} {two-two}       \\
\glt  `Make the door large enough since the cows often enter two by two.'\\


\ex\label{ex:NUM-akee-apple-three-four}

 \gll  a tii banɪɛ̃ ato-toro  wo banɪɛ̃ jaa ana-naasɛ\\
 \textsc{art}  {akee.apple}  {some}   {three-three}  \textsc{foc}  {some}   
\textsc{ident} {four-four}   \\
\glt  `Akee apples have sometimes  three seeds, sometimes four seeds.'\\

\z
\z


%Distribution requires two core arguments: a recepient of the distribution and a
%thing distributed. A distributive event  differs from a giving event by the
%inherent reitatative property of the event, that is distribution involves at
%least two succesive giving action involving either the same thing or the same
%recipent. 
 

%From the data presented in this section, it is hard to establish whether
%Chakali has partial or complete reduplication. 


The reduplication of the numeral {\it álɪ̀ɛ̀} `two' in
(\ref{ex:NUM-door-two-two})
makes the
hearer understand that not only two cows might enter the cattle fence but a
possible sequence of  pairs. Similarly,   example 
(\ref{ex:NUM-akee-apple-three-four}) conveys a proposition which tells us that
the
fruit  {\it tíì}  `Akee apple' (\textit{Blighia sapida}) can reveal sometimes
three
and sometimes
four seeds.


%(As the reference is to fruit tokens, the distributive reading is
%accessible in great part due to the precense of the quantifier banɪa 'some'.)
%\vspace{1.1cm}


\subsubsection{Frequency}
\label{sec:NUM-repet}

When the morpheme {\it bɪ}  (Section
\ref{sec:GRM-preverb-iteration}) is prefixed to a cardinal numeral  stem, it
specifies the exact number of times an event happens. 

 
%(Mourelatos, 1981, p. 205)
%Mourelatos, Alexander (1981). "Events, processes, and states." In Syntax and
%Semantics: Tense and Aspect, edited by P. Tedeschi and A. Zaenen. New York:
%Academic Press.


\ea\label{ex:NUM-repet}{\rm A duty of the male's initiation for  funeral
caretaker}\\
 \gll ja wire ja kɪna ra aka vala go dusie muŋ naval bɪ-toro\\
 \textsc{1.pl} undress  \textsc{1.pl.poss} thing    \textsc{foc}  \textsc{conn} 
walk cross Ducie  \textsc{quant} circuit \textsc{itr-num} \\
\glt  `We undress then walk around Ducie three times.'
\z


The meaning of {\it bɪ}-{\sc num} corresponds to English `times' or  French 
`fois'.  Example (\ref{ex:NUM-repet}) illustrates a  case where  the morpheme 
{\it bɪ} is prefixed to the numeral stem {\it toro} `three' and translates into 
`three times'.



\subsubsection{Ordinals}
\label{sec:NUM-partitive}

Ordinal numerals are seen as those expressions conveying ranks or orders. The
investigation carried out  showed that the language
does not have a morphological marker or unique forms responsible for such a
phenomenon. Chakali expresses ranking and order by other
means.



\ea
\ea\label{ex:thirdmound}
\gll A: lie nɪ ɪ ka gɪla par    \\
    {} {where} \textsc{postp}  \textsc{2.sg.poss}  \textsc{egr} {leave} {hoe}
\\
\glt   `Where did you leave the hoe?'

\gll B: n̩ gɪla a par ra a   pie atoro  gantal  nɪ    \\
    {}  \textsc{1.sg}  {leave} \textsc{art}   {hoe} \textsc{foc} 
\textsc{art} {yam.mound} {three} \textsc{reln}  \textsc{postp} \\
\glt   `I left the hoe behind the third yam mound.'

\z
\z


In example (\ref{ex:thirdmound}),  the expression {\it a pie atoro  gantal nɪ} 
is
best translated as `behind the third yam mound' and not as `behind the three yam
mound'. In the context of B's utterance, there is no  salient set of three
mounds.  

%Notice that (how do I say that I am trying to get the translation of
%English numerals into Chakali)



The word  {\it sɪnsagal} is frequently  used in combination with a numeral to
express a non-specific amount. For example  {\it tʊ́sʊ̀ nɪ̄ sɪ́nsáɣál}  can be
translated into English as  `thousand and something'. In addition,  the word 
{\it sɪnsagal} can be combined with a numeral to identify sibling ranks. In 
(\ref{ex:sibling})  {\it sɪnsagal} is understood as `follower(s)'.  


\begin{exe}
\ex\label{ex:sibling}{\rm Sibling relationship}
\begin{xlist}

\ex\label{ex:sibling-a}
\gll ʊ sɪnsagal batoro jaa-ŋ \\
      \textsc{3.sg.poss} {follower} {three} \textsc{ident-1.sg}  \\
 \glt  `After him/her, I'm the third.'

\ex\label{ex:sibling-b}
\gll n̩ gantal tʊma jaa balɪɛ wa \\
    \textsc{1.sg.poss} {back} {owners} \textsc{ident}  {two}  \textsc{foc}   \\
\glt   `I have two siblings younger than me.' 


\ex\label{ex:sibling-c}
\gll n̩ sʊʊ tʊma jaa balɪɛ wa \\
   \textsc{1.sg.poss} {front} {owners} \textsc{ident}  {two}  \textsc{foc}   \\
 \glt  `I have two siblings older than me.'
\end{xlist}
\end{exe}



 Further, in a situation where a speaker wishes to express the fact that he/she 
won a race by getting to an apriori agreed  goal, a natural way of expressing 
this would be  {\it n̩  jaa dɪgɪmaŋa tɪɪna},  {\it lit:} I-is-1-owner,  `I am 
first'. The second and third (and so on) positions can also be expressed using 
the same construction (e.g. {\it n̩  jaa anaasɛ tɪɪna}, {\it lit:} I-is-4-owner, 
 `I am fourth'). However,  there are other ways to express the same proposition: 
any of the expressions given in (\ref{ex:race}) is appropriate in this context.


\ea\label{ex:race}{\rm Position on a race}
\ea\label{ex:}
\gll a batʃɔlɪ nɪ n̩ na alɪɛ  ra\\   
 \textsc{art}  {race} \textsc{postp}   \textsc{1.sg}  {see} {two}   \textsc{foc}
\\
 \glt  `At the race, I arrived second.'

\ex\label{ex:}
\gll mɪŋ dije\\   
    \textsc{1.sg.st} {eat.\textsc{pfv}}  \\
 \glt  `I arrived  first.' or `I won.'

\ex\label{ex:}
\gll mɪŋ nɪ te sʊʊ, ɪ saɣa\\   
    \textsc{1.sg.st} {postp} {early} {front}  \textsc{2.sg} {be.on}\\
 \glt  `I arrived  first, you followed.'
\z
\z



Finally, the word {\it búŋbúŋ} is translated into `first' and refers to a past 
state, its beginning or origin.

\ea\label{ex:seqevent}
 \gll   búŋbúŋ ní ǹ̩ fɪ́ wàà nʊ̃̄ã̄ sɪ̄ŋ̀\\
first {\postp}  {\sc 1.sg} {\sc pst} {\sc neg} drink alcoholic.drink\\  
\glt  `At first, I was not drinking alcoholic beverage.'
\z


% \subsection{Number verbs (TO DO)}
% \label{sec:NUM-verb}
% 
% {\it kpá pɛ̀} `to add'
% {\it lɛ̀sɪ́tà} [{\it lɛsta}] `subtract'
% {\it bóntí}  `to divide'
% X
% {\it ja } `equal'
% `to count'

 %A large quantity; a multitude,
%determine the number or amount of; count.
% total in number or amount; add up to.
%To constitute a group or number


\subsubsection{Miscellaneous usage of number concept} 
\label{sec:NUM-misc-usage}

In the performance of some rituals or customs, the number concepts 3
and 4 are associated with male and female respectively. Let us illustrate this
phenomenon 
with some examples. The Lobanɪɪ section of Ducie has a funeral song which is 
performed at
the death of a co-inhabitant. The song is repeated three times if the deceased
is a man and four in the case of a woman. When a person is initiated
to {\it Sɪ́gmāá}, a male must drink the black medicine in three successive
occurrences and a female in four.  On the fifth day of the last funeral 
({\it lúsɪ́nnà}), the children of the deceased are given food in a particular 
way
which involves offering the food and pulling  it back repeatedly: three times
for a male and four for a female. The same associations number-sex (i.e. {\it
three-male} and {\it
four-female})
are found in \citet[68-70]{Card27} where it is reported that, among the Kasena,
a woman must stay in her room three days after delivering a boy but four after
delivering a girl. Also,  the navel-string of a boy is twisted three times
 around her finger after being removed, but four times in the case of a
girl.

Two unusual phenomena involving numbers must be included. The first is
also found in neighboring languages (Dagaare, Waali, Buli, and probably 
others). 
The phrase {\it tʃɔpɪsɪ alɪɛ} is used in greetings (Section \ref{}).   It 
literally means `two
days', yet it implies that the speaker has not met the addressee for a long
period  (i.e. days, weeks or years). In other languages, I have been informed 
that one
can say `two months' or `two years', but in Chakali, even if someone has not
seen another person for years it is appropriate to say  {\it tʃɔpɪsɪ alɪɛ} `two
days'. The second concerns the reference to the number of puppies in a litter.
When a
speaker wishes to express the number of puppies a bitch has delivered, then
she/he
must add ten to the actual number. For example,  to express that a dog has given
birth to two puppies, one must say {\it ʊ lʊla fidalɪɛ},  {\it lit.}  `She
give.birth twelve'. 




\subsubsection{Currency}
\label{sec:NUM-currency}

One peculiarity of Chakali appears when numerals are used in the domain of
currency. For example,  in (\ref{ex:70000}) the speaker needs to sell a
grasscutter (\textit{Thryonomys swinderianus}) for the price of seven Ghana
cedis.


\begin{exe}
\ex\label{ex:70000}
\gll kɔsa atoro anɪ matʃeo alɪɛ anɪ fi\\
 hundred.\textsc{pl} three and twenty two and ten\\
\glt `Seven new Ghana Cedis, or seventy thousand old Ghana Cedis' ({\it lit:}
three
hundred and fifty)\\
\end{exe} 


Accounting for the reference to seven Ghana cedis with an expression literally
meaning three hundred and fifty (as was demonstrated in the previous
sections) is done in two steps.  First, Chakali speakers (still) refer
to the old Ghanaian currency (1967-2007), which after years of depreciation was
redenominated (July 2007). Today,  one new Ghana cedi ({\W ₵}) is worth 10,000
old Ghana cedis.\footnote{The term \textit{old} and \textit{new} were especially
used in the period of transition. The redenomination of July 2007 is the second
in the cedis history. The cedi was introduced by Kwame N'krumah in 1965,
replacing the British West African pound (2.4 cedis = 1 pound), but lasted only
two years. Thus,  the first redenomination actually occured in 1967.}  Secondly,
the Chakali word denoting `bag'  is  {\it bʊ̀ɔ̀tɪ́à} 
(\textit{pl.} {\it  bʊ̀ɔ̀tɪ̀sá},  \textit{etym.}  {\it bʊɔ-tɪa} `hole-give').  
There is evidence
that the word has at least one additional sense in the language. In
(\ref{ex:price-market}) the prices of some items are presented.\footnote{The
prices are those recorded at the market in Ducie in
February 2008.}

%\begin{multicols}{2}

\ea\label{ex:price-market}

\ea\label{ex:yamtubers}
\gll bʊɔtɪa matʃeo atoro anɪ fi dɪ  aɲɔ̃\\
bag twenty three and ten and five\\
\glt `15,000' (for  three yam tubers)




\ex\label{ex:groundnutbag}
\gll bʊɔtɪa tʊsʊ\\
bag thousand\\
\glt `200,000' (for a bag of groundnuts)


\ex\label{ex:driedcassava}
\gll bʊɔtɪa kɔsa alɪɛ\\
bag hundred two\\
\glt `40,000' (for a basin of dried cassava)


\ex\label{ex:cassavabag}
\gll bʊɔtɪa kɔsa ŋmɛŋtɛl\\
bag hundred eight\\
\glt `160,000' (for a bag of dried cassava) 


\ex\label{ex:ricebowl}
\gll bʊɔtɪa matʃeo anaasɛ anɪ fi\\
bag twenty four and ten\\
\glt `18,000' (for a bowl of rice) 

\ex\label{ex:Ricebag}
\gll bʊɔtɪa tʊsʊ anɪ kɔwa  aɲɔ̃\\
bag thousand and hundred five\\
\glt `300,000' (for a bag of rice) 


\z
\z
%\end{multicols}


In (\ref{ex:price-market}) the word {\it bʊɔtɪa} initiates  each expression.
Since the
expressions refer solely to the amount of money, it is clear that the word {\it
bʊɔtɪa} does not have the  meaning `bag' but that  the
meaning of a numeral, i.e. 200 can be inferred. The distinction between {\it
bʊɔtɪa}$_{1}$ (=bag)
and {\it bʊɔtɪa}$_{2}$  (=200) is supported by the following observations:  On 
some
occasions where  {\it bʊɔtɪa} is used,  the word cannot refer to `bag' since
there are no potential referents available. In the position it occupies in
(\ref{ex:price-market}) {\it bʊɔtɪa} is usually not pluralized, which is
obligatory for a modified noun. Further, the word {\it kómbòrò} `half' can 
modify
{\it 
bʊɔtɪa}$_{1}$  to mean `half a bag' (i.e. maize, groundnuts, etc), but  the
expression {\it bʊɔtɪa komboro} cannot mean `100 cedis' in the
language.\footnote{This claim was recently challenged by one of my consultants
who recalls his  mother using  {\it bʊɔtɪa komboro} to mean `100 cedis'.  
Compare
this with English  where one can say \textit{half a grand} to mean 500
dollars. The reason why {\it bʊɔtɪa komboro} was originally rejected was perhaps
that 100 old cedis was a very small sum  in 2008 and it was almost impossible
to hear the expression. In 2009,  another informant claimed never to have
heard such an expression to mean 100 old cedis.}  Going back to the form of the
expression given in
(\ref{ex:70000}),
it was also observed that in a conversation in which the reference to money is
understood, {\it bʊɔtɪa}$_{2}$  is often not pronounced. One can use the 
utterance {\it tʊ́sʊ̀}  `thousand' to refer to the price of a bag of 
groundnuts, that is an
amount of
two hundred thousand old cedis.\footnote{While a synchronic account of a sense
distinction for the form {\it bʊɔtɪa} in Chakali is introduced, a diachronic one
is complicated by the reliability of oral sources and a lack of written records.
The origin of a sense distinction of the form {\it bʊɔtɪa},  and its 
equivalent,   is found to be widespread in West Africa.  The lexical
item being discussed here is in Yoruba {\it ʔàkpó}, Baatonum {\it bʊɔrʊ}, 
Hausa
{\it kàtàkù},  Dagbani {\it kpaliŋa}, Dagaare {\it bʊɔra}, Dagaare (Nandom 
dialect)
{\it vʊɔra}, Sisaala {\it bɔ̀tɔ́} and Waali {\it bʊɔra}. Whether the word is
polysemous
in all these languages as it is in Chakali, I do not know. Akan and Gã had
something similar but seem to have lost the reference to currency: a study of
the words {\it  bɔ̀tɔ́} and {\it kotoko}/{\it kɔtɔkɔ}  is needed.} 
Provisionally, I can say that the distinguishing characteristic of {\it
bʊɔtɪa}$_{1}$ is that it is a common noun and refers to `bag' and that {\it
bʊɔtɪa}$_{2}$ is an atomic (and a base) numeral. The latter is  a kind of hybrid
 numeral, a blend of a measure term and a numeral term,   which is only used in
the domain of currency.


 
\subsection{Demonstratives}
\label{sec:GRM-demons}


Unlike the pronominal demonstrative which acts as a noun phrase, 
a demonstrative
within the noun phrase modifies the head noun. The demonstratives in the noun
phrase are identical to the demonstrative pronouns introduced in Section
\ref{sec:GRM-demons-pro}, i.e. ({\it sg.}/{\it pl.}) {\it haŋ}/{\it hama}.  



\begin{exe}
   \ex\label{ex:GRM-dem-sg}{\rm Priest talking to the shrine, holding a kola
nut above it}

\gll  má láá [kàpʊ́sɪ̀ɛ̀ háŋ̀]$_{NP}$ ká já mɔ̄sɛ̄ tɪ̀ɛ̀ wɪ́ɪ́ tɪ̀ŋ bà 
tàà búúrè\\
{\sc 2.pl} take kola.nut {\sc dem} {\sc conn} {\sc 1.pl} plead give matter {\sc
art} {\sc 3.pl.b} {\sc  egr} want\\
\glt   `Take this kola nut, we implore  you to give them what they desire.'

\z

Demonstrative  modifiers are mostly used in spatial deixis, but they do not
encode a proximal/distal distinction. Further, when a speaker uses {\it haŋ}  in
a non-spatial context, he/she tends to ignore the plural form (see example
(\ref{ex:GRM-dem-num}) below). In example (\ref{ex:GRM-dem-sg-non-spatial}), the
 demonstrative is placed before the quantifier,  which is not its canonical
position, as will be  shown in the summary examples in Section
\ref{sec:GRM-NP-sum}.\footnote{The plural form of {\it tɔʊ} `village' in Katua 
is
{\it tɔsɪ}. In the lect of Katua, the noun classes resemble the noun classes of
the Pasaale dialect, especially the lect of the villages  Kuluŋ and Yaala.} 



\begin{exe}
   \ex\label{ex:GRM-dem-sg-non-spatial}

\gll  dɪ́ ʊ̀ nʊ̃́ʊ̃́  dɪ́ [tʃàkàlɪ̀ tɔ́sá háŋ̀ mùŋ]$_{NP}$, dɪ́ bììsáà 
jáá nɪ́hɪ̃̀ɛ̃̂, bánɪ̃́ɛ̃́ 
ká bɪ̀ ŋmá dɪ́ sɔ̀ɣlá jáá nɪ́hɪ̃̀ɛ̃̂ \\
{\sc comp} {\sc 3.sg}  hear {\sc comp} Chakali villages {\sc dem} {\sc quant}
{\sc comp} Biisa {\sc ident} old some {\sc egr} {\sc itr}  say {\sc comp}
Sogla {\sc
ident} old \\
 \glt `He hears that of all  Chakali
settlements, some say that Biisa (Bisikan) is the oldest,  some
also say Sogola is the oldest.' ({\it Katua, 28/03/08, Jeo Jebuni})

\z


% %How does one makes the difference then? but   we notice that by adding the
% %article {\it a} one can capture the meaning of the  proximal/distal
% % distinction. 

The examples in (\ref{ex:GRM-dem}) show that the typical position of  the
demonstrative is after the head noun and before the postposition, after the
numeral,  but before the article {\it tɪŋ}. 


\ea\label{ex:GRM-dem} 
 
 
  \ea\label{ex:GRM-dem-n-postp} 
 \gll [tʃʊ̀ɔ̀sá háŋ̀]$_{NP}$ nɪ́ ǹ̩ǹ̩ dí kʊ̄ʊ̄ rā\\
 morning {\sc dem} {\sc postp} {\sc 1.sg} eat t.z. {\sc foc} \\
 \glt  `This morning I ate T. Z.'

   \ex\label{ex:GRM-dem-num} 
 \gll [nárá bálɪ̀ɛ̀ háŋ̀]$_{NP}$ nā sɛ́wɪ́jɛ́ à mʊ́r\\
person two {\sc dem}  {\sc foc} write {\sc art} story\\
\glt `THESE TWO MEN wrote the story.' 

   \ex\label{ex:GRM-dem-art}
 \gll làà [mʊ́sá záál háŋ̀ tɪ̀ŋ]$_{NP}$\\
 collect Musa fowl  {\sc dem} {\sc art}\\
 \glt `Collect  Musah's  fowl'  


 
\z 
 \z


% \subsection{}
% \label{sec:GRM-}
%gba = too


\subsection{Focus and negation}
\label{sec:GRM-foc-neg}

When the focus is on a noun phrase, the free-standing particle {\it ra} appears
to the right of the noun phrase (see Section \ref{sec:focus-forms} for the
various forms the focus particle can take). The particle {\it lɛɪ} `not'  
negates
a noun phrase, but it is part of the word in the case of a complex quantifier
(see Sections \ref{sec:GRM-quantifier}  and  \ref{sec:classifier}). Focus and
negation particles cannot co-occur together in a single noun phrase.  

\begin{exe}
 \ex\label{ex:GRM-foc-neg} 
 
 \gll [à dìèbísè hámà]$_{NP}$ lɛ̀ɪ́, [hámà]$_{NP}$ rā\\
  {\sc art} cats {\sc dem.pl} {\sc neg}  {\sc dem.pl} {\sc foc}\\
 \glt   `Not these cats, THESE CATS.'
\z

In  (\ref{ex:GRM-foc-neg}), {\it lɛɪ} `not' negates the noun phrase {\it a 
diebise
hama} and {\it ra} puts the focus on the demonstrative pronoun {\it hama},
referring to a different set of cats.  Both focus and negation particles can be
thought as having scope over the noun phrases, functioning as discourse 
particles. 

 Still in need of validation is the contrast offered in 
(\ref{ex:GRM-foc-lenght}),  where one consultant insisted that if the focus 
particle does not appear after the object of {\it kpaga} the subject needs to 
be 
lengthen and display high tone. This appears to co-relate to the distinction 
offered for personal pronoun in Section \ref{sec:GRM-personal-pronouns}.



\ea\label{ex:GRM-foc-lenght}
  \ea\label{ex:GRM-foc-w-lenght}
 \gll wáá/kàláá kpágá bʊ̀ɲɛ̃́\\
 {\sc 3.sg.st}/K.{\foc}  have respect\\
 \glt  `HE/KALA has respect for others'

   \ex\label{ex:GRM-foc-n-lenght} 
 \gll ʊ̀/kàlá kpágá bʊ̀ɲɛ̃́ rá\\
{\sc 3.sg}/K.  has respect {\foc}\\
\glt `He/Kala has RESPECT FOR OTHERS.' 

 \ex\label{ex:GRM-foc-w-lenght-2} 
\gll  wáá/bèléé kpágá záàl\\
 {\sc 3.sg.st}/wild.dog catch fowl\\
 \glt   `IT/WILD DOG catches fowls.'

\ex\label{ex:GRM-foc-w-lenght-2} 
\gll  ʊ̀/bèlè kpágá záál là\\
 {\sc 3.sg}/wild.dog catch fowl {\foc}\\
 \glt  `It/Wild dog catches FOWLS.'

\z
\z



\subsection{Coordination of nominals}
\label{sec:GRM-coord-nom}

\subsubsection{Conjunction of nominals}
\label{sec:GRM-conjunc-nom}



The coordination of nominals is accomplished by means of the conjunction
particle {\it anɪ} ({\it gl.} {\sc conn}).  The vowels of the connective are 
heavily
centralized and the initial vowel is often dropped in fast speech.
The particle can be weakened to [nə], or simply [n̩], when the preceding and
following phonological material is vocalic.  A coordination of two indefinite
noun phrases is displayed in (\ref{ex:GRM-coor-ani}). 

\begin{exe}
 \ex\label{ex:GRM-coor-ani} 
 
 \gll váá ànɪ́ dìèbíè káá válà \\
dog {\sc conn} cat {\sc  egr} walk\\
 \glt  `A dog and a cat are walking.'
\z

 The coordination of a sequence of
 more than two nouns is given in (\ref{ex:GRM-coor-sequen}). It is possible to
repeat the 
connective {\it anɪ}, but a pause between the items in a
sequence is more
frequently found. 

\begin{exe}
 \ex\label{ex:GRM-coor-sequen} 
 
 \gll  bʊ̃́ʊ̃́ŋ, váà ànɪ́ dìèbíè káá válà \\
 goat,  dog {\sc conn} cat {\sc  egr} walk\\
 \glt   `A goat, a dog and a cat are walking.'
\z

When a sequence of  two  modified nouns are conjoined, the head of the second
noun phrase may be omitted if it refers to the same kind of entity as
the first head noun. This is shown in (\ref{ex:GRM-coor-sequen-kind}).


\begin{exe}
 \ex\label{ex:GRM-coor-sequen-kind}
 
 \gll ǹ̩ kpáɣá tàɣtà zèn né ànɪ́ (tàɣtà) ábūmmò  \\
{\sc 1.sg} have shirt large  {\sc foc} {\sc conn} (shirt)  black \\
 \glt   `I got a large shirt and a black shirt.'
\z

If the conjoined noun phrase is definite, the article {\it tɪŋ}
follows both conjuncts. This is shown in (\ref{ex:GRM-qual-conj}) where the
connective appears between two qualifiers.

\begin{exe}
 \ex\label{ex:GRM-qual-conj}
 
 \gll   à kór ábúmmò ànɪ́ ápʊ̀mmá tɪ̀ŋ\\
{\sc art}  bench black {\sc conn} white {\sc art} \\
 \glt   `the black and white chair'
\z


% % 1.I got large and black shirts (one set with two properties). 'n kpaga takta
% % zeng ani abulonso lo. Note: abummo is singular and abulonso is plural. So 
since
% % it's black shirts, it should be takta bulonso.
% % 
% % 2.I got large shirts and black shirts (two sets of more than one shirt) 'n
% % kpaga(kpagi) takta zenie ani abulunso lo. Note that the singular of large is
% % 'zeng' and the plural is 'zenie' so in the above case, large shirts will be
% % 'takta(takdi) zenie. Those in the brackets are the Motigu tone.
% % 
% % 3.I got a large shirt and a black shirt (two sets of one shirt) 'n 
kpaga(kpagi)
% % takta(takdi) zeng ani abummo(abummaa) lo(la).


When the weak personal pronouns are conjoined there are limitations on the 
order 
in which they can appear. The disallowed sequences seem to be caused by two 
constraints. First, consultants usually disapproved   the sequences where a 
singular pronoun is placed after a plural one. Examples are provided in 
(\ref{ex:GRM-conj-const-1}).



\ea\label{ex:GRM-conj-const-1}
\begin{multicols}{2}

\ea\label{ex:GRM-conj-const-1-g}{\rm Acceptable}\\
1.sg {\sc conn} 2.pl $>$ /n̩ anɪ ma/ \\
`I and you ({\it pl})'\\
1.sg {\sc conn} 3.pl.{\sc g}a  $>$   /n̩ anɪ a/ \\
`I and they ({\it hum-})'\\
3.sg  {\sc conn} 2.pl $>$ /ʊ anɪ ma/ \\
 `she and you ({\it pl})'\\
3.sg {\sc conn} 3.pl.{\sc g}b $>$   /ʊ anɪ ba/\\
`she and they ({\it hum+})'

\columnbreak 
\vfill

\ex\label{ex:GRM-conj-const-1-ng}{\rm Unacceptable}\\
2.pl {\sc conn} 1.sg  $>$ */ma  anɪ n̩/\\
3.pl.{\sc g}a   {\sc conn} 1.sg  $>$  */a anɪ n̩/\\
2.pl  {\sc conn} 3.sg $>$ */ma anɪ ʊ/\\
3.pl.{\sc g}b  {\sc conn} 3.sg $>$ */ba anɪ ʊ/\\

\z 
\end{multicols}
 \z

Secondly, the first person pronoun {\it n̩} cannot be found after the 
conjunction, irrespective of the pronoun preceding it. The reason may be a 
constraint on the syllabification of two successive nasals.  In 
(\ref{ex:GRM-coor-sequen-nasal}), it is shown that the vowels of the 
conjunction 
{\it anɪ} either  drop or assimilate the quality of the following vowel. In 
addition, a segment  {\it n} is inserted between the conjunction and the 
following pronoun. 



\begin{exe}
 \ex\label{ex:GRM-coor-sequen-nasal}
/ʊ anɪ ʊ/  3.sg.  {\sc conn} 3.sg. $>$ [ʊ̀nʊ́nʊ̀]  `she and she' \\
/ʊ anɪ ɪ/  3.sg.  {\sc conn} 2.sg. $>$ [ʊ̀nɪ́nɪ̀] `she and you'\\
/n̩ anɪ ʊ/  1.sg.  {\sc conn} 3.sg. $>$ [ǹ̩nʊ́nʊ̀] `I and she'\\
/n̩ anɪ ɪ/  1.sg.  {\sc conn} 2.sg. $>$ [ǹ̩nɪ́nɪ̀]  `I and you' \\
/ɪ anɪ n̩/    2.sg.  {\sc conn} 1.sg. $>$ *[ɪn(V)nn̩]
\z



 If the first person pronoun {\it n̩} were to follow the conjunction, there 
would 
be  (i) no vowel quality to assimilate, and (ii) three successive homorganic 
nasals, i.e. one from the conjunction, one inserted and one from the first 
person pronoun, which would give  rise to a sequence {\it n(V)nn̩}. 

As shown in Table \ref{tab:GRM-conj-pron}, these problems do not arise when the 
strong pronouns ({\sc st}) are used. 



\begin{table}[htb!]
\small
 \caption[Conjunction of pronouns]{Conjunction of pronouns;  weak
pronoun ({\sc wk}) and   strong pronoun ({\sc st}) \label{tab:GRM-conj-pron}}

  \centering
  \begin{Itabular}{lllll}
\lsptoprule 
 & 3.sg. \& 3.sg. & 3.sg. \& 2.sg. & 3.sg. \& 1.sg. &
2.sg. \&1.sg.\\ \midrule

{\sc wk conn wk} &
ʊnʊnʊ & ʊnʊnɪ & \textasteriskcentered & \textasteriskcentered\\

{\sc wk conn wk} &
ʊnʊnʊ & ɪnʊnʊ &  n̩nʊnʊ &  n̩nɪnɪ\\

{\sc wk conn st} &
ʊnɪwa & ʊnɪhɪŋ &  {\it ʊnɪmɪŋ} &  {\it ɪnɪmɪŋ} \\

{\sc st conn wk} & 
wanʊnʊ & hɪnnʊnʊ & mɪnnʊnʊ & mɪnnɪnɪ\\

{\sc st conn st} &
wanɪwa & wanɪhɪŋ &  {\it wanɪmɪŋ}  & mɪnnɪhɪŋ\\
\lspbottomrule
    
  \end{Itabular}
 
\end{table}


\paragraph{Apposition}
\label{sec:GRM-np-apposition}


\begin{exe}
 \ex\label{ex:GRM-coor-appo} 
 
 \gll kùórù bìnɪ̀hã́ã̀ŋ ŋmá tɪ̀ɛ̀ [ʊ̀ ɲɪ́ná kùórù]$_{NP}$ dɪ́ à
báàl párá   à kùó pétùù  (...)\\
chief young.girl say  give {\sc 3.sg.poss} father   chief that 
{\art} man    farm {\art}    farm   finish.{\foc}  (...)  \\

 \glt  `The chief's daughter told her father that the young
man had finished weeding the farm (...)' (CB 014)
\z


There is another conjunction-type of nominal coordination. The noun phrase {\it 
ʊ ɲɪna kuoru} `her father chief'  in (\ref{ex:GRM-coor-appo}) is treated as two
noun phrases in apposition. In this case, apposition is represented as [[ʊ
ɲɪna]$_{NP}$ [kuoru]$_{NP}$]]$_{NP}$.
% in which  the definite noun phrase precedes the
%indefinite one.


\subsubsection{Disjunction of nomimals}
\label{sec:GRM-disjunct-nom}

In a disjunctive coordination, the language indicates a
contrast or a choice by means of a high tone and long {\it káá},    equivalent
to 
English `or'. 
The connective {\it káá}  is  placed between  two disjuncts. This is
shown in (\ref{ex:GRM-disjct}).

\ea\label{ex:GRM-disjct}
\ea\label{ex:GRM-disjct-1}

\gll ɪ̀ búúrè tí rē káá kɔ́fɪ̀\\
        {\sc 2.sg} want tea {\sc foc} {\sc conn} coffee\\
\glt  `Do you want tea or coffee?' 

\ex\label{ex:GRM-disjct-2}

\gll ɪ̀ búúrè tí rē káá kɔ́fɪ̀ rā ɪ̀ dɪ̀ búúrè\\
        {\sc 2.sg} want tea {\sc foc} {\sc conn} coffee {\sc foc} {\sc 2.sg}
{\sc ipfv} want \\
\glt  `Do you want tea or do you want coffee?' 

\z 
 \z



This  connective
 should not be confused with the three conjunctions used to connect verb
phrases and clauses, i.e. {\it aka}, {\it ka} and {\it a} (see Section
\ref{GRM-clause-coord}).   

%It can also be used at the end
%of a yes/no interrogative clause (see
%Section ).


\ea\label{ex:GRM-dij-vp4.5}

\gll ɪ̀ kàá tʊ̀mà tɪ̀ɛ̀ à kùórù ró zàáŋ káá tʃɪ́á\\
        {\sc 2.sg} {\sc fut} work give {\sc art} chief  {\sc foc} today or
tomorrow\\
\glt  `Will you work for the chief today or tomorrow?' 
\z



Example (\ref{ex:GRM-dij-vp4.5}) shows that the same
particle may also occur between
 \is{temporal nominals}temporal nominals. 





\subsection{Two types of agreement}
\label{sec:GRM-agrrement}

Agreement is a phenomenon which operates
across word boundaries: it is a relation between a controller and a
target in a given syntactic domain. In \citet{Corb04, Corb06} 
  agreement is defined as follow: (i) the element which determines the
  agreement is the controller, (ii) the element whose form is determined by
  agreement is the target and (iii) the syntactic environment in which
  agreement occurs is the domain. Agreement features refer to the information
which is shared in an agreement domain. Finally there may be conditions on
  agreement, that  is, there is a particular type of agreement provided certain
  other conditions apply. Chakali has two types of agreement based on animacy.
They are presented in the two subsequent sections. 

\subsubsection{The gender system}
\label{sec:GRM-gender}

% The glosses for the gender values in the examples are indicated
% with {\it {\sc g}a} for residuals and with {\it {\sc g}b} for human. 

Gender is identified as the grammatical encoding of an agreement class.  
Chakali has four domains
in which agreement in gender can be observed; antecedent-anaphor,
possessive-noun, numeral-noun and  quantifier-noun.
The values shared reflect the humanness property of the referent,
dichotomizing the lexicon of nominals into a set of lexemes $a$ (i.e.
human-) and a set $b$ (i.e.  human+), thus {\sc gender} $a$ or
$b$.  It is usually accepted that ``(g)ender is not restricted to
sex-based classifications (`male/female'); other semantic
possibilities include `animate', `small', `insect', `non-flesh food'
and so on'' \citep[293]{Corb00}. Therefore, treating humanness as gender
is not controversial.

In Chakali the values for the feature {\sc gender} are presented in
Table \ref{tab:genders}. A description that specifies the lexemes in those
terms will properly capture agreement in the language.


\begin{table}[htb!]
  \centering
  \caption{Gender in Chakali}
\label{tab:genders}


\subfloat[][Criteria for gender]{
 \begin{tabular}{llll}
\lsptoprule 
{\sc gender} & && Criteria \\ \midrule
\textit{a} &&& \textit{residuals}\\
\textit{b} &&& things that are categorized as human\\    
\lspbottomrule  \end{tabular}
}

\subfloat[][Gender in weak  and strong third-person
pronouns]{
  \begin{tabular}{lll}
\lsptoprule
Pronoun & {\sc wk}  & {\sc st} \\
Grammatical function  &    {\sc s|o}  &  {\sc s} \\ \midrule

{\it 3.sg.}  & {\it ʊ} & {\it wa}\\
{\it  3.pl.a} & {\it a} & {\it awa} \\
{\it  3.pl.b} &  {\it ba} &  {\it bawa} \\
    
\lspbottomrule
  \end{tabular}
}

\subfloat[][Agreement prefix forms]{
 \begin{tabular}{lcc}
\lsptoprule
&\textsc{-hum}=\textsc{g}\textit{a}&\textsc{+hum}=\textsc{g}\textit{b}\\
\midrule
{\it sg.}&{\it a-} &{\it a-}\\
{\it pl.}&{\it a-} &{\it ba-}\\
\lspbottomrule
 \end{tabular}
}


\end{table}



  In addition to
the gender values proposed in Table \ref{tab:genders}(a),  a condition
constrains
the controller to be plural to observe the humanness distinction in
agreement. As Table  \ref{tab:genders}(b)  and   \ref{tab:genders}(c) show, 
the personal pronouns in the language do not distinguish humanness in
 the singular but only in the plural.\footnote{\citet{Brin07c}  argues
that the situation  violates universal
  37 (and perhaps 45) of \citet{Gree63}: ``A language never has more
  gender categories in non-singular numbers than in the singular''.
  Although very rare, four languages, i.e. Fur (Sudan:  Nilo-Saharan, Fur), 
Kiowa (Oklahoma, USA: Kiowa Tanoan, Kiowa-Towa, Kiowa),  Sinhala (Sri
Lanka: Indo-European, Indo-Iranian,
  Indo-Aryan, Sinhalese-Maldivian) and Dagaare are known to display a pronominal
system resembling that of Chakali. The
information was extracted from Dik Bakker's typological database
(http://www.lotschool.nl/Research/ltrc/agreement.htm).  Vagla (vag),
Deg (mzw), Tampulma (tpm), Safaliba (saf), Hanga (hag) and Waali (wlx), to my
knowledge, can also be considered languages which violate Greenberg's
universal 37.} 

The boundary separating human from non-human is subject to conceptual
flexibility. In story telling non-human characters are ``humanized'', sometimes
called personification, as (\ref{ex:antanaH+}) exemplifies: animals talk, are
capable of thoughts and feelings, and can plan to go to funerals. If one
compares the non-human referents in example (\ref{ex:antanaH+}) and
(\ref{ex:domquantH-}), the former reflects personification, while the latter
does
not.


\begin{exe}
  \ex\label{ex:antanaH+}{\rm Domain: antecedent-anaphor}\\
\gll   váá  mã̀ã̀  sʊ́wá.   ʊ̀   ŋmá   dɪ́   ʊ̀  tʃɛ̀ná  
ŋmálɪ́ŋŋmɪ̃̀ɔ̃́ʊ̃̀   dɪ́   
\textbf{bá}  káálɪ̀  ʊ̀ mã̀ã́     lúhò \\
       {dog.{\sc sg}} {mother.{\sc sg}} {die} {he} {said} {\sc comp} {his}
{friend} {bird's name} {\sc comp} {{\sc 3pl.g}{\it b}} {go} {his}
{mother} {funeral} \\
\glt `The Dog's mother died. Dog asks his friend Bird ({\it Ardea purpurea}) to 
accompany him to his
mother's funeral.'  (lit: (...) that \textbf{they} should go to his mother's
funeral.) 
\z





In (\ref{ex:domquant}) the quantifier {\it muŋ} `all'
agrees in gender with the nouns {\it nɪbaala} `men' and
{\it bɔlasa} `elephants'.  The form {\it amuŋ} is
used with non-human, irrespective of the number value, and for human if
the referent is unique. The form  {\it bamuŋ} can only  appear in such a phrase
if the referent is human and the number of the referent is greater than one. In
this example a contrast is being made between
human-reference and animal-reference to show that it is not animacy in general 
but
humanness
which presents an opposition in the language.


\ea\label{ex:domquant}{\rm Domain: Quantifier + Noun}\\


\ea\label{ex:domquantH+}

\gll   nɪ̀-báál-á     \textbf{bā}-mùŋ \\
       {person({\sc g}{\it b})-male-{\sc pl}} {\sc g}{\it b}-{\sc all} \\
\glt `all men' \\

\ex\label{ex:domquantH-}

\gll   bɔ̀là-sá  \textbf{á}-mùŋ\\
        {elephant({\sc g}{\it a})-{\sc pl}}  {\sc g}{\it a}-{\sc all}\\
\glt `all elephants' \\


\z 
 \z

In Section \ref{secːGRM-poss-pro}, it was shown   that the possessive pronouns
have the same forms as
the corresponding weak
pronouns.  In
(\ref{ex:domposs}),  the target pronouns agree with the covert
controller, which is the possessor of the possessive kinship relation.
The nouns referring to goat and human mothers, trigger
{\sc g(ender)}{\it a} and {\sc g(ender)}{\it b} respectively. In cases where 
the possessor is covert the proper
assignment of humanness is dependent on the humanness of the possessed
argument (i.e. `their child' is ambiguous in Chakali unless one can 
retrieve the relevant semantic  information of the possessed entity).

\ea\label{ex:domposs}{\rm Domain: Possessive (possessor) + Noun}\\


\ea\label{ex:dompossH+}

\gll (mããma muŋ na)   \textbf{bà}   bì-sé\\
       mother({\sc g}{\it b}) all {\sc foc}  {\sc poss.3pl.g}{\it b}
{child-{\sc pl}} \\
\glt `their children' (possessor = human mothers) 

\ex\label{ex:dompossH-}

\gll (mããma muŋ na)   \textbf{à}   bʊ̃́ʊ̃́n-á   \\
      mother({\sc g}{\it a}) all {\sc foc}  {\sc poss.3pl.g}{\it a} {goat-{\sc
pl}} \\
\glt `their goats' (possessor = goat mothers) 





\z 
 \z
%\hfill{(Tampulma)}
 

%%%%%%%%%%HERE 05-10-09
Example (\ref{ex:domnum}) displays agreement between the numeral
{\it à-náásɛ̀} `four' and the nouns  {\it bʊ̃́ʊ̃̀nà}
({\sc cl.3}) `goats',  {\it tàátá} ({\sc cl.7})
`languages',  {\it vííné} ({\sc cl.5}) `cooking
pots' and  {\it bìsé} ({\sc cl.1}) `children'. The  numerals that agree
in gender with the noun they
modify are  {\it á-lɪ̀ɛ̀} `two',
 {\it à-tòrò} `three',  {\it à-náásɛ̀}
`four',  {\it à-ɲɔ̃́} `five',
 {\it à-lòrò} `six' and  {\it á-lʊ̀pɛ̀}
`seven'. Here again, animate (other than human), concrete (inanimate) and
abstract entities on the one hand, and human on the other hand do not
trigger the same agreement pattern ({\sc anim} in (\ref{ex:domnumHA}),
 {\sc abst} in  (\ref{ex:domabst}),   {\sc conc} in (\ref{ex:domnumI})  vs. 
{\sc hum} in  (\ref{ex:domnumH+})). Clearly, as shown below, noun
class membership is not reflected in agreement ({\it tàátá} ({\sc cl.7})
`languages' triggers {\sc g}a in (\ref{ex:domabst}) and 
 {\it bìsé} ({\sc cl.1}) `children' triggers {\sc g}b in (\ref{ex:domnumH+})).


\ea\label{ex:domnum}{\rm Domain: Numeral + Noun}\\


\ea\label{ex:domnumHA}

\gll  ǹ̩ǹ̩  kpáɣá  bʊ̃́ʊ̃́n-á  \textbf{à}-náásɛ̀ rā  \\
       {\sc 1sg}  {have}  {goat({\sc g}{\it a})-{\sc pl}}  {{\sc 3pl.g}{\it
a}-four} {\sc foc} \\
\glt `I have four goats.' \\

\ex\label{ex:domabst}

\gll   ǹ̩ǹ̩  ŋmá  tàà-tá \textbf{à}-náásɛ̀ rā  \\
        {\sc 1sg}  {speak}  {language({\sc g}{\it a})-{\sc pl}}   {{\sc
3pl.g}{\it a}-four} {\sc foc} \\
\glt `I speak four languages.' \\



\ex\label{ex:domnumI}

\gll  ǹ̩ǹ̩  kpáɣá  víí-né   \textbf{à}-náásɛ̀ rā \\
        {\sc 1sg}  {have}  {cooking.pot({\sc g}{\it a})-{\sc pl}}   {{\sc
3pl.g}{\it a}-four}  {\sc foc} \\
\glt `I have four cooking pots.' \\



\ex\label{ex:domnumH+}

\gll  ǹ̩ǹ̩   kpáɣá  bì-sé  \textbf{bà}-náásɛ̀  rā  \\
        {\sc 1sg}  {have}  {child({\sc g}{\it b})-{\sc pl}}   {{\sc 3pl.g}{\it
b}-four}  {\sc foc}  \\
\glt `I have four children.' \\



\z 
 \z



Example (\ref{ex:all}) shows that in a coordination construction
involving the conjunction form {\it (a)nɪ},  the targets display
consistently {\sc g}{\it b} when one of the conjuncts is
human-denoting.  In (\ref{ex:alla}) the noun
phrase {\it a} {\it baal} `the man' and the noun phrase
 {\it ʊ  kakumuso} `his donkeys' unite to form the noun
phrase acting as controller.  The noun phrase  {\it a
 baal nɪ ʊ kakumuso} `the man and his
donkeys' triggers {\sc g}{\it b} on targets.  Consequently, the
form of the subject pronoun, the quantifier, the possessive pronoun
and the numeral must expose  {\it ba} ({\sc 3.pl.}{\it b}).
The rule in (\ref{ex:rule}) constrains coordinate noun phrases to
trigger {\sc g}{\it b} if any of the conjuncts is specified as
{\sc g}{\it b}. No test has been applied to verify whether the
alignment of the conjunct noun phrases affects gender
resolution.






\ea\label{ex:all}{\rm Domain: Coordinate structure with {\it nɪ}}\\


\ea\label{ex:alla}

\gll  [à  báál   nɪ̀  ʊ̀ʊ̀  kààkúmò-sō]$_{NP}$  váláà  káálɪ̀  
tàmàlè
rā
\\
      {\sc art} {man} {\sc conn}  {\sc 3.sg.poss} {donkey-{\sc pl}} {walk}  {go}
{Tamale} {\foc} \\
\glt `The man and his donkeys walked to TAMALE' \\

\ex\label{ex:Tamanaphor}

\gll  \textbf{bà}  kʊ̃́ʊ̃́wã́ʊ̃́ \\
       {{\sc 3pl}.{\sc g}{\it b}} tire.{\pfv}.{\foc}\\
\glt `They are tired' \\

\ex\label{ex:Tamquant}

\gll   \textbf{bà}-mùŋ  nã̀ã̀sá tʃɔ́gáʊ́  \\
       {3.{\sc pl}.{\sc g}{\it b}-all} {feet.{\sc pl}} spoil.{\pfv}.{\foc}\\
\glt `All  feet were hurting' \\

\ex\label{ex:Tamposs}

\gll     \textbf{bà}  nã̀ã̀sá  tʃɔ́gáʊ́\\
        {{\sc 3.pl.poss.g}{\it b}}  {feet.{\sc pl}}  spoil.{\pfv}.{\foc}\\
\glt `Their feet were hurting them' \\

\ex\label{ex:Tamnum}

\gll   \textbf{bà}  jáá    \textbf{bà}-ɲɔ̃́  rā\\
        {{\sc 3.pl.g}{\it b}} {\sc ident}
{3.{\sc pl}.{\sc g}{\it
b}-five} {\sc foc} \\
\glt `They were five altogether' \\


\ex\label{ex:rule}

{\sc resolution rule}: When unlike gender values are conjoined
(i.e. {\sc gender} {\it  a} and {\sc gender} {\it b}), the
coordinate noun phrase determines {\sc gender} {\it b} (i.e.
{\sc g}{\it a} + {\sc g}{\it a} = {\sc g}{\it a},
{\sc g}{\it a} + {\sc g}{\it b} = {\sc g}{\it b},
{\sc g}{\it b} + {\sc g}{\it a} = {\sc g}{\it b} and
{\sc g}{\it b} + {\sc g}{\it b} = {\sc g}{\it b}).




\z 
 \z

Examples (\ref{ex:antanaH+}) to (\ref{ex:all}) demonstrate how one can analyse
the humaness distinction as gender. The comparison between humans, animals,
concrete inanimate entities and abstract entities uncovers the sort of animacy
encoded in the language. Section \ref{sec:classifier}  presents a phenomenon 
which
shows
some similarity to gender agreement.



\subsubsection{The classifier system}
\label{sec:classifier}

%get rid of dummy substantive>> classifier
 While there is abundant  literature describing Niger-Congo nominal
classifications and
agreement systems, the grammatical phenomenon  described in
this section  has not received much attention.  Consider the examples in
(\ref{ex:agr1}):      


\ea\label{ex:agr1}



\ea\label{ex:agrA}
\gll  dʒɛ̀tɪ̀ kɪ̀m-bɔ́n  ná\\
  lion.{\sc sg}  {\sc anim}-dangerous.{\sc sg}  {\sc foc} \\
\glt  `A lion is DANGEROUS' (generic reading) 



\ex\label{ex:agrB}
\gll    dʒɛ̀tɪ̀sá kɪ̀m-bɔ́má  rá \\
  lion.{\sc pl}  {\sc  conc; anim}-dangerous.{\sc pl} {\sc foc} \\
\glt  `The lions are DANGEROUS' (individual reading) 



\ex\label{ex:agrD}
\gll   m̩̀ bɪ̀ɛ̀rəsá  nɪ̀-bɔ́má  rá \\
{\sc poss.1.sg} {brother.{\sc pl}}  {\sc hum}-dangerous.{\sc pl} {\sc foc} \\
\glt  `My brothers are DANGEROUS'


\ex\label{ex:agrE}
\gll    bà  jáá  nɪ̀-bɔ́má   rá \\
{{\sc  3pl.g}{\it b}} {\sc ident} {\sc hum}-{dangerous.{\sc pl}} {\sc foc}  \\
\glt  `They are DANGEROUS' (human participants) 



\ex\label{ex:agrF}
\gll   à   jáá   kɪ̀m-bɔ́má  rá\\
{{\sc  3pl.g}{\it a}} {\sc ident}  {\sc  conc; anim}-{dangerous.{\sc pl}}
{\sc foc} \\
\glt  `They are DANGEROUS' (non-human, non-abstract participants) 



\ex\label{ex:agrG}
\gll záɪ́ɪ́   wɪ̀-bɔ́n ná \\
 fly.{\sc nmlz} {\sc abst}-dangerous.{\sc sg}  {\sc foc} \\
\glt  `Flying is dangerous'  \\




\ex\label{ex:agrH}
\gll à tʃígísíí wɪ̀-bɔ́má rá\\
{\sc art}  turn.{\sc pv.nmlz}  {\sc abst}-dangerous.{\sc pl}  {\sc 
foc}
\\
\glt  `The turnings  are DANGEROUS' (repetitively turning clay bowls for
drying) \\


\z 
 \z

The sentences in (\ref{ex:agr1})  are made of two successive noun phrases. The
referent of the first
noun phrase is an entity or a process while the second noun phrase is
semantically headed by a state predicate denoting a property.  Although
speakers prefer the presence of   the  verb {\it jaa} between the two
noun phrases, its  absence is acceptable and does not change the meaning of the
sentence. In these  identificational constructions,  the comment identifies the
topic as having a certain property, i.e. being bad, dangerous or risky. The 
focus marker follows the second noun
phrase, hence  $[$NP1 NP2 ra$]$  means `NP1 is NP2' in which salience or novelty
of information comes from NP2. 


The form of  {\it /bɔm/}   `bad' is determined
by the number value of the first noun phrase. Irrespective of the animacy
encoded in the referent, a  singular noun phrase triggers
the form {\it [bɔŋ]} while a plural triggers {\it [bɔma]} (i.e. {\sc cl.3}).  
The
number
agreement is illustrated in (\ref{ex:agrA}) and
(\ref{ex:agrB}).\footnote{Notice that the nominalized verbal lexemes in
(\ref{ex:agrG}) and (\ref{ex:agrH}) each triggers a different form for 
{\it /bɔm/}. The  form  {\it tʃigisii}  `turning'  is analyzed as a nominalized
pluractional
verb (see Section \ref{sec:GRM-PluralVerb}).}  

Properties do not appear as  
freestanding words in
identificational constructions. To say `the lion is dangerous', the grammar
has to combine the predicate with a dummy substantive, i.e. {\it lit.}  `lion is
{\it
thing}-dangerous',  where {\it thing} stands for the slots where animacy is
encoded. This is represented in (\ref{ex:frame}).  


\ea\label{ex:frame}
 [[ {\it thing}$_{animacy}$-property]$_{NP}$ {\sc foc}]$_{NP}$
\z


In  (\ref{ex:agr1}) there are three dummy substantives:   {\it  nɪ-}, {\it  wɪ-}
and  {\it  kɪn}-.  Each of them has a fully fledged noun counterpart; it can be
pluralized, precede a demonstrative, etc. Those forms are 
{\it 
kɪn}/{\it  kɪna} ({\sc cl.3})  `thing',  {\it  nar}/{\it  nara} ({\sc cl.3})
`person' and {\it 
 wɪɪ}/{\it  wɪɛ} ({\sc cl.4}) `matter, palaver, problem, etc.'.  


\begin{table}[htb!]

  \caption{Classifiers and Nouns   \label{tab:nounclassifier}}
  \centering
  \begin{Itabular}[h]{lllll}
    \midrule 
 Classifier   & Animacy & Noun class  & Sing. & Plur.\\
\midrule  \midrule
   {nɪ-}/{na-} &  $[${\sc hum}$]$ & Class 3 & nár &  nárá\\ 
 {wɪ-} &  $[${\sc abst}$]$ & Class 4 &    wɪ́ɪ́ &   wɪ́ɛ́ \\
  {kɪn}-  &  $[${\sc conc; anim}$]$ & Class 3&    kɪ̀n &   kɪ̀nà\\
 \midrule 
  \end{Itabular}
\end{table}

 %(see {\it forme radicale} in
%\citet[506]{Mane64})

In Table \ref{tab:nounclassifier}, the three possible
distinctions are provided. Also, a dummy substantive is now labelled a {\it
classifier}.  That is because the construction
concordance between the form of
the  classifier and the semantic information encoded in the head of the
first noun phrase reflects one major analytical criterion for  
classifier systems \citep{Dixo86, Corb91, Grin00}. It is clear that the
phenomenon
under investigation shows certain similarities with  
classifier systems, and perhaps could be on a grammaticalization path towards
 being one. Since there are form and sense compatibilities between the
inflecting noun pairs 
and the forms of the expressions preceding the qualitative predicate,  a common 
radical form for each is identified: {\it kɪn}-
{\sc [conc; anim] } `thing, non-human, non-abstract',  {\it nɪ-} {\sc
 [hum] } `person, human being'  and  {\it wɪ-} {\sc  [abst] } `non-concrete,
non-person' are the three classifiers in Chakali. 


All the sentences in (\ref{ex:agr1}) are ungrammatical without a classifier. The
three classifiers  combine with {\it bɔŋ}/{\it bɔma}  to  make  proper
constituents for an identificational construction. They  provide
animacy information, and their forms are determined by the sense properties
relevant
to animacy  encoded in the head of the first noun phrase.  The complex unit made
out of  a property and a licensing classifier is schematically presented
in (\ref{ex:frame2}).  


\ea\label{ex:frame2}\textit{Identificational Construction with a classifier}\\

 [ [$^{topic}$ \ \ {\it head}$_{x}$  ]$_{NP}$ (jaa)[[$^{comment}${\sc
clf}$_{x}$-property]$_{NP}$ {\sc foc}]$_{NP}$]$_{S}$
\z



The structural setting  is the result of a combination of grammatical
constraints which specify that: (i) a property in predicative function cannot
stand on its own, (ii) in predicative function,  a property must be joined with
a classifier, (iii) the merging of the classifier and the property forms a
proper syntactic constituent for an identificational construction, and (iv) the
form of the classifier is dependent on the animacy encoded in the argument of a
qualitative predicate. 

Finally,  classifiers are also found in the formation of the words
meaning  `something' and `nothing'. Consider the examples in 
(\ref{ex:something}) and (\ref{ex:nothing}):


\begin{multicols}{2}
\ea


\ea\label{ex:somethingH}
\gll {ná-mùŋ-lɛ̀ɪ́}\\
 {\sc hum}-all-not\\
\glt `no one'\\
\ex\label{ex:somethingC}
\gll  {wɪ́-mùŋ-lɛ̀ɪ́}\\
 {\sc abst}-all-not\\
\glt `nothing'\\
\ex\label{ex:somethingA}
\gll  {kɪ́n-mùŋ-lɛ̀ɪ́}\\
 {\sc  conc; anim}-all-not\\
\glt `nothing'\\

\z
\z

% 
\ea
 

\ea\label{ex:somethingH}
\gll {nɪ̀-dɪ́gɪ́ɪ́}\\
 {\sc hum}-one\\
\glt `someone'\\
\ex\label{ex:somethingC}
\gll  {wɪ́-dɪ́gɪ́ɪ́}\\
 {\sc abst}-one\\
\glt `something'\\
\ex\label{ex:somethingA}
\gll {kɪ̀n-dɪ́gɪ́ɪ́}\\
 {\sc conc; anim}-one\\
\glt `something'\\

\z
\z

\end{multicols}

As with the role of classifiers in identificational constructions, here again
the classifiers narrows down the tracking of a  referent when one of those
quantifiers is used. The grammar of Chakali arranges the four animacies into
three categories, i.e.  {\sc abst}, {\sc conc; anim} and {\sc hum}.  A
distinction is also made in English between {\sc hum} (i.e. someone, no one) and
 {\sc anim; conc; abst} (i.e. something, nothing), however English does not have
a distinction which captures  specifically abstract entities.

\subsection{Summary}
\label{sec:GRM-NP-sum}

The term nominal in the present context was argued to represent two separate
notions. The first is  conceptual. Nominal stems denote classes of entities
whereas verbal stems denote events. The second notion is  formal. A nominal stem
was opposed to  a verbal stem in noun formation.  As a syntactic unit,  the
nominal  constitutes an obligatory support to the main predicate and was
presented above in  various forms:   as a pro-form, a single noun, or 
 noun phrases
consisting of a noun with a qualifier(s), an article(s), a demonstrative,  among
others.

To summarize, Table \ref{tab:npstruc} lists acceptable
noun
phrases. Certain orders are
favored, but a strict linear order, especially among the qualifiers, needs 
further investigation.   Notice that each
noun phrase in (\ref{ex:GRM-np-list}), except for the weak personal pronoun in
(\ref{ex:GRM-pro}),  may or may not be in focus and may or may not be definite
(i.e. accompanied by the article {\it tɪŋ}). Also,  the column
{\sc head} in Table \ref{tab:npstruc} is not only represented in the
examples by a noun;  example (\ref{ex:GRM-dhq}) is headed by a demonstrative
pronoun. Needless to say, this list of possible distributions of nominal
elements
within the noun phrase is not exhaustive. Again, caution should be taken since
the examples in (\ref{ex:GRM-np-list}), particularly those towards the end of
the list, are the result of elicitation. The order of appearance in 
table  \ref{tab:npstruc} may be   interpreted  as  an approximation of the 
frequency of each kind of noun phrase. (see manuscript p.278)

% -Dakubu
% You can do a better job of generalizing.  It looks as though the order of most
% elements is quite regular and only the qualifiers vary.  Even those could
% probably be subclassified according to how many places they can occur



\begin{table}[htp] \footnotesize
\caption{Noun phrase members and linear order \label{tab:npstruc}}
  \centering
%\begin{small}

\begin{Qtabular}{p{1cm}p{.6cm}p{.6cm}p{.6cm}p{.6cm}p{.7cm}p{.6cm}p{.6cm}
p{.6cm}p{.9cm}p{.6cm}}
    \lsptoprule


\textsc{art/poss} & \textsc{head}&\textsc{qual}&\textsc{qual}&\textsc{num}
 & \textsc{quant} &  \textsc{dem} & \textsc{quant} & \textsc{art} &
\textsc{foc/neg} &  ex. \\[1ex] \midrule
&$\surd$&&&&&&&&& \ref{ex:GRM-pro} \\
&$\surd$&&&&&&&&($\surd$)& \ref{ex:GRM-h} \\
$\surd$&$\surd$&&&&&&&&($\surd$)& \ref{ex:GRM-ah}\\
$\surd$&$\surd$&&&&&&&$\surd$&($\surd$)& \ref{ex:GRM-aha}\\
$\surd$&$\surd$&&&&&&&&($\surd$)& \ref{ex:GRM-ph} \\
$\surd$&$\surd$&&&&&&&$\surd$&($\surd$)& \ref{ex:GRM-pha} \\
&$\surd$&&&&&$\surd$&&&($\surd$)& \ref{ex:GRM-dhq} \\

&$\surd$&&&&&$\surd$&$\surd$&&($\surd$)& \ref{ex:GRM-hdq} \\

&$\surd$&&&&&$\surd$&&&($\surd$)& \ref{ex:GRM-hd} \\

&$\surd$&&&&$\surd$&&&&($\surd$)& \ref{ex:GRM-hq-all} \\
&$\surd$&&&&$\surd$&&&&($\surd$)& \ref{ex:GRM-hq-many} \\
&$\surd$&&&$\surd$&&&&&($\surd$)& \ref{ex:GRM-hn} \\
$\surd$&$\surd$&$\surd$&&$\surd$&&&&&($\surd$)& \ref{ex:GRM-ahqln} \\
$\surd$&$\surd$&$\surd$&&$\surd$&$\surd$&&&&($\surd$)& \ref{ex:GRM-ahqlnd} \\
$\surd$&$\surd$&$\surd$&$\surd$&$\surd$&&&&&($\surd$)& \ref{ex:GRM-ahqlqln} \\

$\surd$&$\surd$&$\surd$&&&$\surd$&&&&($\surd$)& \ref{ex:GRM-ahqlq} \\
$\surd$&$\surd$&$\surd$&$\surd$&&$\surd$&&&&($\surd$)& \ref{ex:GRM-ahqlqlq} \\
$\surd$&$\surd$&$\surd$&$\surd$&&&$\surd$&&&($\surd$)& \ref{ex:GRM-ahqlqlqd} \\
$\surd$&$\surd$&$\surd$&&$\surd$&$\surd$&&&&($\surd$)& \ref{ex:GRM-phqlnq} \\
\lspbottomrule
  \end{Qtabular}


%\end{small}
\end{table}

\ea\label{ex:GRM-np-list} 
 
 
  \ea\label{ex:GRM-pro} 
ɪ̀   {\rm `you'} \\ {\sc head}

  \ex\label{ex:GRM-h} 
hã́ã̀ŋ  {\rm  `woman'} \\ 
{\sc head} 

  \ex\label{ex:GRM-ah} 
à hã́ã̀ŋ  {\rm  `the woman'}  \\  
{\sc art1} {\sc head} 

  \ex\label{ex:GRM-aha}
  à hã́ã́ŋ tɪ̀ŋ  {\rm  `the woman'}   \\  
{\sc art1} {\sc head} {\sc art2}

  \ex\label{ex:GRM-ph} 
ʊ̀ʊ̀ hã́ã̀ŋ  {\rm  `his woman'}  \\ 
 {\sc poss}  {\sc head}

  \ex\label{ex:GRM-pha} 
ʊ̀ʊ̀  hã́ã́ŋ tɪ̀ŋ  {\rm  `his woman'} \\
 {\sc poss} {\sc head} {\sc art2}  

 \ex\label{ex:GRM-dhq}
hámā mùŋ {\rm   `all these'}  \\  
{\sc head} {\sc quant}  

 \ex\label{ex:GRM-hdq}
 nɪ̀hã́ã́ná hámā mùŋ  {\rm  `all these women'}  \\ 
  {\sc head} {\sc dem} {\sc quant}  

  \ex\label{ex:GRM-hd} 
hã́ã́ŋ háŋ̀  {\rm  `this woman'}  \\ 
  {\sc head} {\sc dem}  

  \ex\label{ex:GRM-hq-all} 
 nɪ̀hã́ã́ná  mùŋ    {\rm  `all women'}  \\  
{\sc head} {\sc quant}  

\ex\label{ex:GRM-hq-many} 
 nɪ̀hã́káná   {\rm  `many women'} \\   
{\sc head-quant}  

  \ex\label{ex:GRM-hn} 
nárá bálɪ̀ɛ̀   {\rm  `three person'} \\  
  {\sc head} {\sc num}  

  \ex\label{ex:GRM-ahqln} 
à nɪ̀hã́ã́ná pɔ́lɛ̄ɛ̀ bálɪ̀ɛ̀  {\rm  `the two fat women'}  \\ 
 {\sc art1} {\sc head} {\sc qual} {\sc num}  

  \ex\label{ex:GRM-ahqlnd}
  à nɪ̀hã́ã́ná bálɪ̀ɛ̀ hámà  {\rm  `these two women'} \\ 
 {\sc art1} {\sc head} {\sc num} {\sc dem}  

  \ex\label{ex:GRM-ahqlqln}
à nɪ̀hã́ã́ná  ɲʊ́lʊ́má pɔ́lɛ̄ɛ̀ bálɪ̀ɛ̀  {\rm  `the two fat blind women'} 
 \\
  {\sc art1} {\sc head}  {\sc qual} {\sc qual}  {\sc num}  

\ex\label{ex:GRM-ahqlq} 
à nɪ̀hã́ã́ná pɔ́lɛ̄ɛ̀ káná   {\rm  `many fat women'}  \\  
{\sc art1} {\sc head} {\sc qual} {\sc quant}  

 \ex\label{ex:GRM-ahqlqlq} 
à nɪ̀hã́ã́ná pɔ́lɛ̄ɛ̀  ɲʊ́lʊ́má  káná   {\rm  `many fat blind women'}   
\\ 
{\sc art1} {\sc head} {\sc qual} {\sc qual} {\sc quant}  

  \ex\label{ex:GRM-ahqlqlqd} 
à nɪ̀hã́ã́ná pɔ́lɛ̄ɛ̀  ɲʊ́lʊ́má  káná  hámà   {\rm  `these many fat 
blind 
women'}  \\ 
{\sc art1} {\sc head} {\sc qual} {\sc qual} {\sc quant}  {\sc dem}

 \ex\label{ex:GRM-phqlnq} 
m̩̀m̩̀ párá áwíríjé átòrò bánɪ́ɛ́  {\rm  `some of my three good 
hoes'} \\ 
{\sc 1.sg.poss} {\sc head} {\sc qual} {\sc num} {\sc quant}

%\ex\label{ex:GRM-}
 
\z 
 \z



\section{Verbals}
\label{sec:GRM-verbals}

% 
Any expression which can take
the place of  the predicate {\sc p} in  (\ref{ex:GRM-clause-frame}) is 
identified as \is{verbal} \textit{verbal}.


\begin{exe}
  \exp{ex:GRM-clause-frame}\label{ex:GRM-clause-frame-2}
 {\sc adj}  $\pm$ {\sc s|a}  $+$ {\sc p} $\pm$ {\sc o} $\pm$ {\sc adj} 
\end{exe}


The term  can also refer to a semantic notion at the lexeme 
level. The language is analyzed as exhibiting two types of verbal lexeme: the 
{\it stative} lexeme and the {\it active} lexeme were both shown in Section 
\ref{sec:GRM-der-agent} to take  part in 
nominalization processes. The verbal stem in (\ref{ex:verb-VP})  must be 
instantiated with a verbal lexeme. 

\ea\label{ex:verb-VP}
[[{\it preverb}]\textsubscript{EVG} [[{\it stem}]-[{\it 
suffix}]]\textsubscript{verb}]\textsubscript{VG}
\z


In addition, the term  can refer to the 
whole of the verbal constituent, including the verbal modifiers. The verbal 
group  \is{verbal group} (VG) illustrated in (\ref{ex:verb-VP})
consists of linguistic slots which encode   various aspects of an event  which
may be realized in an utterance. A free standing verb is the minimal requirement
to satisfy the role of a predicative expression. The verbal modifiers, which
are called preverbs (Section \ref{sec:GRM-precerv}),  are grammatical items 
which specify the event
according to various  semantic distinctions. They precede the  verb(s) and take
part in the expanded verbal group \is{expanded verbal group} (EVG). The 
expanded verbal group
identifies  a domain which excludes the main verb, so a  verbal group
without preverbs would  be equivalent to a verb or a series of verbs (see SVC in
Section \ref{sec:GRM-multi-verb-clause}).\footnote{The term and notion are 
inspired from
analyses of the verbal system of Gã \citep{Daku70}. A verbal group is unlike 
the
(traditional) verb phrase in that it does not include its internal argument,
i.e. direct object. I am aware of the obvious need to unify the descriptions of
the nominal constituent and the verbal constituent.} 


While a verbal stem provides the core meaning of the predication,  a suffix may 
supply information on  aspect, whether or not the verbal constituent is in focus 
and/or the index of participant(s) (i.e. {\sc o}-clitic, Section 
\ref{sec:GRM-morph-opro}).  Despite there being little focus on tone and 
intonation, attention on the tonal melody of the verbal constituent is necessary 
since this also affects the interpretation of the event. These characteristics 
are presented below in a brief overview of the verbal system. 

% 
% one melodies affecting not
% only elements of the verbal constituent but elements immediately preceding or
% following it, and (iii) affixes
%without its participant(s) and other peripheral expressions.



\subsection{Verbal lexeme}
\label{sec:GRM-verb-lexeme}


\subsubsection{Syllable structure and tonal melody}
\label{sec:GRM-verb-syll-und-tone}

There is a preponderance  of open syllables of type CV and CVV, and the  common 
syllable sequences found among the verbs are CV, CVV, CVCV, CVCCV, CVVCV, and 
CVCVCV.   In 
the dictionary,  monosyllabic verbs make up approximately 12\% of the verbs, 
bisyllabic 65\%,  and trisyllabic  22\%.  All segments 
are attested in onset position word initially, but only {\it m, t, s, n, r, l, 
g, 
ŋ} and {\it w} are found in onset position word-medially in bisyllabic verbs, 
and only {\it  m, t, s, n,  l} and {\it g} are found  in onset position 
word-medially in trisyllabic verbs.   All trisyllabic,  CVVCV,   
and CVCCV verbs 
have one of the front vowels (\{e, ɛ, i, ɪ\}) in the nucleus of their last 
syllable.  The data suggests that {\sc atr}-harmony is operative, but not   
{\sc ro}-harmony,  in these three environments, e.g. {\it fùòlì} `whistle'. 
There is no restriction on vowel quality for the monosyllabic or bisyllabic 
verbs and both harmonies are operative.

 Table \ref{tab:GRM-verb-tone-melody} 
presents  verbs which are classified based on their syllable structures 
and  tonal melodies.  Despite the various attested melodies, instances of low 
tone CV verbs,    CVV verbs other than low tone,  and rising or falling CVCV, 
CVCCV, and CVVCV verbs are marginal. 


%\clearpage
\begin{table}[htb]
\renewcommand{\arraystretch}{0.8}
\centering
\caption{Tonal melodies on verbs  \label{tab:GRM-verb-tone-melody}}
        
\begin{Itabular}{llll}
\lsptoprule
Syllable type &  Tonal melody  & Form & Gloss\\ [1ex] \midrule

CV     &  H    &  pó  &  divide  \\
    &  H    & pɔ́    &plant  \\
    &  L   &tɔ̀   &cover \\ [0.5ex]
\midrule

CVV   &  L    & pàà   &   take\\
& L    &  tʊ̀à &  argue  \\
    & H    &  kíí   &  forbid  \\
 & LH    &wòó    &  vacant (be)  \\

\midrule

CVCV  &  H   &    hẽ́sí  &  announce  \\
     &  H  &    kúló  &  tilt \\
&  L    &    bìlè  &     put  \\
     &  L    &    hàlà  &  fry\\
     &  HL    &    lúlò  &  leak \\
     &  HM    &    pílē &  cover\\[0.5ex]
     
     
\midrule

CVCCV   & H & bóntí  &  divide\\
    & H  & kámsɪ́&  blink \\
    &L &sùmmè  &  beg \\
    &L & zèŋsì  &  limp\\
        &HL & lɔ́gsɪ̀  & scoop  \\[0.5ex] 
        
\midrule

 CVVCV & H& píílí & start \\
    &H& tɪ́ásɪ́    & vomit   \\
    &L &kààlɪ̀  & go \\
    &L &bùòlì  &  sing\\
    &LM& jʊ̀ɔ̀sɪ̄ & possessed  (be)\\[0.5ex] 
    
\midrule

CVCVCV  & H  & zágálɪ́ &  shake\\    
      & H  & vílímí  &  spin\\
       & L & hàrɪ̀gɪ̀&  try\\
      & L  &dʊ̀gʊ̀nɪ̀  &  chase\\ [0.5ex]

\lspbottomrule
\end{Itabular}      
\end{table} 
 
 The high (H) and low (L) register tones are assumed and two minimal pairs are 
identified:  one is {\it télé} `lean on' and {\it tèlè} `reach', but this 
may turn out to be a kind of  `aspectual switch' (from stative to process, or 
vice-versa), similar to the category switch discussed in Section 
\ref{sec:GRM-der-cat-switch}, and the other is {\it pɔ̀} `protect' and  {\it 
pɔ́} `plant'.  The mid tone (M)  is not contrastive and only describes a verb's 
intonation in the sentence frames compared.
 

\subsubsection{Verbal state and verbal process lexemes}
\label{sec:GRM-verb-stative-active}

A general distinction
between stative and non-stative events  is made: {\it verbal state} (stative
event) and {\it verbal process} (active event) 
lexemes are assumed. A verbal state lexeme can be identificational,
existential, possessive,  qualitative, quantitative, cognitive or  locative, and
refers more or less to a state or condition which is static, as opposed to
dynamic. The `copula' verbs {\it jaa} and {\it dʊa} (and its allolexe {\it 
tuwo})
are treated as subtypes of verbal stative lexemes since they are the only verbal
lexemes which cannot function as a main verb in  a perfective intransitive
construction (see Section \ref{sec:GRM-verb-perf-intran}). Their meaning and
distribution was introduced in the sections concerned with the identificational
construction (Section \ref{sec:GRM-ident-cl}) and existential construction
(Section \ref{sec:GRM-loc-cl}).  The possessive verb
{\it kpaga} `have'  is treated as  a verbal state lexeme as well (see possessive
clause in Section   \ref{sec:GRM-poss-cl}).  A qualitative verbal state lexeme
establishes a relation between an entity and a quality. Examples are given in
(\ref{ex:GRM-v-stat-qual}).

%locative clause \ref{sec:GRM-loc-cl}

\ea\label{ex:GRM-v-stat-qual}{\rm Qualitative verbal state lexeme}\\
%{\it Descriptive:}\\
 {\it bòró}  `short'  $>$ {\it à dáá bóróó} `The tree is short'\\
{\it gòrò}    `curved'  $>$ {\it à dáá góróó} `The wood is curved'\\
{\it jɔ́gɔ́sɪ́}    `soft'   $>$ {\it   à bìé bàtɔ́ŋ jɔ́gɔ́sɪ̀jɔ̀ʊ̄}  `The 
baby's
skin is soft'
\z

Similarly, a quantitative verbal state lexeme  establishes a relation between an
entity and a quantity. Yet, in (\ref{ex:GRM-v-stat-quant}), the subject of   
{\it maase} is the impersonal pronoun {\it a} which refers to a situation and 
not an
individual. The verb {\it hɪ̃ɛ̃}  `age' or `old'  is a quantitative verbal state
lexeme since it  measures  objective maturity between
two individuals, i.e. {\it mɪŋ hɪ̃ɛ̃-ɪ}, {\it lit.} {\sc 1.sg.st} age-{\sc
2.sg.wk}, `I am older than you'. 


\ea\label{ex:GRM-v-stat-quant}{\rm Quantitative verbal state lexeme}\\
%{\it Descriptive:}\\
 {\it kánà}  `abundant'  $>$ {\it bà kánã́ʊ̃́} `They are plenty (people)' 
\\
{\it mààsɪ̀} `enough'  $>$   {\it à máásɪ́jʊ́} `It is sufficient'\\
{\it hɪ̃̀ɛ̃̀} `age' $>$ {\it mɪ́ŋ hɪ̃́ɛ̃́ɪ̃̀} `I am older than you'
\z

Cognitive verbs such as {\it liise} `think',  {\it kʊ̃ʊ̃} `wonder, 
{\it kisi} `wish',    {\it tʃii} `hate', etc.  are also treated as verbal state
lexemes. 

Verbal process lexemes denote non-stative events. They are often partitioned
along the
(lexical) aspectual distinctions of  \citet{Vend57}, i.e. activities, 
achievements, accomplishments. Such verbal categories did not formally emerge, 
so I am not in a position to categorize the verbal process lexemes at this 
point 
in the research (but see \citealt[51]{Bonv88} for a thorough description of a 
Grusi verbal system), although Section \ref{sec:GRM-verb-suffix} suggests that 
there is a system of verbal derivation  that  uses verbal process lexemes which 
needs to be uncovered.  Thus, verbs which express that the participant(s) is 
actively doing something, undergoes a process, performs an action, etc. all 
fall 
within the  set of verbal process lexemes. 



\subsubsection{Complex verb}
\label{sec:GRM-complex-verb}

A complex verb is  composed of more than one verbal lexeme. For
instance, when {\it laa} `take' and {\it di}
`eat' are brought together in a SVC (Section \ref{sec:GRM-multi-verb-clause}),
they denote separate taking and eating event. A complex verb denotes a single 
event.

\ea\label{ex:cpx-verb-laa-di}
\ea
 \gll ǹ̩ láá kúòsò díūū \\
{\sc 1.sg} take G.  eat.{\foc}  \\
\glt `I believe in God.'

\ex
 \gll  ǹ̩ láá bìé dʊ́ʊ̄ \\
{\sc 1.sg} take child put.{\foc}  \\
\glt `I adopted a child.'
\z 
 \z
 
 The sequences  {\it laa}+{\it di} `believe'  and {\it laa}+{\it dʊ} `adopt'  
are  non-compositional, and less literal. Also, unlike complex stem nouns, but 
like SVCs, the elements which compose a complex verb must not necessarily be 
contiguous,  as  (\ref{ex:cpx-verb-laa-di}) shows. Other examples, among others, 
 are {\it zɪ̀mà síí}, {\it lit.} know raise, `understand',  {\it kpá tā}, 
{\it lit.}  take abandon, `drop' or `stop', and {\it gɪ̀là zɪ̀mà}, {\it lit.} 
allow know, `prove'.



\subsubsection{Verb forms and aspectual distinction}
\label{sec:GRM-verb-word}

The inflectional system of Chakali verbs displays  few verb
forms and is closer  to neighbor Oti-Volta languages than, for instance,  a
`conservative' Grusi language like Kasem \cite[51]{Bonv88}.\footnote{Dagbani is
described as a language where the ``inflectional system  for verbs is relatively
poor''  \cite[96]{Olaw99}. It has an imperfective suffix {\it -di}
\cite[97]{Olaw99} and  an imperative suffix {\it -ma}/{\it mi} 
\cite[101]{Olaw99}.
\citet[81]{Bodo97} writes that Dagaare has four verb forms: a dictionary
form, a perfective aspectual form, a perfective intransitive aspectual form and
an imperfective aspectual form. Also for Dagaare, \citet{Saan03}  talks about
four forms: perfective A and B, and Imperfective A  and B.}  Besides the
derivational suffixes (Section \ref{sec:GRM-deri-suff}), the verb in Chakali is
limited to two
inflectional suffixes and one assertive suffix:  (i) one signals negation in the
negative imperative clause (i.e.  {\it  kpʊ́} `Kill',  {\it tíí kpʊ̄ɪ̄} `Don't
kill'),  (ii) another attaches to some verb stems in the perfective intransitive
only, and (iii)  the other signals assertion and puts the verbal constituent in
focus. Since the negative imperative clause has already been presented in
Section
\ref{sec:GRM-imper-clause}, the perfective and imperfective intransitive
constructions are discussed next.  Both are recurrent clauses in data
elicitation. The former may contain both the perfective
suffix and the assertive suffix simultaneously, while the latter  displays the
 verb, with or without the assertive suffix.
 
 
 
\paragraph{Base form of a verb}
\label{sec:GRM-base-verb}

The form of the verb displayed in the dictionary is called the base form.  It  
is identified as the segmental sequence and melody which  would appear in a
positive imperfective transitive clause (Section \ref{sec:GRM-trans-intran}). 


\ea\label{ex:GRM-base-form}{\rm  positive imperfective transitive = base form}
\gll  
bàà kʊ́ɔ́rɪ̀ sɪ̀ɪ̀máá rà \\
3.\pl make food {\sc foc}\\
\glt `They are making food' \\
$\rightarrow$ {\rm kʊɔrɪ  (HL)}
 \z

 This sentence frame is one that does not affect the segmental sequence and 
melody the verb. The base form can also correspond  to a  verb elicited in 
isolation, although consultant are generally not at ease with verbs in 
isolation, unless they are framed in an utterance.

\paragraph{Perfective intransitive construction}
\label{sec:GRM-verb-perf-intran}

As its name suggests, a \is{perfective intransitive construction} perfective 
intransitive construction  lacks a 
grammatical
object and implies an event's completion or its 
reaching point.  In the case of \is{verbal state}verbal state,
the  \is{perfective}perfective  implies that the given state has been reached, 
or 
that the entity in subject position   satisfies the property encoded in
the \is{verbal state lexeme}verbal state lexeme. In 
(\ref{ex:GRM-intperfc-frame}),  two suffixes 
are
attached on  one verbal process stem and one  verbal state 
stem (see Section \ref{sec:nasalization-verb-suffix}
for the general phonotactics involved).\footnote{The presence of  a schwa
({\it ə}) in a CVCəCV surface form, as in (\ref{ex:GRM-intperfc-frame-state}), 
is explained in Sections \ref{sec:epenthesis} and \ref{sec:PHO-weak-syll}.}


\ea\label{ex:GRM-intperfc-frame}{\rm Perfective intransitive construction}\\


\ea\label{ex:GRM-intperfc-frame-process}{{\it  Verbal process:} {\sc s}  $+$
{\sc p} }\\
\gll àfɪ̀á díōō\\
A. {di-j[{\sc -lo, -hi, -ro}]-[{\sc +hi,+ro}]}\\

\glt `Afia ate.'

\ex  àfɪ̀á wá díjē `Afia didn't eat'

\ex\label{ex:GRM-intperfc-frame-state}{{\it  Verbal state:} {\sc s}  $+$ {\sc p}
}\\
\gll à dáá télèjōó\\
{\art} daa  {tele-j[{\sc -lo, -hi, -ro}]-[{\sc +hi,+ro}]}\\
\glt `The stick leans'

\ex à dáá wà télə̀jē `The stick doesn't lean.' %check

\z 
 \z

The first suffix to attach is the perfective suffix, i.e. -j[{\sc -lo, -hi, 
-ro}] or simply /jE/. Although it appears on every (positive and negative) stem 
in (\ref{ex:GRM-intperfc-frame}),  it does not surface on all verb stems. The 
information in Table \ref{tab:GRM-perf-suff} partly predicts whether or not a 
stem will surface with a suffix, and if it does, which form this suffix will 
have.


\begin{table}[htb]
 \centering
\caption{Perfective intransitive suffixes
\label{tab:GRM-perf-suff}}
\begin{Itabular}{p{2cm}p{2cm}p{2cm}}
\lsptoprule
Suffix /-jE/ & Suffix /-wA/ & No suffix  \\[1ex]
\midrule

CV &  CVV & CVCV\textsuperscript{1} \\
 CVCV\textsuperscript{2} & & \\ 

 \lspbottomrule
\end{Itabular}
\end{table} 

Table \ref{tab:GRM-perf-suff} shows that, in a perfective intransitive
construction, a CV stem must
be suffixed with {\it -jE} and  a CVV verb with {\it -wA}. The examples in
(\ref{ex:GRM-jE-wA}) are negative in order to prevent the assertive
suffix from appearing (see Section \ref{sec:GRM-focus} on why negation and the
assertive suffix cannot co-occur).


\ea\label{ex:GRM-jE-wA}


\ea{\it CV}\\
po   $>$  àfíá wá    pójē    `Afia didn't divide'  \\
pɔ     $>$ àfíá wá   pɔ́jɛ̄   `Afia didn't  plant'\\
pu     $>$ àfíá wá  pújē   `Afia didn't  cover'  \\
pʊ     $>$ àfíá wá  pʊ́jɛ̄   `Afia didn't  spit'  \\
kpe     $>$ àfíá wá  kpéjē   `Afia didn't  crack and
remove'\\
kpa     $>$ àfíá wá  kpájɛ̄   `Afia didn't  take'  

\ex{\it CVV}\\
tuu $>$ àfíá wá  tūūwō   `Afia didn't  go down'\\
tie $>$  àfíá wá   tīēwō `Afia didn't chew'\\
sii  $>$  àfíá wá  sīīwō   `Afia didn't  raise'\\
jʊʊ   $>$  àfíá wá  jʊ̄ʊ̄wā  `Afia didn't  marry'\\
tɪɛ $>$  àfíá wá tɪ̄ɛ̄wā  `Afia didn't  give'\\
wɪɪ $>$  àfíá wá  wɪ̄ɪ̄wā  `Afia is not  ill'
    

\z 
 \z

The surface form of the perfective suffix which attaches to CV stems  is 
predicted by the {\sc atr}-harmony rule of Section
\ref{sec:vowel-harmony}. Notice that  {\sc ro}-harmony does not operate
in that domain. 

\begin{Rule}\label{PHO-rule-perf-wa}{Prediction  for perfective intransitive 
-/wA/ suffix}\\
If the vowel of a CVV stem is
{\sc +atr},
the vowel of the suffix is {\sc +ro}, and if the vowel of a CVV stem is {\sc
-atr}, the vowel of the suffix is {\sc -ro}.\\
-/wA/ $>$  $\alpha${\sc ro}$_{suffix}$  /  $\alpha${\sc atr}$_{stem}$   
\end{Rule}

The CVV stems display  harmony between the stem
vowel(s) and the suffix vowel which is easily captured by a variable feature
alpha notation, as shown in Rule (\ref{PHO-rule-perf-wa}), which  assumes that 
the segment [{\it o}] is the
[{\sc +ro, +atr}]-counterpart of [{\it a}]. 

% % 
% % Notice also that I perceived the
% % same tonal melody in all clauses, raising doubts  on the tonal melodies of 
the
% % `citation forms' offered in Table \ref{tab:GRM-verb-tone-melody}.

Predicting  which of 
set CVCV\textsuperscript{1} or set CVCV\textsuperscript{2} in Table 
\ref{tab:GRM-perf-suff}  a
stem falls  has proven unsuccessful. Provisionally,  I suggest that a CVCV
stem must be stored with such an information. One piece of evidence
supporting this claim comes from
the minimal pair {\it tèlè} `reach' and  {\it télé} `lean against':  the
former displays CVCV\textsuperscript{2} (i.e. tele-jE),  whereas the latter 
displays CVCV\textsuperscript{1}
(i.e. tele-\O).  The data shows that a  CVCV stem with round vowels is less 
likely to
behave like a CVCV\textsuperscript{1} stem, yet {\it púmó} `hatch' is a 
counter-example, i.e.
{\it a zal wa puməje} `the fowl didn't hatch'. The CVCCV, CVVCV, and CVCVCV 
stems
have  not been investigated, but {\it kaalɪ} `go', a common  CVVCV verb, takes
the
/-jE/ suffix [to do].  


\paragraph{Imperfective intransitive construction}
\label{sec:GRM-verb-perf-intran}

The imperfective  conveys the unfolding of an event, and it is often used to
describe an event taking place at the moment of speech. In addition, the
behavior of the egressive marker {\it ka} (Section \ref{sec:GRM-EVC-egr-ingr})
suggest that the imperfective may be interpreted as a progressive event.  The 
imperfective is indicated by the base
form of a verb. As in the perfective intransitive, the assertive suffix may be
found attached to the verb stem. 


\ea\label{ex:GRM-assert-suff}
[[{\it verb stem}]-[{\sc +hi,+ro}]]$_{verb \ in \ focus}$
\z

Again, the constraints licensing the combination of the verb stem and the vowel
features  shown in (\ref{ex:GRM-assert-suff})   are (i) none of the other
constituents in the clause are in focus, (ii) the clause does not include
negative polarity items, and (iii) the clause is intransitive, that is, there is
no grammatical object. [to do: note p. 199, + non-pronominal subject)

% , as opposed to an
% event perceived as bounded (i.e. perfective) or a hypothetical event (i.e.
% imperative)



\ea\label{ex:GRM-pos-neg-take}
\ea\label{ex:GRM-ipfv-out-pos}{\rm Positive}\\
 ʊ̀ kàá kpá  {\rm `She will take'}\\
   ʊ̀ʊ̀ kpáʊ̄    {\rm `She  is taking/takes'}

\ex\label{ex:GRM-ipfv-out-neg}{\rm Negative}\\
 ʊ̀ wàá kpā  {\rm  `She will not take'} \\
   ʊ̀ʊ̀   wàà   kpá {\rm `She  is not taking/does not take'}

   
   \ex\label{ex:GRM-ipfv-out-nfoc}
    \textasteriskcentered kalaa kpaʊ {\rm Kala is taking/takes'}\\
      \ex\label{ex:GRM-ipfv-out-stpro}
 \textasteriskcentered waa kpaʊ {\rm `SHE is taking/takes'}\\
       \ex\label{ex:GRM-ipfv-out-obj}
  \textasteriskcentered ʊ kpaʊ a bɪɪ  {\rm `She  is taking/takes the 
stone'} \\
    \ex\label{ex:GRM-ipfv-out-neg}
      \textasteriskcentered   ʊʊ   waa   kpaʊ {\rm `She  is not taking/does 
not take'}

\z 
 \z

In (\ref{ex:GRM-pos-neg-take}), the forms of the verb in the
intransitive imperfective take the assertive suffix to signal that the verbal
constituent is in focus, as opposed to the nominal argument. [to do] The 
assertive suffix cannot appear
when the subject is in focus (\ref{ex:GRM-ipfv-out-nfoc}) or when the strong
pronoun is used as subject (\ref{ex:GRM-ipfv-out-stpro}), when a grammatical
object follows the verb  (\ref{ex:GRM-ipfv-out-obj}), or when the negation
preverb {\it waa} is present  (\ref{ex:GRM-ipfv-out-neg}).



\paragraph{Intransitive vs. transitive}
\label{sec:GRM-trans-intran}


Many verbs can occur in either  intransitive or transitive clauses. The subject 
of the intransitve ({\sc s}) in (\ref{ex:GRM-clause-core-intrans}) and 
(\ref{ex:vp26.14.}) correspond to the subject of the transitive ({\sc a}) in  
(\ref{ex:GRM-clause-core-trans}) and (\ref{ex:vp26.15.}), and the same verb is 
found with and without an object ({\sc o}).


\begin{multicols}{2}
\ea\label{ex:GRM-clause-core}


 \ea\label{ex:GRM-clause-core-intrans}
\glll kàlá díjōō \\
       {\sc s} {\sc p}\\
 Kala eat.{\sc pfv.foc} \\
\glt  `Kala ATE.' 
\ex\label{ex:GRM-clause-core-trans}
\glll kàlá dí sɪ̀ɪ̀máá rā\\
        {\sc s} {\sc p}  {\sc o} {} \\
Kala eat.{\sc pfv}  food {\sc foc}\\
\glt  `Kala ate FOOD' 



\ex\label{ex:vp26.14.}
\glll ʊ̀ʊ̀ búólùū \\
{\sc s} {\sc p}\\
       {\psg} sing.{\ipfv.\foc} \\
\glt  `He is SINGING.' 

\ex\label{ex:vp26.15.}
\glll  ʊ̀ʊ̀ búólù būōl lō \\
{\sc a} {\sc p} {\sc o} {}\\
       {\psg}  sing.{\ipfv} song {\foc}    \\
\glt  `He is singing a SONG.' 

\z
 \z
 \end{multicols}


It is possible to promote a prototypical theme argument to the subject position.
However,  informants have difficulty with some nominals in the subject
position of
intransitive clauses.   The topic needs further investigation, although it is
certainly related to a semantic anomaly.  The data in
(\ref{ex:GRM-intran-theme-subj}), where the  prototypical {\sc o}(bject) is in
 {\sc a}-position, illustrates the problem. In order to concentrate on the 
 activities of  `goat
beating'- and `tree climbing'  and turn the two clauses
(\ref{ex:GRM-int-th-su-out-1}) and (\ref{ex:GRM-int-th-su-out-2}) into
acceptable utterances,  the optimal solution is to use the
impersonal pronoun {\it ba} in subject position  (see impersonal
pronoun in
Section \ref{sec:GRM-impers-pro}).



\ea\label{ex:GRM-intran-theme-subj}

\ea
à bʊ̀ɔ̀ káá hírèū  {\rm `the hole is being dug'}
\ex\label{ex:GRM-int-th-su-out-1}
\textasteriskcentered a bʊ̃ʊ̃ŋ   kaa maŋãʊ̃  {\rm  `the goat is being beaten' }
> {\it 
bàà máŋà à bʊ̃́ʊ̃́ŋ ná}
\ex\label{ex:GRM-int-th-su-out-2}
\textasteriskcentered a daa kaa zɪnãʊ̃  {\rm   `the tree is being climbed'}   
> {\it 
bàà zɪ́ná à dáá rá}


\z 
 \z

%[to do: include image spectrogram]
Given that  the inflectional system of the verb is rather poor, and that the 
perfective
and assertive suffixes occur only in intransitive clauses,  how does one
encode a basic contrast like the one between a transitive perfective and
transitive imperfective? The paired examples in (\ref{ex:tra-pfv}) and
(\ref{ex:tra-ipfv})  illustrate 
 relevant contrasts. \nolinebreak 


\begin{multicols}{2}
\ea\label{ex:tra-pfv}{\rm Transitive perfective}\\

  \ea\label{ex:tra-pfv-eat}
ǹ̩ dí kʊ̄ʊ̄ rā\\
 `I ate T. Z..' 
 \ex\label{ex:tra-pfv-plant}
ǹ̩ pɔ́ dāā rā\\
`I planted a TREE.'
 \ex\label{ex:tra-pfv-cover}
ǹ̩ tʃígé vìì rē\\
`I covered a POT.' 
 \ex\label{ex:tra-pfv-tie}
ǹ̩ lómó bʊ̃́ʊ̃́ŋ ná\\
`I tied a GOAT.' 
 \ex\label{ex:tra-pfv-carry}
m̩̀ mɔ́ná díŋ né\\
`I carried  FIRE.' 

\z 
 \z

\ea\label{ex:tra-ipfv}{\rm Transitive imperfective}\\

 \ea\label{ex:tra-ipfv-eat}
ǹ̩ǹ̩ dí kʊ́ʊ́ rá\\
`I am eating T. Z..' 
 \ex\label{ex:tra-ipfv-plant}
m̩̀m̩̀ pɔ́ dáá rá\\
`I am planting a TREE.' 
 \ex\label{ex:tra-ipfv-cover}
ǹ̩ǹ̩ tʃígè vìì rē\\
`I am covering  a POT.' 

 \ex\label{ex:tra-ipfv-tie}
ǹ̩ǹ̩ lómò bʊ̃̄ʊ̃̄ŋ nā\\
`I am tying  a GOAT.' 
 \ex\label{ex:tra-ipfv-carry}
m̩̀m̩̀ mɔ́nà dīŋ nē\\
`I am carrying  FIRE.' 

\z 
 \z
 
\end{multicols}


Each pair in the verbal frames of  (\ref{ex:tra-pfv}) and (\ref{ex:tra-ipfv}) 
presents fairly regular patterns:  the high tone {\it versus} the falling tone 
on the CVCV verbs is one instance, the systematic change of the tonal melodies 
on the grammatical objects in the two CV-verb cases, and the length of the 
pronoun in the imperfective are identified. The data suggest that it is the 
tonal melody, and not exclusively the one associated with the verb, which 
supports aspectual function in this comparison. When the verb is followed by an 
argument, both perfective and the imperfective are expressed with the base form 
of the verb.  However,  the tonal melody alone  can determine whether a 
phonological string is to be understood as a bounded event which occurred in 
the 
past or an unbounded event unfolding at the moment of speech.



Tonal melody is crucial in the following examples as well. The examples in
(\ref{GRM-pfv-inter}) are three polar questions (see Section
\ref{sec:GRM-interr-polar}), one perfective  and two
imperfective. The two first have the
same segmental content, and the last contains the egressive preverb {\it kaa}
with a rising tone indicating the future tense.  In order to signal a polar
question, each has  an extra-low tone and is slightly lengthened at the end of
the utterance. 

\ea\label{GRM-pfv-inter}


\ea\label{GRM-pfv-inter-pfv}
\glll {\T  } {\T    } {\T   } {\T    } {\T   }\\
 ɪ   teŋesi  a  namɪ̃ã  raa \\
          {\sc 2.sg} {cut.{\sc pv}} {\sc art} {meat} {\sc foc}\\
\glt `Did you cut the meat (into pieces)?'\\



\ex\label{GRM-pfv-inter-impf}

\glll {\T   } {\T     } {\T    } {\T    } {\T   }\\
ɪ   teŋesi  a  namɪ̃ã  raa \\
          {\sc 2.sg} {cut.{\sc pv}} {\sc art} {meat} {\sc foc}\\
\glt `Are you cutting the meat (into pieces)?'\\


\ex\label{GRM-pfv-inter-impf-fut}

\glll {\T  } {\T    } {\T     } {\T    } {\T    } {\T   }\\
ɪ  kaa teŋesi  a  namɪ̃ã  raa \\
          {\sc 2.sg} {\sc ipfv.fut} {cut.{\sc pv}} {\sc art} {meat} {\sc foc}\\
\glt  `Will you (be) cut(ting) the meat (into pieces)?'\\
 
\z 
 \z

The only distinction perceived between (\ref{GRM-pfv-inter-pfv})  and
(\ref{GRM-pfv-inter-impf}) is a pitch difference on the third syllable of the
verb. The tonal melody associated with the verb in 
(\ref{GRM-pfv-inter-impf-fut}) is the
same as the one in (\ref{GRM-pfv-inter-impf}).







\paragraph{Ex-situ subject imperfective particle}
\label{sec:GRM-ipfv-part}

One topic-marking strategy is to prepose a non-subject constituent to the
beginning of the clause.  In  (\ref{ex:GRM-foc-top}),  the focus particle may or
may not
appear after the non-subjectival topic. Notice that one effect of this 
topic-marking strategy is that the particle {\it dɪ} appears between the subject
and
the verb when the non-subject constituent is preposed and when the clause is
used to describe what is happening at the moment of speech. \nolinebreak

\begin{multicols}{2}
\ea\label{ex:GRM-foc-top}
 \ea\label{ex:GRM-foc-top-chew-prog-1}{\rm Imperfective}\\
\gll  sɪ́gá (rá)  ʊ̀ dɪ̀  tíē   \\
 bean  ({\foc}) {3\sg} {\ipfv} chew\\
\glt `It is BEANS he is chewing'


 \ex\label{ex:GRM-foc-top-chew-perf-1}{\rm Perfective}\\
\gll  sɪ́gá (rá) ʊ̀   tìè     \\
 bean  ({\foc}) {3\sg}  chew \\
\glt `It is BEANS he chewed'

 \ex\label{ex:GRM-foc-top-go-prog-2}{\rm Imperfective}\\
\gll   wàà (rá) ʊ̀ dɪ̀  káálɪ̀   \\
Wa    ({\foc}) {3\sg} {\ipfv} go\\
\glt `It is to WA that he is going'


 \ex\label{ex:GRM-foc-top-go-perf-2}{\rm Perfective}\\
\gll   wàà (rá)  ʊ̀ kààlɪ̀    \\
Wa   ({\foc}) {3\sg}  go\\
\glt `It is to WA that he went'
\z 
 \z
\end{multicols}

The position of {\it dɪ} in  
(\ref{ex:GRM-foc-top-chew-prog-1}) and 
(\ref{ex:GRM-foc-top-go-prog-2}), that is between the subject and the verb, is 
generally occupied by linguistic items called  {\it preverbs},  to which the 
discussion turns in Section \ref{sec:GRM-precerv}.  Provisionally, the particle 
{\it dɪ} may be treated as a preverb constrained to occur with  a preposed  
non-subject constituent and an imperfective aspect.\footnote{I do not treat 
topicalization in this work, although the left-dislocation strategy in 
(\ref{ex:GRM-foc-top}) is the only one I know to exist.}


\paragraph{Subjunctive}
\label{sec:GRM-subjunctive}
% [to do: wish, potential event, dependent clause] 
% [high tone on subject]
%\citep{scuh03}


\subsection{Preverb particles}
\label{sec:GRM-precerv}

Preverb particles  encode various event-related
meanings. They are part of the verbal domain  called the expanded verbal group
(EVG), discussed in (\ref{sec:GRM-verbals}) and schematized in  
(\ref{ex:verb-VP}). This domain  follows the 
subject and precedes the main verb(s) and is generally accessible  only to a 
limited set of linguistic items. These grammatical morphemes are
not verbs, in the sense that they do not contribute to SVCs as verbs do,  but as
`auxiliaries'. Still,  some of the preverbs may historically derive from verbs, 
and  some others may synchronically function as verbs.  Examples of the latter
are the egressive particle {\it ka} and ingressive particle {\it wa},  which are
discussed in Section \ref{sec:GRM-EVC-egr-ingr}. 

Nevertheless, given the data available,  it would not be incorrect to analyze 
some of the preverbs  as additional SVC verbs.  However, we will see  that a 
preverb differs from a verb in that it exposes functional categories,  cannot 
inflect for the perfective or assertive suffix,  and never takes  a complement, 
such as a grammatical object, or cannot be modified by  an adjunct. But again,  
a first verb in a SVC and a preverb are categories which can be hard to 
distinguish. Structurally and functionally, many of them may be analyzed as 
grammaticalized verbs in series. These characteristics are not special to 
Chakali; similar, but not identical, behavior are described for Gã and Gurene 
\citep{Daku07b, Daku08}.



\subsubsection{Egressive and ingressive particles}
\label{sec:GRM-EVC-egr-ingr}


The egressive particle {\it ka(a)} ({\it gl.} {\sc egr})   `movement away from
the
deictic centre'  and   the ingressive
particle {\it wa(a)} ({\it gl.} {\sc ingr})  `movement towards
the deictic centre' are  assumed to derive from the  verbs
{\it kaalɪ} `go' and  {\it waa} `come'.\footnote{A discussion on some aspects of
grammaticalization of  `come' and `go' can be read in  \citet{Bour92}. In the
literature, egressive  is also known as  {\it itive} (i.e. away from the
speakers,  `thither')  and  ingressive  is  known as {\it ventive} (i.e. towards
the speakers,   `hither'). }  Table
\ref{tab:deict-pre-verb} shows that  {\it kaalɪ} `go' and {\it waa} `come',  
like 
other verbs, change forms (and are acceptable) in these paradigms,  but {\it 
ka(a)}  is  not.


\begin{table}[h]
\centering
\caption{Deictic verbs and preverbs \label{tab:deict-pre-verb}}

\begin{Itabular}{lllll}
\lsptoprule
Verb & $\sigma$  & Aspect & Positive & Negative\\[1ex] \midrule


{\it waa} `come' & CVV   & {\sc pfv}   &  ʊ̀ wááwáʊ́   & ʊ̀ wà wááwá\\
          &&& `she came' & `she didn't come'\\

       &   & {\sc ipfv}   &  ʊ̀ʊ̀ wááʊ̄  & ʊ̀ wà wáá\\
          &&& `she is coming' & `she is not
coming'\\[1ex] \midrule




{\it kaalɪ} `go' & CVVCV   & {\sc pfv}   &  ʊ̀ káálɪ́jʊ́   & ʊ̀ wà 
káálɪ́jɛ́\\
          &&& `she went' & `she didn't go'\\

       &   & {\sc ipfv}   &  ʊ̀ʊ̀ káálʊ̄ʊ̄  & ʊ̀ wà káálɪ́\\
          &&& `she is going' & `she is not
going'\\[1ex] \midrule


{\it ka}  & CV   & {\sc pfv}   &  *ʊ kaʊ   & *ʊ wa kajɛ\\
        

       &   & {\sc ipfv}   &  *ʊ kaʊ  & *ʊ wa ka\\
        
\lspbottomrule


\end{Itabular}         
\end{table}


If the verbs {\it kaalɪ} `go' and  {\it waa} `come'
occur in a SVC,  they surface as {\it ka} and {\it wa} respectively. In
(\ref{GRM-prev}),  both verbs take part in  a two-verb SVC in which they are
first in the sequence.


\ea\label{GRM-prev}

\ea\label{GRM-prev-SVC-ka}
\glll gbɪ̃̀ã́           bààŋ         té      \textbf{kà}           sáŋá  
 à   
píé  {(...)} \\
monkey     quickly   early   go     sit    {\art}
yam.mound.{\sc pl}   {(...)} \\
{} [[{\it pv} {\it pv}]$_{EVG}$  {\it v} {\it v}]$_{VP}$ {} {}
{(...)}\\
\glt `Monkey quickly went and sat on the (eighth) yam mounds (...)'  (LB 012)

\ex\label{GRM-prev-SVC-wa}
\glll    ŋmɛ́ŋtɛ́l   làà nʊ̀ã̀  nɪ́    ká  ŋmá dɪ́    ʊ́ʊ́  \textbf{wá}  
ɲʊ̃̀ã̀ nɪ́ɪ́ \\
spider collect mouth {\postp}  {\conn} say {\comp}  
{\sc 3.sg}  come   drink water\\
 {} {} {} {} {}  {} {} {} {\it v} {\it v} {} \\
\glt  `(Monkey went to spider's farm to greet him.)  Spider accepted
(the
greetings) and (Spider) asked him (Monkey) to come and drink water.'  (LB 011)

 
\z 
 \z


Because they derive from deictic verbs (historically or synchronically),  the
preverbs have the potential to indicate non-spatial  `event movement'  to or 
from a deictic centre.
This phenomenon is not uncommon cross-linguistically. \citet[62]{Nico07}
maintains  that when a movement verb becomes a tense marker, it may be reduced
to a verbal affix and its meaning can develop ``into meaning relating temporal
relations between events and reference times''. In Chakali, the  preverb {\it 
kaa} contributes   temporal information to an expression. Consider in
(\ref{exe:GRM-crack-remove-attach}) the distribution and contribution of  {\it 
kaa} to  the clauses headed by the verbs {\it kpe} `crack a shell and remove a
seed from it' (henceforth `c\&r') and {\it mara} `attach'.\footnote{In Gurene 
(Western Oti-Volta), it
is the ingressive particle which has a similar role. The ingressive  is 
commonly used before the verb, and can, among other things,  express future
tense \citep[see][59]{Daku07b}.}



\ea\label{exe:GRM-crack-remove-attach}
%\begin{multicols}{2}

\ea

 ʊ̀ kàá kpē  {\rm  `She will c\&r'} \\
   ʊ̀ʊ̀ kpéū  {\rm     `She  is c-\&r-ing/c-s\&r-s'}\\
   ʊ̀ kpéjòō   {\rm `She   c-\&r-ed'}\\
   kpé       {\rm  `C\&r!'}
\ex
 ʊ̀ kàá mārā   {\rm   `She will attach'}\\
   ʊ̀ʊ̀ máráʊ̄     {\rm  `She  is attaching/attaches'}\\
  ʊ̀ márɪ̀jʊ̄    {\rm `She   attached'}\\
   márá     {\rm  `Attach'}

%\end{multicols}
\z
\z

When the preverb particle {\it kaa} is uttered with a rising pitch it situates 
the event in the future. The preverb particle {\it kaa} can also be used to 
express that an event is
ongoing at the moment of speech, which I call the present 
progressive.   However,  when it is used to describe what is happening
now, {\it kaa} can only appear when the subject is not a pronoun and its tone
melody differs from that of the future tense. These contrasts are given in
(\ref{exe:GRM-kaa-attach}).

\ea\label{exe:GRM-kaa-attach}
 ʊ̀ kàá mārā   {\rm `She will attach'}\\
   ʊ̀ʊ̀ máráʊ̄      {\rm  `She  is attaching'}\\
wʊ̀sá kàá mārā   {\rm  `Wusa will attach'} \\
wʊ̀sá káá   máráʊ̄  {\rm  `Wusa is attaching'} \\
\textasteriskcentered  wʊ̀sá   máráʊ̄   {\rm    `Wusa is
attaching'}
\z

The paradigm in  (\ref{exe:GRM-kaa-attach}) shows that when the preverb 
particle {\it kaa} appears with a rising tonal melody it  expresses the future 
tense, but  in
order to convey that a situation is ongoing at the time of speech (i.e. present
progressive), the preverb particle {\it kaa} has a high tone. Thus, it is the
tonal melody on {\it kaa} which distinguishes between the future and the present
progressive (both treated as imperfective),  plus the fact that pronouns cannot
co-occur with the preverb particle {\it kaa} in the present progressive. 


In (\ref{x:GRM-tone-ipfv-L}) {\it kaa}'s melody is shown to be affected by   the 
pitch  of   the  preceding  noun {\it bie} (LH) `child' and the demonstrative 
{\it haŋ} (HL) `this'. 


\ea\label{x:GRM-tone-ipfv}
\ea\label{x:GRM-tone-ipfv-H}
\gll à bìé káá bīlīgī ʊ̀ʊ̀ nàál kɪ̀nkán nà\\
{\art} child {\sc ipfv} touch {\sc poss.3.sg} grand.father {\sc quant} {\sc 
foc} \\
\glt `The child touches his grand-father.'

\ex\label{x:GRM-tone-ipfv-L}
\gll à bìè háŋ̀ kàà bīlīgī ʊ̀ʊ̀ nàál kɪ̀nkán nà\\
{\art} child {\sc dem} {\sc ipfv} touch {\sc poss.3.sg} grand.father {\sc 
quant} {\sc 
foc} \\
\glt `This child touches his grand-father.'

\z
\z



Although little evidence is available, the preverb {\it wa} may also be used to
express a sort of hypothetical  mood.  In  (\ref{ex:GRM-prev-wa-hypo}), the
preverb {\it wa} should be seen as contributing a supposition, or a hypothetical
circumstance where
someone would be found calling the number 8. 
%[to do: check both wa and waa]

\ea\label{ex:GRM-prev-wa-hypo}

\gll ŋmɛ́ŋtɛ́l   ŋmā    dɪ̄,    kɔ̀sánáɔ̃̀,   tɔ́ʊ́tɪ̄ɪ̄nā  ŋmá 
dɪ́,    námùŋ   wá    jɪ̀rà ŋmɛ́ŋtɛ́l sɔ́ŋ,    bá  kpáɣʊ́ʊ̄    wàà 
bá kpʊ́\\
spider     say   {\comp}   buffalo    land.owner say {\comp}  anyone   
{\ingr}   call eight     name {\sc 3.pl.hum+}    catch.{\sc 3.sg} {\foc}
 {\sc 3.pl.hum+}  kill\\
\glt `Spider told Buffalo that landowner said anyone who calls the number 8
should be brought to him to be killed.' (LB 009)
\z



Finally, the example in (\ref{ex:GRM-verb-ta})  intends to show that some
elders of Ducie and
Gurumbele use  {\it ta}  instead of {\it ka(a)},  as a variant of the
preverb.\footnote{I gathered that  (i)   {\it ta} is not a different
preverb (Gurene is said to have  a preverb {\it ta}  signifying intentional
action  (M. E. K. Dakubu, p.\ c.),  and  (ii)   {\it ta} can be heard in  Ducie 
and
Gurumbele from people of the oldest generation, but somebody suggested to me
that {\it ta} is the common form in Motigu (Mba Zien, p.\ c.).  The distinction
is  in need of further research. } 
 

\begin{exe}
   \ex\label{ex:GRM-verb-ta}{\rm Priest talking to the shrine, holding a kola
nut above it}

\gll  má láá kàpʊ́sɪ́ɛ̀ háŋ̀ ká jà mɔ́sɛ́ tɪ̀ɛ̀ wɪ́ɪ́ tɪ̀ŋ bà 
\underline{tà/kàà} búúrè\\
{\sc 2.pl} take kola.nut {\sc dem} {\sc conn} {\sc 1.pl} plead give matter {\sc
art} {\sc 3.pl.}b {\sc  egr} want\\
\glt   `Take this kola nut, we implore  you to give them what they desire.'

\z


Unfortunately, since the relation between tense, aspect, and tonal melody is not
well-understood at this stage of research, the  egressive {\it ka}   and the
ingressive  {\it wa} are  broadly glossed as {\egr} and {\ingr} respectively, 
but
can also be associated with composite glosses such as {\ipfv .\fut} or  {\ipfv
.\pres}  in cases where a distinction is clear.




\subsubsection{Negation preverb}
\label{sec:GRM-verb-neg}
%check negative concord with nobodu, no one, all, nothing


%  Their
% lengths may vary depending on the speech rate, but  they are always long
% in 

There are three different particles of negation in the language:  the forms 
{\it lɛɪ} and {\it tɪ}   were discussed in Sections \ref{sec:GRM-imper-clause} 
and  \ref{sec:GRM-foc-neg}  respectively.  The negative preverb particle {\it 
wa(a)} precedes the verb and is used in the verbal group (in non-imperative
mood). The same form is found in both  main and dependent clauses [to do: ex]. 


\ea\label{ex:GRM-neg-pres-fut}

\ea
\gll ʊ̀  wàá pɛ̀ \\
   {\sc 3.sg}  {\neg} add\\
\glt  `She will not add.'

 \ex 
\gll  ʊ̀ʊ̀ wàà pɛ́\\
     {\sc 3.sg} {\neg} add \\
\glt  `She is not adding.'


 \ex 
\gll  ʊ̀ wà pɛ́jɛ̄\\
     {\sc 3.sg} {\neg} add \\
\glt  `She didn't  add.'

\z 
 \z

 The examples in (\ref{ex:GRM-neg-pres-fut}) show that a tonal quality on the 
negation particle and following verb  distinguishes between the present 
progressive and  the future,  as the preverb {\it kaa} does (see example 
\ref{exe:GRM-kaa-attach}). The length of the negation particle can also 
function as a cue.

\ea\label{ex:neg-quant-any}
\ea\label{ex:neg-quant-any-1}
\gll námùŋ wà ná-ŋ̀ \\
 {\clf}.all {\neg} see-{1.\sg}\\
\glt  `Nobody saw me.' ({\it lit.} everyone not see me) 

\ex\label{ex:neg-quant-any-2}
\gll  ǹ wà ná námùŋ  \\
  {1.\sg}  {\neg}   see  {\clf}.all\\
\glt  `I did not see anyone.' ({\it lit.} I not see everyone) 

\z 
 \z

 Example (\ref{ex:neg-quant-any}) shows that when the negation particle {\it 
wa(a)} and a quantifier appear in the same clause the quantifier is  in the 
positive. 




\ea\label{ex:GRM-neg-come}
 
  
\ea\label{}
\gll ʊ̀ wà wá dī\\
{\sc 3.sg} {\neg} come eat\\
\glt `She did not come to eat.'
\ex\label{}
\gll ʊ̀ wàá wà dí\\
{\sc 3.sg} {\neg} come eat \\
\glt `She will not come to eat.'

\z 
 \z
 
 The negative preverb  always precedes the verb {\it waa} `come'. Although 
length (CV or CVV) is  hard to differentiate in natural speech, the examples in 
(\ref{ex:GRM-neg-come}) suggest that the tonal melody and length  establish 
meaning differences.

Assertion and negation seem to avoid one another and constrain the grammar  in
the following way:  {\it If a clause is negated,  none of its constituents can
be in focus.} In Section \ref{sec:GRM-personal-pronouns},  it was shown that (i)
negation cannot co-occur with the strong pronouns, and (ii) negation cannot
co-occur with an argument of the predicate in focus, i.e. with {\it ra} or one 
of
its variants having scope over the noun phrase. The third non-occurrence of
negation concerns  the assertive form of the verb (Section
\ref{sec:GRM-focus}).  Consider the forms of the verb {\it mara} `attach' in the
two paradigms in (\ref{ex:GRM-verb-neg-foc}).\footnote{The particle {\it la} in
Dagaare has similar constraints. \citet[94]{Bodo97} calls it a {\it factitve}
particle.}





   \ea\label{ex:GRM-verb-neg-foc}

   \ea\label{ex:GRM-verb-neg-foc-pos}{\rm Positive}\\

 ʊ̀ kàá mārā   {\rm  `She will attach'}\\
   ʊ̀ʊ̀ máráʊ̄  {\rm     `She  is attaching/attaches'}\\
  ʊ̀ márɪ̀jʊ̄    {\rm `She   attached'}\\

  \ex\label{ex:GRM-verb-neg-foc-neg}{\rm Negative}\\

 ʊ̀ wàá mārā  {\rm `She will not attach'}\\
   ʊ̀ʊ̀ wàà márá    {\rm  `She  is  not attaching/does not attach'}\\
  ʊ̀ wà márɪ̀jɛ̄   {\rm  `She   did not attach'}\\


\z 
 \z

The paradigms in (\ref{ex:GRM-verb-neg-foc})  suggest that the negation particle 
and the assertive suffix are in complementary distribution. 

\subsubsection{Tense, aspect, and mood preverbs}
\label{sec:GRM-verb-neg}


\paragraph{fɪ}

The preverb {\it fɪ}   is identified with two different but interrelated 
meanings.  First, as (\ref{ex-preverb-fi-neut}) shows, the preverb {\it fɪ}  
({\it gl.} {\sc pst}) is a neutral past tense particle (i.e.  as opposed to the 
specific  {\it dɪ} of  Section \ref{sec:GRM-preverb-three-int-tense}), and the 
event referred to in the past can no longer be in effect in the present.


\ea\label{ex-preverb-fi-neut}

\ea
\gll ʊ̀ jáá  ǹ̩ǹ̩ tʃítʃà rā \\
  {\sc 3.sg} {\ident}    {\sc 3.sg.poss}  teacher {\foc}  \\
\glt  `He is my TEACHER.' 

\ex
\gll   ʊ̀  fɪ̀ jáá  ǹ̩ǹ̩ tʃítʃà rà\\
        {\sc 3.sg} {\pst} {\ident}    {\sc 3.sg.poss}  teacher {\foc}  \\  
\glt  `He was my TEACHER.' 

\z 
 \z 

 Secondly, the preverb {\it fɪ}   ({\it gl.} {\sc mod}) can have  deontic
meaning.  


\ea\label{ex-preverb-fi-deonc}

\ea\label{ex-preverb-fi-deonc-pos}
\gll ʊ̀ fɪ̀ɪ́ jàà  ǹ̩ǹ̩ tʃítʃà rā \\
  {\sc 3.sg}  {\mod}  {\ident}    {\sc 3.sg.poss}  teacher {\foc}  \\
\glt  `He should have been my TEACHER.' 

\ex
\gll    ʊ̀ fɪ̀ wáá jàà  ǹ̩ǹ̩ tʃítʃà \\
        {\sc 3.sg} {\mod} {\neg} {\ident}    {\sc 3.sg.poss}  teacher   \\  
\glt   `He should not have been my teacher.'  


\ex
 ʊ̀  fɪ̀ jáá  ǹ̩ǹ̩ tʃítʃà  rā  {\rm `He was my TEACHER.'}
\ex
 ʊ̀  fɪ̀ wà jáá  ǹ̩ǹ̩ tʃítʃà    {\rm `He was not my teacher.'}

\z 
 \z 


In (\ref{ex-preverb-fi-deonc}),  its presence still conveys  past
tense, but in addition it expresses that the situation did not really occur, yet
it was objectively supposed to occur or subjectively expected to occur or
awaited. The lengthening of the preverb {\it fɪ} in the positive  is not
accounted for, but I suspect it  signals the imperfective. Compare the first two
sentences in (\ref{ex-preverb-fi-deonc}) with the last two  which convey the
neutral past. 


The positive sentence in (\ref{ex-preverb-fi-deonc-pos}) can receive  a
translation along these lines:  In a desirable possible world, he was my
teacher, but it is not what happened in
the real world. 

\ea\label{ex-preverb-fi-pure-deonc}

\ea\label{ex:GRM-vp11.2}
\gll m̩̀m̩̀ mɪ̀bʊ̀à fɪ́  bɪ̀rgɪ̀ \\
    {\sc 1.sg.poss} life    {\mod}  delay    \\
\glt  `May I live long!' 

\ex\label{ex:GRM-vp11.3}
\gll tɪ̀ɛ̀ m̩̀m̩̀ mɪ̀bʊ̀à bɪ́rgɪ̀ \\
      give {\sc 1.sg.poss} life delay  \\
\glt  `Let me live long!' 

\z 
 \z 

Finally, the preverb {\it fɪ}  in (\ref{ex-preverb-fi-pure-deonc}) still conveys
 deontic modality, where the speaker prays or asks permission for a 
situation. Notice, however,  that it cannot refer to a past event. The two
sentences
in (\ref{ex-preverb-fi-pure-deonc}) have a corresponding meaning. Example
(\ref{ex:GRM-vp11.3}) is framed in an imperative clause (see \is{optative} {\it 
optative} in 
Section
\ref{sec:GRM-imper-clause}). 


\paragraph{Preverb three-interval tense}
\label{sec:GRM-preverb-three-int-tense}

Chakali encodes  in  preverbs  a type of time categorization  known as 
three-interval tense  \citep[366]{Fraw92}. It is possible to express that an 
event occurred specifically yesterday, as opposed to earlier today and the day 
before yesterday, i.e. {\it hesternal tense} ({\it gl.} {\sc hest}), or 
specifically tomorrow, as opposed to later today and the day after tomorrow, 
i.e. {\it crastinal tense}  ({\it gl.}  {\sc cras}). The hesternal tense 
particle {\it dɪ}/{\it de} ({\it gl.} {\sc hest})  refers to the day preceding 
the speech time.  It has the temporal nominal  counterpart  {\it dɪare (tɪŋ)} 
`yesterday'.  



\ea\label{ex:vp2.11.a} 
\gll {(dɪ̀àrè tɪ̀n)} ʊ̀ nɪ́ ʊ̀ tʃɛ̀ná dɪ́ wāāwā  {(dɪ̀àrè tɪ̀n)}\\
{(yesterday)}    {\sc 3.sg} {\sc conn} {\sc 3.sg.poss} friend
{\sc hest}  come.{\sc pfv} {(yesterday)} \\ 
\glt  `He arrived with his friend yesterday.'
 \z



In (\ref{ex:vp2.11.a}),  the  phrase {\it dɪare 
tɪn} `yesterday' is optional,  and  when it is used it must be expressed at the 
end or at the beginning of the clause.


\ea\label{ex:vp4.5} {\it Will you work for the chief today or tomorrow?}\\
\gll  ǹ̩ tʃɪ́ kàá tʊ̀mà tɪ̄ɛ̄ʊ̄ rà, záàŋ,  ǹ̩ kàá hɪ̃̀ɛ̃̀sʊ̀ʊ̄ \\
    {\sc 1.sg} {\sc cras}  go  work give.{\sc 3.sg} {\sc foc},
today,   {\sc 1.sg}  {\sc egr} rest.{\sc foc} \\
\glt  `I shall work for
him tomorrow, today,  I shall rest.' 
 \z

The crastinal tense preverb {\it tʃɪ} ({\it gl.} {\cras})  in  (\ref{ex:vp4.5})
functions as future particle,  but is limited to the day following the event
time.
In that sentence the event time referred to follows  the utterance
time by one day.  The temporal nominal counterpart  of {\it tʃɪ} is {\it  tʃɪa} 
`tomorrow'. As
for the hesternal tense and the corresponding nominal,  the  nominal may or
may not co-occur with the crastinal tense particle. 




The hesternal tense particle {\it dɪ} is homophonous with the ({\it ex-situ
subject}) imperfective particle  {\it dɪ} discussed in Section
\ref{sec:GRM-ipfv-part}.  In addition, the question arises as to whether the
crastinal tense  is inherently future, and if so, whether or not it can
co-occur with the future-encoding egressive preverb discussed in Section
\ref{sec:GRM-EVC-egr-ingr}. Consider their distribution and meaning in the
examples given in (\ref{ex:GRM-prev-dist}).


\ea\label{ex:GRM-prev-dist}

\ea\label{ex:GRM-prev-dist-chew-presprog}{\rm Imperfective}\\
\gll  sɪ́gá (rá)  ʊ̀ dɪ̀  tíē   \\
 bean  ({\foc}) {3\sg} {\ipfv} chew\\
\glt `It is BEANS he is chewing'

 \ex\label{ex:GRM-prev-dist-chew-past}{\rm Perfective/Past}\\
\gll  sɪ́gá (rá) ʊ̀   tìè     \\
 bean  ({\foc}) {3\sg}  chew \\
\glt `It is BEANS he chewed'


 \ex\label{ex:GRM-prev-dist-chew-past}{\rm Hesternal past}\\
\gll  sɪ́gá (rá) ʊ̀ dɪ́    tìè     \\
 bean  ({\foc}) {3\sg} {\hest}  chew \\
\glt `It is BEANS he chewed yesterday'


 \ex\label{ex:GRM-prev-dist-chew-past-pro}{\rm Hesternal past progressive}\\
\gll  sɪ́gá (ra) ʊ̀ dɪ́ɪ́    tīè     \\
 bean  ({\foc}) {3\sg} {\hest}  chew \\
\glt `It is BEANS he was chewing yesterday'

 \ex\label{ex:GRM-prev-dist-chew-futprog}{\rm Future (progressive)}\\
\gll  sɪ́gá (rá) ʊ̀  kàá   tíē     \\
 bean  ({\foc}) {3\sg} {\fut}  chew \\
\glt `It is BEANS he will be chewing / will chew'

 \ex\label{ex:GRM-foc-top-chew-crasfutprog}{\rm Crastinal future 
(progressive)}\\
\gll  sɪ́gá (rá) ʊ̀ tʃɪ́  kàá   tìè     \\
 bean  ({\foc}) {3\sg} {\cras} {\fut}   chew \\
\glt `It is BEANS he will be chewing / will chew tomorrow '


\z 
 \z 
 
 
A specific tonal melody associated with  the sequence {\it dɪ tie} can express 
either a present progressive, as in (\ref{ex:GRM-prev-dist-chew-presprog}),  or 
a hesternal past, as in  (\ref{ex:GRM-prev-dist-chew-past}). Lengthening the 
hesternal past particle allows one to express the tense associated with the 
particle, in addition to indicating  progressive 
(\ref{ex:GRM-prev-dist-chew-past-pro}). This strategy seems to correspond 
semantically  to the apparent syntactic anomaly *{\it dɪ dɪ},  {\it lit.} {\sc 
 hest} {\sc ipfv}.  The example in (\ref{ex:GRM-foc-top-chew-crasfutprog}) 
shows that the crastinal tense particle and the egressive particle signaling  
future  can co-occur.  Inserting the imperfective particle {\it  dɪ} between 
the egressive particle and the verb in  (\ref{ex:GRM-prev-dist-chew-futprog}) 
and (\ref{ex:GRM-foc-top-chew-crasfutprog}) is  unacceptable. It is unclear 
whether these two examples must be interpreted as progressive or not.  



\paragraph{te}
\label{sec:GRM-preverb-te}

Lacking a corresponding verb to capture its meaning, the verb {\it te} is 
glossed
with
the English adverb `early'. Even though  it is attested as main verb,  {\it te}
can  function  as a preverb and it is indeed more common to find it in that
function. 


 \ea\label{ex:GRM-prev-early}
 \ea
\gll  ɪ̀ téjòō\\
     {\sc 2.sg} early.{\foc}   \\
\glt  `You are early.'


\exp{GRM-prev-SVC-ka}
\glll gbɪ̃̀ã́           bààŋ         té      kà           sáŋá   à   
píé  {(...)}\\
monkey    quickly   early   go     sit    {\art}
yam.mound.{\pl}   {(...)} \\
{} {\it pv} {\it pv}  {\it v} {\it v} {}  {} {} \\
\glt `Monkey quickly went and sat on the (eighth) yam mounds (...)'  (LB 012)
\z
\z




The main verb {\it te}
and the preverb {\it te} are shown respectively in  
(\ref{ex:GRM-prev-early}).  It contributes a manner, 
one in which the
event is carried out before the expected or usual time.  It cannot be used  in 
adjunct positions (Section
\ref{sec:GRM-adjuncts}).  



\paragraph{zɪ}
\label{sec:GRM-preverb-after-then}

The preverb {\it zɪ} is marginal in the corpus.\footnote{There is another
similar particle, {\it ze}  ({\it gl.} {\sc exp}),  which is still not
understood: (i) it occurs after the noun phrase, and  (ii) its meaning
corresponds to
 `expected (by both the speaker and the
hearer, or only by the speaker)'. It informs that the referent of
the noun phrase was anticipated before the utterance time (or relative time) by
the speaker and hearer (or only the speaker).  Consider  the following
example:

\ea\gll ba ze  waawaʊ \\
{\sc 3.pl.b} {\sc  exp} come.{\sc pfv}\\
`They (the expected people) have come.'
\z
}  



\ea
\ea\label{ex:GRM-prev-zi-1}  {\rm A father is giving a sequence of tasks to
his son}

 \glll tʊ̀mà  à  zɪ̃́ɛ̃́  mʊ̃́ã̀  ká  kà  tʊ̀mà  kùó   àká   zɪ́ kà  
tʊ̀mà à  gár  \\
  {work} \textsc{art}    {wall} {before}  \textsc{conn} {\sc egr}    {work}
{farm} 
\textsc{conn} {after}  {go} {work} \textsc{art} {cattle.fence}\\  
{} {} {} {} {}  {} {} {} {} {\it pv} {\it v} {\it v} {} {}\\
\glt  `First repair the wall, then go and farm, then repair the cattle fence.'


 \ex\label{ex:GRM-prev-zi-2}
  \glll kààlɪ̀ dɪ̀á ká zɪ́ kààlɪ̀ kùó\\
go  house and then go farm\\
{} {} {}  {\it pv} {\it v} {}\\
 \glt `Go to the house and then go to the farm.'


\z 
 \z



There is no corresponding
verb in the language.   It is used to express an order of events,  
in such case words such as {\it mʊã}  
`before' and 
{\it zɪ} `after' and the connective {\it ka/aka}  `and/then'
are used, as (\ref{ex:GRM-prev-zi-1}) shows. However,  as
(\ref{ex:GRM-prev-zi-2}) illustrates,  the preceding event may be presupposed, 
so  it is not necessarily uttered.







\paragraph{baaŋ}
\label{sec:GRM-preverb-baang}

 The preverb  {\it baaŋ}  ({\it gl.} {\sc mod})  is primarily modal and is  
usually translated into English `must', `immediately', `quickly'  or `just'. 


\ea\label{sec:GRM-prev-bg-must}
\ea\label{ex:GRM-7.17}
\gll  kùórù ŋmá dɪ́ ǹ̩ kàá bààŋ bɔ́ bʊ̃́ʊ̃́ná  fí rē \\
 chief say {\comp} {\sc 1.sg} {\fut} {\mod}   pay  goat.{\pl} ten {\sc foc} \\
\glt  `The chief says that I must pay him ten goats.' 

\ex\label{ex:GRM-14.3}
\gll  ɪ̀ɪ̀ kàá bààŋ jáʊ́ rā\\
{\sc 2.sg} {\fut} {\mod} do.{\sc 3.sg} {\foc}\\
\glt  `You must do it.'

 \z 
 \z
 
 First, the examples in (\ref{sec:GRM-prev-bg-must}) show that  the preverb  
{\it baaŋ} conveys an obligation and the notion of temporality is secondary. 


\ea\label{sec:GRM-prev-bg-time}
\glll   {(...)} à kpá ʊ̀ʊ̀ néŋ à sàgà ʊ̀ʊ̀ nɪ̄ dɪ́ ʊ̀ bààŋ té 
bɛ̀rɛ̀gɪ̀ dʊ̃́ʊ̃̀\\
     {(...})  {\conn}  take {\sc 3.sg.poss} arm {\conn} {be.on} {\sc
3.sg}  {\postp} {\conn} {\sc 3.sg} {\mod} {early} turn.into python \\
{} {} {} {} {} {} {} {} {} {} {} {\it pv} {\it pv} {\it v} {} \\

\glt  `(...) then put his hand on her  and quickly turned into
a python.' (Python story 025)
\z
      
      Secondly, as illustrated in (\ref{sec:GRM-prev-bg-time}),  the preverb  
{\it baaŋ} can express an  abrupt or
swift   manner. 




 \ea\label{ex:GRM-prev-bg-excerpt}
\ea\label{ex:FUS-mod}
\gll  ʊ̀ zɪ́má dɪ́ jà kàá ŋmá ʊ̀ʊ̀ wɪ́ɛ́ rá ʊ̀ʊ̀ bààŋ tʃùò dúò\\
 {\sc 3.sg}  know {\sc comp}  {\sc 1.pl}  {\fut} talk   {\sc 3.sg.poss} matter 
{\foc} {\sc 3.sg} {\mod} lie sleep  \\
\glt  `He knew that we would talk about him, so he quickly slept.'

\ex
\gll kàwàá bààŋ tàrɪ̀ kééééŋ \\
pumpkin just creep {\dxm}\\
\glt `A pumpkin just crept like that ...' 

\ex
\gll à kùò ní ʊ̀ bààŋ jírúú kéŋ néé à wà kʊ̀ʊ̀ \\
{\sc art} farm {\postp}  {\sc 3.sg} {\mod} call.{\ipfv} {\sc dxm} {\foc} {\sc 
conn} {\sc ingr} tire  \\
\glt `At the farm he kept calling (for someone) but got tired (gave up).'


\ex
\gll díŋ bààŋ jàà dìŋtʊ́l̀\\
fire  just {\ident} flame\\
\glt `The fire suddenly became flame.'

 \z 
 \z

 Finally, the preverb  {\it baaŋ} may act as a discourse particle used mainly 
to 
emphasize or intensify the action carried out, reminiscent of  the use of 
`just' 
in some English registers.  It is often translated in text as `immediately', 
`suddenly', `then',  or simply `just'. Examples are given in 
(\ref{ex:GRM-prev-bg-excerpt}).


\paragraph{ŋma}
\label{sec:GRM-desiderative mood}

As an independent verb {\it ŋmá} means `say'. The same verb can also 
function in a construction [NP {\it ŋma} [NP VP]]  conveying a desiderative 
mood,  corresponding to the English modal expression `want to'.

% \ea\label{ex:GRM-dsdrtv}
% \ea\label{ex:dsdrtv-1}
% \gll \\
%      \\
% \glt  `' 


\ea\label{ex:dsdrtv-2}
\gll ŋ̀ ŋmá [ŋ́ káálɪ̀ dùsèè tʃɪ̄ā]\\
  {\sc 1.sg} say     {\sc 1.sg} go D. tomorrow   \\
\glt  `I want to go to Ducie tomorrow.'
\z 
%  \z 
 
 Notice the high tone on the  {\sc 1.sg} pronoun subject  
in (\ref{ex:dsdrtv-2}),  which suggests that the embedded clause is in the 
subjunctive mood (Section 
\ref{sec:GRM-subjunctive}).

\paragraph{bɪ}
\label{sec:GRM-preverb-iteration}

The examples in (\ref{ex:GRM-prev-bi}) show that  the preverb particle {\it bɪ}
expresses iteration, but also the single repetition of an event, and follows the
negation particle. 

% bɪ kuor ŋma
%  repeat
%  bɪ pɪlɪ
% start again
% start


\ea\label{ex:GRM-prev-bi}

\ea\label{ex:vp33.2.}
\gll ʊ̀ bɪ́ kʊ̀ɔ̀rɛ̀ sã̀ã̀ ʊ̀ʊ̀ dɪ̀à rá \\
 {\sc 3.sg}     {\itr} make build {\sc 3.sg.poss} house {\foc}    \\
\glt  `He rebuilt his hut.' 


\ex\label{ex:GRM-vp10.4}
\gll à bìtʃèlíí bɪ́ sīīú\\
 {\art}  child.fall   {\itr} raise.{\foc}    \\
\glt  `The fallen child gets up again.' 



\ex\label{ex:vp10.4.}
\gll ʊ̀ wà bɪ́ tùwō \\
       {3.\sg} {\neg} {\itr} be.at\\
\glt  `She is not here again.' 

\z 
 \z 


Unlike other preverbs,  {\it bɪ} may also appear within noun phrases to express
frequency time. This is shown in (\ref{ex:GRM-vp19.2.}) (see Section
\ref{sec:NUM-repet}).



\begin{exe} 
\ex\label{ex:GRM-vp19.2.}
\gll  ǹ̩ jáà  káálɪ̀ ùù pé rè tʃɔ̀pɪ̀sɪ̀ bíí mùŋ \\
{\sc 1.sg} {\hab} go {\sc 3.sg.poss} end {\foc}  day.break {\itr} all\\
\glt  `I do visit him every day.' 

\z 



 \paragraph{bra}
\label{sec:GRM-preverb-return}

The preverb {\it bra} has a corresponding verb with the same form. It is
primarily a motion verb which conveys a change of direction. 


\ea\label{ex:GRM-verb-bra}

\ea
\gll brà à káálɪ̀\\
return {\conn} go\\
\glt `Go back.' (Hearer coming towards speaker, speaker ask hearer to turn and 
go back)

\ex
\gll brà àká tʃáʊ̀\\
return {\conn} leave.{\sc 3.sg}\\
\glt `Return and leave him.' (Speaker ask hearer to turn and go away from the 
person the hearer is with)

\z 
 \z

The examples 
in (\ref{ex:GRM-verb-bra}) present the verb {\it bra} in imperative clauses
separated by the connectives {\it a} and {\it aka}. 

\ea\label{ex:vp33.1.}
\gll ʊ̀ brá tʊ̀mà à tʊ́má tɪ́ŋ kà wà wíré kéŋ̀ \\
 {\sc 3.sg}  {again}  {work} {\art} {work}   {\art} {\egr} {\neg} well {\dxm}\\
\glt  `He redid the work that was badly done.'
\z


When {\it bra} functions as 
a preverb, as in (\ref{ex:vp33.1.}),  it loosely keeps its motion sense and
conveys in addition a sort of repetition. It differs from the morpheme {\it bɪ}
introduced in
 Section \ref{sec:GRM-preverb-iteration} because it does not mean that an
action is
done
repeatedly.  Instead, the preverb {\it bra} is associated with actions done 
`once
more', `over again',  or `anew'.

%regloss ka



\paragraph{ja}
\label{sec:GRM-preverb-hab}

The preverb {\it ja(a)} ({\it gl.} {\sc hab})  indicates habitual aspect. It 
expresses that the subject's referent is accustomed to, familiar with, or 
routinely do the action described by the predicate.


\ea\label{ex:GRM-prev-hab-do}
\gll tʃɔ̀pɪ̀sɪ̀ bɪ́-múŋ̀ ʊ̀ʊ̀ jáà jááʊ̄ \\
 day.break {\itr}-all {\sc 3.sg} {\hab} do.{\sc 3.sg}\\
\glt `He does it every day.'
\z 

 A variation in  length and intonation suggest  an (im)perfective aspectual 
distinction. In   (\ref{ex:GRM-prev-hab-do})  there is a  vowel sequence {\it 
aa} pronounced with a falling intonation. Compare this with the examples in 
(\ref{ex:GRM-prev-hab}). 



% % 
% % \ea\label{ex:habitual}
% % \ea\label{ex:}
% % \gll  ǹ̩  já kààlɪ̀ kùó\\
% % {\sc 1.sg} {\sc hab} go farm \\
% % \glt `I do go to the farm.'
% % 
% % \ex\label{ex:}
% % \gll  ǹ̩  jáà káálɪ̀ kùó\\
% % {\sc 1.sg} {\sc hab} go farm \\
% % \glt `I have been going to the farm.'
% % 
% % \z 
% %  \z
% %  
 

\ea\label{ex:GRM-prev-hab}
\ea\label{ex:GRM-prev-hab-do-pfv}
\gll  Kàlá já tùgòsì bísé ré\\
 K.  {\hab}  beat.{\sc pl} child.{\sc pl} {\sc foc} \\
\glt `Kala beat children.' (He used to do it.)

\ex\label{ex:GRM-prev-hab-impv}
\gll  Kàlá jáà túgósì bísé ré\\
K.  {\hab}  beat.{\sc pl} child.{\sc pl} {\sc foc} \\
\glt `Kala beat children.' (He regularly does it.)

\z 
 \z

The aspectual distinction in (\ref{ex:GRM-prev-hab}) is reflected by the 
preverb's vocalic length and intonation, but also on the following verb's 
intonation.




\paragraph{ha}
\label{sec:GRM-preverb-yet}

The morpheme {\it ha} ({\it gl.} {\sc mod}) is similar in meaning to the 
English 
morpheme `yet'. 


% The morpheme  {\it ha} is circumscribed to the expanded verbal group, although I 
% translate as `yet'  another expression in (\ref{ex:yet-conn}) which functions as 
% connective. The expression {\it haalɪ}, also found in many West African 
% languages -- ultimately from Mande family --  is not frequent in the data 
% available. 


\ea
\ea\label{ex:vp32.24}
\gll ʊ̀ʊ̀ háá díūū \\
     {3.\sg}  {\mod} eat.\foc  \\
\glt  `He is still eating.' 


\ex\label{ex:vp20.3.2.}
\gll ʊ̀ há wà díìjē \\
 {3.\sg}  {\mod} {\neg} eat.{\pfv}   \\
\glt  `He has not eaten yet.'


\ex\label{ex:vp21.2.1.}
\gll bà ɲíné ʊ̀ʊ̀ gɛ̀rɛ̀gá rá àká ʊ̀ʊ̀ háá wɪ̄ɪ̀ \\
 {\sc 3.pl.hum+} look {\sc 3.sg.poss} sickness {\foc} {\conn}  {\sc 3.sg}
{\mod} ill \\
\glt  `He has been cared for to no avail; he is still ill.' 


\ex\label{ex:vp20.1.1.}
\gll ʊ̀ há  wà wāā báàŋ múŋ̀ \\
       {3.\sg} {\mod}  {\neg} come {\dem} {\quant}\\
\glt  `He does not come here (ever).' 


\ex\label{ex:vp20.3.1.}
\gll ʊ̀ há wà wááwá \\
       {3.\sg} {\mod}   {\neg} come.{\pfv} \\
\glt  `He has not come yet.' 

% 
% \ex\label{ex:yet-conn}
% \gll ʊ̀ jɪ́rʊ́ʊ́ sāŋā mūŋ̀ ká ʊ̀ há wà wááwá \\
%        {3.\sg}  call.{\sc 3.sg} time all {\conn}   {\sc 3.sg} {\sc mod}  {\neg}
% come.{\pfv} \\
% \glt  `He called her long time ago, yet she has not
% come.' 

\z 
 \z
 
 It is used when an event is or was anticipated and a speaker considers or 
considered probable the occurrence of the event. As  for the English `yet', it 
is frequently found in negative polarity. In such cases the morpheme {\it ha} 
indicates that the event is expected to happen and the negative marker {\it wa} 
indicates that the event has not unfolded or happened at the referred time. In 
the cases where {\it ha} is found in a positive polarity,  it  conveys a 
continuative aspect, similar to English `still',  as in (\ref{ex:vp32.24}) and 
(\ref{ex:vp21.2.1.}). 



\paragraph{tu and zɪn}
\label{sec:GRM-preverb-up-down} 

The verbs {\it tuu} and {\it zɪna} are motion
expressions making reference to two opposite paths. 


\ea\label{ex:GRM-verb-up-down}
\ea
\gll ǹ̩ zɪ́nà sàl lá ḿ̩ páá tʃùònò\\
{\sc 1.sg} go.up flat.roof {\foc} {\sc 1.sg} take.{\pv} shea.nut.seed.{\pl}\\
\glt  `I go up on the roof to collect my shea nuts.'

\ex
\gll ǹ túú dɪ̀à rá\\
{\sc 1.sg} go.down house {\foc}\\
\glt I went down to the house.'
\z 
 \z

When they are used as main
predicate, as in example (\ref{ex:GRM-verb-up-down}),  they denote `go down' and
`go up' and  surface as {\it tuu} and {\it zɪna} respectively. 




\ea\label{ex:GRM-preverb-up-down}
\ea\label{ex:GRM-preverb-up}
\gll zɪ́ná tʃɔ́  à káálɪ̀  \\
      {go.up} run {\conn} go  \\
\glt  `Go up,  run, and leave'  (*Run upwardly and go)

\ex\label{ex:GRM-preverb-down}
\gll tùù tʃɔ́  à káálɪ̀\\
      {go.down} run {\conn} go \\
\glt  `Go down, run, and leave'  (*Run downwardly and go)

\z 
 \z
 
The verbal morphemes {\it tuu} and {\it zɪn} in (\ref{ex:GRM-preverb-up-down}) 
are not treated as preverbs, but first verbs in SVCs.  As explained at the 
beginning of  Section \ref{sec:GRM-precerv}, more criteria are required to be 
considered in order to categorize verbals  of that particular kind.


% % % 6
% % %  ́
% % % The directional particles he (‘itive’, related to the homophonous verb meaning
% % % ‘go’ (departure from
% % %  ́
% % % deictic center or indexically determined location)) and va (‘ventive’, related
% % % to the homophonous verb meaning
% % % ‘come’ (arrival at deictic center or indexically determined location)) belong
% % %to % the class of preverbs of Ewe.
% % % These are forms that mark functional categories such as aspect, modality, and
% % % voice on verbs. Preverbs differ
% % % from verbs in that they do not head VPs, do not inflect for habitual aspect,
% % %and % do not take NP or PP
% % % complements (cf. Ameka 1991, 2005a,b, Ansre 1966).
% % 
% % 
% % % The particle  {\it ja} is
% % % polyfunctional:  when it precedes a main verb it  means  either `do'   to
% % % emphasize the event or conveys an habitual reading, or as, in the present
% % %case,
% % % it links two noun phrases. The latter case is glossed in example
% % %(\ref{ex:agrE})
% % % and (\ref{ex:agrF}) as {\sc ident}. 
% % 
% % 
% % % --Dakubu
% % % I wonder whether what you call IPFV is an egressive particle? such a particle
% % % derived from 'go' is quite common.  If it is incompletive / progressive this
% % % might have to do with the tone pattern



\subsection{Verbal suffixes}
\label{sec:GRM-verb-suffix}


In Section \ref{sec:GRM-verb-word}, two suffixes were introduced: the 
perfective 
intransitive suffix and the assertive suffix. It was shown that the perfective 
intransitive suffix surfaces either as {\it -jE}, {\it -wA} or {-\O} depending 
on  the verb stem.  The assertive suffix appears  in the imperfective and 
perfective  intransitive construction if  (i) none of the other constituents in 
the clause are in focus, (ii) the clause does not include negative polarity 
items, and (iii) the clause is intransitive, that is, there is no grammatical 
object. Also,  as mentioned in Section \ref{sec:GRM-imper-clause},  the suffix 
{\it -ɪ}/{\it -i} appears in the negative imperative. 

In this section,  the incorporated object index  ({\sc
o}-clitic), the pluractional  suffix, and  other derivative suffixes whose
functions are not fully understood are introduced.


\subsubsection{Incorporated object index}
\label{sec:GRM-morph-opro}


The object index  is represented as being incorporated into the verb,  and 
together they form a phonological word (e.g.  {\it wʊ̀sá tɪ́ɛ́ń nā} < {\it 
wʊ̀sá tɪɛ-n̩ na}  `Wusa gave-{\sc 1.sg} {\sc foc}').  For that reason I refer 
to this incorporated object index as the {\sc o}-clitic. Given the constraints 
governing the appearance of the perfective intransitive suffix and the assertive 
suffix, it is obvious that the {\sc o}-clitic cannot coexist with any of them. 
Recall that the  weak subject and object  pronouns are identical (see Section 
\ref{sec:GRM-personal-pronouns}).

\begin{table}[!htb]
\centering
\caption{Incorporated object index on  CV(V) stems\label{tab:object-clitic}}

\subfloat[tɪɛ `give']{
\begin{Itabular}{ll}
 wʊ̀sá tɪ́ɛ́-ń̩ nā & `Wusa gave ME'\\
wʊ̀sá tɪ́ɛ́-ɪ́ rā & `Wusa gave YOU'\\
 wʊ̀sá  tɪ́ɛ́-ʊ́ rā &  `Wusa gave HER'\\
 wʊ̀sá tɪ́ɛ́-já rā &  `Wusa gave US'\\
 wʊ̀sá tɪ́ɛ́-má rā & `Wusa gave YOU'  \\
 wʊ̀sá tɪ́ɛ́-á rā & `Wusa gave THEM'  \\
 wʊ̀sá tɪ́ɛ́-bá rā &  `Wusa gave THEM'   \\
\end{Itabular} 
}
\quad
\subfloat[tie `cheat']{
\begin{Itabular}{ll}
 wʊ̀sá tíé-ń̩ nē & `Wusa cheated ME'\\
 wʊ̀sá tíé-í rē & `Wusa cheated YOU'\\
 wʊ̀sá tíé-ú rō &  `Wusa cheated HER'\\
 wʊ̀sá tíé-já rā &  `Wusa cheated US'\\
 wʊ̀sá tíé-má rā & `Wusa cheated YOU' \\
 wʊ̀sá tíé-á rā & `Wusa cheated THEM'\\
 wʊ̀sá tíé-bá rā &  `Wusa cheated THEM'\\
\end{Itabular} 
}
\quad
\subfloat[tie `cheat']{
\begin{Itabular}{ll}
 wʊ̀sá tíé-jé rē &  `Wusa cheated US'\\
 wʊ̀sá tíé-mé rē & `Wusa cheated YOU' \\
 wʊ̀sá tíé-é rē & `Wusa cheated THEM'\\
 wʊ̀sá tíé-bé rē &  `Wusa cheated THEM'\\
\end{Itabular} 
}
\quad
\subfloat[po `divide']{
\begin{Itabular}{ll}
 wʊ̀sá pó-jé rē &  `Wusa divided US'\\
 wʊ̀sá pó-mó rō & `Wusa divided YOU' \\
 wʊ̀sá pó-á rā & `Wusa divided THEM'\\
 wʊ̀sá pó-bé rē &  `Wusa divided THEM'\\
\end{Itabular} 
}
\end{table}

Table \ref{tab:object-clitic} shows that the {\sc atr}-harmony 
operates in the domain produced by the {\sc
o}-clitic merging with a CV or CVV stem, but may or may not affect the
plural pronouns, as Tables \ref{tab:object-clitic}(b) and 
\ref{tab:object-clitic}(c) display. The form of the focus particle is determined
by
the preceding material (i.e. the phonological word  verb+{\sc
o}-clitic) and the harmony rules introduced in
Section
\ref{sec:focus-forms}.  The irregularities in Table \ref{tab:object-clitic}(d)
are not accounted for.  I did perceive rounding throughout in conversations
(i.e.  {\it wʊ̀sá poma ra} $>$ {\it wʊ̀sá pomo ro} `Wusa divided you.{\sc 
pl}'), 
but
I was unable to get a consultant  produce it in an elicitation session. Table
\ref{tab:object-clitic}(d) should be seen as displaying various renditions,
i.e. with and without {\sc atr-}harmony or {\sc ro-}harmony.


A CVCV stem differs from a CV or CVV stem by exhibiting vowel apocope and/or 
vowel
coalescence.  Table \ref{tab:object-clitic-CVCV} provides paradigms for {\it 
kpaga} `catch' and {\it goro} `(go in) circle'. 



\begin{table}[!htb]
\centering
\caption{Incorporated object index on  CVCV stems
\label{tab:object-clitic-CVCV}}

\subfloat[kpaga `catch']{
\begin{Itabular}{ll}
 wʊ̀sá kpáɣń̩ nā & `Wusa caught ME'\\
 wʊ̀sá kpáɣɪ́ɪ́ rā & `Wusa caught YOU'\\
 wʊ̀sá kpáɣʊ́ʊ́ rā &  `Wusa caught HER'\\
 wʊ̀sá kpáɣə́já wā &  `Wusa caught US'\\
 wʊ̀sá kpáɣə́má wā & `Wusa caught YOU' \\
 wʊ̀sá kpáɣáá wā & `Wusa caught THEM'\\
 wʊ̀sá kpáɣə́bá wā &  `Wusa caught THEM'\\
\end{Itabular} 
}
\quad
\subfloat[goro `(go in) circle']{
\begin{Itabular}{ll}
wʊ̀sá górń̩ nō & `Wusa circled ME'\\
 wʊ̀sá góríí rē & `Wusa circled YOU'\\
 wʊ̀sá górúú rō &  `Wusa circled HER'\\
 wʊ̀sá górə́já wā/rā &  `Wusa circled US'\\
 wʊ̀sá górə́má wā/rā & `Wusa circled YOU'\\
 wʊ̀sá góráá wā/rā & `Wusa circled THEM'\\
 wʊ̀sá górə́bá wā/rā &  `Wusa circled THEM'\\
\end{Itabular} 
}
\end{table}

The schwas ({\it ə}) in {\it kpaɣəja} and  {\it gorəja} are perceived as 
fronted,
and the ones in {\it kpaɣəma} and {\it gorəma}  as rounded. Although this is
certainly due to the following consonant, they are so weak that they can only be
heard when they are carefully pronounced (see Section \ref{sec:PHO-weak-syll}). 
The paradigm in Table  
\ref{tab:object-clitic-CVCV}(b) can also be uttered in the plural as 
{\it górójé rē} ({\sc 1.pl}),  %
{\it górémá rā} ({\sc 2.pl}), %
{\it góráá rā} ({\sc 3.pl.-h}), and %
{\it górébá rā} ({\sc 3.pl.+h}). 
 The focus particle {\it wa} is a
variant of {\it ra}. Some consultants  agree that these forms are in free 
variation,
yet the {\it wa} form coexists only with  the plural in the paradigms elicited.
Nonetheless, such paradigm elicitations are particularly subject to
unnaturalness.\footnote{I personally believe that the alteration is
determined by some kind of sandhi, not number. As to why {\it wa} appears only 
in
the plural, a scenario may be that (i) first, I install a routine by starting
with `ME' and ending with `THEM', (ii) in the process of eliciting, the passage
from third singular to first plural triggers  a different verb shape, i.e.
CVCVV/CVCN  to CVCVCV, and (iii)  although formally identical to the verb forms
of the singular, the reason why {\it wa} follows the third plural non-human 
could
be explained by psychological habituation.}

\subsubsection{Pluractional suffixes}
\label{sec:GRM-PluralVerb}


A pluractional verb is defined as a verb which can (i) express the repetition of
an event,  (ii)   subcategorize for a plural object and/or  plural subject,
and/or  (iii)  be marked by the pluractional suffix {\it -sI}, a derivative 
suffix whose  vowel quality is always high and
front
and  {\sc atr} value determined by the stem vowel(s).\footnote{An exposition of
the
`plural verbs' in Vagla can be found in \citet{Blen03}. \citet[viii]{daku07}
calls a similar morpheme `iterative' (i.e. Gurene {\it -sɛ}).  Among the West
African
languages, it is the pluractional verbs in Hausa which have received most
attention \citep[see][]{Jose08}. [Storch forthcoming]}  According to (i) above, 
the iterativeness may
affect the interpretation of the number of participants of an event. Consider
the contrasts between the 
sentences in (\ref{ex:GRM-pv-cut}), where none of the arguments are in the
plural (i.e. contra (ii)).


\ea\label{ex:GRM-pv-cut}
  
    \ea\label{GRM-pv-cutsg}
\gll   ǹ̩  téŋé  à nàmɪ̃̀ã̀  rā  \\
       {\sc 1.sg} {cut}  {\sc art} {meat} {\sc foc}\\
\glt `I cut a piece of meat (i.e.  made a cut in the flesh or cut into two
pieces).'

\ex\label{GRM-pv-cutpl}
\gll    ǹ    téŋé-sí  à nàmɪ̃̀ã̀  rā \\
          {\sc 1.sg} {cut-{\sc pv}} {\sc art} {meat} {\sc foc}\\
\glt `I cut the meat into pieces.'

 
\z 
 \z

In  (\ref{GRM-pv-cutpl}),  the formal distinction on the verb `cut',  compared
to (\ref{GRM-pv-cutsg}),  causes  the event to be interpreted as one which
involves the repetition of the `same'  sub-event.  The word {\it namɪ̃ã} `meat'
is allowed in both the contexts of (\ref{GRM-pv-cutsg}) and
(\ref{GRM-pv-cutpl}), although one may argue that the word {\it namɪ̃ã} is
inherently
plural but grammatically singular,  and that the word is appropriate in both
contexts. Despite the fact that  `meat' has indeed a plural form, i.e. {\it 
nansa}, it is probably the mass term denotation of {\it namɪ̃ã} which 
makes (\ref{GRM-pv-cutpl}) acceptable. 


\ea\label{GRM-pv-turn}
  
    \ea\label{GRM-pv-turnsg}
\gll   ǹ̩  tʃígé  à  hɛ̀ná  rá  \\
        {\sc 1.sg} {turn} {\sc art} {bowl.\sg} {\sc foc}\\
\glt `I turn (upside down) the bowl.'

 \ex\label{GRM-pv-turnpl1}
\gll   ǹ̩  tʃígé-sí  à  hɛ̀nsá  rá   \\
         {\sc 1.sg}   {turn-{\sc pv}} {\sc art} {bowl.\pl} {\sc foc}\\
\glt `I turn (upside down) the bowls (one after the other).'


 \ex\label{GRM-pv-turnpl2}
\gll {(?)}  n̩  tʃige-si   a  hɛna  ra \\
       {}  {\sc 1.sg}    {turn-{\sc pv}} {\sc art}  {bowl.\sg}  {\sc foc}\\
\glt `I turn (upside down in a repetitional fashion) the bowl.'

\z 
 \z

In (\ref{GRM-pv-turn}), however,  the 
grammatical object of a
pluractional verb {\it tʃigesi} `turn iteratively' or `put on face
down iteratively'  must refer to individuated entities. Comparing  
(\ref{GRM-pv-turnsg}) and (\ref{GRM-pv-turnpl2}) with 
(\ref{GRM-pv-turnpl1}),   the pluractional verb cannot coexist with a singular
noun as grammatical object due to the fact that  some `turning' events cannot be
conceived as affecting the same object in a repetitive fashion. However, in
(\ref{GRM-pv-beat}) the `beating' can affect  one or several
individuals. 


\ea\label{GRM-pv-beat}
  
    \ea\label{GRM-pv-beat.sg}
\gll   ǹ̩   túgó  à bìè  rē  \\
            {\sc 1.sg}  {beat} {\sc art} {child.\sg} {\sc foc}\\
\glt `I beat the child.'

\ex\label{GRM-pv-beat.pl1}
\gll   ǹ̩    túgó-sí  à bìsé  ré   \\
         {\sc 1.sg}   {beat-{\sc pv}} {\sc art} {child.\pl} {\sc foc}\\
\glt ` I beat the children.'


\ex\label{GRM-pv-beat.pl2}
\gll    ñ̩̀    túgó-sí  à  bìè  rē   \\
          ñ̩̀  {\sc 1.sg}    {beat-{\sc pv}} {\sc art}  {child.\sg} {\sc foc} 
\\
\glt `I beat the child (more than once, over a short period of time).'


 
\z 
 \z

Whereas  (\ref{GRM-pv-beat.pl2})
has a possible interpretation, two language consultants
could not assign a meaning to (\ref{GRM-pv-catchout}) below. 




\ea\label{GRM-pv-catch}
  
    \ea\label{GRM-pv-catchsg}
\gll    ŋ̩̀  kpágá  à  zál  là  \\
         {\sc 1.sg}   {caught} {\sc art} {chicken.\sg} {\sc foc}\\
\glt `I caught a chicken.'


 \ex\label{GRM-pv-catchpl1}
\gll    ŋ̩̀    kpágá-sɪ́  à  zálɪ́ɛ́ rà  \\
       {\sc 1.sg} {caught-{\sc pv}} {\sc art} {chicken.\pl} {\sc foc}\\
\glt `I caught chickens (i.e. in repeated actions).'


 \ex\label{GRM-pv-catchpl2}
\gll   ŋ̩̀     kpágá  à  zálɪ́ɛ́ rà   \\
       {\sc 1.sg}  {caught} {\sc art} {chicken.\pl} {\sc foc}\\
\glt `I caught chickens (i.e. in one move).'


 \ex\label{GRM-pv-catchout}
\gll (?)     ŋ̩  kpaga-sɪ  a  zal  la  \\
     {}   {\sc 1.sg}  {caught-{\sc pv}} {\sc art} {chicken.\sg} {\sc foc}\\
\glt `I caught a chicken (i.e. after unsuccessful attempts until finally
succeeding with
one particular chicken).'

 
\z 
 \z


A pluractional verb usually denotes an action, but not a state. Therefore, in
(\ref{GRM-pv-catch}), the sense of {\it kpaga}$_{1}$  is related to `catch', 
and 
not
to the  possessive sense of the verbal state lexeme   {\it kpaga}$_{2}$
`have'.\footnote{Though I like to treat {\it dʊasɪ} as a counterexample.  The
pluractional verb {\it dʊasɪ} `be in a row'  may be  derived from the 
existential
predicate {\it dʊa} `be on/at/in'.  For instance, the verbs {\it tele} `lean'   
and {\it telege} `lean' are determined 
by
the number value ({\it sg.}/{\it  pl.})  of the subject.  If more examples like
these  arise, {\it pluractional} would then loose its literal 
signification.} Beside {\it /-sI/}, the suffix {\it 
/-gE/} may also turn a verbal process lexeme into a pluractional verb, e.g.   
{\it tɔtɪ} `pluck' $>$ {\it  tɔrəgɛ} `pluck iteratively' and  {\it keti} 
`break'  $>$
{\it kerigi} `break iteratively'.

\ea\label{ex:GRM-kpa-paa}
  
    \ea\label{ex:GRM-kpa}
\gll kà kpá zál háŋ̀ tà\\
go take.{\sc pl} fowl.{\sg} {\dem} let.free\\
\glt `Go and take this fowl away.'
      \ex\label{ex:GRM-paa}
\gll kà páá zálɪ́ɛ́ hámà tà\\
go take.{\sc pl} fowl.{\pl} {\dem}.{\pl}  let.free\\
\glt `Go and take these fowls away.'
 
\z 
 \z


Finally, a pluractional verb must not necessarily display the
suffixation pattern
described above. This is confirmed by the pair {\it kpa}/{\it paa} `take'  in
(\ref{ex:GRM-kpa-paa}).




\subsubsection{Possible derivational suffixes}
\label{sec:GRM-deri-suff}


\citet[37]{Daku09} and \citet[69]{Bonv88} identify some derivational suffixes 
in 
Gurene and Kasem respectively, but write that their signification is hard to 
establish.  However, their descriptions indicate that  derivational suffixes 
mainly encode aspectual distinctions.

As mentioned in Section \ref{sec:GRM-verb-syll-und-tone}, about 90\% of the
verbs are monosyllabic or bisyllabic, and  only the consonants {\it m,
t, s, n,  l} and {\it g} are found  in onset position word-medially in
trisyllabic verbs. This situation could suggest that 10\% of the verbs in the
current lexicon are the product of verbal derivation, and that the consonants
found  in onset position word-medially in trisyllabic verbs are part of
derivational suffixes. However, apart from the pluractional suffix discussed in
the previous section,  it is impossible at this stage of the research to
establish a systematic mapping between the third syllable of a trisyllabic verb
and a meaning.  


\ea
    \ea\label{ex:plur-ex}
\gll   ʊ̀ wʊ́rɪ́gɪ́ à hàɣlíbíé ré\\
{\sc 3.sg} scatter {\sc art} block.{\sc pl} {\sc foc}\\ 
\glt `He scattered the mud blocks.' (they were piled and packed)

    \ex\label{ex:}
\gll  ʊ̀ wʊ́rá à hàɣlíbíí ré\\
 {\sc 3.sg} move {\sc art} block  {\sc foc} \\
\glt `He moved a mud block.' (they are uneven, but still piled)
\z 
 \z
 
 
 
\begin{table}[!htb]
\small
\centering
\caption{Possible derivational suffixes\label{tab:GRM-der-suff}}

\begin{tabular}{lllll}
\lsptoprule

 &&&{\it -gV}&\\\midrule

wʊ̀rà {(v)}& `move, shift' & $>$ & wʊ̀rɪ̀gɪ̀ {(v)}& `scatter'\\
tàrà  {(v)}& `support' & $>$ &tàràgɛ̀ {(v)}& `pull' \\
%bɪla {(v)}& `turn repetitively' & $>$ & bɪlgɪ {(v)}& `clean' \\
brà {(v)}& `return' & $>$ & bɛ̀rɛ̀gɪ̀  {(v)}& `change direction'\\\midrule

&&&{\it -mV} &\\\midrule

ɲàgà  {(v)} & `be sour' &$>$ & ɲàgàmɪ̀  {(v)}& `ferment' \\
víl {(n)} &`well' & $>$ &vílímí {(v)} & `whirl' \\
 mɪ̀là {(v)} & `turn round' & $>$ &mɪ̀lɪ̀mɪ̀ {(v)}& `turn'\\[1ex]\midrule

&&&{\it -lV}&\\\midrule
 kàgà {(v)}& `choke'& $>$ & kàgàlɛ̀ {(v)} & `lie across' \\
 \lspbottomrule
\end{tabular}
\end{table}


The  example  provided in (\ref{ex:plur-ex}) and Table \ref{tab:GRM-der-suff} 
presents  some indications that {\it m, l} and {\it g}, i.e. CVCV\{m, l, g\}V, 
are involved in some kinds of derivation, although the glosses (and part of 
speech categories) assigned to them clearly indicate that the next step would be 
to determine their exact meaning (and category).\footnote{The verb pair {\it go} 
`round'  and {\it goro}  `(go in) circle'  is  manifestly a derivation as well, 
i.e. CV $>$ CV-rV.}


\section{Grammatical Pragmatic and Language Usage}
\label{sec:GRM-adjuncts}


\subsection{Manner deictics {\it keŋ} and {\it nɪŋ}}
\label{sec:GRM-adv-pro}
%manner deixis

Chakali has a two-term exophoric system of manner deixis \citep{koen12}. 
The expressions  {\it keŋ} and {\it nɪŋ} are treated as  two manner deictics  
({\it gl.} {\sc dxm}).  Manner is a cover term since the content dimension 
appears to cover degree and  quality as well. Consider the 
examples in (\ref{ex:GRM-dxm-}).

\ea\label{ex:GRM-dxm-}
 \ea\label{ex:GRM-dxm-}
\gll {keŋ}/{nɪŋ}  \\
{\sc dxm} \\
\glt `That's the way to do it (manner)'  

  \ex\label{ex:GRM-dxm-}
\gll {keŋ}/{nɪŋ}  \\
{\sc dxm} \\
\glt `The snake was that/this big (degree)'  

  \ex\label{ex:GRM-dxm-}
\gll {keŋ}/{nɪŋ}  \\
{\sc dxm} \\
\glt `Kala is like that (quality) [depictive gesture]'  

\z 
 \z   

The expressions  {\it keŋ} and {\it nɪŋ} are very frequent  and bring
to mind the  English  `like this/that',  that is,  an expression which 
refers to something extralinguistic yet in  the context of the utterance. 
Example (\ref{ex:GRM-adv-pro-keng-ning}) illustrates this point.


\ea\label{ex:GRM-adv-pro-keng-ning}

 \ea\label{ex:GRM-adv-pro-ning}
\gll bàáŋ ɲʊ̃̀ã̀sá káá sìì báŋ̀ nɪ̄ nɪ̏ŋ\\
{\q}  smoke  {\egr} rise {\dem} {\postp} {\dxm}\\
\glt `What smoke is rising here like this?'  
%(Python story 059)
  \ex\label{ex:GRM-adv-pro-keng}
 \gll bàáŋ káá jāā kȅŋ?\\
  {\sc q} {\sc egr} do  {\dxm}\\ 
 \glt `What is doing like that?' (Reaction to a sound coming from inside a pot)

\z 
 \z   

The meaning difference between   {\it nɪŋ}  and {\it 
keŋ} seem to be 
motivated by the way they  encode a sort of psychological saliency on a
proximal/distal dimension. This distinction needs more evidence than the one I
provide,  but consider the conversation between A and B in
(\ref{ex:GRM-adv-pro-keng-AB}). 


\ea\label{ex:GRM-adv-pro-keng-AB}

 \ea\label{ex:GRM-adv-pro-A}
\gll A: nɪ́n nā bààbá ŋmȁ\\
 {} {\dxm} {\foc} B. say\\
\glt `Is this what Baaba said?'

  \ex\label{ex:GRM-adv-pro-B}
 \gll B: ɛ̃̀ɛ̃́ɛ̃̀ kén{\T ꜜ} né ʊ̀ ŋmá\\
 {} yes {\dxm} {\foc} {\sc 3.sg} say\\
\glt `Yes, that is what he said.'
 
 
\z 
 \z   

Similarly,  the (fictional) discourse excerpt in
(\ref{ex:GRM-adv-kapok}) concerns a father (A) addressing his son (B) on the
topic of  how to ignite kapok fiber. The sentence (\ref{ex:GRM-adv-kapok-A-2})
is accompanied with a demonstration on how to strike a cutlass on a stone.


\ea\label{ex:GRM-adv-kapok}
 
 
 \ea\label{ex:GRM-adv-kapok-A-1}
\gll A: kpá kóŋ à ŋmɛ̀nà díŋ\\
{}  take kapok {\conn} ignite fire\\
\glt `Take some kapok and start a fire.'

 \ex\label{ex:GRM-adv-kapok-B}
\gll B:  ɲɪ̀nɪ̃̀ɛ̃́ bà já kà ŋmɛ̀nà\\
{} {\q} {\sc 3.pl} do {\egr} ignite\\
\glt `How does one ignite.' 

 \ex\label{ex:GRM-adv-kapok-A-2}
\gll  A: ŋmɛ̀nà nɪ́ŋ̀\\
{} ignite {\dxm}\\
\glt `Ignite like this.'

 \ex\label{ex:GRM-adv-kapok-A-3}
 \gll  A: tʃɪ́á dɪ̀ tʃɪ́ wááwá ŋmɛ̀nà kéŋ̀\\
{} tomorrow {\conn} {\cras} come.{\pfv} ignite {\dxm}\\
\glt `Tomorrow when you come, ignite like that.'
 
\z 
 \z  
 
In the context of (\ref{ex:GRM-adv-kapok}), at the farm the next day, the boy
(B) would tell a colleague: {\it ken ne ba ja ŋmɛna},  {\it lit.} like.that they
do ignite, `that is how one ignites'. 


\ea\label{ex:GRM-ning-prop-2}
 \gll nɪ́ŋ lɛ̀ɪ́ ʊ̀ʊ̀ dɪ̀à háŋ̀ já dʊ̀\\
 {\dxm} {\sc neg}  {\sc 3.sg.poss} house {\sc dem} {\sc hab} be\\
\glt  `This is not how his room used to be.'
%(Python story 078)

\z


In (\ref{ex:GRM-ning-prop-2}), {\it nɪŋ} refers to the condition of  the room,
which is not a manner  but a property of the room. 
In addition, {\it keŋ} and 
{\it nɪŋ} can function as  discourse particles, whose
meanings resemble   English `like' in some registers \citep{Muff02}. In
(\ref{ex:GRM-keng-like}), {\it keŋ} is considered superfluous since it does not
contribute to the manner of  motion or the state of the
participant.\footnote{Something identical to the translation of
(\ref{ex:GRM-keng-like}) may be heard in some varieties of  English spoken in 
Wa.
This suggest that Waali and/or Dagaare has at least  similar 
expression.} 
 
 \ea\label{ex:GRM-keng-like} 
 \gll ǹ̩ káálʊ̄ʊ̄ kéŋ̀ \\
 {\sc 1.sg} go.{\ipfv .\foc}  {\sc dxm}\\
 \glt `I am leaving like that'
\z

Also, depending on the intonation associated with it, and whether or not  the
focus
particle  is  present, {\it keŋ} and {\it nɪŋ} can function as
interjections used to convey comprehension or surprise. So a phrase like {\it 
kén nȅȅ} could be roughly translated as `Is that so?', {\it kén nè}   has a
similar function to the English  tag-question `Isn't it?', but {\it kéēèŋ} or
{\it kén né} could be translated as `yes, that is it'. 

Finally, \citet{Mcgi99} presents  {\it nyɛ} and {\it ɛɛ} (variant {\it gɛɛ}) as
demonstrative pronouns in Pasaale, which can also modify an entire clause. The
former
corresponds to `this' and the latter to `that'. At this point, it is a matter of
comparing the two languages and the terminology employed.  Nonetheless, in the
majority of the examples provided by \citet{Mcgi99}, Chakali {\it keŋ} and {\it 
nɪŋ} seem to have the same function. 


\subsection{Spatial deictics {\it bááŋ̀} and  {\it dé}}
\label{sec:GRM-deic-adv}
 

A speaker-subjective,  two-way contrast  exists to locate entities in space. 
The spatial deixis demonstrative  {\it bááŋ̀} designates the location of the 
speaker, while 
the spatial deixis demonstrative  {\it dé} designates  where the
speaker is not located. They represent what is known as the `proximal' and
`distal' 
dimensions of  place deixis. In (\ref{ex:deic-adv-prox}) and
(\ref{ex:deic-adv-dist}),  they are translated as `here' and 'there'
respectively, and glossed {\sc dxl}, standing for `locative deixis'. In these
two examples  the postposition {\it nɪ} is optional.  The  spatial deixis 
demonstrative cannot occur clause initially, as  (\ref{ex:deic-adv-prox-out})  
and  (\ref{ex:deic-adv-dist-out}) show. 


\ea\label{ex:vp}


\ea\label{ex:deic-adv-prox}
\gll wa ban (nɪ)\\
     come {\dem} {\postp} \\
\glt  `Come here'
\ex\label{ex:deic-adv-prox-out}
\textasteriskcentered baŋ wa 

 \ex\label{ex:deic-adv-dist}
\gll ʊ  dʊa de (nɪ) \\
       {\psg}  be.at  {\dem}  {\postp}\\
\glt  `He is there'

\ex\label{ex:deic-adv-dist-out}
\textasteriskcentered de ʊ  dʊa 

\z 
 \z

Notice that unlike the single demonstrative  modifier discussed in Section 
\ref{sec:GRM-demons},   {\it bááŋ̀} and   {\it dé} 
encode a proximal/distal distinction.



\subsection{Focus}
\label{sec:GRM-focus}

Since the notion of focus has been discussed separately in connection with
nominals and verbals, this section offers a basic overview of what has been
stated.  \citet[326]{Dik97} writes that   ``the focal information in a
linguistic expression is
that
information which is relatively the most important or salient in the given
communicative setting''.  In Chakali, we saw  two ways in which a
speaker can integrate focal information, and both of them put `in focus' a
constituent.\footnote{The  terminology employed in the literature is probably
the result
of  complex and still obscure phenomena. For instance, for the
post-verbal particle {\it lá} in Dagaare, \citet{Bodo97} uses the term
`factitive' and `affirmative' particle interchangeably, \citet{Daku05} uses
`(broad- and narrow-)  focus' and glosses it either as {\sc aff} or {\sc foc},
and
\citet{Saan03} uses post-verbal particle and glosses it as {\sc aff}. In-depth
accounts of focus in Grusi languages can only be found in \citet{blas90}, 
but see also \citet{Mcgi99}.
 Anne Schwarz has worked extensively on the topic in some Gur and Kwa
languages \citep{Schw10}.}   The first
encodes focal information in a particle which  always
 follows a nominal, i.e. {\it ra} and variants. Its  phonological shape is
determined by the
preceding phonological material (see Sections \ref{sec:focus-forms} and 
\ref{sec:GRM-foc-neg}). The second, which was called the assertive suffix, takes
the form of vowel features which
are suffixed onto the verb  (see Sections \ref{sec:GRM-verb-perf-intran} and 
\ref{sec:GRM-verb-suffix}). It was claimed that  the assertive suffix surfaces
only if (i) none of the other constituents in the
clause are in focus, (ii) the clause does not include negative polarity items,
and (iii) the clause is intransitive.
The second criterion (ii) is applicable to the particle {\it ra} as well: thus
focal
information can only exist in affirmative clauses, negation automatically
prevents information from being in focus.\footnote{\citet[94]{Bodo97} writes
(for
Dagaare) that
``[the factitive particle {\it lá}] is in complementary distribution with the
negative polarity particles, as one would expect of an affirming particle".}  In
 (\ref{ex:GRM-focus}),  the
examples illustrate  how the  focal information is
encoded when the object (\ref{ex:GRM-focus-obj}), the subject
(\ref{ex:GRM-focus-subj}) and the predicate  (\ref{ex:GRM-focus-pred}) are
considered the most important piece of information. 


\ea\label{ex:GRM-focus}

 \ea\label{ex:GRM-focus-obj}{\rm Focus on object: What has the man chewed?}\\
\gll   à báál tíē sɪ́gá rá\\
      {\art} man chew bean {\foc} \\
\glt `The man chewed BEANS'

\gll  kàlá tíē sɪ́gá rá\\
      K.   chew bean {\foc}\\
\glt `Kala chewed BEANS'

\ex\label{ex:GRM-focus-subj}{\rm Focus on subject: Who has chewed the beans?}\\
\gll   à báál là  tíē sɪ́gá   \\
       {\art} man {\foc} chew bean    \\
\glt `The MAN chewed beans'



\gll  kàláá tíē sɪ́gá\\
      K.   chew bean\\
\glt `KALA chewed beans'

\ex\label{ex:GRM-focus-pred}{\rm Focus on predicate: What happened?}\\
\gll    à báál tíéwóó  \\
   {\art} man chew.{\pfv .\foc}    \\
\glt `The man CHEWED'


\z 
 \z

The focus particle does not differentiate between  grammatical functions and 
appears to be optional.  Also,  the assertive suffix is quite rare 
in narratives.  \citet[94]{blas90} is the only author to my knowledge 
who identifies the presence  of  evidentiality --  hearsay, more precisely -- 
in Gur languages. According to her the morpheme {\it rɛ} in Sissala refers to 
reported or inferred information. This raise the question as to what extent the 
focus particle and the assertive suffix provide evidential information. 



\subsection{Linguistic taboos}
\label{sec:GRM-ling-taboo}

A linguistic \is{taboo} taboo is defined here as the avoidance of
certain words on certain occasions due to  misfortune associated with those
words. 
These circumstances depend on belief; they can be widespread or marginal. The
avoidance of certain words may depend on the time of the day or action carried
out. For instance, not only  is sweeping  not allowed when someone eats, but
uttering the word {\it tʃãã} `broom' is also forbidden. Also, mentioning
certain animal names is excluded as they may either be tabooed by someone
present, due to his/her animal totem and/or its meat is forbidden (Section 
Section \ref{sec:SOC-religion}),  or attract the animal's attention, i.e the 
belief that the  animal may feel it is called out. The strategy is to substitute 
a word with another. 


\ea\label{ex:GRM-taboo-synonyms}{\rm Taboo synonyms}\\

 {\it bɔ̀là} $\leftrightarrow$ {\it sèl-zèŋ́} {\rm (animal-big)},  {\rm or} 
{\it néŋ-tɪ̄ɪ̄nā} {\rm (arm|hand-owner)}  {\rm `elephant'} \\
{\it dʒɛ̀tɪ̀} $\leftrightarrow$ {\it ɲú-zéŋ-tɪ̄ɪ̄nā} {\rm 
(head-big-owner)} {\rm  `lion'}\\
{\it bʊ́ɔ̀mánɪ́ɪ́} $\leftrightarrow$ {\it ɲú-wíé-tɪ̄ɪ̄nā} {\rm 
(head-small-owner)}  {\rm  `leopard'}\\
{\it váà} $\leftrightarrow$ {\it nʊ̃̀ã̀-tɪ́ɪ́ná} {\rm (mouth-owner)}  
{\rm  `dog'}\\
{\it kɔ́ŋ} $\leftrightarrow$ {\it nɪ́ɪ́-tɪ́ɪ́ná}  {\rm 
(water-owner)}  {\rm  `cobra'} \\
{\it gbɪ̃̀ã́} $\leftrightarrow$ {\it néŋ-gál-tɪ̄ɪ̄nā} {\rm 
(arm|hand-left-owner)} {\rm  `monkey'}\\
{\it hèlé} $\leftrightarrow$ {\it mùŋ-zɪ́ŋ-tɪ̄ɪ̄nā} {\rm
(back-big-owner)} {\rm  `type of squirrel'}\\
{\it tébíŋ̀} $\leftrightarrow$ {\it bà-tʃɔ́g-ɪ́ɪ́}  {\rm
(place-spoil-{\sc nmlz})},  {\rm or} {\it 
sáàŋkárá} {\rm (Vagla loan})   {\rm  `night'}\\
{\it ɲʊ́lʊ́ŋ} $\leftrightarrow$ {\it ɲú-bɪ́rɪ́ŋ-tɪ́ɪ́ná}   {\rm
(head-full-owner)} {\rm  `blind'}\\
{\it tʃã́ã́} $\leftrightarrow$ {\it kɪ̀m-pɪ̀ɪ̀g-ɪ́ɪ̀}  {\rm
(thing-mark-{\sc nmlz})}   {\rm  `broom'}\\
{\it búmmò} $\leftrightarrow$ {\it dóŋ}  {\rm  (dirt)}  {\rm  `black'}\\
{\it dʊ̃̀ʊ̃̀wìé} $\leftrightarrow$ {\it mábíé-wāá-tèlè-púsíŋ} {\rm
(sibling-will.not-reach-meet.me)}
{\rm  `type of snake'}\\

\z


The examples in (\ref{ex:GRM-taboo-synonyms})  are called \is{taboo synonyms} 
taboo \is{synonym} synonyms; the word on the left of the arrow is the word 
avoided and the one on the right is its substitute(s).   The substitutes  are 
majoritarily complex stem nouns with a transparent descriptive meaning. Most 
of them use the stem {\it tɪɪna} `owner of', e.g. {\it neŋ-tɪɪna}, {\it lit.} 
arm|hand-owner.of,  `elephant',  the one with a big arm. 


\subsection{Ideophones\is{ideophone} and iconic strategies}
\label{sec:GRM-onoma}

 In Chakali, ideophones typically suggest the description of an abstract 
property or the manner in which an event unfolds.  The majority of ideophones 
functions like  qualifiers (Section \ref{sec:GRM-qualifier}) or adjunct 
adverbials  (Section \ref{sec:GRM-adjuncts}). Ideophones  tend to appear with a 
low tone. 
                               

\ea\label{ex:GRM-ideo}

 \ea\label{ex:GRM-ideo-dxm}
\gll  à díŋ káá dīù gàlɪ̀gàlɪ̀gàlɪ̀/pèpèpè\\
{\sc art} fire {\sc ipfv} eat  {\ideo}\\
  \glt `The fire is burning at an increasing rate.'


 \ex\label{ex:GRM-ideo-qual}
\gll à dʊ̃́ʊ̃́ síè jáá wə̀rwə̀rwə̀r\\
{\sc art} python eye {\ident} {\ideo} \\
  \glt `The python's eyes are glittery.'

   \ex\label{ex:GRM-ideo-}
\gll à dáánɔ́ŋ márá bɪ̄jʊ̄ʊ́ lìgèlìgèlìgè\\
{\sc art} tree.fruit well ripe.{\sc pfv}  {\ideo} \\
  \glt `The fruit is perfectly riped.'
  
  
     \ex\label{ex:GRM-ideo-}
\gll à sìbíé wàà márá bɪ̀ɪ̀ à dʊ́ nɪ̄ŋ wùròwùròwùrò\\
{\sc art} beans  {\sc neg} well ripe {\sc conn} be   {\sc dxm} {\ideo}\\
  \glt `The bambara beans are not well cooked, they are still hard.'
  

\z 
 \z

The translations into English in (\ref{ex:GRM-ideo}) were not 
tested for consistency across many speakers.

An onomatopoeia is a type of \is{ideophone}ideophone which not only suggests 
the concept   it expresses with sound, but imitates  the actual sound of an 
entity or event.  Examples of onomatopoeia\is{onomatopoeia} are {\it púpù} 
`motorbike', {\it tʃétʃé} `bicycle', {\it tʃɔ̀kɔ̃́ɪ̃́ tʃɔ̀kɔ̃́ɪ̃́} `sound of a 
guinea fowl',  {\it krrrr} `sound of running',  {\it pã̀ã̀} `sound of an 
eruption caused by lighting a fire',  {\it gbàgbá}  `duck',\footnote{The 
word for `duck' is probably borrowed from Waali. This bird was introduced 
recently and was hard to find in the villages visited.}   and {\it 
kpókòkpókòkpókò} `sound of knocking on a clay pot'.


 Similarly, an iconic strategy to convey an amplified meaning or the idea of
continuity is to lengthen the sound of an existing word. 


 \ea\label{ex:GRM-lenght}
   \gll  kawaa sii tarɪ keeeeeeeŋ, aka dʊa  ba dɪanʊã nɪ\\
pumpkin rise {creep} {\dxm} {\conn} {be.at} {\sc 3.sg.poss} door {\postp}\\
\glt `The pumpkin crept, crept, crept, and crept up to their door mat.'
%\hfill{Python story (line 56)}
\z

In (\ref{ex:GRM-lenght}) the manner deictics {\it keŋ} (Section
\ref{sec:GRM-adv-pro}) is stretched to simulate the extention in time of the
event, i.e. the pumpkin grew until it reached the door.\footnote{An equivalent
meaning may be expressed in some varieties of Gh. Eng.  with the 
adverbial
expression  {\it ãããã}, as in {\it Today I worked ãããã, until night 
time.}}




%busabusa in Waali, from Akan ``unatteactive, dull, repulsive''
% tʃatʃara in Waali  ``a muddy area''
% apelepele in Waali ``clear''
% ganura ``paint/marks on sacks at market''



Reduplication\is{reduplication} of one or two syllables is the general 
structural shape of
\is{ideophone}Ideophones and onomatopoeias. A large set of visual 
perception expressions can be treated as
 ideophonic expressions (Section \ref{sec:GRM-qualifier}), all of which are 
reduplicated expression.  

\ea\label{ex:BCTreduplic}{\rm Visual perception expressions  and 
non-attested stems}\\

\ea {(kɪn|a)-hɔlahɔla}	  [áhɔ̀làhɔ̀là]   *hɔla   
\ex {(kɪn|a)-ahɔhɔla}		[áhɔ̀hɔ̀là]    *hɔla   
\ex {(kɪn|a)-busabusa}	[ábùsàbùsà]	    *busa  
\ex {(kɪn|a)-adʒumodʒumo}	[ádʒùmòdʒùmò]    *dʒumo  
\ex {(kɪn|a)-bʊɔbʊɔna}	[ábʊ̀ɔ̀nàbʊ̀ɔ̀nà]		    *bʊɔna  
\ex {(kɪn|a)-ʔileʔile}	[áʔìlèʔìlè]	    *ʔile 

\z
\z


If we accept that reduplication is a morphological process in which the root or 
stem is repeated (fully or partially), then it is questionable whether one can 
treat most of the naming data as reduplication. It is obvious from Table 
(\ref{ex:BCTreduplic}) that there is a `form-doubling' on the surface, yet such 
expressions  are not made out of attested stems.  


\subsection{Interjection and formulaic language}
\label{sec:GRM-greet}


This section introduces some pieces of formulaic language, which is defined as
conventionalized and idiomatic words or phrases. It usually include greetings,
idioms, proverbs,  etc. \citep{Wray05}. First, common
interjections are introduced in Table \ref{tab:GRM-interj},\footnote{The 
etymology
of {\it ʔàmé} has not been confirmed and {\it gáfrà} is ultimately Hausa. 
The word
{\it ʃɪ́ã̀ã̀} is equivalent to the
function  associated with the action of {\it tʃuuse} in Chakali ({\it tʃʊʊrɪ} in
Dagaare, {\it tʃʊʊhɛ} in Waali, `puf' or `paf'  in Gh.  Eng.  ($<$
English `pout')), which
is a
fricative sound produced by a non-pulmonic, velarized ingressive airstream
mechanism, articulated with the lower lip and the upper front teeth while the
lips are protruded.} then some greetings and idioms are presented. Needless to 
say, since
they are conventionalized and idiomatic, the translation formulaic language is
always a rough  approximation.



\begin{table}[!htb]
\small
\centering
\caption{Selected interjections \label{tab:GRM-interj}}

\begin{tabular}{>{\itshape}lp{8cm}}
\lsptoprule
{\rm Interjection} & Gloss\\[1ex] \midrule
ʔàɪ́  &  no  \\
ʔɛ̃ɛ̃    &yes  \\
gáfrà  &  excuse  \\
tóù  &  o.k.  ({\it from}  Hausa)\\
ʔàmé   & so be it  ({\it etym.}  Amen)  \\
ʔóí  &  indicates surprise\\
fíó  & totally not  \\
ʔánsà  &  1) welcome, 2) thank you    ({\it from} Gonja)\\
ʔĩ́ĩ̀ĩ́  &  
expressing disappreciation of an action
carried out by someone else\\
ʔàwó  &  reply to greetings, a sign of appraisal of the interlocutor's
concerns ({\it from} Gonja)\\
 ʔábà & indicates new and unexpected information\\
sʊ́ɛ̀ɛ̏ & insult when uttered after someone's remark or simply intended at
someone\\ 
\lspbottomrule
\end{tabular}
\end{table}


\subsubsection{Greetings}
\label{sec:GRM-greet}

Crucial and obligatory prior to any communicative exchange, greetings trigger
both attention and respect. When meeting with elders, one should  squat  or bend
forward hands-on-knees  while greeting. Clan names can be used in greetings,
e.g. {\it ɪ́tʃà} `respect to you and to your clan'. In Table
\ref{tab:greetings},  I provide typical greeting lines with some responses.




\begin{table}[!htb]
\small
\centering
\caption{Greetings\label{tab:greetings}}

\begin{tabular}{l>{\itshape}lp{7cm}}
\lsptoprule
Time & {\rm Speaker A} & Following by either speaker A or B\\ \midrule

Morning  & ánsùmōō  & \textit{ɪ̀ sìwȍȍ} `You stood?', \textit{ɪ̄ dɪ̀ 
tʃʊ́àwʊ̏ʊ̏}   `And your lying?', \textit{ɪ̀ bàtʃʊ̀àlɪ́ɪ̀ wīrȍȍ }  `You 
sleeping place was good?' \\[1ex]

Afternoon   & ántèrēē & \textit{ɪ́ wɪ́sɪ́ tèlȅȅ}   `Has the sun reached 
you?' \textit{ ɪ́ dɪ̄à} `And your house?'  \textit{ɪ̄ bìsé mūŋ} `And all 
your children?'\\[1ex]
  

Evening & ɪ́ dʊ̀ànāā &  \textit{ ɪ́  dʊ̄ɔ̄n tèlȅȅ}  `Your evening 
has reached', 
\textit{ɪ́ kùó} `And your farm?'\\
\lspbottomrule
\end{tabular} 
\end{table}



The second singular pronoun {\it ɪ} is replaced by the  second singular plural 
{\it ma}, i.e.  {\it ánsùmōō} $\leftrightarrow$ {\it māānsùmōō},
when there is more than one adressee or when there is  a single person but the
greetings
are intended to the entire house/family: thus  the distinction {\it ɪ}/{\it ma}
does
not correspond to the politeness function of French {\it tu}/{\it vous}. Chakali
 morning and afternoon greetings resemble those of Waali and other languages of
the area.
The response to various greetings such as {\it ɪ́ dɪ̄à} `and your house?',  
{\it ʔánsà} `welcome, thanks' and many others is the multifunctional 
expression {\it ʔàwó},  which is, among other things, a sign of appraisal of 
the
interlocutor's
concerns. The same expression is found in Gonja, but it may have different
functions. I was told that the more extensive the greetings, the more
respect one shows the addressee.  For instance, the elders do not
appreciate the tendency of
the youths to morning-greet as {\it ã̄sūmō}, but prefer something like {\it 
áánsùùmōōō}. 

Other expressions often used are

{\it tʃɔ̄pɪ̄sɪ́ ālɪ̀ɛ̀}  {\it lit.} morning two `two days'
{\it bámùŋ kɔ́rɛ́ɪ́}  {\it lit.}  `all.\textsc{+hum} extent? (Hausa)'
{\it ànɪ́ mà wʊ̀zʊ́ʊ́rɪ́ tɪ̀ŋ} {\it lit.} and your that day (after the death 
of someone, refereing to the funeral day, after not seeing one another)
 

\subsubsection{Idioms}
\label{sec:GRM-idiom}

An idiom is a  composite expression which does not convey the literal  meaning 
of the composition  of its parts. Common among many African languages is a 
strategy by which  abstract nominals are expressed in idiomatic compounds. 
These compounds are made of stems whose meanings are disassociated from their 
ordinary usage.


Some examples have already been provided in section \ref{sec:GRM-qualifier}. In 
Chakali, words identifying mental states and habits/behaviors are often 
idiomatic, e.g. {\S síínʊ̀màtɪ́ɪ́nà} ({síí-nʊ̀mà-tɪ́ɪ́nà}, {\it lit.} 
eye-hot-owner), `wild' or {\S nʊ̀ã̀pʊ̀mmá} ({nʊã-pʊmma}, {\it lit.} 
mouth-white), `unreserved'. Even though the expression {\S síínʊ̀màtɪ́ɪ́nà} 
is made out of three lexical roots, it is a `sealed' expression and is 
associated with the manner in which a person behaves, i.e. a wild person. The 
sequence {\S jaa nʊ̃ã dɪgɪmaŋa} in (\ref{ex:GRM-idiom-mouth}), {\it lit.} 
do-mouth-one,  is also treated as an idiomatic expression.

\ea\label{ex:GRM-idiom-mouth}
   \gll   bà jáá nʊ̃̀ã̀ dɪ́gɪ́máŋá à sùmmè dɔ́ŋà\\
{\sc 3.pl} do mouth one {\conn} help {\recp} \\
\glt `They should agree and help each other.'

\z

Needless to say, it is often difficult to  distinguish between an idiomatic
expression and  an expression in which only one of the  components is use in a
 non-literal sense. [to do {\it di kaɲɪtɪ}]
 
 
% % eat-patience
% % However, it is often difficult to  distinguish between an idiomatic
% % expression from an expression in which one of its component is use in a
% % non-literal sense. For instance, one could treat {\S di kaɲɪtɪ},  {\it lit.}
% % eat-patience,  as an idiom, but the verb {\S di} 
% % Other idiomatic expressions are 
% % {\S}, {\it lit.} , `'
% % {\S}, {\it lit.} , `'
% % {\S}, {\it lit.} , `'
% % {\S}, {\it lit.} , `'






\subsection{Clicks}
\label{sec:GRM-greet}

\citet[151]{Nade89} writes that clicks\footnote{A click may be roughly defined 
as  the release of a pocket of air enclosed between two points of contact in the 
mouth. The air is rarefied by a sucking action of the tongue 
\cite[see][]{Lade93}.}  may be  heard in the Gur-speaking area to  mean an 
affirmative `yes', or `I'm listening'.  This also occurs in the villages where I 
stayed, but I noticed that one click usually means `yes', `I understand' or `I 
agree', whereas two clicks mean the opposite. The click is  palatal and produced 
with the lips closed.




