\chapter{Methods}

 The project places itself under the general field of descriptive linguistics.
 However, one of the main foci being documentation, the research methodology
 puts strong emphasis on text, audio and video data gathered in the field. The
 information is organized with modern tools used in the field of anthropology
 and linguistics and will be effectively archived and maintained. Any
 complementary work in the language area will definitely benefit from the
 attention put on the quality and the organization of the raw data (Moran06).

It is not until recently that an international effort has put focus on endangered languages. These efforts have common and different focus. One common effort is to acknowledge that language is only an oral medium in many endangered languages. Therefore when a language dissapear one cannot dig to expect to find vestige of a culture or a society: it is simply gone forever. If the goal is to preservelanguages and the information it contains, cultural and lingusitics, it is obvious that one should gather as much data as possible. The data can be selected at random, but this makes any future analysis almost impossible, similar to watching a Chinese movie without subtitle (i.e. when one does not understand Chinese).


The efforts in preserving endangered languages can have different focus. Suppose the many of the speakers themselves aren't interested in using their native languages anymore, the goal is therefore only to gather the data and let the speakers decide what they really want. There are no non-native that can save a language. It is the interest and motivation of the native that create a revitalisation. The linguist can help with his academic training, moral responsability, economical possibilities and others factors, but he or she cannot alone save a language


%Another issue is the the relative number of speakers.









