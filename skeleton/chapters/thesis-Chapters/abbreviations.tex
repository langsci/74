%fix small cap vs italic

\chapter*{Abbreviations}
\label{sec-ABB}


\thispagestyle{plain}

 \begin{xtabular}{ll}


{\sc a} & subject of transitive clause\\
$A$ & agent(ive) \\ 
{\sc abl} & ability (modality)\\
 {\sc abst} & abstract (semantic feature)\\
 {\sc advl} & adverb locative \\
 {\sc advt} & adverb time \\
{\sc advm} &  adverb manner\\
{\sc art} &  article\\


CB  & Clever Boy story\\
{\sc clf} & classifier\\
cli & ISO 639-3 code for Chakali\\
 {\sc conc} & concrete, animate, non-human (semantic feature)\\
 {\sc cond} & conditional particle\\
 {\sc conn} &  connective\\
CPS & Containment Picture Series\\
 {\sc cras} & crastinal tense (futur tomorrow) \\

 {\sc dem} & demonstrative \\
 {\sc dist} & distal \\%used at all??
 {\sc distr} & distributive \\
 {\sc dub} & dubitative \\%used at all??
%{\sc dwn} & downward (as in path)\\

{\sc e} & extended argument\\
{\sc egr} & egressive particle\\
Eng. & English\\
{\sc etym} & etymology\\
EVC & extended verb complex\\
{\sc exst} & existential verb\\


 {\sc foc} & focus \\
{\sc fut} &  futur\\


{\sc g}{\it a} & non-human gender  \\
{\sc g}{\it b} &  human gender\\
{\it gl} & glossed as\\

{\sc hab} & habitual \\ %head
{\sc head} & head of phrase \\ %head
 {\sc hest} & hesternal tense (past yesterday) \\
 {\sc hum} & human   (semantic feature)\\

%{\sc icpt} & inceptive\\ %encode intial point of an event
{\sc ident} & identificational verb\\ %encode intial point of an event
{\sc imps} & impersonal\\ %encode intial point of an event
{\sc ingr} & ingressive particle\\
{\it interj} & interjection\\ %like gafra, greetings,...
{\sc ipfv} &  imperfective aspect\\%ongoing, habitual, repeated,
%or generallycontaining internal structure.
%{\sc irr} &  irrealis aspect\\%ongoing, habitual, repeated,


LB  & Law Breaker story\\
{\it lit.} & literal meaning\\

 {\sc mod} &  modality \\

{\sc num} &  numeral\\ %numeral

{\sc o} or {\sc obj} & object of transtive clause\\
{\sc obl} & oblique argument\\
 {\sc ono} & onomatopoeia \\


%\thispagestyle{empty}

\thispagestyle{plain}

{\sc p} or  {\sc pred} & predicate\\
%{\sc path} & path\\
{\sc pfv} & perfective aspect \\%perfective aspect, which marks an action as
%complete
{\sc pl} or {\it pl} &  plural \\
PoS & Part of Speech\\
 {\sc postp} & postposition \\
{\sc pro} & pronoun \\ %head
 {\sc prop} & property  \\ %property
 {\it propn} & proper noun  \\ %property
 {\sc prox} & proximal \\
{\sc psed} & possessed\\ 
{\sc psor} & possessor\\
PSPV&  Picture Series for Positional Verbs \\
{\sc pst} & past  \\

 {\it pv} & preverb particle \\
 {\sc pv} &Pluractional verb \\



{\sc q} & question word, phrase or intonation\\
{\sc qual} &  qualifier\\
{\sc quant} &  quantifier\\

\thispagestyle{plain}

$R$ & recipient \\ 
 {\sc recp} & reciprocal \\
 {\sc reln} & relational noun \\
%{\sc rl} &  realis aspect\\%ongoing, habitual, repeated,

{\sc s} or {\sc subj}& subject of intransitive clause\\
{\sc sg} or {\it sg}  & singular \\  
SPS & Support  Picture Series\\
{\sc st} &strong pronoun\\


{\sc tam} & tense, aspect and mood \\
$T$ & theme \\  
{\sc trm} & topological relation marker \\
TRPS& Topological Relations Picture Series\\
t.z. & staple food. From Hausa {\it  tuo zaafi} (see {\S kʊʊ} in lexicon)\\

%{\sc up} & upward (as in path)\\
{ultm.} & ultimately \\

{\it v} & verb\\

%{\sc w} & weak pronoun\\
{\sc wk}  &weak pronoun\\

 \textasteriskcentered  & ungrammatical expression\\ 
 \textasteriskcentered   & Proto-form\\
x́ & high tone\\
x̀ & low tone\\
-  or $]$& morpheme boundary\\
$[    ]$ & phonetic representation\\
$[  ]_{X}$ &   structure of type X\\
\#  or $]_{wb}$ &  word boundary\\
 \#\# &utterance-final boundary\\
$]_{\sigma}$ &  syllable  boundary\\
X|Y & either X or Y\\
(Y) & optional Y\\
(Y) & covert Y\\
,  & pause\\
%=& clitic boundary\\
< x & from root x\\
%\thispagestyle{empty}

\end{xtabular}
\thispagestyle{plain}
%\thispagestyle{plain}
% 
% The dash symbol (-) is
% used to identify the edges of morphological units (e.g. kæt-s `cats'),  while 
% the hash symbol (\#) separates word units. \footnote{In discussing the
% morphology the symbol $]$ identifies an
% affix/stem/root boundary,  $]_{wb}$  a word boundary and  $]_{\sigma}$  a
% syllable  boundary. In the rule format,   the material following / and
%preceding
% \_ represents the context. The symbol C* indicates a condition where any
%number
% of consonant (usually zero, one or two) can occur. A prose translation is
%given
% under each rule.}  