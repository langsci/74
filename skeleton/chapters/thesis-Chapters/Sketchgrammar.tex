\chapter{Grammar Outline}
\label{sec:gramsketch}


\section{Introduction}
\label{sec:Introduction}


This chapter  provides a broad outline of the grammar and introduces those
aspects needed to understand the subsequent chapters. Further, it acts as a
preliminary grammar of the language, which is and will always be essential for
future description and analysis since it sets forth claims to be confirmed,
rejected, challenged or improved.  Since Chakali  has not yet been  documented,
this chapter may be seen as a step towards an accurate description of the 
grammar. The chapter is divided as follows: First, the common clause structure,
the main elements of syntax and clause coordination and subordination are
presented. Then, the nominal syntax and morphology are introduced,  followed by
the verbal syntax and morphology. Finally, a few peripheral and miscellaneous
linguistic phenomena are examined.  The work is descriptive and employs theory
grounded in traditional grammar but  influenced by  recent work in linguistic
typology. When necessary, the relevant theoretical assumptions are introduced
and the relevant literature provided. Recall that the full list of glossing tags
is available on page \pageref{sec-ABB}.

% Next, we look at
% function words found outside nominal and verbal phrases. 

%consider getting rid of Mcgill and Bonvini sentence

% More `Grusi'-oriented,   the work of \cite{Bonv88} and
% \cite{Mcgi99}
%   have influenced the way we studied the language.
% A publication similar to 
% \cite{Nade98} on Vagla is something we would like to achieve for Chakali in a
% near future.


\section{Clause structure}
\label{sec:GRM-nom}

A  clause which can stand as a complete utterance is
an independent clause. When a grammatically correct clause cannot stand on its
own, it is dependent on  a main clause.  A clause may convey three
sorts of speech act: a statement, a question or a command. The latter two are
encoded in interrogative clauses (section \ref{sec:GRM-interr-clause}) and
imperative clauses  (section \ref{sec:GRM-imper-clause}) respectively.  
{\it Constructions} are treated as clause-types; constructions are persistent
formal and
semantic frames which are conventionalized and display both compositional and
non-compositional characteristics.\footnote{Constructions are also practical
descriptive
and typological  tools. A good example is the system of construction labels
devised in \cite{Hell10} for Gã and Norwegian. For the framework of
Construction Grammar, see \cite{Fill88} for one of the original work and
\cite{Crof01} for a recent approach. } In this section  the components
of
the
common independent  clauses and the major constructions encountered are
presented.  In
section \ref{GRM-clause-coord-subord},  clause coordination and subordination 
are 
introduced. 


\subsection{Declarative clause}
\label{sec:GRM-interr-clause}

Statements are expressed by a series of declarative clause types. The structure
of most common clauses consists of  a simple predicate, one or two arguments and
an optional adjunct. This structure is represented in
(\ref{ex:GRM-clause-frame})


\begin{exe}
\ex\label{ex:GRM-clause-frame}
 {\sc s|a}  $+$ {\sc p} $\pm$ {\sc o} $\pm$ {\sc adj} 
\end{exe}

where the symbol $+$  requires for the presence of the element preceding
and following it,  whereas  $\pm$ means that the term following it may be
optional.
The
symbol {\sc s} stands for the subject of an intransitive clause,  {\sc a} 
for the subject of a transitive clause, {\sc p}  for the predicate,  {\sc
o}  for the object of a transitive clause, and finally {\sc adj}
stands for an adjunct to a clause. The main instantiations  of this common
structure are shown in  (\ref{ex:GRM-cl-fr-inst}).\footnote{The way I represent
the components of a clause is inspired by \citet[31]{Bonv88}.}


\begin{exe}
\ex\label{ex:GRM-cl-fr-inst}
\begin{xlist}
\ex\label{ex:GRM-cl-fr-inst-s-p}
 {\sc s}  $+$ {\sc p} 
\ex\label{ex:GRM-cl-fr-inst-s-p-o}
 {\sc a}  $+$ {\sc p} $+$ {\sc o}
\ex\label{ex:GRM-cl-fr-inst-s-p-adj}
 {\sc s}  $+$ {\sc p}  $+$ {\sc adj} 
 \ex\label{ex:GRM-cl-fr-inst-s-p-o-adj}
 {\sc a}  $+$ {\sc p} $+$ {\sc o} $+$ {\sc adj} 

\end{xlist}
\end{exe}


The predicate ({\sc p})  is represented by a verbal syntactic constituent ({\it
v}) whereas  the arguments ({\sc s, a, o}) are represented by nominal syntactic
constituents   ({\it n}).  The adjunct constituent  ({\sc adj}) may consist of 
words or phrases referring to time, location, manner of action, etc.  (see
section \ref{sec:GRM-adjuncts} on adjunct types).  An
argument may be seen as core or peripheral.  The core
argument of an intransitive clause is realized in the subject position ({\sc
s}), which
precedes the predicate. The core arguments of a transitive clause are realized
in the subject ({\sc a}) and object ({\sc o}), the former preceding and the
latter following the predicate in their canonical positions. These
characteristics are illustrated in 
(\ref{ex:GRM-core-S-A-O}).




%{ {\sc s}  $+$ {\sc p} }\\

%{ {\sc a}  $+$ {\sc p}  $+$  {\sc o}}\\




 \begin{minipage}[h]{12cm}
\begin{exe}

\ex\label{ex:GRM-core-S-A-O}

\begin{xlist}
\begin{multicols}{2}

\ex\label{ex:GRM-core-S-O}{ {\sc s}  $+$ {\sc p} }
\glll afia dijoo\\
 {\sc s}  {\sc p}\\
{\it n} {\it v}\\
`Afia ate.'


\ex\label{ex:GRM-core-S-O}{ {\sc s}  $+$ {\sc p} $+$  {\sc adj}}
\glll afia dijoo kɪŋkaŋ\\
 {\sc s}  {\sc p} {\sc adj} \\
{\it n} {\it v}  \\
`Afia ate a lot.'


\ex\label{ex:GRM-core-A-O}{ {\sc a}  $+$ {\sc p}  $+$  {\sc o}}
\glll afia di sɪɪmaa  \\
 {\sc a}  {\sc p}  {\sc o}\\
{\it n} {\it v} {\it n}\\
`Afia ate food.'


\ex\label{ex:GRM-core-A-O}{ {\sc a}  $+$ {\sc p}  $+$  {\sc o} $+$  {\sc adj}}
\glll afia di sɪɪmaa kɪŋkaŋ  \\
 {\sc a}  {\sc p}  {\sc o} {\sc adj}\\
{\it n} {\it v} {\it n}   \\
`Afia ate food a lot.'


\end{multicols}
\end{xlist}
\end{exe}
 \end{minipage}
\vspace*{15pt}



% Nominal
% and verbal syntactic constituents are discussed in
% section \ref{sec:GRM-nom} and \ref{sec:GRM-verbals} respectively, whereas
% adjuncts are presented in section
% \ref{sec:GRM-adverbs}.

The grammatical relations are primarily determined by
constituent order. Thus, the subject and object functions are not
morphologically
marked,  except that the subject pronouns in {\sc s} and {\sc a} positions  can 
have  strong or  weak forms (see section \ref{sec:GRM-personal-pronouns}). This
is extraneous to the marking of grammatical functions but pertinent to the
emphasis put on  an  event's participant. A peripheral argument  consists of a
constituent foreign to the core predication, that is, an argument which is not
part of the core participant(s) typically attributed to a predicate.  As
peripheral argument,  an  adjunct  ({\sc adj}) may be realized by a single word
or a complex syntactic constituent. Reference to space, manner and time are the
main denotations of peripheral arguments.  Adjuncts are optional to the main
predication and can be added to both intransitive and transitive clauses, as
shown in  (\ref{ex:GRM-clause-peripheral}), and (\ref{ex:GRM-core-S-O}) and 
(\ref{ex:GRM-core-A-O}) above (see
sections \ref{sec:GRM-obl-phrase} and \ref{sec:SPA-postp} for a discussion on
the postposition).


\begin{exe}
\ex\label{ex:GRM-clause-peripheral}
\begin{xlist}

\ex\label{ex:vp26.12.}{\it Manner adverb in intransitive clause}
\gll ʊ ɲʊ̃ã  laɣalaɣa nɪ  \\
      {\psg} drink {\advm} {\postp}    \\
\glt  `He drinks quickly.' 

\ex\label{ex:vp26.13.}{\it  Manner adverb in transitive clause}
\gll ʊ ɲʊ̃ã a nɪɪ  laɣalaɣa nɪ \\
      {\psg} drink {\art} water  {\advm} {\postp}     \\
\glt  `He drinks the water quickly.' 


\end{xlist}
\end{exe}


Adjuncts are usually found following the core constituent(s), but may also be
found at the beginning of a clause. As
shown in (\ref{ex:GRM-pre-adj}), 
reference to time may be found at the beginning of a clause.

%to time (\advt), location (\advl) or manner (\advm).

\begin{exe}
\ex\label{ex:GRM-pre-adj}{{\sc adj} $+$ {\sc s}  $+$ {\sc p}  $+$  {\sc o}  }
\glll  {[tʃʊ̀ɔ̀sá  pɪ̀sɪ̀]}   {à    bìpɔ̀lɪ́ɪ̀}  kpá {ʊ̀ páŕ}\\ 
 {\sc adj}  {\sc s}  {\sc p} {\sc o}\\
{morning   scatter}   {{\art} young.man} take {{3\sg.\poss} hoe}\\

 `The following day the young man took his hoe along...' (CB 005)
\end{exe}

%Discuss adjunct type here

A variation of the prototype  clause in (\ref{ex:GRM-clause-frame}) is a
clause containing an additional core argument.  \citet[116]{Dixo10b} calls  a
clause which contains an
additional core argument, that is,  an extended argument (i.e. {\sc e}), an
{\it extended} (intransitive or transitive) clause. The
difference between an adjunct and an additional core argument is not a clear-cut
one;   still,   the locative phrase in (\ref{ex:GRM-add-arg-e}) is treated as
  an additional core argument of the predicate {\S bile} `put'. In section
\ref{sec:GRM-obl-phrase}, I call a clause constituent whose semantics is
characterised by an  affected or effected object, although realized in a
postpositional phrase, an oblique object phrase. Thus, the extended argument
in (\ref{ex:GRM-add-arg-e}) should be treated as an oblique object. 


\begin{exe}
\ex\label{ex:GRM-add-arg-e}{{\sc a} $+$ {\sc p}  $+$  {\sc o} $+$   {\sc e}}

\glll ŋmɛ́ŋtɛ́l {sìì    à    bìlè}  {ʊ̀  kùó}  {tìwìzéŋ nʊ̀ã̀  nɪ̄}\\
{\sc a} {\sc p}   {\sc o}   {\sc e}\\
spider  	{raise.up   {\conn}    put}  	{{3\sg.\poss} 	farm } 
{road.large     {\reln}  {\postp}}\\

\glt  `Spider went establish his farm by a main road.' (LB 003)

\end{exe}

For the remaining, a ditransitive clause consists of a transitive clause with an
additional core argument.  In Chakali, the verb {\S tɪɛ} `give', a predicate 
that conceptually implies both a Recipient (R)  and a Theme (T) and is typically
associated with ditransitive clauses, restricts its (right-) adjacent argument
in object position as  beneficiary of the situation. The thing transfered
(i.e. Theme) can never follow the verb if the beneficiary of the transfer
(Recipient) is realised. This is shown in (\ref{ex:GRM-arg-e-ditrans}).

\begin{exe}

\ex\label{ex:GRM-arg-e-ditrans}
\begin{xlist}
 \ex\label{ex:GRM-arg-e-ditrans-ben-the-1}
\glll Kala tɪɛ Afia {a lɔɔlɪ} \\
{\sc a} {\sc p} {\sc o}_{R} {\sc e}_{T}\\
K. give A.  {{\art} car}\\

\glt  `Kala gave Afia the car.' 

 \ex\label{ex:GRM-arg-e-ditrans-ben-the-2}
\glll  Kala tɪɛ ʊ {a lɔɔlɪ} \\
{\sc a} {\sc p} {\sc o}_{R} {\sc e}_{T}\\
K. give {\sc 3.sg}  {{\art} car}\\

  `Kala gave her  the car.' 

 \ex\label{ex:GRM-arg-e-ditrans-the-ben-1}
 *Kala tɪɛ a lɔɔlɪ Afia 
 \ex\label{ex:GRM-arg-e-ditrans-the-ben-2}
*Kala tɪɛ ʊ Afia 
\end{xlist}
\end{exe}

The assumption is that the verb {\S tɪɛ} `give'  is transitive and its
extended argument is always the tranfered entity (i.e.
Theme) in a ditransitive clause. This is supported by the extensive use of the 
{\it manipulative serial verb construction} (see section
\ref{sec:GRM-multi-verb-clause}), used as an alternative strategy,  in order to
express transfer of
possession  and information.



\begin{exe}
\ex\label{ex:GRM-m-svc-give}
\glll  Kala kpa  {a lɔɔrɪ / ʊ} tɪɛ Afia  \\
{\sc a} {\sc p} {\sc o}_{T} {\sc p}  {\sc o}_{R}\\
K. take  {{\art} car / 3\sg} give A.\\

\glt  `Kala gave  the car/it to Afia.' ({\it lit.} Kala take the car/it give
Afia.)
\end{exe}

The extended argument in sentence (\ref{ex:GRM-arg-e-ditrans-ben-the-1})  and
(\ref{ex:GRM-arg-e-ditrans-ben-the-2})  above  is the Theme argument of the verb
{\S kpa} `take'   in a manipulative serial verb construction  in
(\ref{ex:GRM-m-svc-give}).   Ditransitive clauses are
very rare in the text corpus despite their grammaticality.  Multi-verb
clauses, which are discussed in section \ref{sec:GRM-multi-verb-clause},
may offer  better strategies to arrange arguments and predicates than
ditransitive clauses. 

The following subsections present various clause types and
constructions which are based on the declarative clause structure introduced
above.  

%whats a text corpus, mentioned in method





\subsubsection{Identificational clause}
\label{sec:GRM-ident-cl}


An identificational clause can express generic and ordinary categorizations, or
assert the identity  of two expressions. Generic categorization involves the 
classification of a subset to a set (e.g. Farmers are hard-working),
whereas an ordinary categorization holds between a specific entity and a generic
set  (e.g.  Wusa is a farmer). The clause can assert the identity of the
referents of two specific entities, a clause also known as equative or identity
(e.g. Wusa is the farmer). The examples in
(\ref{ex:GRM-ident-cl}) illustrate the
distinctions. 


\begin{exe}
\ex\label{ex:GRM-ident-cl}
\begin{xlist}


\ex\label{ex:GRM-ident-gen-cat}{\it Generic categorization}
\gll
 bɔla jaa kɔsasel le\\
 elephant {\ident}  bush.animal {\foc}\\
\glt `The/An elephant is a bush animal.'

\ex\label{ex:GRM-ident-ord-cat}{\it Ordinary categorization}
\gll
wʊ̀sá jáá pàpàtà rā\\
W. {\ident} farmer {\foc}\\
\glt `Wusa is a farmer.'

\ex\label{ex:GRM-ident-tk-id}{\it Identity}
\gll
wʊ̀sá jáá à tɔ́ɔ̀tɪ̀ɪ̀ná\\
W. {\ident} {\art} landlord\\
\glt `Wusa is the landlord.'

\gll
wʊsa jaa a baal tɪŋ ka saŋɛ̃ɛ̃ keŋ \\
W. {\ident} {\art} man {\art} {\egr} sit.{\pfv} {\advm}\\
\glt `Wusa is the man sitting like this.'

\gll
a baal tɪŋ ka saŋɛ̃ɛ̃ keŋ  jaa wʊsa  \\
 {\art} man {\art} {\egr} sit.{\pfv} {\advm} {\ident}   W. \\
\glt `The man sitting like this is Wusa.'


%\ex\label{ex:GRM-ident-}{\it }

\end{xlist}
\end{exe}


The verb {\S jaa}  ({\it gl.} {\ident})
always  occurs between two nominal  expressions,  and, as shown in the 
last two examples in (\ref{ex:GRM-ident-tk-id}),  their order  does not matter,
except for the generic categorization where the order is always [{\it hyponym}
{\S
jaa} {\it hypernym}].  So,  {\S papata ra jaa  wʊsa} `farmer {\sc foc} is Wusa' 
and {\S a
tɔɔtɪɪna  jaa  wʊsa} `landlord {\sc foc} is Wusa'  are  as acceptable as in the
order given
in (\ref{ex:GRM-ident-ord-cat}) and the first example
 in (\ref{ex:GRM-ident-tk-id}).    The verb {\S
jaa} is one of the two `copular'
verbs in the language. The other is {\S dʊa}, which is used in existential
clauses.

%nin na 
%re keng

\subsubsection{Existential clause} 
\label{sec:GRM-loc-cl}

One type of existential clause is the  locative construction, which is
described in detail in chapter \ref{sec:SPA-chap}. Its
two main characteristics are the obligatory presence of the postposition {\S
nɪ},  which signals that the phrase contains the conceptual ground, and the
presence of a locative predicate or the general existential predicate {\S
dʊ̀à}. An example is provided in (\ref{ex:GRM-loc-cl}).

\begin{exe}
\ex\label{ex:GRM-loc-cl}{\it Locative construction}
\gll a baal dʊa a dɪa nɪ\\
{\art}  man be.at {\art} house {\postp}\\
\glt  `The man is at/in the house.'
\end{exe}

The existential predicate {\S dʊa} is glossed `be at', but it is not the case
that it is only used in spatial description. For instance,  adhering to a
religion may be expressed using the existential predicate {\S dʊa} and the
postposition {\S nɪ}, e.g.  {\S ʊ dʊa jarɪɪ nɪ} `he/she is a Muslim', even
though no space reference is involved in such an utterance. 

An existential clause is also used in order to express that something is at
hand, accessible or obtainable. The clause in (\ref{ex:GRM-avail-cl}) is called
here 
the availability construction. It slightly differs from the
locative
construction in (\ref{ex:GRM-no-avail-cl}) because of  the absence of the
postposition
{\S nɪ}.

\begin{exe}
\ex\label{ex:GRM-avail-vs-loc}
\begin{xlist}
\ex\label{ex:GRM-avail-cl}{\it Availability construction}
\gll a molebii dʊa de\\
{\art}  money be.at {\advl} \\
\glt  `There is money (available).'

\ex\label{ex:GRM-no-avail-cl}
 a molebii dʊa de nɪ\\
`The money is there.'
\end{xlist}
\end{exe}


Another use is the attribution of a property ascribed to a participant. The
example in (\ref{ex:GRM-loc-propascr}) reads literally `a sickness is at Wojo', 
i.e. a person named Wojo is sick.  In addition to the clause presented in
(\ref{ex:GRM-loc-propascr}), ascribed property may also be conveyed in a
possessive clause (see section \ref{sec:GRM-poss-cl}). 


\begin{exe}
\ex\label{ex:GRM-loc-propascr}
\gll garaga dʊa wojo nɪ\\
sickness be.at W. {\postp}\\
\glt  `Wojo is sick.'
\end{exe}

 The
verb {\S
dʊa} has an allolexe (i.e. a combinatorial variant) used only in the negative.
Consider (\ref{ex:GRM-allolexe}).

\begin{exe}
\ex\label{ex:GRM-allolexe}
\begin{xlist}
\ex\label{ex:GRM-allolexe-pos}
\gll ʊ  dʊa dɪa nɪ \\
{\sc 3.sg} be.at house {\postp}\\
\glt  `She is in the house.'

\ex\label{ex:GRM-allolexe-neg}
\gll ʊ  wa tuwo dɪa nɪ \\
{\sc 3.sg} {\neg} be.at  house {\postp}\\
\glt  `She is not in the house.'


\ex\label{ex:GRM-allolexe-pos-out}
 \textasteriskcentered ʊ  tuwo dɪa nɪ
\ex\label{ex:GRM-allolexe-neg-out}
 \textasteriskcentered ʊ  wa dʊa dɪa nɪ
\end{xlist}

\end{exe}




\subsubsection{Possessive clause}
\label{sec:GRM-poss-cl}

A possessive clause expresses a relation between  a
possessor and a possessed.  Generally,  the  {\it
have-}construction  is used to convey a possessive relation. It consists of
the verb {\S kpaga} `have',  and two nominal expressions acting as subject and
object; the former being the possessor (\psor) of the relation, while  the
latter being  the possessed
(\psed).

\begin{exe}
\ex\label{ex:GRM-poss-have}
\glll kala kpaga nãɔ̃ ra\\
K. have cow {\foc}\\
  {\psor} {}   {\psed} {} \\
\glt  `Kala has a cow'
\end{exe}

Example (\ref{ex:GRM-poss-have}) says that an animate alienable possession
relates  Kala (possessor) and a cow (possessed).  Since the  {\it
have-}construction does not encode animacy or alienability features,   staple
food can `have' lumps, i.e. {\S kapala kpaga bie}, and someone can `have' a
senior brother, i.e. {\S ʊ kpaga bɪɛrɪ}.  Abstract possession may also be
conveyed using the {\it have-}construction. In (\ref{ex:GRM-poss-have-abst}),
  shame, hunger,  thirst and sickness are conceived as the possessors, the
possessed being the person experiencing these feelings. 



\begin{exe}
\ex\label{ex:GRM-poss-have-abst}
\begin{xlist}
 \ex\label{ex:GRM-poss-have-abst-1}
\gll hɪ̃̀ɪ̃̀sá kpàgà 	à   hã́ã̀ŋ    kɪ̀ŋkáŋ   \\
shame     	have  	{\art} 	woman    much\\
\glt `The woman was ashamed ...' (CB 034)
\ex\label{ex:GRM-poss-have-abst-2}
\gll lʊ̀sá kpágáń̩ nà\\
hunger have.{1.\sg} {\foc}\\
`I am hungry.'
\ex\label{ex:GRM-poss-have-abst-3}
nɪɪɲɔksa kpagan̩ na \\
`I am thirsty.'
\ex\label{ex:GRM-poss-have-abst-4}
garaga kpagan̩ na \\
`I am sick.'
\end{xlist}
\end{exe}

Some characteristics ascribed to animate entitites are expressed by  the
word {\S tɪɪna} `owner' following the possessed.  However it
is an existential clause (\ref{ex:GRM-poss-owner-exist}), rather than the  {\it
have-}construction, which carries the
possessive phrase {\sc psed}{\S -tɪɪna}, as (\ref{ex:GRM-poss-owner})
illustrates.



\begin{exe}
\ex\label{ex:GRM-poss-owner}
\begin{xlist}
 \ex\label{ex:GRM-poss-owner-exist}
\glll ʊ jaa sisɪama-tɪɪna\\
{3\sg} {\ident} seriousness-owner\\
  {\psor} {}   {\psed} \\
\glt `He is serious'

 \ex\label{ex:GRM-poss-owner-have}
\gll ʊ kpaga sisɪama ra\\
{3\sg} have {seriousness} {\foc}\\
\glt `He is serious'
\end{xlist}
\end{exe}

Another way to express possession is by using a non-verb clause which
  exclusively identifies   the possessor. For instance, a speaker may utter {\S
mɪ́n nà} `it is
mine' in order to say that a
certain thing belongs to him or her. This utterance consists solely of the third
singular strong pronoun followed by
the focus particle (see section \ref{sec:GRM-pronouns} on pronouns).


\subsubsection{Non-verb clause}
\label{sec:GRM-noverb}

As its name suggests, a non-verb clause is a clause without verbal elements. 
Its
main function is to identify or assert the (non-) existence of 
something.  The examples in (\ref{ex:GRM-noverb}) assert the (non-) existence
of a
referent with a single nominal expression, followed by the focus particle in
the affirmative and the negative particle in the negative (see section
\ref{sec:GRM-foc-neg} on focus and negation). 



\begin{exe}
\ex\label{ex:GRM-noverb}

\begin{xlist}
\begin{multicols}{2}
 \ex\label{ex:GRM-noverb-aff-1}
\gll fʊ́n ná\\
knife {\foc}\\
\glt `It is a shaving knife.'
 \ex\label{ex:GRM-noverb-aff-poss}
\gll ǹ̩ fʊ́n ná\\
{\sc 1.sg.poss} knife {\foc}\\
 \glt `It is my shaving knife.'
 \ex\label{ex:GRM-noverb-neg-1}
\gll fʊ́n lɛ̀ɪ́\\
knife {\neg}\\
 \glt `It is not a shaving knife.'
 \ex\label{ex:GRM-noverb-neg-poss}
\gll ǹ̩  fʊ́n lɛ̀ɪ́\\
{\sc 1.sg.poss} knife {\neg}\\
 \glt `It is not my shaving knife.'
\end{multicols}
\end{xlist}
\end{exe}


Correspondingly the adverbs {\S keŋ} and {\S  nɪŋ} are also found in non-verb
clause. For instance, {\S kéŋ né} means `That is it!', but the same string is
more often heard as {\S kéŋ nȅȅ} `Is that so/it?',  i.e.
constructed  as a polar question (see section \ref{sec:GRM-interr-polar} on 
polar questions, and section \ref{sec:GRM-adv-pro}  on  the adverbs {\S keŋ} and
{\S  nɪŋ}).




\subsubsection{Multi-verb clause}
\label{sec:GRM-multi-verb-clause}



A multi-verb clause is a clause containing more than one verb. The main type of
multi-verb clause is the serial verb construction (SVC), the definition of which
is still subject to contention. Let us start by stating that the SVC in Chakali
has the following properties: (i) a SVC is a sequence of verbs which act
together as a single predicate, (ii) each verb in the series could make up a
predicate on its own, (iii)  no connectives  surface (coordination or
subordination), (iv)  tense, aspect, mood and/or polarity are marked only once,
(v)  a verb involved in a SVC may be formally shortened,  (vi)  transitivity is
common to the series, so arguments are shared (one argument obligatorily), (vii)
the verbs in the series are not necessarily contiguous, and  (viii) the grammar
does not limit the number of verbs. These characteristics are not uncommon for 
SVCs in West-Africa \citep{Amek05a}. In this section, the SVC in Chakali is
identified using representative examples. 


Even though the construction has more than one
verb, it describes a single event and does not contain  markers of
subordination or coordination. The first sequence of verbs in
(\ref{ex:GRM-mvc-svc}) illustrates the phenomenon.



\begin{exe}
\ex\label{ex:GRM-mvc-svc}
\glll à   	kɪ̀rɪ̀nsá    	m̩̀    	màsɪ̀ 	kpʊ́  	àká  	dʊ̀gʊ̀nɪ̀ tá\\
{\art}	tsetse.fly.{\pl} 	{1.\sg}     	beat 	kill 	{\conn} 
chase  	     let.free\\
 {} {} {}  {} [{\it v} {\it v}]  {} [{\it v} {\it v}]\\
\glt `I beat and killed the tsetse flies, and drove them away.' (CB 023)
\end{exe}

Together,  the verbs {\S masɪ} `beat' and  {\S kpʊ} `kill'  in 
(\ref{ex:GRM-mvc-svc})  constitute a single event.  The same can be said about 
the verbs {\S dʊgʊnɪ} `chase' and {\S ta} `let free' in the second clause
following the connective.   If the clause following the connective   {\S aka}
lacks a subject,  the subject of the preceding clause shares its reference in
the two clauses   (see section \ref{GRM-clause-coord-ka-aka} on the connective 
{\S aka}). What we have in (\ref{ex:GRM-mvc-svc}) is one SVC separated from
another multi-verb clause by the connective {\S aka},  and the three verbs {\S
masɪ},  {\S kpʊ} and {\S dʊgʊnɪ}  share the reference of the  nominal {\S a
kɪrɪnsa} `the tsetse flies' as their Theme argument and {\S m̩̀} as their Agent
argument, i.e. {\sc o} and {\sc s} respectively. The role of  the verb {\S ta}
in the sentence depicted in  (\ref{ex:GRM-mvc-svc}) is discussed at the end of
this section.

Tense/aspect (\ref{ex:GRM-svc-tense}), mood (\ref{ex:GRM-svc-aspect}), and/or
polarity value (\ref{ex:GRM-svc-negation}) are marked only once, usually with
preverb particles. This means that they are not repeated for each verb of which
a predicate is composed. The preverb particles are discussed in section
\ref{sec:GRM-precerv}.


\begin{exe}
\ex\label{ex:GRM-svc-preverb}
\begin{xlist}
 \ex\label{ex:GRM-svc-tense}
\gll  ǹ̩ tʃɪ́ kàá màsɪ̀ 	kpʊ́   à   	kɪ̀rɪ̀nsá rá\\
{1.\sg} {\cras} {\fut.\prog} 	beat 	kill 	 {\art} tsetse.fly.{\pl}
{\foc}\\ 
\glt `I will be beating and killing the tsetse flies tomorrow.'

 \ex\label{ex:GRM-svc-aspect}
\gll  ǹ̩  há màsɪ̀ 	kpʊ́   à   	kɪ̀rɪ̀nsá rá\\
	{1.\sg}  {\mod} 	beat 	kill 	 {\art} tsetse.fly.{\pl}
{\foc} \\
 `I am still beating  and killing the tsetse flies.'

 \ex\label{ex:GRM-svc-negation}
\gll  ǹ̩   wà másɪ́ 	kpʊ́   à   	kɪ̀rɪ̀nsá\\
	{1.\sg}  {\neg} 	beat 	kill 	 {\art}
tsetse.fly.{\pl}\\
 `I did not beat and kill the tsetse flies.'
\end{xlist}
\end{exe}
%wàá will not

SVCs must share at least one core  argument. The example 
(\ref{ex:GRM-arg-sh-objsubj}) is an instance of argument sharing: the two verbs
in the construction share the (referent of the) noun {\S foto} `picture' and
are not contiguous. The
transitive verb {\S tawa} `pierce' takes  {\S foto} as its object, whereas {\S
laga} takes  {\S foto} as its subject. A representation of object-subject
sharing (or switch sharing) appears under the free translation in
(\ref{ex:GRM-arg-sh-objsubj}).
%\footnote{The label {\it object-subject sharing}
%is borrowed from \citet[20]{Osam03}.} 

\begin{exe}
\ex\label{ex:GRM-arg-sh-objsubj}{\it Object-subject sharing}
\glll  hɛmbɪɪ tawa foto laga daa nɪ\\
nail pierce picture hang wood  {\postp}\\
{} {\it v} {}  {\it v} \\
 \glt `A picture hangs from a nail on a wooden pole.'

{\S foto} $<x>$\\
{\S tawa} $<${\sc subj}$ =  y$ ,  {\sc obj}$=x$  $> $\\
{\S laga} $<${\sc subj}$ = x$ , {\sc obl} $= z $  $ >$\\
\end{exe}



Subject-subject and object-object sharing are more common than object-subject
sharing. In example
(\ref{ex:GRM-mvc-svc}), which is repeated below, the nominal expression {\S a 
 kɪrɪnsa} is the shared object of three verbs, i.e. {\S masɪ}, {\S kpʊ} and {\S
dʊgʊnɪ}, whereas the pronoun {\S m̩} is the shared subject for the same three
verbs. However, only {\S masɪ} and {\S kpʊ}  make up the SVC. 

\begin{exe}
\exp{ex:GRM-mvc-svc}{\it Subject-subject and Object-object sharing}
\gll à   	kɪ̀rɪ̀nsá    	m̩̀    	màsɪ̀ 	kpʊ́  	àká  	dʊ̀gʊ̀nɪ̀ tá\\
{\art}	tsetse.fly.{\pl} 	{1.\sg}     	beat 	kill 	{\conn} 
chase  	     let.free\\
\glt `I beat and killed the tsetse flies, and drove them away.'

{\S m̩} $<x>$\\
{\S kɪrɪnsa} $<y>$\\
{\S masɪ} $<${\sc subj}$ =  x$ ,  {\sc obj}$=y$  $> $\\
%{\S  kpʊ} $<${\sc subj}$ = x$ , {\sc obj} $= y $  $ >$\\
{\S dʊgʊnɪ} $<${\sc subj}$ = x$ , {\sc obj} $= y $  $ >$\\
\end{exe}

SVCs often involve two verbs, but there can be three or more verbs involved. 
Examples of three-verb and four-verb sequences are given in
(\ref{ex:GRM-mvc-3-4}). Each of the verbs involved can otherwise act alone as
main
predicate. Notice that the free translations provided do not accommodate well
the idea that
the two examples in (\ref{ex:GRM-mvc-3-4}) are conceived as single event.
In section \ref{GRM-clause-coord-subord},  it will be shown that connectives
are usually present  when one wishes to distinguish events.


\begin{exe}
\ex\label{ex:GRM-mvc-3-4}
\begin{xlist}
\ex
\glll ʊ sii kaalɪ na\\
{3\sg} rise go see\\
{}   {\it v}_{1} {\it v}_{2} {\it v}_{3}\\
`She stood, went and saw'
\ex
\glll ʊ bra tuu tʃɔ kaalɪ\\
{3\sg} turn go.down run go\\
{} {\it v}_{1} {\it v}_{2} {\it v}_{3} {\it v}_{4}\\
`She return, ran downhill (and went)'
\end{xlist}
\end{exe}



A  manipulative serial verb construction \cite[378]{Amek06} is a SVC
which  expresses a transfer of possession (e.g. give, bring, put)  or  
information (e.g. tell). It consists of the verb {\S kpa} `take' and another
verb following it. The example in (\ref{ex:GRM-m-svc-give}), repeated below,
illustrates a transfer of possession. 

\begin{exe}
\exp{ex:GRM-m-svc-give}{\it Manipulative serial verb construction}
%{\it Manipulative serial verb construction}\\
\glll  Kala kpa  {a lɔɔlɪ / ʊ} tɪɛ Afia  \\
K. take  {{\art} car / 3\sg} give A.\\
{} {\it v} {}  {\it v} {} \\
\glt  `Kala gave the car/it to Afia' ({\it lit.} Kala take the car give Afia.)
\end{exe}

Frequent co-locations of the type presented in (\ref{ex:GRM-m-svc-give}) are {\S
kpa wa}, {\it lit.}  take come,  `bring',  {\S kpa kaalɪ}, {\it lit.} take go,
`send', {\S kpa pɛ}, {\it lit.} take add,  `add', {\S kpa ta}, {\it lit.} take
let free, `remove', {\S kpa bile}, {\it lit.} take put,  `put (on)'  and {\S kpa
dʊ}, {\it lit.} take put,  `put (in)'. The two verbs may or may not be
contiguous;  usually the Theme argument of the  verb {\S kpa} `take'  is found
between the two verbs.






Finally, some multi-verb clauses are not  SVCs.  There are a few verbs
which
bear a
relation to the main predication and  contribute  aspects of the phase of
execution or scope of an event.\footnote{These verbs are similar 
to what \citet[108]{Bonv88}
calls {\it auxiliant}.} For instance, a {\it
terminative}  construction describes an event coming to an end or reaching a
termination, and  a {\it relinquishment} construction describes an event whose
result is the release or abandonment of someone or something.  The verbs {\S
peti}
`finish' and {\S ta} `abandon' in (\ref{ex:GRM-mvc-pha-3.1}) and
(\ref{ex:GRM-mvc-pha-help}), together with a non-stative predication, determine
each construction. 



\begin{exe}
\ex\label{ex:GRM-mvc-phase}
\begin{xlist}
\ex\label{ex:GRM-mvc-pha-3.1} {\it Terminative construction} 
\glll laɣalaɣa han nɪ n̩ kʊtɪ a ʔãã peti\\
{\advt} {\dem} {\postp} {1.\sg} {skin} {\art} bushbuck  finish\\
{} {} {} {}  {\it v} {} {} {\it v} \\
\glt `I  just finished skinning the bushbuck.'

\ex\label{ex:GRM-mvc-pha-40.3}
\gll  m̩ peti a tʊma ra \\
{1.\sg} finish {\art} work {\foc}\\
`I have finished the work.'


\ex\label{ex:GRM-mvc-pha-help}{\it Relinquishment construction}
\glll  kpa n̩ neŋ ta \\
take {1.\sg} hand let.free\\
{}  {\it v}  {} {\it v} \\
\glt `Leave my hand' (Let me go!)

\ex\label{ex:GRM-mvc-pha-relish} 
\gll a bʊ̃ʊ̃ŋ ta ʊ bie re \\
{\art} goat abandon {3.\sg.\poss} child {\foc}\\
`The goat abandoned its kids.'
\end{xlist}
\end{exe}

The examples  in (\ref{ex:GRM-mvc-pha-3.1}) and (\ref{ex:GRM-mvc-pha-help}),
which may be called  {\it phasal  constructions},\footnote{The analysis of the
progressive and prospective in Ewe and Dangme in \cite{Amek08} influences the
way I approach and name the phenomenon.}  are treated as multi-verb clauses
since the predication is expressed with more than one verb. Yet, they are not
SVCs because the second verb in each example only specifies aspects of the
process
of the event  and does not contribute to the main predication as verb sequences
in SVCs do. Nonetheless, these verbs can function otherwise as main predicate,
as shown in (\ref{ex:GRM-mvc-pha-40.3}) and (\ref{ex:GRM-mvc-pha-relish}).
Similarly, the verb {\S baga} `attempt to no avail'  conveys
nonachievement, e.g. {\S ʊ buure kisie baɣa}, {\it lit.} he look.for knife fail,
`he did not find the knife',  and the verb {\S na} `see' conveys confirmation or
verification, e.g. {\S sʊɔrɛ dɪsa na}, {\it lit.} smell soup see, `smell the
soup'. Going back to example (\ref{ex:GRM-mvc-svc}) above, the verb {\S ta}
contributes to a {\it relinquisshment} multi-verb construction, similar to
(\ref{ex:GRM-mvc-pha-help}) above, and not to a SVC. 
 
%  In the same
% spirit, the verb {\S baga} conveys nonachievement and the verb {\S na} conveys
% confirmation or verification.

%lexical idoms... put in verbs


\subsubsection{Comparative construction}
\label{sec:GRM-compar-ct}

A comparative construction has the semantic function of assigning a graded
position on a predicative scale to two (possibly complex) objects.
The comparative construction of inaquality can be expressed with the
transitive predicate {\S kaalɪ} `exceed, surpass', whose  two arguments are
the objects compared.\footnote{\cite{Brin05} presents a lexical-functional
grammar (LFG)  account of the comparative construction in Gã, a language also
exhibiting an `exceed'-comparative.}  One of the arguments represents the
standard
against which the other is
measured and found to be unequal.  The nominal expression in subject position is
the {\it comparee}, i.e. the objective of comparison, whereas the
one in object position is the {\it standard}, i.e. the object that
serves as yardstick for comparison \citep{Stas08}. The gradable
predicative scale is verbal and is normally adjacent to  the comparee, but may
be repeated adjacent to the standard. Given that both the scale and the
transitive predicate {\S kaalɪ} are verbs, a comparative construction is  a
 type of multi-verb clause.  If the predicative scale is absent, as in
(\ref{ex:GRM-comp-tr-sca-abs}),  one
may still interpret the construction as a comparative one, in which case both
the
context
and the meaning of  the nominals involved would provide the property on which
the
comparison  is made. These characteristics are illustrated in
(\ref{ex:GRM-comp-tr}).



\begin{exe}
\ex\label{ex:GRM-comp-tr}{\it Comparative transitive construction}
\begin{xlist}
\ex\label{ex:GRM-comp-tr-sca-pres}
\glll wʊ̀sá zéné káálɪ̀ àfìà\\
     W. grow surpass A.\\
[{\it n}]_{comparee} [{\it v}]_{scale} {\it v} [{\it n}]_{standard}\\
\glt `Wusa is taller than Afia.'

\ex\label{ex:GRM-comp-tr-sca-abs}
\glll wʊ̀sá bàtʃɔ́lɪ́ káálɪ́ kàlá bàtʃɔ́lɪ́\\
W.  running surpass K. running\\
[{\it n}  {\it n}]  {\it v} [{\it n} {\it n}]  \\
\glt `Wusa's running is better/faster than Kala's running.'
\end{xlist}
\end{exe}

Another way to compose a comparative construction of inequality is with the
identificational clause bounded with a postpositional phrase.  It is referred
to as a
comparative intransitive construction since the standard is not encoded in the
grammatical object of a transitive verb. Instead, the predicative scale is
embedded in a nominalized property following the identificational verb {\S jaa}
(see section \ref{sec:classifier} on classifiers).


\begin{exe}
\ex\label{ex:GRM}{\it Comparative intransitive construction}

\glll wʊsa jaa nɪ-hɪɛ̃ afia nɪ\\
W.  {\ident} {\clf}-old A. {\postp}\\
[{\it n}]_{comparee}  {\it v} [{\it v}]_{scale}   [{\it n}]_{standard} {} \\
\glt `Wusa is older than Afia.'
\end{exe}

The same  two strategies are used to
express a superlative degree: surpassing or being superior to all others is
explicitly expressed by a phrase containing the pronoun {\S ba} `they, them'.
This
is shown in (\ref{ex:GRM-super}).


\begin{exe}
\ex\label{ex:GRM-super}{\it Superlative construction}
\begin{xlist}
\ex 
\glll wʊsa zene kaalɪ ba\\
W. grow surpass {\sc 3.pl}\\
{} {\it v} {\it v} {} \\
`Wusa is the tallest.'
\ex 
\gll wusa jaa nɪ-hɪɛ̃ ba nɪ\\
W. {\ident} {\clf}-old {\sc 3.pl} {\postp}\\
`Wusa is the oldest.'
\end{xlist}
\end{exe}

A comparison of equality (i.e. X is same as Y) consists of a subject
phrase containing both objects to be  compared joined by the  connective {\S
(a)nɪ} followed by the scale, the verb {\S maase} `equal, enough, ever' and the
reciprocal word {\S
dɔŋa} `each other'  (see section \ref{sec:GRM-recipro-reflex} on reciprocity
 and reflexivity). This is shown in (\ref{ex:GRM-comp-equal}).

\begin{exe}
\ex\label{ex:GRM-comp-equal}{\it  Comparison of equality construction}

\gll wʊsa nɪ afia bɪnsa maase dɔŋa ra \\
W. {\conn} A.  year equal {\recp} {\foc}\\
%[{\it n}]_{comparee}   {\it v} [{\it v}_{scale}]   [{\it n}]_{standard} {} \\
\glt `Wusa is as old as Afia.'
\end{exe}
%old come from where

Finally,  the verb {\S bɔ} in (\ref{ex:GRM-comp-verb}) is a comparative
transitive verb which can be translated with the English comparative adjective
and prepositon `better than'.


\begin{exe}
\ex\label{ex:GRM-comp-verb}
\glll zaaŋ tʊma bɔ dɪare tɪŋ tʊma \\
today work better.than yesterday {\art} work\\
{}  {}  {\it v} {} {}   {}\\
\glt `Today's work is better than yesterday's work'
\end{exe}


\subsubsection{Modal clause}
\label{sec:GRM-compar-ct}

A modal clause is a clause expressing  ability, obligation, possibility, etc. An
ability-possibility construction is a clause containing the word {\S
kɪŋ} immediately preceding the main verb(s).  The construction conveys either
the
physical or mental
ability of something or someone, or    probability or possibility under some
circumstances. The construction is more frequent in the negative, but affirming
an ability or possibility is also possible in the positive using the
construction. The word {\S kɪŋ} is glossed  {\abl} to refer to
`ability'.\footnote{The word {\F kɪŋ} has a  nominal homophone meaning 
`thing' (and a classifier derived from the noun, see section
\ref{sec:classifier}).  Its distribution would suggest that it is a kind of 
preverb (section \ref{sec:GRM-precerv}), although it seems premature to
categorize it.}


\begin{exe}
\ex\label{ex:GRM-modal}{\it Ability-Possibility construction}
\begin{xlist}
\ex\label{ex:GRM-modal-12.2}
\gll ʊ wa kɪŋ wa\\
{3\sg} {\neg} {\abl} come\\
`He is not able to come.'
\ex  
\gll ɪ kaa kɪŋ kaalʊʊ\\
 {2.\sg} {\fut} {\abl} go.{\foc} \\
`You may go.'

\ex\label{ex:GRM-modal-13.1}
\gll ǹ̩ kàá kɪ̀ŋ wàʊ̀ tʃȉȁ\\
 {1.\sg}  {\fut} {\abl} come.{\foc} tomorrow\\
\glt `May I come tomorrow?'
\end{xlist}
\end{exe}

The phrase {\S a bɔnɪɛ̃ nɪ} `perhaps' is an adjunct phrase used when the
occurence of a situation  or an achievement  is in doubt. The word {\S bɔnɪɛ̃}
is not
used in any other context in the text corpus or lexicon. Thus, the dubitative
modality construction is a construction marked by the presence of the phrase
{\S a bɔnɪɛ̃ nɪ} clause-initially.


\begin{exe}
\ex\label{ex:GRM-modal}{\it Dubitative construction}
\begin{xlist}
\ex\label{ex:GRM-modal-45.5}
\gll {à bɔ̀nɪ̀ɛ̃̀ nɪ̀}  dʊ̀ɔ́ŋ kàá wàʊ̀\\
{\dub} rain {\fut} come.{\foc}\\
\glt  `Perhaps it is going to rain.'


\ex\label{ex:GRM-modal-45.3}
\gll {à bɔ̀nɪ̀ɛ̃̀ nɪ̀} ʊ̀ dɪ̀ wááwáʊ̀\\
{\dub}  {3\sg} {\hest} come.{\pfv.\foc}\\
\glt `Perhaps he came yesterday.' 

\end{xlist}
\end{exe}

In some contexts, a speaker may prefer  to use a cognitive verb in a phrase
like {\S
n̩ lisie} `I think (...)'  or
the phrase {\S a kʊ̃ʊ̃ n̩ na}, {\it lit.} it tires me {\sc foc},  `I wonder
(...)' as alternative to the dubitative
construction. 



\subsection{Interrogative clause}
\label{sec:GRM-interr-clause}

An interrogative clause consists either of a clause (i) with an initial
interrogative word/phrase, or (ii) with the absence of an initial interrogative
word but the presence of an extra-low tone at the end of the clause. The former
is called a `content' question and the latter a `polar' question. 

\subsubsection{Content question}
\label{sec:GRM-interr-content}

A content question contains an interrogative word/phrase whose typical position
is clause-initial. In (\ref{ex:GRM-inter-content}), {\S baaŋ} `what' replaces
the complement of the verb {\S jaa}, whereas {\S aŋ} `who'  replaces
the subject constituent of the clause. The inventory of interrogative
words/phrases can
be found in section \ref{sec:GRM-interg-pro}.

\begin{exe}
 \ex\label{ex:GRM-inter-content}
\begin{xlist}
\ex\label{ex:GRM-inter-content-what}
\gll baaŋ ʊ kaa jaa\\
{\q} {3\sg} {\ipfv} do \\
 \glt  `What is he doing?' 
\ex\label{ex:GRM-inter-content-who}
\gll aŋ kaa wa baŋ\\
{\q}  {\ipfv} come here\\
\glt  `Who is coming here?'
\end{xlist}
\end{exe}

When an interrogative word/phrase cannot be located clause-initially,  it is
found at the canonical position of the constituent replaced. There are
rare cases, but in (\ref{ex:GRM-inter-content-who-rev-bear-in}), which is
equivalent to  (\ref{ex:GRM-inter-content-who-rev-bear-ex}),  the question
word {\S aŋ} `who' appears in the object position following the transitive verb
{\S maŋa} `beat' and is slightly lengthened. 


\begin{exe}
 \ex\label{ex:GRM-inter-content-who-rev-bear}
\begin{xlist}
\begin{multicols}{2}
 \ex\label{ex:GRM-inter-content-who-rev-bear-in}
\gll  zɪɛn maŋa aŋŋ\\
 Z. beat {\q} \\
 \glt  `Zien beat who?'

 \ex\label{ex:GRM-inter-content-who-rev-bear-ex}
\gll  aŋ zɪɛn maŋa\\
{\q} Z. beat \\
 \glt  `Who did Zien beat?'
\end{multicols}
\end{xlist}
\end{exe}





\subsubsection{Polar question}
\label{sec:GRM-interr-polar}

A polar question is characterized by an interrogative intonation, consisting of
an extra-low tone at the end of the utterance. Additionally, lengthening  on the
penultimate vocalic segment takes place. The properties differentiating an
assertive clause from a polar question are illustrated in
(\ref{ex:GRM-inter-polar}). The
extra-low tone is represented with a double grave accent (i.e.  ̏). 


\begin{exe}
 \ex\label{ex:GRM-inter-polar}{\it Assertion vs. question}
\begin{xlist}
\begin{multicols}{2}
\ex 
\gll ʊ̀ wááʊ̀\\
{3.\sg} come.{\ipfv .\foc}\\
\glt  `He is coming.'
\ex 
\gll ʊ̀ wááʊ̏ʊ̏ \\
{3.\sg} come.{\ipfv .\foc}\\
\glt `Is he coming?'%22.1.1
\end{multicols}
\end{xlist}
\end{exe}


Possibly common to all Ghanaian languages, the  agreeing response
to a  negative polar interrogative
  takes into account the logical negation, as 
(\ref{ex:GRM-inter-polar-neg-rep}) illustrates. 



\begin{exe}
 \ex\label{ex:GRM-inter-polar-neg-rep}
\begin{xlist}
\begin{multicols}{2}
\ex\label{ex:GRM-inter-polar-neg-rep-S}{\it Speaker}
\gll  ɪ̀ wàá káálɪ̏ɪ̏\\
{2\sg} {\neg} go.{\q}\\
\glt `Aren't you going?'

\ex\label{ex:GRM-inter-polar-neg-rep-A}{\it Addressee}
\gll ɛ̃̀ɛ̃́ɛ̃̀\\
yes\\
\glt `No' ({\it lit.} Yes, I am not going)
\end{multicols}
\end{xlist}
\end{exe}

A negative polar interrogative in English usually asks about the
positive proposition, i.e. with `Aren't you going?' the speaker presupposes 
  that the addressee is going,   while in Chakali it questions the
negative proposition, i.e. with {\S  ɪ̀ wàá káálɪ̏ɪ̏} the speaker's belief
is that the addressee is not going. That is probably why we get `yes' in Chakali
and `no' in English for a corresponding negative polar interrogative.


%Alternative question are expressed with... p or not p

\subsection{Imperative clause}
\label{sec:GRM-imper-clause}

An imperative clause consists of a clause  expressing
direct commands, requests, and prohibitions. It can be an exclusively 
adressee-oriented clause or  can include the speaker as well. This distinction,
i.e. exclusive-inclusive, is rendered in  (\ref{ex:GRM-imperative-exc-inc}). 

\begin{exe}
 \ex\label{ex:GRM-imperative-exc-inc}
\begin{xlist}
%\begin{multicols}{2}
\ex\label{ex:GRM-imperative-exc}{\it Exclusive}
\gll fuuri a diŋ dusɪ\\
blow {\art} fire quench\\
\glt `Blow on this flame (to extinguish it).'

\ex\label{ex:GRM-imperative-inc}{\it Inclusive}
\gll tɪɛ ja muŋ laɣamɛ kaalɪ tɔʊtɪɪna  pe \\
give {1\pl} all gather go landlord end\\
\glt `Let's all go to the chief together.'

%\end{multicols}
\end{xlist}
\end{exe}


In (\ref{ex:GRM-imperative-exc}) the  speaker excludes herself  from the
performers of the action, while in  (\ref{ex:GRM-imperative-inc}) 
she includes herself among the performers. 



\begin{exe}
 \ex\label{ex:GRM-imper-exc-var}
\begin{xlist}
\ex\label{ex:GRM-imper-exc-var-sg}  (dɪ) wa `Come' (sg)
\ex\label{ex:GRM-imper-exc-var-pl}  (ma) wa `Come' (pl)
\ex\label{ex:GRM-imper-exc-var-out}  *dɪ ma wa `Come' (pl)

\end{xlist}
\end{exe}

When an order is given directly to the addressee, as in 
(\ref{ex:GRM-imper-exc-var-sg}), the clause may be introduced with the
 particle {\S dɪ}. Some consultants believe that omitting the
particle may
be perceived as rude. In addressing a command to a group, as in 
(\ref{ex:GRM-imper-exc-var-pl}), the
second person plural subject pronoun  usually appears in its canonical subject
position, but it may be absent if the speaker believes that the context allows a
single interpretation. The example in (\ref{ex:GRM-imper-exc-var-out}) shows
that  the  particle {\S dɪ} and the subject plural pronoun cannot coexist in
an imperative clause.\footnote{However the same string is grammatical if it is
interpreted as reported speech. If A asks `What does he want?', B may reply
{\F dɪ  ma wa} `That you ({\it pl.}) should come'. In this case {\F dɪ}
heads a subordinate clause which introduces indirect speech. }

Example (\ref{ex:GRM-hortative-vp11.3}) expresses a wish of the speaker and no
addressees are called for. Such a meaning is sometimes associated with optative
mood. Similarly but not  identically,  an utterance like the one in
(\ref{ex:GRM-hortativ-vp11.4})  assumes one or more addressees, yet the desired
state of affairs is not in the control of anyone in particular but of everyone. 
 As in (\ref{ex:GRM-imperative-inc}), the  strategy in both cases is
to use the verb {\S tɪɛ} `give'.  
% 
% , and
%  occurs exclusively  with the first person (\ref{ex:vp11.3}) or, implicitly or
% explicitly, the
% third person (\ref{ex:vp11.4}), singular and  plural. 

 

\begin{exe}
 \ex\label{ex:GRM-hortative}
\begin{xlist}
 
\ex\label{ex:GRM-hortative-vp11.3}{\it Optative}
\gll tɪ́ɛ́ m̩̀ mɪ̀bʊ̀à bírgì \\
       give {\sc 1.sg.poss} life delay \\
\glt  `Let me live long! ' 

\ex\label{ex:GRM-hortativ-vp11.4}{\it Hortative}
\gll tɪ́ɛ́ à gʊ̀á píílè \\
     give {\art} dance start     \\
\glt  `Let the dance begin!'


\end{xlist}
\end{exe}


A prohibitive clause consists of  a negated proposition conveying an imperative
(or hortative) mood. It is marked by 
the negative particle {\S tɪ}/{\S te} `not'   ({\it gl.} {\sc neg.imp})
occurring
clause-initially.

\begin{exe}

\ex\label{GRM-neg-imp-vp15.10.}
\gll tɪ́ káálíì, dʊ́ɔ̀ŋ kàá wàʊ̀ \\
     {\neg.\imp} go rain {\fut} come.{\foc}   \\
\glt  `Don't go; it's going to rain.' 
\end{exe}
 
The prohibitive also involves a high front vowel   suffixed to its verb. The
quality of the vowel, i.e. {\S -ɪ}/{\S -i}, is determined by the quality of the
verbal stem.

\begin{exe}
 \ex\label{ex:GRM-neg-imperative}
\begin{xlist}
\begin{multicols}{2}
\ex gó\\
`Move in a circle around'
\ex té   góì\\
`Don't move in a circle around'
\ex  kpʊ́\\
`Kill'
\ex  tɪ́    kpʊ́ɪ̀\\
`Don't kill'
\end{multicols}
\end{xlist}
\end{exe}

In addition, a distinction within the prohibitive can be made
between a prohibition or advice for a future situation  (\ref{ex:GRM-neg-fut}), 
and  for an on-going situation (\ref{ex:GRM-neg-pres}). 
\vspace*{10pt}


 \begin{minipage}[h]{12cm}
\begin{exe}
 \ex\label{ex:GRM-neg-fut-pres}
\begin{xlist}
\begin{multicols}{2}
 \ex\label{ex:GRM-neg-fut}
 tɪ́  wà  kpʊ́ɪ̀\\
`Don't kill' 
 \ex\label{ex:GRM-neg-pres}
 tɪ́    kpʊ́ɪ̀\\
`Don't kill'
\end{multicols}
\end{xlist}
\end{exe}

 \end{minipage}
\vspace*{15pt}




\subsection{Clause coordination and subordination}
\label{GRM-clause-coord-subord}
%embedded clause, subordinate clause) cannot stand alone as a sentence

A relation between two clauses is signaled with or without a morpheme,  and 
various  structures and morphemes  are used to relate clauses.  Two
relations are discussed below: coordination and subordination. 

\subsubsection{Coordination}
\label{GRM-clause-coord}


%conjunct ka, aka, a
The distribution of four clausal connectives which are used in coordinating
clauses is presented: these are {\S a}, {\S ka}, {\S aka} and {\S
dɪ}.\footnote{See \citet[143-149]{Mcgi99} for an account of similar clausal
connectives in Pasaale.}  


\paragraph{Connective a}
\label{GRM-clause-coord-a}

The connective {\S a} `and'  introduces a clause without overt subject.  When it
is used between two clauses, the subject of the first clause must cross-refer to
the covert subject of the second clause  (and subsequent clauses).
It links a sequence of closely related events carried out by the same agent,
and the events are encoded in  verb
phrases denoting temporally distinct events. The example in
(\ref{ex:GRM-coor-vp8.1}) is  an illustration of four consecutive clauses
introduced by the connective  {\S a}.   This phenomenon is often referred to as
`clause chaining'. 

\begin{exe}
\ex\label{ex:GRM-coor-vp8.1}
\gll dɪarɪ tɪŋ n̩ dɪ kaalɪ bɛlɛɛ ra, \textbf{a} \underline{jawa namɪɛ̃},
\textbf{a} \underline{kpa wa dɪa}, \textbf{a}  \underline{wa tɪɛ n̩ hããŋ},
\textbf{a} 
\underline{ŋma tɪɛ n̩ hããŋ} dɪ ʊ tɔŋa. ʊ tɔŋa ja di \\ 
{\advt} {\art} {\sc 1.sg} {\hest} {go} {G.}  {\foc} {\conn}  {buy.meat} 
{\conn} {take.come.home} {\conn} {come.give.my.wife}  {\conn} {say.give.my.wife}
{\comp}  {\sc 3.sg}   {cook}    {\sc 3.sg}   {cook}  {\sc 1.pl} {eat} \\
\glt  `Yesterday I went to Gurumbele,  bought some meat, brought it
home to my wife, told her to cook it. She cooked and we ate.'
\end{exe}


\paragraph{Connective (a)ka}
\label{GRM-clause-coord-ka-aka}

From the data available, it proved impossible to solve the variances between the
connectives {\S ka} and {\S aka}. Thus, since at this stage of research the two
connectives are believed to be in free variation, I will refer to them as {\S
(a)ka}. 

First, the data suggests that either (i) the subject of the clause preceding the
connective is inferred in the second clause, i.e. as for  the connective  {\S a}
above, or (ii) a different subject surfaces in the second clause. Each case is
shown in (\ref{GRM-clause-conn-ka-1-subj}) and
(\ref{GRM-clause-conn-(a)ka-2-subj}) respectively.


\begin{exe}
\ex\label{GRM-clause-conn-ka-1-subj} 
\gll    [ŋmɛ́ŋtɛ́l   làà nʊ̀ã̀  nɪ́]    ká  [ŋmá dɪ́    ʊ̀  wá  
ɲʊ̃̀ã̀ nɪ́ɪ́] \\
spider collect mouth {\postp}  {\conn} say {\comp}  
{\sc 3.sg}  come   drink water\\
\glt  `(Monkey went to spider's farm to greet him.)  \textbf{Spider} accepted
(the
greetings) and (Spider) asked him (Monkey) to come and drink water.'  (LB 011)
 \end{exe}



\begin{exe}
\ex\label{GRM-clause-conn-(a)ka-2-subj} 
\begin{xlist}
\ex\label{GRM-clause-conn-ka-2-subj} 
\gll  [dɪ̀   	 \underline{ɪ̀}     	wá        	pàrà] 	ká   
[\underline{kìrìmá}     wá  	dʊ́mɪ́ɪ́]\\
{\conn} 	 {\sc 2.sg}  	{\ingr}     {farm ({\it v})}   	{\conn} 
tstse.fly.{\pl}        {\ingr}  bite.{\sc 2.sg} \\
\glt  `When  \textbf{you} are doing the weeding and  \textbf{tsetse flies} bite
you (...)' (CB 003) 

\ex\label{GRM-clause-conn-aka-2-subj} 
\gll  [dɪ́   \underline{námùŋ}  	tɪ́  bɪ́ wà   jɪ́rà kɪ̀ŋkùrùgìé
ŋmɛ́ŋtɛ́l sɔ́ŋ] àká  [\underline{ɪ̀}    jɪ̀rà 	kéŋ]  \\
{\comp} 	anyone 	           {\neg} 	{\itr} {\ingr}  
   call         enumeration          
eight    name    	{\conn} 	{\sc 2.sg} 	call   	{\adv} \\
\glt  `(The monkey said:  ``They said) that \textbf{anyone} should not say the
number eight and \textbf{you} have called the number eight''.' (LB
017) 
\end{xlist}
\end{exe}


Secondly, the connective {\S (a)ka} `and'  may encode a `logical' or `natural'
sequence of events.   For instance, in (\ref{GRM-clause-conn-ka-1-subj}), 
someone traveling (or coming from the road) expects to be offered
water to drink
after the greetings are exchanged. 



Thirdly, the  connective {\S (a)ka}
may also suggest a causal relation between interdependent clauses. In 
(\ref{GRM-clause-conn-a}), it is the counting of the mounds which caused Spider
to be confused, which can be seen as an unexpected outcome.  

% , but the left-hand conjunct in the {\S
% (a)ka}-construction in (\ref{GRM-clause-conn-a})  behaves somehow  like a
% subordinate clause.

% Once again, the fact that the subordinate causal
% clause is a VP adjunct, thus a syntactic constituent of the matrix sentence,
% explains its
% behaviour concerning focus marking structures. The impossibility of focusing
%the
% second segment of a Justification construction through negation reinforces its
% syntactic
% peripheral status.


\begin{exe}
\ex\label{GRM-clause-conn-a} 
\gll   ʊ́  wà  ŋmɛ́ŋtɛ́l 	já  	      kùró 
àká  	bùtì. \\
{\sc 3.sg} {\foc}   spider do count 	  
{\conn}  confuse\\
\glt  `(Because) he himself (Spider) did count and he became confused'
(LB  007) 
 \end{exe}


Nevertheless the connective {\S (a)ka} can introduce a clause denoting an event
which is not necessarily related to the event of the previous clause. It looks
as if  the connective {\S (a)ka} in (\ref{GRM-clause-conn-ka-then}) is used to
integrate an unrelated event to  the overall situation.   

\begin{exe}
\ex\label{GRM-clause-conn-ka-then} 
\gll [{nansa su bara muŋ.}] ka [dʊ̃ʊ̃ tɪŋ ŋma dɪ kɪndɪgɪɪ dʊɔ a dɪa nɪ]\\
{meat fill place all} {\conn} python {\art} say {\comp} something is  {\art}
house {\postp}\\
\glt `Meat was all over the place. Then,  Python said: ``there is something in
the room''.'
 \end{exe}

\begin{exe}
\ex\label{GRM-clause-conn-ka-transition} 
\gll  [à   	bìpɔ̀lɪ́ɪ̀   	sìì      	   tʃɪ́ŋá]     àká      
[ŋmá,  
ámɪɛ̃̀   ɪ̀               ɲɪ́ná  {...} \\
{\art} young.man      raise   stand {\conn} said,  {\adv}   {\sc 2.sg.poss} 
father  {...}\\
\glt `The young man stood up and said:  ``So, when your father (...)''.' (CB
010)
 \end{exe}

Notice that the `standing' and `saying' events in
(\ref{GRM-clause-conn-ka-transition}) are strictly transitional, but this is not
the case in (\ref{GRM-clause-conn-ka-then}). The connective {\S ka} in
(\ref{GRM-clause-conn-ka-then}) opens a sentence which marks a shift from a
scene description (i.e.  `there was meat all over the place') to a character's
intervention (i.e. `Python speaking').  In comparing
(\ref{GRM-clause-conn-ka-then}) and (\ref{GRM-clause-conn-ka-transition}), and
the other examples provided in this section, it was impossible to distinguish 
{\S ka} from   {\S aka}, and predict their  distribution.


\paragraph{Connective dɪ}
\label{GRM-clause-coord-di}
The clausal connective {\S dɪ} `and' or `while'  is homophonous with a
complementizer particle (section \ref{GRM-clause-comp-di}), a connective used in
conditional constructions (section \ref{GRM-clause-subord}),   and a preverb
particle signaling the imperfective aspect (section \ref{sec:GRM-ipfv-part}). It
connects two clauses which encode different events, yet these events must be
interpreted as occurring simultaneously.  A clause introduced by the connective
{\S dɪ} has no overt subject, instead the subject is inferred, as it has the
same referent as the subject of the preceding clause. Three examples are
provided in (\ref{GRM-clause-conn-di}). Some examples of the clausal connective
{\S dɪ} in the corpus may be argued to convey intention or purpose, e.g.
(\ref{GRM-clause-conn-di-3}). 

\begin{exe}
\ex\label{GRM-clause-conn-di}
\begin{xlist}
\ex\label{GRM-clause-conn-di-vp22.4.9.}
\gll líé ʊ̀ kààlɪ̀ dɪ̀ wá \\
     {\q} {\sc 3.sg} go {\conn} come  \\
\glt  ` Where is he coming from?' ({\it lit.} where he left and  come)

\ex\label{GRM-clause-conn-di-vp47.2.9.}
\gll kpá sɪ̀ɪ̀má háŋ dɪ̀ káálɪ̀  \\
    take   food {\dem}  {\conn} go \\
\glt  ` Take this food away! ({\it lit.} take this food and go)

 \ex\label{GRM-clause-conn-di-3}
\gll ʊ dʊa ja gantal nɪ dɪ wa\\
 {\sc 3.sg} be.at {\sc 1.pl.poss} back {\postp} {\sc conn} come\\
`She is following us'
\end{xlist}
 \end{exe}



\subsubsection{Subordination}
\label{GRM-clause-subord}

The morpheme {\S tɪŋ} is mainly used as a  determiner in noun phrases  (see
section \ref{sec:GRM-np-def}).  However, there are instances where the discourse
following {\S tɪŋ} must be treated as subordinated and related to the noun
phrase which  {\S tɪŋ} is part. One may argue that the morpheme {\S tɪŋ} can
function  as a relativizer. Consider (\ref{GRM-clause-subord-rel}).

\begin{exe}
\ex\label{GRM-clause-subord-rel}
\gll  kúrò [píé tɪ́ŋ]_{NP} \underline{ʊ̀ kà tɔ́ à kùò nɪ́ kéŋ̀}
tɪ̀ɛ̀ʊ́\\
	 count 	yam.mound.{\sc pl}  {\art} 	{{\sc
3.sg}-{\egr}-cover-{\art}-farm-{\postp}-{\adv}}         give.{\sc 3.sg}\\
\glt  `(Spider_{x} asked Buffalo to) count  for him_{x} the yam mounds which
he_{x} covered at the farm.' (LB 006)
 \end{exe}

In (\ref{GRM-clause-subord-rel}), the phrase {\S ʊ̀ kà tɔ́ à kùò nɪ́
kéŋ̀} is (i) in apposition to the noun phrase {\S píé tɪ́ŋ}, and (ii)
in a subordination relation with the noun phrase.\footnote{Examples 
LB 004, LB 012, CB 019 and  CB 026 in the appendix display the same sort of
subordination.}


In a conditional construction like the one in
(\ref{GRM-clause-subord-if-vp46.11}), the subordinate
clause is headed by the particle {\S dɪ},  whereas the main clause follows
the subordinate clause. Proverbs are typically  conditional
constructions.  An example is given in (\ref{GRM-clause-subord-proverb}).


\begin{exe}
\ex\label{GRM-clause-subord-di}
\begin{xlist}
\ex\label{GRM-clause-subord-if-vp46.11}

\gll dɪ̀ ǹ̩ fɪ̀ tú kààlɪ̀ dē, bà kàá  tùgúǹ nō \\
      {\sc conn}  {\sc 1.sg} {\mod} {go.down} go {\adv} {\sc 3.pl.h+} {\fut}
beat.{\sc 1.sg} {\foc} \\
\glt  `If I was to go down there, they will beat me. ' 

\ex\label{GRM-clause-subord-proverb}
\gll dɪ̀ ɪ̀ zíŋ wā zɪ̀ŋà,  ɪ̀ wàá kɪ̀ŋ gáálɪ́ díŋ nɪ̄ \\
  {\sc conn} {\sc 2.sg}  tail {\ingr} long  {\sc 2.sg} {\neg} {\abl} be.over
fire
{\postp}\\
`If you have a long tail, you cannot cross fire.'

\end{xlist}
 \end{exe}
 

Adverbial expressions are used as connectives in similar clause-relating
functions. The subordinate clause of a concessive construction is introduced by
the expression  {\S anɪ amuŋ}, {\it lit.} and-all, `despite',  `in spite
of', `although' or `even though'. A subordinate clause  which conveys a
consequence or a justification of the proposition in the main clause  is
introduced by the expressions {\S aɲuunɪ} or {\S awɪɛ}, {\it lit.} the-head-on
 and  the-matter,  respectively,  `therefore' or
`because'. Examples are shown in (\ref{GRM-clause-conces-consec}).

\begin{exe}
\ex\label{GRM-clause-conces-consec}
\begin{xlist}
\ex\label{GRM-clause-conces}

\gll ʊ̀ wááwáʊ́ {ànɪ́ ámùŋ} dɪ́ ʊ̀ wɪ́ɪ́ʊ̀ \\
       {\sc 3.sg} come.{\pfv .\foc} {\conn} {\comp}  {\sc 3.sg} sick.{\foc}\\
\glt  `He came in spite of his illness.' 


\ex\label{GRM-clause-consec-1}

\gll n̩ wa kpaga sakɪr, {aɲuunɪ} n̩ dɪ vala nããsa \\
{\sc 1.sg} {\neg} have bicycle {\conn} {\sc 1.sg}   {\ipfv} walk
leg.{\pl}\\
\glt `I don't have a bicycle, therefore I am
walking.'

\end{xlist}
 \end{exe}
 

\paragraph{Complementizer dɪ}
\label{GRM-clause-comp-di}


Example (\ref{GRM-clause-comp-inds}) shows that the complementizer {\S dɪ}
introduces indirect speech. 
\begin{exe}
 \ex\label{GRM-clause-comp-inds}
 \gll  	kùórù 	bìnɪ̀hã́ã̀ŋ        	ŋmá 	dɪ́   	ɛ̃̀ɛ̃́ɛ̃́ɛ̃̀  \\
 chief 	young.girl 	say 	{\comp}  	yes\\
 \glt  `The chief's daughter answered yes.'  (CB 011) 
 \end{exe}


Direct speech is usually introduced by a speech
verb only, e.g. {\S ŋma (tɪɛ)} `say (give)',   {\S tʃagalɪ} `teach, show,
indicate', {\S hẽsi} `announce', etc.  This is shown in 
(\ref{GRM-clause-comp-inds}) with {\S hẽsi}.

\begin{exe}
 \ex\label{GRM-clause-comp-ds}
 \gll tɔʊtɪɪna ŋma dɪ ba hẽsi \underline{ma ka para kuo}\\
 landlord say  {\comp} {\sc 3.pl.g}b  announce {{\sc 2.pl} {\egr} farm farm}\\
 \glt  `The landowner says that they announced:  ``You go and work at the
farm''.' 
 \end{exe}

In (\ref{GRM-clause-comp-2-int}),  the complementizer {\S dɪ} introduces a
clause which conveys the intention of the event in the main clause. In a literal
sense, the husband {\S lala} `open'  the wife {\it in order to} have her {\S
sii} `raise up'. In (\ref{GRM-clause-comp-2-pur}) it is shown that a purpose (or
an intention) can be encoded when {\S dɪ} introduces the goal.



\begin{exe}
 \ex\label{GRM-clause-comp-2}
\begin{xlist}
 \ex\label{GRM-clause-comp-2-int}
\gll  tʃʊ̀ɔ̀sá   pɪ́sɪ̀,    ʊ̀   báàl    tɪ̀ŋ té    lálá    à hã́ã̀ŋ   
 dɪ̀  ʊ́  	     síí 	     dùò 	 nɪ̀.\\
	morning scatter          {\sc 3.sg.poss}	 husband  {\art} 
early  wake.up 
{\art}  wife         {\comp}      {\sc 3.sg}  	     raise.up    asleep	
{\postp} \\
 \glt  `Early in the morning her husband woke up the wife from sleep.' ({\it
lit.} that she should/must stand up)  (CB 030)

 \ex\label{GRM-clause-comp-2-pur}
 \gll ʊ kaalɪ dɪ ʊ ka ɲʊ̃ã nɪɪ   \\
{\sc 3.sg} go   {\comp}  {\sc 3.sg}   {\egr} drink  water  \\
 \glt  `He went to have a drink of water.' 

\end{xlist}
  \end{exe}






 \paragraph{Clause apposition}
 \label{GRM-dep-comp-clause}
% 
In (\ref{GRM-clause-appo}) it is shown that a desire can be
encoded by two clauses in apposition. 
% 
 \begin{exe}
 \ex\label{GRM-clause-appo}
 \gll ja buure nɪɪ ra ja ɲʊ̃ã\\
{\sc 2.pl} want water {\foc} {\sc 2.pl} drink\\
 \glt  `We want some water to drink.' 
 \end{exe}



\subsubsection{Summary}
\label{GRM-clause-summary}

This section gave an overview of the  common clause structure, the main
elements of syntax and clause coordination and subordination. Next, aspects of
the nominal syntax and morphology are
introduced.

%\begin{exe}
% \ex\label{GRM-clause-appo}
% \gll    \\
% \\
% \glt  `' 
%  \end{exe}

% 
% \ex\label{ex:vp19.2.}
% \gll anɪ a muŋ dɪ ʊ dia boloo , n̩ ja ka lɪ ʊ pe re tʃʊɔpɪsɪ bɪ muŋ \\
%         \\
% \glt  ` Although his room is far away, I visit him every day.  ' 




\section{Nominals}
\label{sec:GRM-nom}


The term `nominal'  identifies  a formal and functional  syntactic level and
lexemic level. At the syntactic level, a noun phrase is a nominal  which can
either function as core or peripheral argument.  Its composition may
vary from a single pronoun to a noun with modifier or series of
modifiers. At the lexeme level, two categories of lexemes are assumed:
nominal and verbal. These two types correspond roughly to the semantic division
{\it entity} and {\it event}, but do not correspond to the syntactic categories
{\it noun} and {\it verb}. That is because lexemes are assumed to not be
specified for syntactic category. The diversity  of forms and functions of
nominals is presented below. 


\subsection{Noun phrases}
\label{sec:GRM-verb-phrases}

A noun phrase (NP)  consist of a nominal head, and optionally, its dependent(s).
In this section,  the internal components of noun phrases and the roles these
components have within the noun phrase are described. First,   indefinite and
definite noun phrases are considered. Then, the elements which can be found in
the noun phrase are introduced. 

\subsubsection{Indefinite noun phrase}
\label{sec:GRM-np-indef}

Indefinite noun phrases are used when ``the speaker invites the addressee to
construe a referent [which conforms with] the properties specified in the term''
\citep[184]{Dik97}.  In Chakali, a noun standing alone can  constitute a noun
phrase. Such a 
noun phrase can be interpreted as indefinite, i.e. the noun phrase is a
non-referring expression,  or   generic, i.e. the noun phrase 
denotes  a kind or class of entity  as opposed to an individual.  In rare cases,
a
definite noun phrase can be interpreted from a single noun  (i.e. lacking  an
  article). Each interpretation is obviously dependent on the context of
the utterance in which the noun occurs.

\begin{exe}
 \ex\label{GRM-np-type-indef}{\it  N = NP}
\begin{xlist}
 
\ex\label{GRM-np-indef-1}
\gll  kala jawa   [pɪɛŋ]_{NP} na\\
     Kala buy mat {\sc foc}\\
\glt  `Kala bought a MAT' 


\ex\label{GRM-np-indef-2}
\gll  [dʒɛ̀tɪ̀]_{NP} kɪ́m-bɔ́n  ná \\
	{lion.{\sc sg}} {\sc clf}-dangerous.{\sc sg} {\sc foc} \\
\glt  `A lion is DANGEROUS'

\end{xlist}

\end{exe}

In (\ref{GRM-np-type-indef}),  the noun phrase {\S  pɪɛŋ}  describes any mat
and
is interpreted as a novelty in the hearer's knowledge of Kala, while
{\S dʒɛtɪ} describes the entire class of lions. 

Noun phrases  containing the
numeral {\S dɪ́gɪ́ɪ́} `one'  may be translated as
English `a certain'.    The
expression {\S badɪgɪɪ} can be translated as `one of them', `someone' or
`anyone'  (e.g. {\S ʊ wa ja badɪgɪɪ}, {\it lit.} he-not-be-one.of.them, `he is
an
illegitimate child'). 

%Other strategies to introduce a specific object or set of
%objects that is believed to be new to the addressee is the 


\subsubsection{Definite noun phrase}
\label{sec:GRM-np-def}

Definite noun phrases are employed when ``the speaker invites the addressee to
identify a referent which he (the speaker) presumes is available to the
addressee'' \citep[184]{Dik97}. A definite noun phrase may consist of  a single
pronoun in subject position, as
shown in (\ref{GRM-np-type-pro}).

\begin{exe}
 \ex\label{GRM-np-type-pro}{\it pro = NP}
\gll  [ʊ]_{NP} sʊwa \\
      {\sc 3.sg} die\\
\glt  `She died'
\end{exe}


A possessive noun phrase is always definite. A possessive pronoun followed by a
noun is analyzed as a succession of a noun phrase and a noun. Thus,  the noun
phrase in (\ref{GRM-np-type-pro-n})  is analyzed as a sequence of the noun
phrase {\S ʊ} and the  noun  {\S mãã}. 


\begin{exe}

 \ex\label{GRM-np-type-pro-n}{\it pro + N = NP}
\gll   [ʊ mãã]_{NP} ŋma dɪ ʔoi \\
      {\sc 3.sg.poss} mother say  {\sc comp} {\sc interj}\\
\glt  `Her mother said, `Oi'.'
\end{exe}


The treatment  NP+N for possessive noun phrase is   motivated  by the
possibility of  recursion of  an attributive possession relation. The expression
{\S
lɔɔlɪgbɛrbɪɪ}_{N} `car 
key' is the head in the three possessive noun phrases   {\S karedʒa
lɔɔlɪgbɛrbɪɪ}
`Kareja's car key', {\S karedʒa ɲɪna lɔɔlɪgbɛrbɪɪ} `Kareja's father's car
 key' and {\S karedʒa ɲɪna bɪɛrɪ lɔɔlɪgbɛrbɪɪ}
`Kareja's father's senior brother's car  key'.  Notice that in these examples
  the nominal head consists of the right-most element in the noun phrase. The
compound noun  {\S  lɔɔlɪgbɛrbɪɪ}  `car-key', and correspondingly,
the head of the left daughter of the  possessive noun phrase is the right-most
element, e.g.   [[karedʒa ɲɪna \underline{bɪɛrɪ}]_{NP}
[lɔɔlɪgbɛrbɪɪ]_{N}]_{NP},
{\it lit.} `the key of the brother of the father of
Kareja'. The syntactic tree in (\ref{ex:GRM-poss-def-tree}) illustrates the
structure of this definite noun phrase.

%add lɔɔlɪgbɛrbɪɪ to dict

\begin{exe}
\ex\label{ex:GRM-poss-def-tree}
\Tree[-1]{  &&&&&\Kq{NP} \Bq{dlll}\Bq{drrr} &&&&\\
&&\Kq{NP} \Bq{dll}\Bq{drr} &&&&&& \Kq{N}\Bq{dl}\Bq{dr} &\\
\Kq{karedʒa} &&&& \Kq{NP} \Bq{dl}\Bq{dr} &&& \Kq{lɔɔlɪ}  &&\Kq{gbɛrbɪɪ}\\
&&& \Kq{ɲɪna} && \Kq{bɪɛrɪ} &&&&}
\end{exe}





\paragraph{Articles a and tɪŋ}
\label{sec:GRM-np-def-articles}
%This treatment predicts that the head of a possessive noun phrase 

There are two articles in Chakali; one which encodes specificity and the other
definiteness. The first one is the 
article {\S a} ({\it gl.} {\sc art1}) and the other is  {\S tɪŋ}  ({\it
gl.} {\sc art2}).  

The  article {\S a} is translated with the English article
`the'.\footnote{A pre-nominal article is not found in Tampulma or Pasaale. The
fact that
Waali and Dagaare use an identical  article suggests that the definite article
{\F a}
is a contact-induced innovation.} It must precede the head noun and cannot
co-occur with the possessive pronoun.  In the context of
(\ref{GRM-np-type-det1}), the speaker assumes that the hearer is informed about
Kala's interest in buying a mat. 


\begin{exe}
 \ex\label{GRM-np-type-det1}{\it  a + N = NP}
\gll  kala jawa  [a pɪɛŋ]_{NP} na\\
     Kala buy {\sc art1}  mat {\sc foc}\\
\glt  `Kala bought the MAT' 
\end{exe}


The type of mat,  its color or the location where Kala bought the mat and so on
are not necessarily shared pieces of information between the speaker and hearer
in (\ref{GRM-np-type-det1}).  The only information the speaker believes they
have in common is Kala's interest in purchasing a mat. The article {\S
a} is treated as a functional word which makes the noun phrase specific but not
necessarily
definite.  When a noun phrase is  specific, the speaker should have a particular
referent in mind whereas the addressee may or may not share this knowledge.


The article {\S tɪŋ}  ({\it gl.} {\sc art2}) can also be seen to correspond to
English `the',  but a preferable paraphrase would be `as referred previously' or
 `this (one)'.  The article {\S  tɪŋ} appears when the speaker knows that the
hearer will be able to identify the referent of the noun phrase. In that sense,
the referent is familiar.\footnote{In the giveness hierarchy of
\citet[278]{Gund93}, the status {\it familiar} is reached when ``the addressee
is able to uniquely identify the intended referent because he already has a
representation of it in memory.''}   When {\S tɪŋ} follows a noun, the referent
must either have been mentioned previously or the speaker and addressee have an
identifiable referent in mind. Thus, compared to the examples
(\ref{GRM-np-type-indef}) and (\ref{GRM-np-type-det1}) above, a proper
interpretation of example (\ref{GRM-np-type-det2}) requires that both the
speaker and addressee have a particular mat in mind. In terms of word order, the
article  {\S a}  initiates the noun phrase and  the article {\S tɪŋ}  belongs
near the end of the noun phrase. 
 

\begin{exe}
 \ex\label{GRM-np-type-det2}{\it   (a +) N + tɪŋ = NP}
\gll  kala jawa  [a pɪɛŋ  tɪŋ]_{NP} na\\
     Kala buy {\sc art1}  mat {\sc art2} {\sc foc}\\
\glt  `Kala bought the MAT'
\end{exe}

Consider the slight meaning difference between
(\ref{GRM-np-type-det2-a}) and (\ref{GRM-np-type-det2-b}).


\begin{exe}
 \ex\label{GRM-np-type-det2-ab}
% \vspace{-12pt}
 \begin{xlist}
  \ex\label{GRM-np-type-det2-a}
\gll ɲɪnɪɛ̃ ɪ ɲɪna ka dʊ\\
    {\sc q} {\sc 2.sg.poss} father {\sc  egr} be  \\
\glt  `How is your father?'

  \ex\label{GRM-np-type-det2-b}
\gll ɲɪnɪɛ̃ ɪ ɲɪna tɪŋ ka dʊ \\
   {\sc q} {\sc 2.sg.poss} father {\sc art2} {\sc  egr} be \\
\glt  `How is your father?'
  
 \end{xlist}
\end{exe}


Both sentences may be translated with `How is your father?'. However, whereas 
the sentence (\ref{GRM-np-type-det2-a}) can request  a general description
of the father (i.e. skin color, size, general health, etc.), the sentence
in (\ref{GRM-np-type-det2-b}) asks for a particular aspect of the
father's condition which both the speaker and the addressee are aware of, for
instance the father's sickness. As sketched above, the article {\S tɪŋ}  in
(\ref{GRM-np-type-det2-b}) establishes that a particular disposition of the
father is known  by both the speaker and the addressee,  and the speaker
asks, with the question word {\S ɲineã} `how',   for details. 

The two  articles {\S a} and {\S tɪŋ}  are not in complementary distribution.
The article {\S tɪŋ} may occur following the head of a possessive noun phrase,
although it is not attested  following a weak pronouns. When the articles {\S
a} and {\S tɪŋ} co-occur,  language consultants could omit
the preposed {\S
a}  without affecting the interpretation of the
 proposition. 

Notice that any of the terms {\it article}, {\it determiner} and
{\it demonstrative} could have been used to identify {\S a} and {\S tɪŋ}.
Similar forms/functions are labelled differently in the
literature; for instance \citet[47]{Bodo97} calls the prenominal {\S a} in
Dagaare an article (`the')  and  {\S nyɛ} a demonstrative (`this').  Yet, an
analysis of the paradigms in (\ref{GRM-np-det-dem}),  in which Central Dagaare
is
included for illustration,\footnote{Central Dagaare as it is spoken in Nadowli,
Sombo, Dafiama, etc. Thanks to John Gaanaa for providing the examples. } has
never been offered. It is shown that a sequence of two demonstratives (in
\citeauthor{Bodo97}'s term)  can occur in postnominal postition.


% \begin{minipage}[h]{12cm}
\begin{exe}
 \ex\label{GRM-np-det-dem}{\it Corresponding  noun phrases in Chakali and
C.  Dagaare} 
\begin{xlist}
\begin{multicols}{2}

 \ex\label{GRM-np-det-dem-dag}{\it Central Dagaare}\\
à bíé   `the child'\\
à bíé ŋà   `the child this'\\
à bíé  ŋánɛ́ɛ́ ŋà  `the child this this'\\
 \textasteriskcentered  a bie  ŋa ŋanɛɛ  

 \ex\label{GRM-np-det-dem-cli}{\it Chakali}\\
a bie   `the child'\\
a bie tɪŋ   `the child this'\\
a bie haŋ tɪŋ  `the child this this'\\
 \textasteriskcentered a bie  tɪŋ haŋ  

\end{multicols}
 \end{xlist}
\end{exe}
% \end{minipage}
%\vspace*{15pt}





The question raised by paradigm  (\ref{GRM-np-det-dem}) is  whether one
should call {\S tɪŋ} a demonstrative (and not an article) based  on the English
glosses  supplied by  \citet[47]{Bodo97}, or show that Chakali and Central
Dagaare stack determiner-like function words in an unusual way and that this is
a problem for the general description of noun phrases.\footnote{The
demonstrative {\F haŋ} `this' is presented in section \ref {sec:GRM-demons}.}
The latter position is
chosen and a description of  {\S a} and {\S tɪŋ} is provided above. The
noun phrase {\S a bie haŋ tɪŋ} means `this child of which we (both speaker and
hearer)   have a familiar/common representation'.  The discourse implications
 of  {\S a} and  {\S tɪŋ}  need 
further study. 

Now that the indefinite and definite noun phrases have been presented, the
subsequent sections introduce the elements which can compose  either  indefinite
or  definite noun  phrases.


% The position taken here is
% that only article and demonstrative exist in the language; as opposed to the
% latter,  the former lacks deictic power and cannot function as noun
% phrase on its own.  


\subsection{Nouns}
\label{sec:GRM-noun}

In this section the elements admitted in the
schematic representation (\ref{sec:GRM-noun-strcut}) are discussed.

\begin{exe}
\ex\label{sec:GRM-noun-strcut}
[[ {\sc lexeme}]$_{stem}$ - [{\sc noun class}]]$_{n}$
\end{exe}

A stem may have 
nominal or verbal lexeme status. The latter has either a state (i.e. stative) or
a process (i.e. active) meaning.  A stem can be either atomic or complex and a
noun class suffix may be overt or covert.  In a
 process which turns a lexeme into a noun-word,  the noun class provides the
syntactic category {\it noun}. 




\subsubsection{Noun classes}
\label{sec:GRM-noun-classes}

The accepted view is that ``the Gurunsi languages, and indeed all Gur languages,
had historically a system of nominal classification which was reflected in
agreement. The third person pronominal forms and other parts of speech were at a
certain time a reflection of the nominal classification''  \citep{Nade89}.
 Similar affirmations are present in \cite{Mane69b, Waa71, Nade82, Nade98,
Tcha07}.  In this section and in section
\ref{sec:GRM-gender}, it is suggested that
an eroded form of this `reflection' is still observable in Chakali.
\cite{Brin07c} claims that in Chakali inflectional class
(i.e. noun class) and agreement class (i.e. gender) should be distinguished and
analyzed as separate phenomena at a synchronic level.

 The identification of noun classes is based on non-syntagmatic evidence; noun
class is a type of inflectional  affix, independent of agreement
phenomena, where the values of number
and class are exposed. In Chakali, as in all  other Southwestern Grusi  
languages,\footnote{\citet[136]{Nade98} state that ``[i]n
Vagla most traces
of this [noun-class system where paired singular/plural noun affixes correlate
with concording pronouns and other items] system have been lost. The
morphological declensions of nominal pluralization have not yielded to a clear
analysis''.  Even though the authors do not attempt to allot nouns into classes,
Marjorie Crouch's field notes (1963, Ghana Institute for Linguistics, Literacy
and Bible Translation (GILLBT)) present seven classes. Nominal classifications
are proposed for other SWG languages (number of classes for each language in
parenthesis): Sisaala of Funsi in \cite{Rowl66} (2), Sisaala-Pasaale in
\cite{Mcgi99} (5) and Isaalo in \cite{Mora06} (4).  The number of classes is of
course determined by the linguist's analysis.\label{foot:noun-class}}  the
values are exposed by
suffixes: number refers to either singular or plural, and class can be regarded
as phonological and/or semantic features encoded in the lexemes for the
selection
of the proper pair of singular and plural suffixes. This will be considered in
section \ref{sec:GRM-sem-ass-crit}. 



 \begin{table}[!h]
 \caption{The five most frequent noun classes \label{tab:GRM-synop-nc}}
   \centering
   \begin{Gtabular}{lccccc}

 \Hline
             &  {\sc cl.1} & {\sc cl.2}  & {\sc cl.3} & {\sc cl.4} & {\sc cl.5} 
 \\  [1ex] \hline
{\sc sing} & -V&  \O&  \O& -V  & \O \\
{\sc plur} & -sV& -sV & -V & -V  & -nV\\ 
 \Hline
   \end{Gtabular}
 \end{table}



One method used to identify the noun classes of a language appears in
\citet[23]{Rowl66}. The author writes that ``[t]he nouns in Sissala may be
assigned to groups on the basis of the suffixes for singular and plural''. 
 According to this definition, there are nine noun 
classes, of which four are rare.   A synopsis is displayed in table 
\ref{tab:GRM-synop-nc}, and each
of them is discussed below. 




% Notice that some of these representations can either be projected by one
%lexeme
% or by the combination of  one lexeme and a vowel suffix. The distinction is
% particularly needed, for instance,  in the decription of nominal morphology
% where a suffix is
% added to a stem to form a singular or a plural. For example we describe the
%word
% {\S bie} `child' as being formed by the stem {\S bi}    and a singular suffix
% vowel, but the word {\S taa} `language'  as being formed by the stem {\S taa} 
 
% and a  zero-suffix for singular.  Noun classes are discussed in section
%\ref{}. 

 
 %bring to chapter on morphology of nominal classes
%Consider the paradigmatic pair  {\S bie} ``child'' and {\S bise} ``children''.
% Let's assume that this particular nominal inflection add one mora  to the root
% and further that  {\S bi} is a monomoraic root. In inflecting for singular a
% vowel is suffixed to the root as in \ref{}. The mora in (\ref{child+sg}b) and 
% (\ref{child+sg}c) are nominal class suffixes. The composition in \ref{}
%reflect
% the situation where a mora is added by inflection and the two resulting moras
% are syllabify as one syllable in the case of ``child'' but two in the case of
% ``children'' 


\paragraph{Class 1}
\label{sec:class1}

Class 1 allows a variety of stems:  CV, CVC, CVVCV and CVCV are possible.
It gathers the nouns whose singular is formed by a single vowel
suffix {\S -V} and plural by a
light syllable {\S -sV}.


\begin{table}[h]

\caption{Class 1 \label{tab:freq-noun-class-1}}
\centering
%\subfloat[][{\sc class 1}]{
 \begin{Gtabular}{lllll}
  \Hline
{\sc class} & Stem    & {\sc sg.} &   {\sc pl.} & Gloss \\ [1ex] 
\hline
{\sc cl.1}  &   va   &  váà   &  váꜜsá  & dog \\ 

%{\sc cl.1}  &  hɛn   &  hɛ̀ná   &  hɛ̀nsá  & bowl \\
%{\sc cl.1}  &  da   &  dáá   &  dààsá & tree\\

{\sc cl.1}  &  pɛn   &  pɛ̀ná   &  pɛ̀nsá  & moon\\
{\sc cl.1}  &  gun   &  gùnó   &  gùnsó  & cotton \\
{\sc cl.1}  &  tʃuom   & tʃùòmó  & tʃùònsó   & togo hare\\
{\sc cl.1}  &  bi   &  bìé   &  bìsé  & child\\
{\sc cl.1}  &  gbiegie   &gbìègíè   &  gbìègísè  & type of hawk  \\

  \Hline
 \end{Gtabular} 
 %}

\end{table} 

 The quality of the vowels of the singular and plural is
determined by
the quality of the stem vowel and the harmony rules in operation. The rules were
stated in section \ref{sec:vowel-harmony} and correspond to the noun class
realization rules given in 
(\ref{ex:GRM-Hrules}).


\begin{exe}
  \ex\label{ex:GRM-Hrules}
\begin{xlist}
\ex\label{ex:mod-front-suffix}
-(C)V_{nc} $>$ [ $\beta${\sc ro},  {\sc +atr}, {\sc -hi}]  / [ $\beta${\sc ro}, 
{\sc +atr}] C* \_  \\

A noun class suffix vowel becomes {\sc +atr} if preceded by a {\sc +atr}
stem vowel, and shares the same value for the
feature {\sc ro}  as the one specified on the preceding (stem) vowel. A noun
class suffix is always {\sc -hi}.\\



 \ex\label{ex:low-suffix}
-(C)V_{nc} $>$ {\sc +lo}  / {\sc -atr} C* \_ \\

A noun class suffix vowel becomes {\S +lo} if the preceding stem vowel is either
{\S
ɪ}, {\S ɛ}, {\S ɔ}, {\S ʊ} or {\S a}.\\



\end{xlist}
\end{exe}


 



 
 \paragraph{Class 2}
\label{sec:class2}
 
Table \ref{tab:freq-noun-class-2} displays  nouns assigned to
class 2. Typically, this class consists of nouns whose stems are CVV or CVCV.
While the singular form  displays no overt
suffix,  {\S -sV} is suffixed onto the stem to form the plural.  

\begin{table}[h]


\caption{Class 2 \label{tab:freq-noun-class-2}}
\centering
%\subfloat[][{\sc class 2}]{
 \begin{Gtabular}{lllll}
  \Hline
{\sc class} & Stem    & {\sc sg.} &   {\sc pl.} & Gloss \\ [1ex] 
\hline

%{\sc cl.2}  &nãã &  nã̀ã̀    &    nã̀ã̀sá & leg \\
{\sc cl.2}  &  daa   &  dáá   &  dààsá & tree\\
%{\sc cl.2} &  tii     &  tìì   &  tìsè  & akee tree  \\
{\sc cl.2}  &  bɔla    &    bɔ̀là   &  bɔ̀làsá  &  elephant\\
%{\sc cl.2}  &  bʊɔ    &  bʊ̀ɔ́   &  bʊ̀ɔ́sá  &  hole\\
%{\sc cl.2}  &  joŋ   &  jòŋ́   &  jósó  & slave  \\
%{\sc cl.2} &  ziŋ  &  zíN   &  zísé  & tail \\
{\sc cl.2} &  kuoru    &  kùórù   &  kùórùsó  & chief \\
{\sc cl.2} &tomo &tòmó & tòmòsó& type of tree\\

%{\sc cl.2} &  ŋmɛŋ   &  ŋmɛ̀ŋ   &  ŋmɛ̀sà  & rope \\
%{\sc cl.2}  &   tuto    &tútò   &  tùdùsó  & mortar  \\
{\sc cl.2} &  bele  &    bèlè  &  bèlèsé &type of wild dog  \\
{\sc cl.2} & tii   &  tíì    &  tísè & type of tree \\
%put as type of tree
  \Hline
 \end{Gtabular} 
% }

\end{table}



 The rules in  (\ref{ex:GRM-Hrules}) capture the majority of the
singular/plural pairs of class 1 and 2. However, it is insufficient in some
cases, that is, there are cases which raise uncertainty in the allotment of
the pairs into one class or the other. Consider the examples in
table \ref{tab:uncer-noun-class}.


\begin{table}[h]
\caption{Uncertain class 1 or 2 \label{tab:uncer-noun-class}}
\centering
%\subfloat[][{\sc class 3}]{
 \begin{Gtabular}{lllll}
  \Hline
 {\sc sg.} &   {\sc pl.} & Gloss \\ [1ex] 
\hline
dʊ̃́ʊ̃̀  & dʊ̃́sꜜá	& 	python\\
kìrìmá & kɪ̀rɪ̀nsá & tsetse fly\\
kɔ̀wɪ̀ɛ́	 & kɔ̀wɪ̀sá	 & soap\\
lɛ́hɛ́ɛ́	& lɛ̀hɛ̀sá&	cheek\\
%hõõ	& hõsa	& grasshopper\\
  \Hline
 \end{Gtabular} 


\end{table}


 Two questions are raised by looking at the data in table
\ref{tab:uncer-noun-class}: (i) What is the stem of these nouns?  (ii) Is
there a good reason to favor final vowel deletion instead of insertion, e.g.
 /kɪrɪma/ vs. /kɪrɪm/?
Addressing  the first question, consider the first pair of words of table
\ref{tab:uncer-noun-class}, i.e. {\S dʊ̃ʊ̃}  and {\S dʊ̃sa}. On the one hand, if
 {\S dʊ̃} is treated as   the stem and  the word for `python' is assigned to
class
1,   the refutation of the rule in   (\ref{ex:GRM-Hrules}) must be explained,
i.e.
vowel suffixes are always {\sc -hi}.  On the other hand, if  the stem
is  {\S dʊ̃ʊ̃},  a deletion rule which reduces the length of the 
vowel, i.e. {\S /dʊ̃ʊ̃-sa/}  $\rightarrow${\S [dʊ̃́sꜜá]},  must be stated.
Such a decision  would
assign
the word for `python' to class 2.  The decision taken here is to respect the
rule in
(\ref{ex:GRM-Hrules}), which is empirically supported, and assume an {\it ad
hoc} deletion rule. The deletion rule may be explained by general prosody,
something which is not considered here. The word pairs in table
\ref{tab:uncer-noun-class} are assigned the following classes: `python' is in
class 2 and the last stem vowel is deleted in the plural, `tsetse fly' is in
class 1 and its stem is /kirim/, `soap' is in  class 1 and its stem is /kɔwɪ/,
and finally  `cheek' is in class 2 and the last stem vowel is
deleted in the plural.



 \paragraph{Class 3}
\label{sec:class3}

Nouns in class 3 generally have a sonorant coda consonant, i.e. {\S l}, {\S n}, 
{\S r}, etc. Class 3 contains nouns whose singular forms have no overt
suffix and plural forms  which have a single vowel as suffix. As for class 1 and
2, the
plural vowel suffix of class 3 is determined by the harmony rule given in
(\ref{ex:GRM-Hrules}).



\begin{table}[h]
\caption{Class 3 \label{tab:freq-noun-class-3}}
\centering
%\subfloat[][{\sc class 3}]{
 \begin{Gtabular}{lllll}
  \Hline
{\sc class} & Stem    & {\sc sg.} &   {\sc pl.} & Gloss \\ [1ex] 
\hline

{\sc cl.3}  &  nɔn    &  nɔ́ŋ   &  nɔ́nã́  &  fruit\\
%{\sc cl.3}  &  ɲiŋ    &  ɲíŋ   &  ɲíŋá  &  tooth \\
%{\sc cl.3}  &  par    &  pár   &  párá  &  hoe\\
%{\sc cl.3  &  kUr    &  k\'Ur    &  k\'Ur\'U  & bench  \\
{\sc cl.3}  &  hããn    & hã́ã̀ŋ   &  hã́ã̀nà  & woman \\
{\sc cl.3}  &  gɔŋ    &  gɔ́ŋ     &  gɔ́ŋá  & river  \\
%{\sc cl.3}  &  hamoŋ    &  hàmõ̀ŋ     &  hàmõ̀nà  & child  \\
{\sc cl.3}  &  nar    &  nár   &  nárá  &  person\\
%{\sc cl.3  &  bUɔ ŋ   &  bʊʊ̀ŋ    &  b\'Uná  & goat \\
%{\sc cl.3}  &  tɔn    &  tɔ́ŋ    &  tɔ́ná  & skin/book  \\
%{\sc cl.3}  &  sɔŋ    &  sɔ́ŋ    &  sɔna  &  name\\
{\sc cl.3}  &  ʔol     &  ʔól    &  ʔóló  & type of mouse \\
%{\sc cl.3}  &  ʔul     &  ʔúl    &  ʔúló  &  navel\\
{\sc cl.3}  & butet    &   bùtérː  &  bùtété & turtle \\
{\sc cl.3}  &   sel  &   sélː  & sélé  & animal \\
  \Hline
 \end{Gtabular} 
 %}
 

\end{table}
 
 
 
 \paragraph{Class 4}
\label{sec:class4}

The major characteristic of class 4 is that all the stems have a final
syllable consisting of  {\sc [+hi, -ro]} vowel(s) (see table
\ref{tab:freq-noun-class-4}) .  Class 4 is analyzed in the following way: in
both  the singular and the plural, a  vowel is added to the stem, i.e. V]\# $>$
V]-V\#. The suffix vowel of the singular is always an exact copy of the stem
vowel.  If the stem vowel is {\sc [+atr, +hi]} the plural suffix vowel is {\S
-e},
 and if the stem vowel is  {\sc  [-atr, +hi]}, the  plural suffix vowel  {\S
-a}.
This low vowel is then raised due to the height of the stem vowel. In normal
speech, one can perceive either  {\S -a} or {\S -ɛ} in that position. Given the
rules in (\ref{ex:GRM-Hrules}),  class 4 is certainly the most problematic in
terms of uniformity. However,  class 4 is productive. A similar noun class was
found in both Tampulma, Vagla and  Dɛg. 


 
 \begin{table}[h]
\caption{Class 4 \label{tab:freq-noun-class-4}}
\centering

 \begin{Gtabular}{lllll}
  \Hline

{\sc class} & Stem    & {\sc sg.} &   {\sc pl.} & Gloss \\ [1ex] 
\hline

{\sc cl.4}  &  begi   &  bégíí    &  bégíé  & heart \\
{\sc cl.4}  &  si   &  síí    &  síé  & eye\\
{\sc cl.4}  &fili &fílíí&fílíé&bearing tray\\
{\sc cl.4}  &  bɪ   &  bɪ́ɪ́    &  bɪ́á  & stone \\
{\sc cl.4}  &  wɪ   &  wɪ́ɪ́    &  wɪ́ɛ́  & matter, thing  \\
{\sc cl.4}  &  wɪlɪ   & wɪ́lɪ́ɪ́   &  wɪ́lɪ́ɛ́  & star \\
  \Hline
 \end{Gtabular}
\end{table} 



Class 4 also includes nominalized verbal lexemes.  In section
\ref{sec:GRM-verb-act-stem},  it is shown that one way to make  a noun from a
verbal lexeme is
to suffix a  high-front vowel to the verbal stem. For instance,  the lexeme  {\S
zɪn} may be translated into English `drive', `ride' or `climb'. In the word  
{\S kɪ́nzɪ̀nɪ́ɪ́} `horse', {\it lit.} thing-riding, the suffix  -[{\sc +hi,
-ro}]  is added to the verbal lexeme {\S zɪn} making it nominal.
Consequently,
the plural of {\S kɪ́nzɪ̀nɪ́ɪ́} `horse'  is {\S kɪ́nzɪ̀nɪ́ɛ́}. The sequences {\S
-ie} and {\S -ɪɛ} of class 4  often coalesce and may be  perceived as {\S -ee}
and {\S -ɛɛ}
respectively. 
 
 
 \paragraph{Class 5}
\label{sec:class5}


 The monosyllabic stems of class 5  nouns can either be CVV or CVC. Class 5
consists of nouns which  form their singular with no overt suffix and form their
plural with the suffix {\S -nV}. The quality of the consonant is determined by
the stem and the place assimilation rules introduced in section
\ref{sec:focus-forms}, some of which are repeated in  (\ref{GRM-cl-5}). The
vowel of the plural suffix is determined by the stem vowel and the rules in 
(\ref{ex:GRM-Hrules}). 
 


\begin{exe}
\ex\label{GRM-cl-5}
{\it Class 5 suffix -/nV/ surfaces -[lV] if the  coda consonant of the stem is
[l].}\\
{\sc -/[nasal]V/}_{nc} $>$  {\sc -/[lateral]V/}_{nc} /  {\sc [lateral]} \_\\

\end{exe}


 
 \begin{table}[h]
 \caption{Class 5 \label{tab:freq-noun-class-5}}
\centering
 \begin{Gtabular}{lllll}
  \Hline
{\sc class} & Stem    & {\sc sg.} &   {\sc pl.} & Gloss \\ [1ex] 
\hline

{\sc cl.5}  &  zɪŋ    &  zɪ̀ŋ́    &  zɪ́nná  &  type of bat \\
{\sc cl.5}  &hʊ̃ŋ&hʊ̃̀ŋ́&hʊ̃́nná& farmer or hunter gear\\
{\sc cl.5}  &  kuo    &  kùó   &  kùónò  & farm \\
{\sc cl.5}  &  ɲuu    &  ɲúù   &  ɲúúnò  & head  \\
%{\sc cl.5}  &  sũũ    &  sũ̀ũ̀   &  sũ̀ũ̀nó  &  guinea fowl \\
{\sc cl.5}  &  vii    & víí   &  vííné &   type of cooking pot\\
{\sc cl.5}  &diŋ&díŋ & dínné & fire \\
{\sc cl.5}  &pel &pél & péllé & burial specialist\\

  \Hline
 \end{Gtabular}
\end{table} 
 

 \paragraph{Nasals in noun classes}
\label{sec:gene-sum}


 
Apart from the singular of class 4,  much of the same vocalic morpho-phonology
is found in all classes. This was reduced to the two rules in
(\ref{ex:GRM-Hrules}). Furthermore, in all the noun classes, the nasal
consonants surface differently depending on the phonological context. The rules
in  (\ref{ex:GRM-nrules}) predict the observed outputs and are derived from the
nasal assimilation rules in section \ref{sec:internal-sandhi-nasal-place}.
 
\begin{exe}
  \ex\label{ex:GRM-nrules}\textit{Possible outputs of  nasals}\\
\begin{xlist}
\ex\label{ex:GRM-Nrules}
 C[{\sc +nasal}]\   $>$ ŋ / \_ \# \\
 /hããn/  $>$     [hã́ã̀ŋ]  `female' {\sc cl.3.sg}   


\ex\label{ex:GRM-Msrules}
 /m/ $>$ n / \_  C [{\sc -labial, -velar}] \\
 /tʃuom/   $>$ [tʃùònsó]   `togo hares'  {\sc cl.1.pl}  

\ex\label{ex:GRM-NGsrules}
 /ŋ/ $>$ n / \_  C [{\sc -labial, -velar}] \\
/kɔlʊ̃ŋ/ $>$  [kɔ̀lʊ̀nsá]  `wells'   {\sc cl.2.pl} 

\end{xlist}
\end{exe}

The rule in  (\ref{ex:GRM-Nrules})  says that  any nasal consonant occurring
word finally becomes [ŋ]. The rule in (\ref{ex:GRM-Msrules}) changes a bilabial
nasal into an alveolar when it precedes a non-labial and non-velar consonantal
segment. The rule in (\ref{ex:GRM-NGsrules}) changes a velar nasal into an
alveolar in the same environment.

 
 \paragraph{Generalization and summary}
\label{sec:gene-sum}

While the method proposed suggests that one should look for pairs of forms, the
present classification treats phonologically empty suffixes as `exponents'. What
counts as a noun class is the paradigm determined by the  inflectional
pattern of the lexeme. The five  most frequent pairs were presented in tables
\ref{tab:freq-noun-class-1} to \ref{tab:freq-noun-class-5} and the exponents are
gathered in  table \ref{tab:GRM-nc-exponent}.\footnote{The percentage is based
on a list
of 978 singular/plural pairs  (lexicon 02/10/10 version). The five classes in
table \ref{tab:GRM-nc-exponent} make up 88\% of the nouns which are assigned a
class in the lexicon.} 
%see all numbering/percentage as it was changed

 \begin{table}[!h]
 \caption{The five most frequent noun classes   \label{tab:GRM-nc-exponent}}
   \centering
   \begin{Gtabular}{lccccc}
 \Hline
             &  {\sc cl.1} & {\sc cl.2}  & {\sc cl.3} & {\sc cl.4} & {\sc cl.5} 
 \\  [1ex] \hline
{\sc sing} & -V&  \O&  \O& -V  & \O \\
{\sc plur} & -sV& -sV & -V & -V  & -nV\\ \hdashline
                &      8\%&     32\%  &     23\% &   17\%   & 8\%\\
 \Hline
   \end{Gtabular}
 \end{table}



In practice the most productive and regular patterns are those recognized as
noun classes. However, some words do not fit perfectly into the patterns
described
above but are not totally alien to genetically related languages and the
reconstructions of Proto-Grusi in \cite{Mane69a, Mane69b} and
Proto-Grusi-Kirma-Tyurama  in \cite{Mane82}.   In fact, there are more
possibilities
and surfaces forms when the  classes  ({\it sg./pl.})  {\S \O/\O},  {\S \O/ta},
{\S \O/ma} and {\S
ŋ/sV} are included in the classification. Examples are given  in table
\ref{tab:GRM-less-pro-nc}.  
 
\begin{table}[h]
\caption{Noun classes 6, 7, 8 and 9 \label{tab:GRM-less-pro-nc}}
\centering
  \begin{Itabular}{lllll}
  \Hline
{\sc class} & Stem    & {\sc sg.} &   {\sc pl.} & Gloss \\ [1ex] 
\hline

{\sc cl.6}  & dʒɪɛnsa & dʒɪ́ɛ̀nsá & dʒɪ́ɛ̀nsá & twin\\
{\sc cl.6}  &kapʊsɪɛ &  kápʊ̀sɪ́ɛ́ & kápʊ̀sɪ́ɛ́ & kola nut\\
{\sc cl.6}  & kpibii & kpìbíí & kpìbíí & louse\\[0.2ex] \hline

{\sc cl.7}  & kuo & kúó &kùòtó  &roan antelope\\
{\sc cl.7}  &kie  &kìé & kìété & half of a bird \\
{\sc cl.7}  &fɔʊ̃ &fɔ́ʊ̃̀& fɔ̀tá & baboon \\[0.2ex] \hline
%{\sc cl.7}  &  taa    &  tàá    &  tàátá  & language  \\
{\sc cl.8}  & naal &náàl&nááləmà&grand-father\\
{\sc cl.8}  &ɲɪna &ɲɪ́nà&ɲɪ́námà &father\\
{\sc cl.8}  &  hɪɛŋ & hɪ́ɛ́ŋ &hɪ́ɛ́mbá &relative\\[0.2ex] \hline

 {\sc cl.9}  &  jo   &jóŋ̀ & jósò  & slave  \\
{\sc cl.9}  & zi &zíŋ̀ &zísè &tail\\
{\sc cl.9}  & ŋmɛ&ŋmɛ́ŋ̀&ŋmɛ́sà&rope\\

  \Hline
 \end{Itabular} 

\end{table} 


 The nouns in class 6 do not formally differentiate singular and plural.   Those
in class 7 mark their plural with the suffix {\S -tV} and  class 8 with the 
suffix {\S -mV}.  The singular exponent of class 7 and 8 is covert. Finally,
the nouns of class 9 have a suffix {\S -ŋ} in the
singular and {\S -sV} in the plural. In table \ref{tab:l-leasttive-class},  the
percentage of occurence of the less productive noun classes 6, 7, 8
and 9 is given.
 
  
 \begin{table}[h]
   \caption{Less productive  noun classes 
\label{tab:l-leasttive-class}}
   \centering
   \begin{Gtabular}{lcccc}

  
 \Hline
&  {\sc cl.6} 	& {\sc cl.7}   	&  {\sc cl.8} 	 & {\sc cl.9}    \\
[1ex] \hline
    {\sc sing} 	& \O 		& \O 		 & \O & -N \\
{\sc plur} 	& \O & -tV  & -mV & -sV\\ \hdashline
                &      7\%&            1.8\% &   0.9\%   & 0.8\%\\
 \Hline
     
   \end{Gtabular}
 \end{table}

In addition, there are pairs which can only imperfectly be reduced  to the nine
classes presented until now. However, the problem lies in the stem and not in
the inflectional patterns. For example the color terms ({\it sg./pl.}) {\S
pʊ̀mmá}/{\S pʊ̀lʊ̀nsá} `white' and {\S búmmó}/{\S bùlùnsó} `black'  do
not have comparable pairs and do not fit the noun classes described above. One
would expect *{\S pʊmmasa} to be the plural form for  `white' (also *tɪɪnama for
{\S tɪ̀ɪ̀ná}/{\S tʊ́mà} `owner'). Other examples are the  pairs {\S tɪ́ɛ̀}/{\S
tɛ́sà} `foetus' and {\S túò}/{\S tósó} `bow'. Also here, one expects  the
last vowel to delete in each of the plural forms instead of the penultimate one.
Moreover,  inconsistent class assignment across speakers, across villages, and
surprisingly different forms (predominantly in the plural) from the same speaker
on different elicitation sessions do arise, but the latter case rarely occurs. 
 

%also tʊ̀ɔ́nɪ́ã̀ tʊ̀ɔ́nsà  type of genet
%also kùòlíè  kùòlúsò  type of tree 
%apocope?
%  hɔ̃́ʊ̃̀
%  type of grasshopper
% \cl 2
%  hɔ̃́sà


 
\paragraph{Semantic assigment criteria}
\label{sec:GRM-sem-ass-crit}

Several authors have presented their views on  the semantic classification of
nominals.   The general idea is that there must be an underlying system which
can explain, first, why some words display identical number morphology, and
second, how these words are related in `meaning'. \citet[23]{Tcha07} shows that
Tem organizes its nominals on the basis of semantic values such as humanness,
size and countability. \citet[41]{Awed07} argues that nominal groupings  in
Kasem should take into consideration phonological and semantic characteristics,
in addition to other more cultural factors.  Similarly, \cite{Assi07}
argues at length on the shortcoming of traditional semantic rules and argues for
abandoning them. 

The semantic value of the noun class suffixes has proven difficult to
establish. It is possible that there are analogies in class assignment based on
semantic criteria but it is more likely that synchronically (i) the phonological
shape of the stem triggers the suffix type, and that (ii) some classes can be
identified as residues of former semantic assignment. Let us comment on each
point: 


\begin{enumerate}
\item[(i)]

Most class 3 nouns have  a sonorant consonant in the coda position,
the stems of  class 4 nouns must have their last vowel specified for  [{\sc -hi,
-ro}] and a typical class 2 noun is either   CVV or CVCV.  These are some of
the characteristics  described for the noun classes. It seems that the
phonological
shape of the stem plays a role in class assignment and that there is no
productive class
where most of its  members is assigned to a particular semantic domain.  Using
four features of the animacy hierarchy
of  \cite{Comr89}, i.e.  human $[${\sc
hum}$]$, animal (exclude human) or other-animate
and insects $[${\sc anim}$]$, concrete inanimate $[${\sc conc}$]$ and abstract
(inanimate) $[${\sc abst}$]$,  \cite{Brin08} shows that the noun
classes  do not encode any of these distinctions. Such
distinctions may have
been  expected given the nominal classification of other Gur languages. For
instance in Dagaare, an Oti-Volta language and linguistic `neighbor' of
Chakali, \citet[124]{Bodo94} presents the Class 2 (V/ba) as ``unique in that it
is the only class that has exclusively [+human] nouns in it''. From a
diachronic point of view, this suggests that Chakali has dropped all animacy
distinctions in the noun class system while preserving one distinction in
agreement (see section \ref{sec:GRM-gender}).



 \item[(ii)]
Geographically and genetically, languages related to Chakali had noun class
systems whose classifications were based, at least partially, on semantic
criteria. To my knowledge, the most conservative system today  within Grusi is
Tem (see {\it identification sémantique} in \cite{Tcha07}). When and how the
speakers of Chakali  classified nouns based on semantic criteria is impossible
to know,  but traces can be detected in  the   {\it less productive noun
classes}, that is class 6, 7, 8 and 9.
Some members of class 6 consist of
nouns with mass or abstract  denotations, i.e. rice,  louse, struggle, profit,
etc.  Class 7 also contains mass and abstract nouns, i.e. oil,  honey, water and
 taboo, but also bush animals such as bushbuck, waterbuck, baboon, roan
antelope and hartebeest. Class 7 represents 1.8\% of the noun sample
(see table \ref{tab:l-leasttive-class}) and  mass/abstract nouns and bush
animals
represent each  
30\% of class 7 membership. Class 8 is likely to be the class where kinship and 
human
classification terms were assigned, 
as
mother, father and `owner of' are among
remnant
members of that class.  Finally, a  common trait of class 9 may be
`elongated things', since words referring to  rope, arm, tail and ladder are
members. Yet, only eight nouns are
assigned to class 9. Despite the arbitrary nature of the semantic
assignment of class 9,  \citet[94]{Mane75} maintains that there are Oti-Volta
languages which show relics of  the Proto Oti-Volta class {\S *ŋu- *u},
which is  itself a remnant of Proto-Gur class 3   according to
\citet[11]{Mieh06}, and that this class contains ``les noms du bâton, du pilon,
du balai, de la corde, de la peau et du chemin''.  Although these nouns seem
to
 denote `elongated  things',   Manessy claims that they cannot contribute to
an hypothesis. Generally, howewer, the fact that members of classes 6, 7, 8 and
9 are
similarly clustered in other languages suggests that these classes are remnants
of a more productive semantic assignment system. %six nouns
Beside semantic
domains, the simple empirical fact that homonyms are found with
different suffixes excludes a purely phonologically-based class assignment.
There is no way a speaker can correctly pluralize the stems {\S tii} `type of
tree' and {\S tii} `type of ants' based entirely on their (segmental)
phonological shape.\footnote{I put segmental in parenthesis since  homonyms {\it
with the same tonal melody} belonging to two different
classes have not yet been  found. The pair {\F pól} ({\sc cl.5}) `river' and
{\F pól}
({\sc cl.3}) 
`vein' may be treated as one example, but their meanings point to a
common etymology. Nevertheless, \cite{Bonv88}, \cite{Awed07} and \cite{Tcha07}
provide data to support a similar claim.} 
 

\end{enumerate}

A combination of both (i) and (ii) seems consistent with the data
observed. Chakali speakers seem to acquire the noun classes as French or Dutch
speakers acquire
the grammatical gender of inanimate entities which lack natural gender.

%mostly semantically arbitrary
%a healthy mix of lexical storage and 
%rule-based semantic and phonologica prediction

Finally, class assignment in complex stem nouns indicates  that the 
denotation of a word plays no role in determining its noun
class (see section \ref{sec:GRM-com-stem-noun}). 
The class of a complex stem noun is always determined by the rightmost stem.
Given that compounding is highly productive,  this purely  formal
process suggests that semantic criteria in
noun class assignment are inoperative.

\paragraph{Tone patterns of noun classes}
\label{sec:GRM-tone-p}

Some tonal melodies are identified. One of them is the  general tendency for
the singular and  plural words in a pair to display the same tonal melody. For
instance, a HL melody may be associated with both the singular and the plural,
e.g.  {\S zíŋ̀}/{\S zísè} `tail' ({\sc cl.9}) 
and   {\S lʊ́l̀}/{\S
lʊ́là} `biological relation'  ({\sc cl.3}). These cases are tonally regular. 
Another common pattern is when a singular noun displays a H melody, but the
plural a LH melody, e.g.  {\S dáá}/{\S dààsá} `tree' ({\sc cl.2}). While it
 seems that  the
plural suffix -{\S sV}  depresses a preceding H,  it does not do so in class 9
nouns.
The majority of class 4 nouns in the data available are high tone irrespective
of the number of moras and they are all tonally regular. Some cases involving
singular CVC words with moraic
coda exhibit the deletion of a low tone;  {\S zɪ̀ŋ́}/{\S
zɪ́nná} `bat' ({\sc cl.5}),   {\S gèŕ}/{\S gété} `lizard' ({\sc cl.3}) and
{\S sàĺ}/{\S sállá} `flat roof' ({\sc cl.3})  have a LH tonal melody in the
singular but  H in the plural. The downstep rule (section
\ref{sec:tone-intonation})  predicts that a high tone preceded by a low tone is
perceived as lower than a preceding high
tone, e.g. {\S váà} {\I HL},  {\S váꜜsá} {\I HꜜH}  `dog' ({\sc cl.1}).  In
spite of some variations,  it seems that there are recurrent
melodies. Representative examples are presented in
table \ref{tab:GRM-tm-nc-1-5}.
 
%  the singular and plural
% suffixes  -{\S V} is H if the preceding tone is H, e.g.  {\S píí}/{\S píé}
% `yam mound' ({\sc cl.4}).
 %fix table 
 \begin{table}[htb!]
   \caption{Tonal melodies in noun classes 1-5
\label{tab:GRM-tm-nc-1-5}}
   \centering
   \begin{Itabular}{lp{1cm}lp{1cm}ll}

 \Hline
{\sc class}    &  Tone  melody {\it sg.}  &   Singular   &  Tone  melody  {\it
pl.} &   Plural & Gloss
\\ [1ex]

\hline

{\sc cl.1} 	& 	HL   & váà & HꜜH & 	váꜜsá	& dog\\
	&   	LH &  gùnó&	LH& gùnsó	& cotton\\
& HL & tʃíníè & HL &  tʃínísè  & type of climber\\
	&  	L &  dɪ̀gɪ̀nà	& LH & dɪ̀gɪ̀nsá&	ear\\[0.2ex] \hline


{\sc cl.2}  &	H	&	 síé &	LH	 &
sìèsé&	face\\
&	L 	&	bɔ̀là 	&		LH&	 bɔ̀làsá	&
elephant\\
&	LH &	tòmó	&	LH &	tòmòsó	&	type of
tree\\
& 	LH 	& sòntògó & 		LH & 	sòntògòsó & base \\
&	HL &	júò	&	HLH 	&	júòsó	&
quarrel \\[0.2ex] \hline


{\sc cl.3} & 	H & 	hóg	& 	H 	&	 hógó	& 	bone\\
&      H 	&	sɔ́nná	&	H  &	sɔ́nnəsá
& lover\\
		&	LH &	gèŕ	&	HH	&	gété	
& lizard\\
		&	LH &	pààtʃák	&	LH 	&  pààtʃàgá 
& leaf\\[0.2ex] \hline


{\sc cl.4} &	H &	síí	&	H& síé	&	type of dance\\
		&	H &	bégíí		&	H & bégíé
& heart\\
& H & tʃɪ̃́ɪ̃́  & H & tʃɪ̃́ã́ & dawadawa seed\\
& LH& sòkìé & LH & sòkìété & type of tree \\[0.2ex] \hline




{\sc cl.5}&H &	víí	&	H &	vííné
& cooking 
pot\\
	      &	L &	bɔ̀g	&	L &	bɔ̀ɣənà 	&
type of tree\\
	      &	HL &	bámpɛ̀g	&	HL&	 bámpɛ̀gənà	&
half of nut\\
	      &	LH &	kùó	&	LHL &	kùónò	&	farm\\
	      &	LH &	 zɪ̀ŋ́	&	H&	 zɪ́nná	&	bat\\
	      &	L &	tʃàl 	&	LH&	 tʃàllá	&	blood\\
		&	LH &	sàĺ	&	H &	sállá	  &	flat
roof\\

  
 \Hline
   \end{Itabular}
 \end{table}





\paragraph{Noun class reconstruction}
\label{sec:GRM-noun-class-recons}
The numerical labeling of the noun classes in table \ref{tab:GRM-nc-exponent}
and \ref{tab:l-leasttive-class} is arbitrary. Given  the state of the
documentation on nominal classifications in other SWG languages, and the fact
that  almost all singular suffixes
have disappeared in today's SWG languages, a reconstruction  is practically
impossible. Nonetheless,  some 
observations  on similarities between the noun class
system in Chakali and other SWG noun class systems can be put forward. The
information sources are my field notes on neighboring languages, the
reconstruction of the
noun class suffixes of Grusi in \cite{Mane69a, Mane69b},  and the reconstruction
of noun classes in Gur in \cite{Mieh06}; the latter being for the most part an
up-date and synthesis of Manessy's work \citep{Mane69a, Mane69b, Mane75, Mane79,
Mane82, Mane99}. Needless to say, the following statements are first
impressions.

%need name of pasaale girl

Field notes on neighboring languages, supported with unpublished
material produced
by GILLBT's staff,\footnote{\label{ft:GRM-naden-donate}In 2008, Tony Naden gave
me  a copy of his ongoing
Vagla and Dɛg lexicons. I am also 
indebted to: Kofi Mensa (New
Longoro) for Dɛg, Modesta
Kanjiti  (Bole) for Vagla and Dɛg, Pastor Alex Kippo (Tuossa) for Vagla and 
Yusseh Jamani (Bowina) for Tampulma.}  provided relevant information on
the
(dis-)similarities of Chakali with other SWG languages. As in all SWG languages,
a typical Vagla noun class is characterized by  suffixation.
The most frequent plural markers in Vagla are {\S -zi}, {\S -nɪ} and
{\S -ri}. The pattern found in Chakali  class 4 is similar to one found in
Vagla, i.e.
({\it sg.}/{\it pl.}) {\S bàmpírí}/{\S bàmpíré} `chest',  {\S hūbí}/{\S
hūbé}  `bee' and   {\S gíngímí}/{\S gíngímé} `hill'.  In Dɛg,   the most
frequent plural markers are mid vowel suffixes, often rounded,  and
the {\S -rV}, {\S -nV} and  {\S -lV} suffixes, of  with which the vowel 
harmonizes
in
roundness and {\sc atr} with the stem vowel. Both Vagla and Deg display 
miscellaneous classes which are characterized by  a simple difference
in vowel quality between the last vowel of the singular and the plural, e.g. Dɛg
{\S dala}/{\S dale} `cooking place',  Vagla  {\S gbɔ́nā}/{\S gbɔ́nī}. Attested
alternations  ({\it sg.}/{\it pl.}) in Vagla are {\S -i}/{\S -e},  {\S -i}/{\S
-a},  {\S -a}/{\S -i}, 
{\S -u}/{\S -a},  {\S -o}/{\S -i} and  {\S -e}/{\S -i},   and in Dɛg {\S -a}/{\S
-e}, {\S -e}/{\S -a}, {\S -i}/{\S -e}, {\S -o}/{\S -i} and  {\S -i}/{\S
-a}.\footnote{These singular/plural pairings are extracted from the Vagla and
Deg lexicons (fn. \ref{ft:GRM-naden-donate}) and are not exhaustive.}  The noun
classes of Tampulma and Pasaale
correspond more to those of Chakali. Tampulma has at least
the
following class suffix pairs ({\it sg.}/{\it pl.}): {\S \O}/{\S -V}, {\S -i}/{\S
-e}, {\S
\O}/{\S -nV},  {\S  \O}/{\S -sV}, {\S  -V}/{\S -sV},  {\S -hV}/{\S -sV} and  
{\S \O}/{\S -tV}.
Tampulma displays similar harmony rules to those found in Chakali. Apart from
the singular suffix {\S -hV}, all the noun class suffixes in Tampulma are
manifested in Chakali.  Correspondingly, Pasaale reveals  pairs and
harmony rules similar to those of Chakali and Tampulma.\footnote{As mentioned in
footnote
\ref{foot:noun-class}, the number of noun classes is determined by the
linguist's analysis.  \citet[5-12]{Mcgi99} is a good example of 
the consequence of analyzing noun classes differently. For instance,  
 \citet[7]{Mcgi99} postulate a subclass  ({\it sg.}/{\it pl.})  {\F -l/-lA} for 
word pairs like {\F baal/baala} `man', {\F gul/gulo} `group', {\F
miibol/miibolo} `nostril' and  {\F mɔl/mɔlɔ} `stalk'. If these words were part
of the
Chakali data, they would have been alloted to class 2 ({\it -\O/-V}), that is,
I would have treated the /l/ as a coda consonant of the stem instead of a noun
class
suffix consonant. In addition, whereas I derive the quality of the vowel
entirely from harmony rules,  \citeauthor{Mcgi99} assume archiphonemes, like A
and E, which surface depending on harmony rules.}

It is important to keep in mind that the analysis in \cite{Mane69a, Mane69b} is
based on a very
limited set of SWG data,  most of the data being extracted from
\cite{Bend65}. He
stresses often the tentative nature of his claims and  sets forth more than one
 hypothesis on several occasions. The Chakali plural suffix of class 8 {\S -mV}
may be treated as a descendant of the Proto-Grusi Class {\S *B_{1}A}
\citep[32]{Mane69b}, class 9 {\S -ŋ} as a descendant of the Proto-Grusi Class
{\S
*NE}   \citep[37,41]{Mane69b}, class 1 {\S -V} as a descendant of the
Proto-Grusi
Class {\S *K_{1}A}  \citep[39]{Mane69b}, classes 1, 2 and 9 {\S -sV} as 
descendants
of the Proto-Grusi Class  {\S *SE}  \citep[39]{Mane69b} and class 7 {\S -tV} as
a descendant of the Proto-Grusi Class {\S *TE/O}  \citep[43]{Mane69b}. The vowel
suffixes of class 1 and 4 may also descend from the Proto-Grusi Class {\S *YA}
\citep[34]{Mane69b}. 

In consulting \citet[7-22]{Mieh06}, Chakali's  most frequent plural suffix  {\S
-sV}, found in class 1, 2 and 9, would seem to correspond to Proto-Gur Class 13
*{\S
-sɪ}, the plural suffix of class 5 {\S -nV} to  Proto-Gur Class 2a *{\S -n.ba}
or Proto-Gur Class 10 *{\S -ni}, class 7 {\S -tV} to Proto-Gur Class 21 *{\S
-tʊ} and class 8  {\S -mV}  to Proto-Gur Class 2 *{\S -ba}. The singular suffix
{\S -ŋ} would correspond to Proto-Gur Class 3 *{\S -ŋʊ}.


Needless to say, these observations  deserve further investigation. Even though
there is  literature to support the reconstruction of the Gur classes, little
can be done in the SWG
area unless descriptions of  nominal classifications in the languages  Winyé,
Vagla, Tampulma, Phuie,  Deg  and the dialects of Sisaala  are made available
(see  
figure \ref{fig:Gur-tree}).
A synthesis of these
descriptions could be
compared to  ``better-documented'' nominal classfications of Grusi languages 
such as Kasem (Northern Grusi, \cite{Awed79, Bonv88, Awed03}),  Lyélé (Northern
Grusi, \cite{Delp79}),  Lama  (Eastern Grusi, \cite{Arit87, Ours89}), Kabiyé
(Eastern Grusi, \cite{Tcha07}),  Chala   (Eastern Grusi, \cite{Klei00}) and Tem
(Eastern Grusi, \cite{Tcha72, Tcha07}), to evaluate the Proto-Grusi noun class
suffixes of \cite{Mane69b} and Proto-Gur of \cite{Mieh06}, and to reconstruct
the nominal classifications of SWG  languages.


\subsubsection{Atomic stem nouns}
\label{sec:GRM-sim-bas-noun}

The notion of stem in the present context refers to the host of a noun class
suffix or the  host of a nominalizer, i.e. the element which conveys the lexical
meaning and  to which affixes attach. A stem can be either irreducible or
reducible morphologically: they are referred to as atomic  and complex stem
respectively.  Complex stems are presented in  section
\ref{sec:GRM-com-stem-noun}.   An atomic stem is always a  nominal or a verbal
lexeme.  A verbal lexeme may either be of the type `state' or `process'. Three
types of nominalization formation (i.e. nominalizers) are attested: suffixation,
prefixation and reduplication.  

%A few prefixes are attested. Some are treated as classifiers (see section
%\ref{sec:}). Other formal means for nominalization are suffixation and
%reduplication. 

\paragraph{Nominal  stem}
\label{sec:GRM-nom-stem}
A nominal stem is a predicate denoting a class of entities.   Nouns composed by
the combination of  a nominal stem and a noun class affix are the most common. A
nominal stem can be juxtaposed with various noun class affixes, yielding forms
with
different meanings. For instance, the lexeme {\S baal} is associated with the
general meaning `man'. In a context where the lexeme is used in the singular,
{\S baal} can mean either `a man' or `a husband'. Given the same context but
used in the plural, the lexeme {\S baal} is disambiguated by the 
plural suffix it takes;  {\S baala} `men'  ({\sc cl.3}) and  {\S baalsa}
`husbands'  ({\sc cl.2}). Another
example is the lexeme {\S natɔʊ} `shoe'. The word {\S natɔwa} refers to `a pair
of shoes' whereas {\S natɔʊsa} refers to `pairs of shoes' or `unsorted shoes'.
Evidence from other Grusi languages suggests that the situation where    lexemes
are found in different noun classes was certainly a   more common
phenomenon than it is today \cite[126-128]{Bonv88}. This may coincide
with semantically richer
noun class suffixes. In addition, for many noun classes the singular forms are
not overtly marked and the plural forms are by and large less frequent. This
situation makes it difficult to provide the necessary evidence which would
demonstrate that nominal stems are found together with different noun classes.  

Nominal stems exist in opposition to the verbal ones. To classify a stem in such
a dichotomy, the linguistic test carried out consists of placing the stem in
several core predicative positions, i.e. positions where an
argument must appear. If the stem is perceived as grammatical in the given
context by 
language consultants, it cannot be nominal. For instance, in French the word
{\it bille}  `marble' cannot take a nominal argument in a non-genitive
predication, e.g. *Marie billait/a billé  `Mary marbled/has
marbled'.\footnote{\label{ft:GRM-fre-eng-deri}French and English are not 
appropriate languages to use
for the point I want to make  because of their preference  for the categorial
derivation n$>$v. I believe that
Chakali cannot as easily derive a verb from a noun.}


The examples in
(\ref{GRM-nom-or-verb}) illustrate a simple classification procedure. It uses a
frame where the predicate is in the 
perfective aspect and  the same 
predicate, as opposed to the argument,  is in focus.

\begin{exe}
 \ex\label{GRM-nom-or-verb}
 \begin{xlist}
\ex /di/  `eat' $\rightarrow$ {\I ʊ  dijoo} |{\sc 3.sg} eat.{\sc pfv.foc}| `he
ate'
  \ex /kpeg/  `hard' $\rightarrow$ {\I ʊ kpegeo}  |{\sc 3.sg} hard.{\sc
pfv.foc}| `he is strong'
   \ex /sɪama/  `red' $\rightarrow$ *{\I ʊ sɪamao}, but   {\I ʊ sɪareo}   |{\sc
3.sg} red.{\sc
pfv.foc}|  `it is
red'
   \ex /bi/  `child' $\rightarrow$  *{\I ʊ bio} 
 % \ex //
 \end{xlist}

\end{exe}


The test displayed in (\ref{GRM-nom-or-verb}) shows that  {\S di} and  {\S
kpeg} are verbal,  whereas {\S sɪama} and {\S bi} are not. In
chapter \ref{sec:COL-chap}, it will be shown that color terms change forms
depending on whether
they occur  in a nominal or verbal context.  


\paragraph{Verbal process stem}
\label{sec:GRM-verb-act-stem}

%Somewhere should be included a section on derivative affixes (and/or other
%types of nominalizer) 

Verbal process stems are predicates denoting non-stative events, that is,
processes or actions. These predicates are embedded in  nominalizations of the
type `agent of X' and `action of X' , where X replaces the meaning of the
verbal stem. Table \ref{tab:GRM-nom-process} displays  two types of
nominalization
formation  involving verbal process stems. 

\begin{table}[htb!]

\centering
\caption{Examples of nominalization of verbal process stem
\label{tab:GRM-nom-process}}
 \begin{Itabular}{llll}
 \Hline
Sem. value & Verb. process stem & Nmlz & Form\\
 \hline

Agent of X &  gʊɔ `dance' &  -/r/ & gʊɔr `dancer'\\
Agent of X &  kpʊ  `kill' &   -/r/  & kpʊra  `killer'\\
Agent of X &   buol   `sing' &  reduplication &   buolbuolo  `singer'\\
Agent of X &   summe `help' &  reduplication &   susumma `helper'\\[1ex]\hline

Action of X  &  gʊɔ `dance' &  -/[{\sc +hi, -bk}]/ & gʊɔɪɪ `dancing'\\
Action of X &  kpʊ  `kill' &  -/[{\sc +hi, -bk}]/  & kpʊɪɪ  `killing'\\
Action of X  &   buol   `sing' &-/[{\sc +hi, -bk}]/  & buolii    `singing'  \\
Action of X  &  summe `help'  &-/[{\sc +hi, -bk}]/  &  summii  `helping'
\\
\Hline
 
 \end{Itabular} 

\end{table} 


% The first column describes in prose the meaning of each nominalization,
%the second column provides the stem, the third column provides  the
%nominalization formations and the fourth provides the  translation.


 In table \ref{tab:GRM-nom-process}, the column entitled semantic value (i.e.
Sem. value)  identifies the meaning of the verbal nominalization. In such a
context, `agent of X'  refers to the instigator or doer of the state of affairs
denoted by  the predicate X and
the nominalization is generally accomplished by the suffix -/{\S
r(a)}/.  However,  there are some expressions with the equivalent
agentive denotation which
do
not suffix    -/{\S r}/  to the  predicate, e.g.  {\S ʔɔra} `to sew' vs.
{\S ʔɔta} `sewer' and {\S maŋa} `to beat' vs.  {\S kɪŋmaŋana} `drummer'.  The
singular forms  are given in the fourth column: 
the plural of agent nominals  of
this type, i.e. nominalized by the suffix -/{\S
r}/, is made by  a single vowel
suffix  ({\sc cl.3}) whose surface
form  depends on harmony rules.\footnote{One language consultant
had a problem retrieving the plural of some agent nouns. He often repeated the
singular entry
for the plural. I interpret this as  either a situation where agent nouns do
not show differences in the singular and plural ({\sc cl. 6}), or different {\it
sg.}/{\it pl.}
forms exist but he could not retrieve them. The pair {\F kpʊra}/{\F kpʊrəsa}
`killer(s)'   is unusual.  The word {\F
sãsaar} means
`woodcarver' and not `car driver'  even though {\F sãã} can mean both `carve'
and `drive vehicle'.  People usually use  {\F lɔ́ɔ́lɪ̀sã́ã́r}, or the English
word {\F dərávɛ̀}, 
 which is common all over Ghana, to refer to a `car driver'. } Another verbal
nominalization process conveying `agent of X' is reduplication. The evidence
suggests that  only the first syllable  is reduplicated.

The second nominalization process is  interpreted as `action of X' or
`process of X' and
consists of the suffixation of a  high front vowel to the verbal
 stem.\footnote{The nominalization `the process X' is often not distinguishable 
from
`the result of a process X'.  Does `dancing'  refer to `the process of dance', 
`the result of the process of dance' or both?} The surface form of the vowel
depends
on the quality of the stem vowel and {\sc atr}-harmony  (see
rule \ref{RULE-atr} in section \ref{sec:vowel-harmony}). 
Consider example (\ref{ex:vp36.1.}).


\begin{exe}
\ex\label{ex:vp36.1.}
\glll ʊ̀ pílè wáɪ́ɪ́ rā \\
ʊ pile wa-ɪ-ɪ ra \\
        {\sc 3.sg} start come-{\sc nmlz}-{\sc cl.4} {\sc foc}\\
\glt  `He is beginning to come'
\end{exe}

 The final vowels in the
words referring to `the process of X' are analyzed as a sequence of two
vowels: first a nominalizer suffix (i.e. {\sc nmlz})  on the verbal stem,  and
second,  a noun class suffix.  Such nominalized verbal stems are alloted to
noun class 4;  their singular suffix is a copy of the {\sc nmlz} vowel,
and their  plural suffix is the low vowel {\S a}, raised to a mid height, e.g.
{\S pɛrɪɪ}/{\S pɛrɪɛ} `weaving(s)'  ($<$ {\S pɛra} `weave', see class 4
in section \ref{sec:class4}).

%The infinite forms are always prefixed by the low vowels. The transcription
%shows harmony in some instances but it is a topic to further investigate. 


\paragraph{Verbal state stem}
\label{sec:GRM-verb-state-stem}

Verbal state stems are  predicates denoting static events. They generally
function  as verbs, but they can take the role of attributive modifiers in
noun phrases, referred to as  `qualifiers' in section \ref{sec:GRM-qualifier}.
In that role, their semantic value is similar to the value of adjectives in
English: they denote a property  assigned to a referent.  To function as  a
qualifier, some verbal state predicates must be nominalized. 

As with verbal process stems,  verbal state stems are found in nouns which
have
been nominalized by suffixation of a  high front vowel, i.e. `the state of X'.
For instance, the state predicate {\S kpeg} has a general meaning which can be
translated into English as `hard' and `strong'. The expression {\S kpegii} in
{\S a tebul kpegii dʊa de} `The hard table is there' functions as qualifier in
the noun phrase {\S a tebul kpegii} {\it lit.} `the table hard'. 


\begin{exe} 
 \ex\label{exːGRM-v-sta-p-hard}{\it Verbal state stem {\S kpeg}  `hard' in
complex stem nouns}
 \begin{xlist}
 \ex\label{exːGRM-v-sta-p-hard-head}
{\I ɲúúꜜkpég} $<$ {\sc head-hard}
`stubborness' 
%pl. ɲuukpegse

 \ex\label{exːGRM-v-sta-p-hard-arm}
{\I nékpég} $<$ {\sc arm-hard} `stingy' 

 \ex\label{exːGRM-v-sta-p-hard-tree}
{\I dààkpég} $<$ {\sc wood-hard} `strong wood' 
 \end{xlist}
\end{exe}
 

Howewer, verbal state stems are usually found in complex stem nouns (see
section \ref{sec:GRM-com-stem-noun}).  Examples  are provided in
(\ref{exːGRM-v-sta-p-hard}) using the same  state predicate {\S kpeg} for
the sake of illustration.  Notice that only (\ref{exːGRM-v-sta-p-hard-tree}) has
a literal meaning.





%In a nominal context, the form in occurs in a  modifier position, i.e. ,  and
%the form in is found in compound formation, i.e. .



% Nevertheless there are  verbal state predicates which cannot be
% nominalized. For instance, the verbs {\S dʊa} and  {\S jaa} cannot
%list verb state predicate
%locative; dua
%identificational; ja


 


\subsubsection{Complex stem nouns}
\label{sec:GRM-com-stem-noun}

A complex stem noun is a type of compound word.
A  complex stem noun, as opposed to an atomic one,  is formed by the
combination of two stems (XY). Either X or Y in a  XY-complex stem noun may be 
atomic or complex.  Nominal stems ({\sc ns}), verbal state stems ({\sc ss}) and
verbal process stems ({\sc ps}), together with a single noun class  suffix 
(and/or other
types of nominalizer) are
the linguistic elements which take part in the
formation of complex stem nouns. 


\begin{exe}
 \ex\label{exːGRM-cplx-stm}
 \begin{xlist}
 %\ex\label{exːGRM-cplx-stm-SS-AS} {\I deŋlii}  `straight'  (deŋ|lii
%= `single'|`thither' 
 % $>$  {\sc ss} + {\sc ps} + {\sc cl.4}.sg)

  \ex\label{exːGRM-cplx-stm-NS-NS-1}%
{\I nébíí}  `finger' \\  %
ne-bi-i  $>$  {\sc arm-seed} \\%
 {\sc ns} + {\sc ns} + {\sc cl.3.sg}
 

  \ex\label{exːGRM-cplx-stm-NS-NS-2}
 {\I pàtʃɪ̀gɪ̀búmmò} `secretive' \\ %
patʃɪgɪ-bummo  $>$  {\sc stomach-black}  \\  %
 {\sc ns} + {\sc ns} (+ {\sc cl.1.sg})

 \ex\label{exːGRM-cplx-stm-NS-SS}
 {\I ŋmɛ́ŋhʊ̀lɪ́ɪ̀} `dried okro' \\ %
ŋmɛŋ-hʊl-ɪ-ɪ $>$   {\sc okro-dry}  \\  %
 {\sc ns} + {\sc ss} + {\sc nmlz}+ {\sc cl.4.sg}
 
 \ex\label{exːGRM-cplx-stm-PS-PS}
 {\I jawadir} `business person' \\ %
jawa-di-r  $>$ {\sc buy-eat-agent} \\  %
 {\sc ps} + {\sc ps} + {\sc nmlz} (+ {\sc cl.3.sg})

 \end{xlist}
\end{exe}

%Nominal stems ({\sc ns})
%Verbal state stems ({\sc ss}) 
%Verbal process stems ({\sc ps})

%In (\ref{exːGRM-cplx-stm-SS-AS}), the noun  {\S deŋlii} consists of a verbal
%state stem followed by a verbal process stem. The complex stem gets the
%noun class 4 suffixes.

  In (\ref{exːGRM-cplx-stm-NS-NS-1}) and (\ref{exːGRM-cplx-stm-NS-NS-2}),  all 
stems are nominal. In 
(\ref{exːGRM-cplx-stm-NS-SS}),  the verbal state stem {\S hʊl} `dry'  follows
a nominal stem,  and  in  (\ref{exːGRM-cplx-stm-PS-PS}) both stems are of the
type verbal
process.  In these stem appositions, it is the noun class suffix of
the rightmost stem which appears. Further, stems are lexemes, as opposed to
nouns or verbs.  This is readily apparent in  (\ref{exːGRM-cplx-stm-NS-NS-1})
and
(\ref{exːGRM-cplx-stm-NS-NS-2}), in which the leftmost stems {\S ne} and
{\S patʃɪgɪ}
would appear as {\S neŋ} and {\S patʃɪgɪɪ} if they were full-fledged nouns.
Thus, although complex stem nouns contain more than one stem, there is only  one
noun
class associated with the noun and it is always the noun class associated with
the rightmost stem.  This was mentioned in section \ref{sec:GRM-sem-ass-crit}
to support the claim  that
semantic criteria in
noun class assignment may be  nonexistant. 


If  stems are treated as lexemes, there is still a problem in accounting for the
 `reduced' form of  some lexemes when they occur in stem appositions. That
is, the first stem  of a complex stem noun is often reduced to a single
syllable in the case of a polysyllabic lexeme, or a monosyllabic lexeme of the
type CVV is reduced to CV. For
example,  {\S lúhò}  and  {\S lúhòsó} are respectively the singular and
plural forms for 
`funeral' ({\sc cl.2}).  The expectation is that
when the lexeme takes part in position X of a XY complex stem noun,
it should exhibit its lexemic form, i.e.   {\S luho}. Yet, the word for `last
funeral'  is {\S lusɪnna}, {\it lit.} funeral-drink,  and not {\S
*luhosɪnna}.  Not all  lexemes get reduced in that particular environment.
Nevertheless, it is  more common (and visible)  for polysyllabic lexemes or
monosyllabic ones built on a heavy syllable. Moreover, some lexemes are more
frequent in that environment than others.

%, a phenomenon which may be called {\it construct state}.
%Although like suffixes it occurs at the end of a sequences
%of stems, we assign it the status of verbal process stem.
%Apart mainly functioning as a verb, the lexeme {\S lɪɪ}`out, thither' is
%found
%in noun formation. The lexeme has a wide meaning


The relation between the stems in a complex stem noun is asymmetric.  The
relation is defined in terms of what the referents of the stems and the complex
noun as a whole have to do with each other.  As in a syntactic relation
between a head and a modifier, one of the stems modifies while the other stem is
modified. The semantic relations between the stems  are of two types:
 `completive' modification and  `qualitative' modification. These distinctions
are discussed in  sections \ref{sec:GRM-comp-completive} and
\ref{sec:GRM-comp-quality} below.



\paragraph{Completive modification}
\label{sec:GRM-comp-completive}

A completive modification in a complex stem noun XY can translate as `Y of X' of
which Y is the head. For instance {\S sììpʊ́ŋ}   `eyelash', {\it lit.}
eye-hair, is a kind of hair and not a kind of eye. And {\S ʔɪ̀lnʊ̃̀ã̀}
`nipple', {\it lit.} breast-mouth, is most likely seen as a kind of orifice than
as  a kind of breast.  In both cases, the noun class is suffixed to the
rightmost stem, incidentally to the head of the morphological construction, i.e.
{\S sììpʊ́ŋ}/{\S sììpʊ́ná} {\sc (cl.3)} and {\S ʔɪ̀lnʊ̃̀ã̀}/{\S
ʔɪ̀lnʊ̃̀ã̀sá} {\sc (cl.2)}. As mentioned earlier,  either X or Y  in a complex
noun XY can be complex. The word {\S nèpɪ́ɛ̀lpàtʃɪ́gɪ́ɪ́} `palm of the hand'
is an example of two completive modifications. It consists of a complex stem {\S
nepɪɛl} `hand', which is composed of  {\S ne} `arm' and {\S pɪɛl} `flat', and
the atomic stem {\S patʃɪgɪ} `stomach', yielding in turn  `flat of hand' and
then `inside of flat of hand'. 


\paragraph{Qualitative modification}
\label{sec:GRM-comp-quality}

A qualitative modification in a complex stem noun is the same as the  syntactic
modification  noun-modifier. The difference lies in the formal status of the
elements: when the
relation is held at a syntactic level the elements are words, whereas at the
morphological level they are lexemes. As mentioned earlier,  either X or Y  in a
complex noun XY can be complex. For instance, the word {\S nebiwie} consists of
the combination of {\S ne}  `arm' ({\sc cl.9}) and {\S bi} `seed'    ({\sc
cl.4}), then the combination of {\S nebi} `finger' and {\S wi} `small'. The 
noun class  of {\S wi} `small'  is {\sc cl.1}, so the singular and plural
forms for the word `little finger' are {\S nèbìwìé} and  {\S nèbìwìsé}
respectively. The first relation involved is a completive modification, i.e.
`seed  of arm', while the second is a qualitative one, i.e. `small seed  of
arm' or `small finger'.  A qualitative modification in a complex noun XY can
translate as `X has
the property Y'  of which X is the head. Therefore, unlike many languages,  it
is not necessarily the head of the morphological construction which determines
the type of inflection.


\begin{table}[htb!]

\centering
\caption[Distinction between completive and qualitative
modification]{Distinction between completive and qualitative
modification using /daa/ `tree' or `wood'.  Abbreviations: {\sc h}= head,  {\sc
m}=
modifier, {\sc ns}= nominal stem, {\sc ss}= verbal state stems,  {\sc ps}=
verbal process stem, \label{tab:GRM-complet-and-qualit}}
\begin{Itabular}{l|lllll}
\Hline
&  \multicolumn{3}{c}{Structure}    &		Stems &  Word\\  \cline{2-4}
    & Lex. type &  Function & Semantic& &\\[1ex]\hline
%\multirow{4}{5mm}{\begin{sideways}\parbox{15mm}{Completive}\end{sideways}}
\multirow{4}{5mm}{\begin{sideways}\parbox{20mm}{Completive}\end{sideways}}

& {\sc ns-ns} & {\sc m-h}&{\sc whole-part}&/daa/-/luto/&   dààlútó \\
& &&&`tree'-`root'&`root of tree' \\[1ex]

&{\sc ns-ss}&{\sc m-h}&{\sc whole-part}&/daa/-/pɛtɪ/ &dààpɛ́tɪ́ \\
&&&&`tree'-`end'& `bark'\\[1ex]

&{\sc ns-ns}&{\sc m-h}&{\sc whole-part}&/kpõŋkpõŋ/-/daa/&kpõ̀ŋkpõ̀ŋdāā\\
&&&&`cassava'-`wood' &`cassava plant'\\[1ex] \hline


 \multirow{4}{5mm}{\begin{sideways}\parbox{20mm}{Qualitative}\end{sideways}}

&{\sc ns-ns}&{\sc h-m}&{\sc thing-charac}&/daa/-/sɔta/& dààsɔ̀tá \\
&&&&`tree'-`thorn'& `type of tree'\\[1ex]

&{\sc ns-ns}&{\sc h-m}&{\sc thing-charac}&/ɲin/-/daa/& ɲíndáá\\
&&&&`tooth'-`wood'&  `horn'\\[1ex]

&{\sc ps-ns}&{\sc h-m}&{\sc purpose-thing}&/tʃaasa/-/daa/&tʃáásádàà \\
&&&&`comb'-`wood'&`wooden comb'\\[1ex]

\Hline
\end{Itabular} 
\end{table} 


The examples in table \ref{tab:GRM-complet-and-qualit} illustrate the
distinction between the completive and qualitative modification. The form {\S
daa}  conveys either the meaning `tree' or `wood'. Both meanings may function as
head or as modifier.  If the head stem follows its modifier, it is a completive
modification, and vice-versa for the qualitative modification. A semantic
relation between the stems may  be a whole-part relation, a characteristic added
to define an entity or a purpose  associated with an entity. 

So far,   XY-complex stem nouns were assumed to be  endocentric compounds whose
head is X in qualitative modification and the head is Y in completive
modification.  However, a word such as {\S patʃɪgɪbummo} `secretive', {\it lit.}
stomach-black, suggests that some XY-complex stem nouns may  either lack a head
or have more than one head. These possibilities are not ignored, but in this
particular case the complex stem noun may be seen as involving  the abstract
senses of {\S patʃɪgɪɪ} and {\S bummo}, that is  `essence' and 
`subtle/restrained' respectively, making {\S patʃɪgɪbummo} a qualitative
modification  which can be formulated literally as `subtle/restrained essence',
i.e.   a property applicable to humans. Thus, the stem {\S patʃɪgɪɪ} is treated
as the head, and {\S bummo} as the stem functioning as the qualitative modifier.
Another example is {\S dààdùgó}. This word consists of the stems {\S daa}
`tree' and  {\S dugo} `infest'   and refers to a type   of insect. Unlike the
analyzed expressions displayed in  table \ref{tab:GRM-complet-and-qualit}  none
of the stems can be treated as the head of the expression and the meaning of the
whole noun cannot be transparently  predicted from its constituent parts. This
leads me to provisionally consider the expression {\S dààdùgó}  as an
exocentric compound, i.e. a complex stem noun without a head.


 
\paragraph{Compound or circumlocution}
\label{sec:GRM-comp-vs-circum}

For a few expressions,  it is hard to tell whether they are compounds, i.e. the
results of  morphological operations, or circumlocutions, i.e.  the results of
syntactic operations \cite[165]{Alla01}. Clear cases of circumlocution
nevertheless exist. For instance,  the word {\S kpatakpalɪ} `type of hyena'  is
treated by one language consultant as {\S kpa ta kpa lɪɪ} {\it lit.}
 `take let.free take leave'.\footnote{Yet {\F kpatakpari} is the word for
`hunting trap' in Gonja \citep{Rytz66}.}  Another example is {\S
sʊ́wàkándíkùró} `parasitic plant'. This expression refers to a type of
parasitic plant lacking a  root which grows upon and survives from the
nutrients provided by its  hosts. The word-level expression originates from the
sentence  {\S sʊwa ka n̩ di kuoro}, {\it lit.}  die-and-I-eat-chief, `Die so
that I can become the chief'. It is common to find names of individuals being
constructed in this way: the oldest woman in Ducie is known as {\S n̩wabɪpɛ}, 
{\it lit.}  {\S n̩ wa bɪ pɛ}  `I-not-again-add'. Since two successive husbands 
died early,  she used to say that she will never marry again. For that reason
people call her {\S n̩wabɪpɛ}.  



\subsubsection{Derivational morphology}
\label{sec:GRM-der-morph}

A derivational morpheme is an affix which combines with a stem to form a word.
The meaning it carries combines with the meaning of the stem.     By definition,
a derivational morpheme is a bound affix, and  thus 
cannot exist on its own as a word. This property keeps apart complex
stem nouns and derived nouns. Yet, the distinction between a bound affix and
a lexeme is not obvious, mainly because some bound affixes were probably lexemes
at a previous stage, or still are today. 
% \paragraph{}
% \label{sec:GRM-der-}


\paragraph{Maturity and sex of animate entities}
\label{sec:GRM-der-matur}


The specification of the maturity
and sex of an animate entity is accomplished in the following
way: male, female, young and adult are organised in morphemes encoding
one or two distinctions. These morphemes are suffixed to the rightmost stem.
To distinguish between male and female, the morphemes ({\it sg.}/{\it pl.}) {\S
wal/wala} `male'  and {\S nɪɪ/nɪɪta} `female'  are used as
(\ref{exːGRM-sex-ent}) illustrates.


\begin{exe}
 \ex\label{exːGRM-sex-ent}
 \begin{xlist}
 \ex\label{exːGRM-sex-en} {\I bɔla-wal-\O} /  {\I bɔla-wal-a} \\
elephant-male-sg / elephant-male-pl ({\sc cl.3})
 \ex\label{exːGRM-sex-en}  {\I bɔla-nɪɪ-\O}  / {\I bɔla-nɪɪ-ta}\\
elephant-female-sg / elephant-female-pl ({\sc cl.7})
 
 \end{xlist}
\end{exe}


The language employs two strategies to express the distinction between  the
adult animal and its young, which is  called here 
`maturity'.  The first
is to simply add the morpheme {\S -bi} `child'  to the head,
e.g. {\S bɔlabie/bɔlabise} `young elephant(s)'. In the second strategy 
both the sex and maturity distinctions are conveyed by the morpheme.  This is
shown in table \ref{GRM-maturity-sex}. 

%componential-

\begin{table}[htb!]

\caption{Morphemes encoding maturity and sex of animate entities}
\centering
 \begin{Itabular}{lcc}
\Hline
&\textsc{male}&\textsc{female}\\
\hline
\textsc{young}& -w(a|e)lee & -lor\\
\textsc{adult} & -wal & -nɪɪ\\
\Hline

  
 \end{Itabular} 

\label{GRM-maturity-sex}
\end{table} 

Some examples are more opaque than others. For instance, the onset consonant of
the morpheme {\S wal/wala} `male' may surface as a bilabial plosive,  e.g. {\S
bʊ̃́ʊ̃̀mbāl} `male goat'.  One can also observe a difference in form between
the word {\S pìèsíí} `sheep',   {\S pèmbál}  `male sheep' and  {\S
pènɪ̀ɪ́} `female sheep'. The words displayed in the first three rows of table
\ref{tab:GRM-matur-sex-ex} show the least transparent derivations.  The
annotation of tone is a first impression and the symbol - indicates that the
consultants could not associate a word for the intended meaning. 


%\ex\label{exːGRM-sex-en}  {\I bɔlɛbie}  /  {\I bɔlɛbise}
 %\ex\label{exːGRM-sex-en}  {\I bɔlɛbiwal}  /  {\I bɔlɛbiwala}
%. a case of qualitative
%modification


\begin{table}[htb!]

\caption{Maturity and sex of animals}
\centering
 \begin{Itabular}{llllll}
\Hline
Animal & Generic & \multicolumn{2}{c}{Adult} 
& \multicolumn{2}{c}{Young}  \\ \cline{3-4} \cline{5-6}
 && \multicolumn{1}{c}{Male} &  \multicolumn{1}{c}{Female} & 
\multicolumn{1}{c}{Male} &   \multicolumn{1}{c}{Female}\\
\hline


fowl  			&zál̀& zímꜜbál		& zápúò&
zímbéléé  & zápúwìé\\

sheep  		&píésíí &pèmbál		& pènɪ̀ɪ́&
pémbéléè&pélòŕ\\

goat 		&bʊ̃́ʊ̃̀ŋ & bʊ̃́ʊ̃́mbál 	& bʊ̃̀nɪ̀ɪ́ &  bʊ̃̀mbéléè & 
bʊ̃̀ʊ̃̀lòŕ\\

pouched rat		&sàpùhĩ́ẽ̀ & sàpúwál		& sàpúnɪ̀ɪ̀&
 sàpúwáléè& sàpúlòr\\

antilope  		&ʔã́ã́ &ʔã̀ã̀wál	
&ʔã́ã́nɪ́ɪ́&ʔã̀ã̀wéléè&ʔã̀ã̀lòr\\

dog  		&váà  &váwàl		&vánɪ̀ɪ̀&
váwáléè&válòŕ\\

cat  			&dìébìé &díèbəwál	 
&díèbənɪ̀ɪ̀&díèbəwáléè&díèbəlòr\\

cow 			&nã̀ɔ̃́ &nɔ̃̀wál	
&nɔ̃̀nɪ̀ɪ́&nɔ̃̀wáléè&nɔ̃̀lòŕ\\

elephant  	&bɔ̀là &bɔ̀lwàl		& bɔ̀lənɪ̀ɪ́&  -  &bɔ̀llòŕ\\

guinea fowl  	&sũ̀ṹ&sũ̀wál		&sũ̀nɪ̀ɪ́& - & -\\
bush mouse  	&ʔól&ʔólwál		&	ʔólnɪ̀ɪ́ & - & - \\
%house mouse&dàgbòŋó&dagboŋowal	&dagboŋonɪɪ& - & - \\
lizard 		&gèŕ &géwál		&génɪ̀ɪ́&  -& - \\

\Hline
 \end{Itabular}
\label{tab:GRM-matur-sex-ex}
\end{table} 

%The plural of the transparent forms are... For hen, sheep and goat,...



\paragraph{Inhabitant of ...}
\label{sec:GRM-inhabitant-of}

To express `I am from X',  where `be from X' refers  to the place where
someone is born and/or the place where someone lives, the verb {\S lɪ̀ɪ̀} is
used, e.g. {\S sɔɣla n̩ lɪɪ} `I am from Sogola'.  

Expressions with the meaning `Inhabitant of X'  can be  noun words referring to
this same idea, that is  `being from X'. The examples in table
\ref{tab:inhabitant-of} show that the meaning is captured in suffixes {\S
-(l)ɪɪ/(l)ɛɛ/la} which display vowel qualities in the singular and plural
similar to those found in noun class 4. 


\begin{table}[htb!]

\caption{Inhabitant of ... \label{tab:inhabitant-of}}
\centering
 \begin{Itabular}{lll|lll}
\Hline
Location & sg. & pl. & Location & sg. & pl.  \\[1ex] \hline
Chakali 	& tʃàkálɪ́ɪ́ 	& tʃàkálɛ́ɛ́    &
Katua	&kàtʊ́ɔ́lɪ́ɪ́ 	&kàtʊ́ɔ́lɛ́ɛ́ \\
Motigu 	&mòtígíí	&mòtígíé    &%
Tiisa		&	tíísàlí	&tíísàlá\\
Ducie 	&dùsélíí	&dùséléé&%
Chasia	&tʃàsɪ́lɪ́ɪ́		& tʃàsɪ́lɛ́ɛ́	\\
  Bulinga	& búléŋíí & búléŋéé&%
Wa		&	wáálɪ́ɪ́	&wáálà\\
Gurumbele&grʊ̀mbɛ̀lɪ́lɪ́ɪ́& grʊ̀mbɛ̀lɪ́lɛ́ɛ́&%
Tuosa	&	tʊ̀ɔ̀sálɪ́ɪ́	& tʊ̀ɔ̀sálá\\
\Hline

 \end{Itabular}
\end{table} 
 

%({\S mũ̀sóró} `type of food ingredient') , 
%({\S dárà} `type of game')



\paragraph{Category switch}
\label{sec:GRM-der-cat-switch}

The
phenomenon called `category switch' refers to a derivational process whereby two
words with  related meanings and composed of the same segments change category
based entirely on their tonal melody. Examples are provided in 
(\ref{exːGRM-der-cat-switch}).


\begin{exe}
 \ex\label{exːGRM-der-cat-switch}
tʊ̀mà  ({\it v}) `work'  $\leftrightarrow$ tʊ́má ({\it n}) `work'\\
gʊ̀à ({\it v})  `dance' $\leftrightarrow$ gʊ̀á ({\it n}) `dance'\\
jɔ̀wà ({\it v}) `buy'   $\leftrightarrow$ jɔ̀wá  ({\it n})  `market'\\
mʊ̀mà ({\it v}) `laugh'  $\leftrightarrow$ mʊ̀má ({\it n})  `laughter'\\
gòrò ({\it v}) `circle'  $\leftrightarrow$ góró ({\it n})  `bent'
\end{exe}


\paragraph{Agent- and event-denoting nominalizations}
\label{sec:GRM-der-agent}

Agent- and event-denoting nominalizations were discussed in
section \ref{sec:GRM-verb-act-stem} in connection with the licensing of verbal
stems in atomic noun formation. Apart from their roles in complex stem nouns, it
was shown that both verbal state and verbal process stems undergo these two
nominalizations processes in order to function as atomic nouns.
The two
processes are summarized in (\ref{exːGRM-der-agent}) and 
(\ref{exːGRM-der-action}). Notice in (\ref{exːGRM-der-agent}) that the two
agent-denoting nominalizations can occur on a single stem. Also, the
noun class does not seem to be  determined by the suffix
-[r].  While one consultant prefers that all agent nouns end as ({\it
sg.}/{\it pl.}) {\S -r/-rV}, another consultant varies between {\S -r/-rV} 
({\sc cl.3}) and  {\S -ra/-rəsV} ({\sc cl.1}).  In addition, there is another
agent-denoting word formation which simply adds the word {\S kʊɔrɪ} `make' to
the noun denoting the product, e.g. {\S nã̀ã̀tɔ̀ʊ̀kʊ́ɔ́rá} / {\S
nã̀ã̀tɔ̀ʊ̀kʊ́ɔ́rəsá} ({\sc cl.1}) `shoemaker(s)' $<$ {\S nããtɔʊ} ({\it n})
`shoe' + {\S kʊɔrɪ} ({\it v}) `make'.


\begin{exe}
 \ex\label{exːGRM-der-agent}{\it Agent nominalization}
\begin{xlist}
\ex\label{exːGRM-der-agent-suffix}
{\it A verb stem takes the suffix -[r]  to express agent-denoting
nominalization.} \\
{\S  sʊ̃̀ã̀sʊ́ɔ́r} / {\S sʊ̃̀ã̀sʊ́ɔ́rá} ({\sc cl.3}) `weaver(s)' \\
 $<$  {\S sʊ̃̀ã̀} ({\it v}) `weave'\\
{\S  lúlíbùmmùjár} / {\S lúlíbùmmùjárá} ({\sc cl.3})  `healer(s)' \\
$<$ {\S lulibummo} ({\it n}) `medicine' + {\S ja} ({\it v}) `do'


 \ex\label{exːGRM-der-agent-redup}
{\it A verb stem gets partially reduplicated   to express
agent-denoting nominalization.} \\
{\S  sùsùmmà} / {\S sùsùmmə̀sá} ({\sc cl.3})  `helper(s)' \\
$<$ {\S sùmmè} ({\it v}) `help' \\
{\S  sã́sáár} / {\S sã́sáárá} ({\sc cl.3})  `carver(s)' \\
$<$ {\S sã̀ã̀} ({\it v}) `carve' 

\end{xlist}
\end{exe}


\begin{exe}
 \ex\label{exːGRM-der-action}{\it Event nominalization}\\
{\it A verb stem takes the suffix -/[{\sc +hi, -bk}]/   to express
event-denoting
nominalization.} \\
{\S lʊ́lɪ́ɪ́ } / {\S  lʊ́lɪ́ɛ́} ({\sc cl.4}) `giving birth'  \\
 $<$ {\S  lʊla} ({\it v}) `give birth'\\
{\S kpégíí} / {\S   kpégíé}  ({\sc cl.4})  `hard'  or
`strong' \\
 $<$ {\S  kpeg} ({\it v}) `hard' or `strong'\\

\end{exe}

\subsubsection{Proper nouns}
\label{sec:GRM-prop-noun}

% 
% Are these the names these people were given by their families? they look like
% nick-named if they are not actually titles.  The whole paragraph sounds
%strange. %  Either get more ethnographic information on naming or cut some of
%this out.

%Proper nouns  are  usually characterized by their  inability  to inflect for
%number and to co-occur with articles and modifiers. 

As a rule,   proper nouns
have  unique referents:  they  name people, places, spirits and so on.  So in
the
area where Chakali is spoken, there is only one river named {\S gorogoro}, only
one hill named {\S dɔ̀lbɪ́ɪ́}, one
village
named {\S mòtīgú},  only one shrine named {\S dàbàŋtʊ́lʊ́gʊ́}, etc.  
Nevertheless more than one person can have the same
name, and the same applies to a lesser extent to villages. For instance,
{\S
sɔɣla} may refer to the Chakali village situated between Tuosa and Motigu, or to
a Vagla village situated at the junction of the Bole-Wa and Damongo-Wa road. To
identify the former, one must say {\S tʃakalsɔɣla}. 


A  Chakali person may bear two or three names: his/her father's name, the name
of his/her grandfather or great-grandfather, and his own (common) name. In the
case of the (great-) grandfather's name, it is a feature of the newborn or an
external sign which suggests the child's name.  The common name may be changed
in
the course of one 's life. Today, regardless of whether a  person is Muslim or
not,
common names are mainly from Arabic, Hausa and Gonja origins, probably due to
the
Islamization of Chakali (see section \ref{sec:SOC-why-cli-still}).


Common names among the  elders (over 50 years old) consist of the name of a
non-Chakali village,  together with {\S nàà} `chief'. In Tuosa, Ducie and
Gurumbele, one finds one or more Kpersi Naa, Mangwe Naa, Jayiri Naa, Wa Naa, 
Sing Naa,  Busa Naa, etc. The next generation (below 50 years old) tend to
have either `Muslim' names or `English-title' names. Common Muslim names are
Idrissu, Fuseini, Mohamedu, Ahmed, Mohadini, etc.  Typical 
`English-title' names are {\S spɛ́ntà} `inspector',  {\S dɔ́ktà} `doctor', {\S
títʃà} `teacher', etc. Apart from `teacher',  which can identify actual
teachers in communities in which schools are present, none of the individuals
are actual teachers, doctors or inspectors. The same can be said about the older
generation, none of them are/were chief of those places. The villages are not
Chakali villages and these individuals have no connections with the villages
used in their names. It seems that these common names are trendy  nicknames
peers  assign to each other. One consultant claims that the elders can be
ranked in terms of power and influence according to their nicknames. 

In Chakali society, one may have two additional names, a drumming name and a
Sigu name. A drumming name is used in drummed messages sent to other villages
about weddings or deaths,  while a {\S  sigu} name is a name one receives
when initiated to the  {\S sigmaa} society. 

Because of their pragmatic function,  proper nouns  are rarely observed in a
plural form, but some contexts may allow this. In
(\ref{ex:GRM-propn-noun-plur}), the proper name {\S Gbolo} takes the plural
marker {\S -sV}.\footnote{The context of (\ref{ex:GRM-propn-noun-plur}) makes
sense when one understands that the name `Gbolo' has a particular meaning.  It
is understood that when a couple has a  fertility problem,  it is common to
travel to the community of Mankuma and to consult their shrine. If the woman
gets pregnant after the visit, they must return to Mankuma to appease the
shrine. Subsequently, the child must be named `Gbolo' and automatically acquires
 the Red Patas monkey as  taboo.}


  \begin{exe}
   \ex\label{ex:GRM-propn-noun-plur}
\gll  gbolo-so ba-ŋmɛna ka dʊa dusee nɪ\\
gbolo.({\sc g.}b)-{\sc pl}  {\sc g.}b-{\sc q} {\sc  egr} exist Ducie {\sc
postp}\\
\glt  `How many Gbolos are there in Ducie?' 
    \end{exe}

Finally, circumlocution is a common process found in names of people and dogs 
 (e.g. the example of {\S n̩wabɪpɛ}, 
{\it lit.}  {\S n̩ wa bɪ pɛ}  `I-not-again-add', was given in section
\ref{sec:GRM-comp-vs-circum}).   A few examples of dog names are given in
(\ref{ex:GRM-propn-dog-name}).


  \begin{exe}
   \ex\label{ex:GRM-propn-dog-name}{\it Examples of dog names}
 \begin{xlist}
% \ex\label{ex:GRM-propn-dog-name-} 

 \ex\label{ex:GRM-propn-dog-name-1} 
{\I jàsáŋábʊ́ɛ̃̀ɪ̀} `Let's keep peace'\\ 
ja-saŋa-bʊɛ̃ɪ   $>$  we-sit-slowly  
 \ex\label{ex:GRM-propn-dog-name-2} 
{\I ǹ̩nʊ̀ã́wàjàhóò}   `I will not open my mouth again'  \\ 
 n̩-nʊã-wa-ja-hoo  $>$ my-mouth-not-do-hoo 
 \ex\label{ex:GRM-propn-dog-name-3} 
{\I kùósòzɪ́má}   `God knows' \\  
kuoso-zɪma $>$ god-know
\end{xlist}
\end{exe}



% In many folktales, animals are the characters and their acts evoke
% human beings' behaviors. There is a tendency in folktales to switch between
% using the animal name as common noun and using it as proper noun. (example)


%1. born in village but moved from there
%2. not necesarily chief but from royal family
%3. gan naa= `more than a chief' grand father of John
%4. pure nick name

\subsubsection{Loan nouns}
\label{sec:GRM-borr-noun}


 A loan noun,  or more generally a loanword, can be defined as  ``a word that at
some point came into a language by transfer from another language''
\citep[58]{Hasp08}.  When a word is found in both Chakali and in another
language, many loan scenarios are conceivable. However,  for some semantic
domains such as   bicycle or car parts, school material and so on, the past and
present sociolinguistic situations  suggest that Chakali is
the recipient language and Waali, English, Hausa, Akan, and Dagaare are the
donor
languages.  Loan scenarios differ and are harder to establish when other SWG
languages are involved. It is often unfeasible to demonstrate whether the same
form/meaning in two languages was inherited from a common ancestor, or  
borrowed by one and subsequently passed on to other SGW languages. Moreover, it
may be unwise to assume that in all cases Chakali is  the recipient language,
especially for loan words in domains which  were in the past fundamental in
Chakali lifestyle,  but to a lesser degree for neighboring ethnic groups. 
Thus, Chakali as a donor language can be evaluated in a wider Grusi-Oti Volta
genesis, or  at a micro-level where the influence of Chakali on Bulengi is
established.

It is unlikely that Chakali borrowed from English through contact. And Ghanaian
English, in Wa town and Chakali communities,  is not an effective mode of
communication, at least in social spheres where Chakali men and women 
interact (see discussion in section \ref{sec:SOC-unesco}).  Nonetheless, the
situation is different for school children  who are
exposed to Ghanaian English on a regular basis. I believe that Ghanaian English
spoken by Waali, Dagaare or Chakali native speakers is the only potential
variety of English which can function as a donor language. Words ultimately  
from
English origin are: {\S ʔàbəlù} `blue', {\S ʔásɪ̀bɪ́tɪ̀} `hospital',    {\S
dɔ́ktà} `doctor', {\S bàlúù} `balloon', {\S bátərbɪ́ɪ́} `battery (-stone)',
{\S bɛ́lɛ́ntɪ̀} `belt',  {\S təráádʒà} `trouser',  {\S détì} `date',    {\S
mɪ́ntɪ̀} `minute',   {\S dʒánsè} `type of dance',  {\S kàpɛ́ntà}
`carpenter', {\S kɔ́lpɔ̀tɛ̀} `coal pot', {\S kɔ́tà} `quarter',  {\S lɔ́ɔ́lɪ̀}
`lorry (any four-wheel vehicle)',   {\S sákə̀r} `bicycle',  {\S pɛ̀n} `pen',
{\S sùkúù} `school',   {\S tʃítʃà} `teacher' and many more.  There is 
a recurrent falling tonal melody (i.e. HL) among the loan nouns of  ultimately
English origins. Many of them,  if not all, can be found in other
languages of the area \citep{Sisa75, Daku07}.\footnote{I neither own nor have 
had access to the published English-Vagala dictionary, the  
Waali dictionary manuscrits (there are two to my knowledge), or {\it Durand, J.
B. 1953. Dagaare-English Dictionary. Navrongo: St. John Bosco's Press}. It is
common to hear these words in Wa. I did not carefully look into the Vagala and
Dɛg lexical database (see footnote \ref{ft:GRM-naden-donate}).}

People are aware of the linguistic fragility of Chakali; some of the
language
consultants confirm that many people do not bother making  the effort to use
Chakali and that many
prefer to use Waali expressions.  The knowledge and interest of our
language consultants in their language made  it possible  to reduce the number
of
Waali
expressions in the  lexicon of chapter \ref{sec:LEX-lexicon}.  Despite that,
when a word is found both in
Waali and Chakali, it is not automatically classified as borrowed from Waali,
yet it is suspected or confirmed to be non-Chakali.  Examples such
{\S dʒɪ̀ɛ̀rá} `sieve', {\S dʒùmbùrò} `type of medicine', {\S gbàgbá}
`duck', {\S kákádùrò} (Akan)  `ginger', {\S kàpálà} `fufu', {\S
káṹ}  `mixture
of sodium carbonate' , {\S nāsárá}  (Hausa) `white man', and  {\S
sɛ̀nsɛ́nná}
`prostitute' are some of the Waali/Chakali nouns  found in transcribed texts, or
by chance. 

The weekdays are from Arabic (probably via Hausa or Oti-Volta languages).
Vagla and
Tumulung Sisaala, but  not Dɛg, use similar expressions \citep[60]{Nade96}: 
{\S atanɛa} `Monday', {\S atalata} `Tuesday', {\S alarba} `Wednesday',   {\S
alamussa} `Thursday',  {\S ardʒɛmaa} `Friday' , {\S asɛbɛtɛ} `Saturday'    and
{\S allahadi} `Sunday'.  The expressions for the lunar months seem to be
borrowed from Waali, but Dagbani and Mamprusi have similar expressions. In
these
Oti-Volta languages, some of the names  correspond to important festivals, i.e.
1, 3, 7, 9, 10 and 12 below. In Chakali, only {\S dʒɪ́mbɛ̄ntʊ̀} is celebrated
and  considered the first month.\footnote{Dagbani {\F buɣum} and Waali {\F
dʒɪmbɛntɪ} are both treated as first month by the speakers of these
languages.} The lunar months are: {\S dʒɪ́mbɛ̄ntʊ̀} `first month
(1)', {\S sífə̀rà}  `second month (2)', {\S dùmbá} `third month (3)', {\S
dùmbáfúlánààn} `fourth month (4)',  {\S dùmbákókórìkó} `fifth month
(5)', {\S kpínítʃùmààŋkùná} `sixth month (6)', {\S kpínítʃù} `seventh
month (7)', {\S ʔàndʒèlìndʒé} `eighth month (8)', {\S sʊ́ŋkàrɛ̀} `ninth
month (9)', {\S tʃíŋsùŋù} `tenth month (10)', {\S dùŋúmààŋkùnà}
`eleventh month (11)' and {\S dùŋú} `twelth month (12)'.  It was understood
that
these terms and concepts are not known by the majority.\footnote{Thanks to Tony
Naden, who helped me clarify some issues relating to this topic.} 

\subsection{Pronouns and pro-forms}
\label{sec:GRM-pronouns}

The difference between pronouns and pro-forms depends on whether they can be 
anaphors of nominal arguments.  A pronoun is a type of pro-form. In
section \ref{sec:GRM-adv-pro}, we shall have a look at another type of
pro-form. In this section, the personal, impersonal,
demonstrative and possessive pronouns are introduced, followed by the
expressions used to convey reciprocity and reflexivity. 


\subsubsection{Personal pronouns}
\label{sec:GRM-personal-pronouns}

The  personal pronouns are presented in table
\ref{tab:GRM-pers-pro}. They 
  do not encode a gender distinction in the singular. In the
plural, an animacy  distinction is made between non-human and  human. They are
glossed {\sc 3.pl.g}a and {\sc  3.pl.g}b  respectively (see section
\ref{sec:GRM-gender} for the role of gender in agreement).  


\begin{table}[h]
 \caption{Weak and Strong personal pronouns \label{tab:GRM-pers-pro}}
  \centering
  \begin{Itabular}{lll}
\Hline 
Pronoun & weak form ({\sc wk})   & strong form ({\sc st})\\
Gram. function  &    {\sc s|a} and {\sc o}  &  {\sc s|a} \\[1ex]
\hline
{\sc 1.sg.} &  N &   mɪŋ\\
{\sc 2.sg.}  &   ɪ& hɪŋ \\
{\sc  3.sg.}  &  ʊ&  wa\\
{\sc 1.pl.}  &   ja&  jawa \\
{\sc 2.pl.} &    ma &   mawa \\
{\sc  3.pl.g}a &  a  &   awa \\
{\sc 3.pl.g}b  &   ba&   bawa \\
    
\Hline
  \end{Itabular}
\end{table}

  
  The first person singular pronoun,  represented with an `N' in table 
\ref{tab:GRM-pers-pro},  is a syllabic nasal which assimilates its
place feature from the following phonological material (see section
\ref{sec:ext-nasal-place}). The  distinction between weak ({\sc wk}) and strong
({\sc st})  is relevant
  when pronouns function as subject. Their proper use is conditioned
by the emphasis  placed on the participant of the event or the
  event itself, and by the polarity of the clause in which they
  appear.  





  \begin{exe}
  \ex\label{ex:GRM-weak-strong-arg}
   \begin{xlist}
 
   \ex\label{ex:}
\gll   mɪŋ      jawa   kɪnzɪnɪɪ\\
     {\sc 1.sg.st}  buy  horse\\
\glt  ` I bought a horse. ' 
   
   
   \ex\label{ex:}
\gll    n̩    jawa kɪnzɪnɪɪ ra\\
     {\sc 1.sg.wk}  buy  horse  {\sc foc}  \\
\glt  `I bought a HORSE. ' 
   
   
   \ex\label{ex:}
\gll    n̩    wa jawa kɪnzɪnɪɪ\\
    {\sc 1.sg.wk} {\sc neg} buy  horse \\
\glt  `I did not buy a horse. ' 
  
   \ex\label{ex:GRM-out-STR-FOC-buy}
    *mɪŋ      jawa   kɪnzɪnɪɪ ra
   \ex\label{ex:GRM-out-STR-NEG-buy}
     *mɪŋ    wa  jawa   kɪnzɪnɪɪ 
 
     
   \end{xlist}
  \end{exe}




  \begin{exe}
  \ex\label{ex:GRM-weak-strong-verb}
   \begin{xlist}

   
   \ex\label{ex:}
\gll    n̩   petijo\\
    {\sc 1.sg.wk}  {terminate.{\sc pfv.foc}}  \\
\glt  `I finished. ' 


   \ex\label{ex:}
\gll    mɪŋ   petije\\
     {\sc  1.sg.st} {terminate.{\sc pfv}} \\
\glt  ` `I finished. '  

   \ex\label{ex:GRM-out-STR-FOC-finish}
  *mɪŋ   petijo

   \ex\label{ex:GRM-out-STR-NEG-finish}
  *mɪŋ   wa petije
         
   \end{xlist}
  \end{exe}

Thus, strong pronouns cannot co-occur in a sentence in which another
constituent is in focus, that is a nominal phrase  flanked by the focus
marker  or   a
verb ending with the assertive suffix vowel   {\sc -[+ro,  +hi]}    (see
examples 
(\ref{ex:GRM-out-STR-FOC-buy}) and (\ref{ex:GRM-out-STR-FOC-finish})). In
addition,  in
sentences 
where a negative operator occurs, strong pronouns are disallowed, as  
(\ref{ex:GRM-out-STR-NEG-buy}) and   (\ref{ex:GRM-out-STR-NEG-finish}) show.

  
  %clitizised pronoun in objetc position here. Bring a CVVV example
  
  
 \subsubsection{Impersonal pronouns}
 \label{sec:GRM-impers-pro}

An impersonal pronoun does  not refer to a particular person or thing. The form
{\S a} is treated as an impersonal pronoun  in some
particular context.

               
\begin{exe} 
\ex\label{ex:imps-pro-sing}
\gll a maasejo keŋ\\
     {\sc 3.sg.imps}  enough.{\sc pfv.foc} {\sc adv}\\
\glt `That's enough' or `That's it' or `Stop'
\end{exe}

Example (\ref{ex:imps-pro-sing}) is treated as a type of impersonal
construction. It is characterized by its  subject position being  occupied by 
the
pronoun {\S a}, which may be seen as referring to the situation,  but not to any
participant. The example in  (\ref{ex:imps-pro-sing}) may
be appropriate in contexts involving pouring  liquids or giving food on a plate,
or when people are quarrelling. In these hypothetical contexts, using the
personal
pronoun {\S ʊ} instead of  the  impersonal pronoun  {\S a} would be
unacceptable.  

The language does not have a passive construction as one finds in English.
Nonetheless,  an argument  can be demoted  by placing it in object position.
This is shown in (\ref{ex:GRM-vp23.3.}).

\begin{exe}
\ex\label{ex:GRM-vp23.3.}
\gll ka a namɪã?  ba tieu ro\\
      {\sc q} {\sc art} meat {\sc 3.pl.g}b  eat.{\sc pfv}.{\sc 3.sg.obj} {\sc
foc}\\
\glt  `Where is the meat? It has been eaten.'
\end{exe}


This type of impersonal
construction illustrated in  (\ref{ex:GRM-vp23.3.})  is characterized by the
personal pronoun {\S ba} ({\sc 3.pl.g}b)  in
subject position. In this context,  the subject cannot be seen as filling the
role of agent and the pronoun {\S ba} does not refer to anyone/anything in 
particular. Therefore,  the pair {\S a}/{\S ba} is treated  as the singular and
plural impersonal pronouns, only when they occur in impersonal constructions, 
as shown above.






\subsubsection{Demonstrative pronouns}
\label{sec:GRM-demons-pro}


In the examples (\ref{ex:GRM-demons-pro-reply-1}) to
(\ref{ex:GRM-demons-pro-quest}),  the demonstrative pronouns  
function as noun phrases. All the examples below were accompanied with
 pointing gestures when uttered.



\begin{exe}
\ex\label{ex:GRM-demons-pro-reply-1}{\it Replies to the question: Which cloth
has she chosen?}
 \begin{xlist}
 
  \ex\label{ex:GRM-demons-pro-reply-1sg} 
 \gll han na\\
   {\sc dem.sg} {\sc foc}\\
 \glt `It is this one' 
   
      \ex\label{ex:GRM-demons-pro-reply-1pl}
       \gll hama ra\\
   {\sc dem.pl} {\sc foc}\\
 \glt `It is these ones' 
 \end{xlist}
\end{exe}



\begin{exe}
\ex\label{ex:GRM-demons-pro-quest}{\it The speaker asks the addressee whether
he
had moved a certain object.} 
 \gll ɪ̀ já hàn nȁ\\
  {\sc 2.sg} do {\sc dem.sg}  {\sc foc}\\
 \glt `You did THIS?' 

\end{exe}


\begin{exe}
\ex\label{ex:GRM-demons-pro-quest}{\it Explanation on how the fingers cooperate
when they scoop t.z. from a bowl.}

 \gll hama ka zɪ pɛjɛ a wa zɪ ja tiise  haŋ\\
 {\sc dem.pl} {\sc egr} then add.{\sc pfv} {\sc conn} come then do support {\sc
dem.sg} \\
 
\glt `These (two fingers) are then added,
and then they come to support  this one.' 

\end{exe}



The expressions {\S hàŋ} ({\it sg.}) and {\S hàmà}
({\it pl.}) are employed for spatial deixis, specifically as proximal
demonstratives, corresponding to English `this' and `these' respectively. The
language does not offer another set for distal demonstratives.



\subsubsection{Interrogative words}
\label{sec:GRM-interg-pro}


Interrogative constructions are of two types:  {\it yes/no} interrogatives
and  {\it pro-form} interrogatives (see section
\ref{sec:GRM-interr-clause}). The former
type, as the dichotomy suggests, requires  
a `yes' or a `no' answer.  A {\it pro-form} interrogative  uses  an
interrogative word which identifies the sort of information requested. In
Chakali,  some interrogative words may be treated as pronouns, while others may
be treated as the combination of a noun and a pronoun.  Table
\ref{tab:GRM-interg-pro} gives a list of interrogative words, together with an
approximate English translation,  the sort of information requested by each  and
a link to an illustrative example of {\it pro-form} interrogatives. The examples
are listed in (\ref{ex:GRM-interg-pro}). The question words are glossed as  {\sc
q}.


\begin{table}[htb!]

 \caption{Interrogative pronouns \label{tab:GRM-interg-pro}}
  \centering
  \begin{Gtabular}{llll}
\Hline 
Pronoun & Gloss  & Meaning requested & Example  \\[1ex] \hline
baaŋ & what &  non-animate entity, event & \ref{ex:vp1.11.a}\\
 aŋ & who & animate entity & \ref{ex:vp2.5}\\
 lie & where & location & \ref{ex:vp9.25}\\
ɲɪnɪɛ̃ & why/how & condition, reason& \ref{ex:vp22.4.1.}\\
(ba/a)weŋ  & which &  entity, event & \ref{ex:vp22.4.4.}\\
 (ba/a)ŋmɛna & (how) much/many & entity, event & \ref{ex:vp22.4.10.}\\
 saŋa weŋ & when & time & \ref{ex:vp22.4.15.}\\
\Hline
  \end{Gtabular}
 
\end{table}


  
  \begin{exe}
  \ex\label{ex:GRM-interg-pro}
   \begin{xlist}
 
\ex\label{ex:vp1.11.a}
\gll baaŋ ɪ ka ja \\
{\sc q} {\sc 2.sg} {\sc egr} do\\
\glt `What are you doing?' 


\ex\label{ex:vp2.5}
\gll aŋ ɪ ka na a tɔʊ nɪ \\
     {\sc q} {\sc 2.sg}  {\sc egr}  see {\sc art} village {\sc postp} \\
\glt  `Whom did you see in the village?' 

%check comp here
\ex\label{ex:vp9.25}
\gll lie nɪ dɪ tʃʊɔlɪɪ ka dʊɔ \\
        {\sc q} {\sc postp} {\sc comp} sleeping.room   {\sc egr} exist \\
\glt  `Where is the room for sleeping?' 



\ex\label{ex:vp22.4.1.}
\gll ɲɪnɪɛ̃ ɪ ja ka jaaʊ \\
      {\sc q}  {\sc 2.sg} {\sc hab}   {\sc egr} do.{\sc 3.sg.obj} \\
\glt  `How do you do it?' 



\ex\label{ex:vp22.4.4.}
\gll aweŋ ɪ ka kpaɣa \\
      {\sc q}   {\sc 2.sg}  {\sc  egr}    catch  \\
\glt  `Which one did you catch?' 


\ex\label{ex:vp22.4.10.}
\gll aŋmɛna ɪ ka kpagasɪ \\
         {\sc q}    {\sc 2.sg}  {\sc  egr}  catch.{\sc pv}  \\
\glt  `How many of them did you catch?' 

\ex\label{ex:vp22.4.15.}
\gll {saŋa weŋ} ɪ ka waa \\
       {\sc q} {\sc 2.sg} {\sc  egr}    come \\
\glt  `When are you coming?' 
  
   \end{xlist}
  \end{exe}


When the question word {\S lie} `where' is  followed by the locative
postposition {\S
nɪ},  a request for a particular location is interpreted. This question word can
also be
followed by the noun  {\S pe} `end' in which case it should be interpreted
as
`where-towards' or `where-by', e.g. {\S lie pe ɪ ka vala} `Where did he go
by?'.  Another form used to request information on a location is {\S ká(á)}.
This form is neither specific to Chakali nor to location {\it per se}:
the languages Waali and Dagaare use it for the same purpose and the
form is even used to request other types  of information. For instance, {\S
káá tʊ́má} means `how is work?' in the three languages. It might be that
Chakali borrowed the form from Waali.  It was
employed consistently in an experiment I carried out, which is discussed  in
section \ref{sec:SPA-exper1}. Example  (\ref{ex:GRM-vp23.3.}),  repeated below, 
illustrates the use of {\S ka(a)} as interrogative word.

\begin{exe}
\exp{ex:GRM-vp23.3.}
\gll ka a namɪã?  ba tieu ro\\
      {\sc q} {\sc art} meat {\sc 3.pl.b} chew.{\sc pfv}.{\sc 3.sg.obj} {\sc
foc}\\
\glt  `Where is the meat? It has been eaten.'
\end{exe}





When they stand alone as interrogative words, the
expressions {\S weŋ} and {\S ŋmɛna}, roughly
corresponding to English `which' and `how much/many', must be prefixed by either
{\S a-} or {\S ba-} reflecting a distinction between non-human and human 
entities respectively (see section \ref{sec:GRM-gender}). The expression {\S
saŋa weŋ} in (\ref{ex:vp22.4.15.}) is
literally translated as `time which'.     The question word {\S
baaŋ} can be used together with {\S wɪɪ} to correspond to English `why', i.e.
{\S baaŋ wɪɪ ka wa ɪ dɪ wii}? `Why are you crying?'.  The expression {\S baaŋ
wɪɪ} is equivalent to English `what matter'. 





%\begin{exe}
%\ex\label{ex:vp23.1.}
%\glll  àná ká tūgùù?\\
 %aŋ ka tuga-u\\
    %{\sc qw}   {\sc  egr} beat-{\sc 3.sg.obj}\\
%\glt  `By whom is he being beaten.'
%\end{exe}

%The question words may be followed by the focus particle. This is shown in
%example (\ref{ex:vp23.1.}) with the question word {\S aŋ}  `who'.




\subsubsection{Possessive pronouns}
\label{secːGRM-poss-pro}

The possessive pronouns are displayed in table \ref{tab:posspro}. 

\begin{table}[h!]
  \caption{Possessive pronouns \label{tab:posspro}}
  \centering
  \begin{Itabular}{ll}
\Hline 
Pronoun 			&  Form  \\
Gram. function  	&   Possessive  \\[1ex] \hline
{\sc 1.sg.poss} 		& N \\
{\sc 2.sg.poss}  		&   ɪ\\
{\sc  3.sg.poss}  		&  ʊ \\
{\sc 1.pl.poss}  		&   ja  \\
{\sc 2.pl.poss} 			&    ma \\
{\sc  3.pl.a.poss}		&  a  \\
{\sc 3.pl.b.poss}  		&  ba \\
    
\Hline
  \end{Itabular}
\end{table}

These pro-forms are used in the possessor slot ({\sc psor}), but never in
the possessed slot ({\sc psed}) of an attributive possessive relation. This
is shown in (\ref{ex:vp7.15}). 

\begin{exe}
\ex\label{ex:vp7.15}
\glll a kuoru ŋma dɪ ʊ hããŋ tʃɔjɛʊ \\
{} {} {}  {} {\sc psor} {\sc psed}  {}\\
     {\sc art} chief say {\sc comp} {3.sg.poss} wife ran.{\sc pfv.foc}   \\
\glt  `The chief says that his wife ran away.' 
\end{exe}

The  weak personal pronouns and the possessive pronouns have the same forms, the
differences between the two being their respective syntactic positions and their
argument structures. First, the weak pronoun normally precedes a verb while the
possessive pronoun normally precedes a noun. Second, the weak pronoun is an
argument of a verbal predicate while the possessive pronoun can only be the
possessor in a possessive attributive construction. 
%tone differences


\begin{exe}
\ex\label{ex:vp7.15}
\glll                                                                    
{\I mɪ́nꜜná}  \\
 mɪŋ na\\
{\sc 1.sg.st.} {\sc foc}\\

\glt  `It is MINE.' 
\end{exe}


Phrasal possessives, as in English `mine, yours, etc.', are expressed with the
strong personal pronoun  in a verbless identificational
construction. This is shown in (\ref{ex:vp7.15}).




\subsubsection{Reciprocity and reflexivity}
\label{sec:GRM-recipro-reflex}


Reflexive and reciprocal pronouns do not exist in Chakali.  Instead,
reciprocity and reflexivity  are
encoded in  the nominals {\S dɔŋa}    and {\S tɪntɪn}, which are glossed in the
texts as {\sc
recp}  and {\sc refl} respectively.   Reciprocity is illustrated in
(\ref{ex:GRM-recipro}) and reflexivity in (\ref{ex:GRM-reflex}). 

% They are categorized nominals since {\S
% dɔŋa}  is believed to 

  \begin{exe}
  \ex\label{ex:GRM-recipro}
   \begin{xlist}
   
\ex\label{ex:vp24.1.}
\gll a nɪbaala balɪɛ kpʊ dɔŋa wa \\
     {\sc art} men two kill    {\sc recp}  {\sc foc} \\
\glt  `The two men killed EACH OTHER.' 

\ex\label{ex:vp24.2.}
\gll ja kaa kpʊ dɔŋa wa \\
      {\sc 1.pl} {\sc fut} kill  {\sc recp}   {\sc foc} \\
\glt  `We will kill  EACH OTHER.' 

\ex\label{ex:vp24.3.}
\gll a hamõwise ka juo dɔŋa \\
     {\sc art} children {\sc  egr} fight {\sc recp}   \\
\glt  `The children are fighting against one another.' 
 
   \end{xlist}
  \end{exe}



 \begin{exe}
        \ex\label{ex:GRM-reflex}
	\begin{xlist}
   
\ex\label{ex:vp25.1.}
\gll a baal kpʊ ʊ tɪntɪŋ \\
      {\sc art} man kill  {\sc 3.sg.poss} {\sc refl.sg} \\
\glt  `The man killed himself.' 

\ex\label{ex:vp25.2.}
\gll ja kaa kpʊ ja tɪntɪnsa wa \\
     {\sc 1.pl}  {\sc fut} kill {\sc 1.pl.poss}  {\sc refl.pl} {\sc foc} \\
\glt  `We shall kill OURSELVES	.'

\ex\label{ex:vp25.4.}
\gll a bie kpa kisie dʊ ʊ tɪntɪŋ daŋɪɪ \\
       {\sc art} child take knife put     {\sc 3.sg.poss}   {\sc
refl.sg} wound.{\sc nmlz}\\
\glt  `The child wounded himself with his knife.' 
  
   \end{xlist}
  \end{exe}




%Nonetheless, when fixed expressions containing {\S a}
%co-occur with {\S tɪŋ} none of them can be deleted, e.g. {\S a wozuri  tɪŋ} `on
%that day'??.
%do exist (above) and do down as verbs (reflexive examples)
%what `do' means?

\subsection{Qualifiers}
\label{sec:GRM-qualifier}



% In the naming data the great
% majority of the 1560 expressions referring to the 62 tiles have the form  {\S
% a-X} or {\S kɪn-X}. In section \ref{sec:gramsketch} and \ref{sec:gramsketch}
% respectively  these prefixes were described as  (i)
%  affixing  on property- and state-denoting predicates, (ii) encoding a
%selection
% of
% semantic features of the referent to which the property modifies,  and (iii)  
%  as  purely grammatical prefixes, the former renders a property into a
% qualifier and the latter into a noun. Thus the words {\S
% a-pʊmma} `white'  and  {\S kɪn-pʊmma} `white'  are syntactically qualifiers
%and
% nouns
% respectively. 
%  Also
% recurrent in the naming data is the focus marker {\S ra}, introduced in
%section
% \ref{sec:gramsketch}, following the expressions {\S a-X} or {\S
% kɪn-X}. The three frames are illustrated in (\ref{ex:most-frame}).
% a- prefix

Since qualifiers display singular/plural pairs (as  do nouns) and verbs do
not inflect for number, qualifiers are  treated  as  nominals. Examples are
presented
in (\ref{ex:GRM-qual}).


\begin{exe}
\ex\label{ex:GRM-qual}
 \begin{xlist}
  \ex\label{ex:GRM-qual-red}
sɪ̀àmá {\it sg.}, sɪ̀ànsá {\it pl.}   ({\sc cl.}1) `red'
  \ex\label{ex:GRM-qual-bad}
 bɔ́ŋ̀ {\it sg.}, bɔ́má {\it pl.}  ({\sc cl.}3)  `bad'
  \ex\label{ex:GRM-qual-real}
dɪ́ɪ́ŋ {\it sg.}, dɪ́ɪ́má {\it pl.} ({\sc cl.}3) `true, real' 

  \end{xlist}
\end{exe}


As shown in (\ref{ex:GRM-qual}), qualifiers agree in number with the head of the
noun phrase. 

\begin{exe}
\ex\label{ex:GRM-qual-agree}
 \begin{xlist}
  \ex\label{ex:GRM-qual-agree-sg}

\gll a nɪhããŋ pɔlɪɪ\\
{\art} woman.{\sc cl.3.sg} fat.{\sc cl.4.sg}   \\
\glt `The fat woman'


  \ex\label{ex:GRM-qual-agree-pl}

\gll a nɪhããna pɔlɛɛ\\
{\art} woman.{\sc cl.3.pl} fat.{\sc cl.4.pl}   \\
\glt `The fat women'

  \end{xlist}
\end{exe}


Many qualifiers are assigned to noun class 4, the
reason being that qualifiers are often nominalized verbal stems, e.g. {\S
hʊlɪɪ/hʊlɪɛ} ({\it qual}) `empty' $<$ {\S hʊl}  ({\it v})  `dry'.  Examples are
provided in  (\ref{ex:GRM-qual-cl4}).


\begin{exe}
\ex\label{ex:GRM-qual-cl4}
 \begin{xlist}
  \ex\label{ex:GRM-qual-cl4-call}
jɪra `call' $>$ jɪ́rɪ́ɪ́  {\it sg.},  jɪ́rɪ́ɛ́  {\it pl.} `calling'
\ex\label{ex:GRM-qual-cl4-give-birth}
lʊla `give birth' $>$ lʊ́lɪ́ɪ́ {\it sg.},   lʊ́lɪ́ɛ́ {\it pl.} `giving birth'
\ex\label{ex:GRM-qual-cl4-die}
sʊwa `die' $>$ sʊ́wɪ́ɪ́ {\it sg.},  sʊ́wɪ́ɛ́ {\it pl.} `corpse'

  \end{xlist}
\end{exe}



Nonetheless, the two categories, noun and qualifier, are differentiated by the
following characteristics: (i)  a qualifier must be semantically verbal (i.e. 
denoting a state or an event),
a noun must not necessarily be, and (ii) while a qualifier
modifies a noun,  a  noun functions as  the
nominal argument of the qualifier. The asymmetry is reflected in
(\ref{ex:GRM-qual-hot}).

\begin{exe}
 
 \ex\label{ex:GRM-qual-hot}{\it  /nʊm/ `hot'}
 \begin{xlist}
  \ex\label{ex:GRM-qual-hot-cmp-stem}
  \glll nɪ́ɪ́nʊ̀ŋ nà \\
 nɪɪ-nʊŋ na \\
     water-hot {\sc foc}\\
  \glt `It is HOT WATER.'

 \ex\label{ex:GRM-qual-hot-head}
  \glll  à nɪ́ɪ́ nʊ́mã́ʊ̃́\\
 a  nɪɪ nʊma-ʊ\\
      {\sc art} water hot-{\sc pfv.foc}\\
  \glt `The water is HOT.'

 \ex\label{ex:GRM-qual-hot-qual}
  \glll  à nɪ́ɪ́ nʊ́mɪ́ɪ́ dʊ̀à dē\\
 [a nɪɪ nʊm-ɪ-ɪ]_{NP} dʊa de\\
  {\sc art} water hot-{\nmlz}-{\sc cl.4} exist {\sc advl}\\
  \glt `The hot water is there.'
  \end{xlist}
\end{exe}

In (\ref{ex:GRM-qual-hot-cmp-stem}) the stem {\S nʊm} `hot' is part of the
complex stem noun {\S nɪ́ɪ́nʊ̀ŋ} `water-hot' (see section
\ref{sec:GRM-com-stem-noun}).  In this morphological configuration, a
qualitative
modification is  established  between the stem {\S nʊm} and the stem {\S nɪɪ}.
In (\ref{ex:GRM-qual-hot-head}), {\S nʊm}  functions as a verbal predicate in
the
intransitive clause, and the definite noun phrase {\S a nɪɪ} `the water'
occupies
the argument position. In (\ref{ex:GRM-qual-hot-qual}) the stem {\S nʊm} is
nominalised and the singular of  noun class 4 is suffixed. The word {\S
nʊ́mɪ́ɪ́} may be translated as  `the result of heat'. It is treated as a
qualifier since {\S nɪɪ} `water' is  (the head of) the argument of the
predicate, and {\S dʊa} is a predicate which needs   one core argument. Since 
{\S nʊm}  can neither function as main predicate nor as head noun of the
argument phrase, and since {\S nʊm}  is understood to be a property of the
entity
and not of the event, then {\S nʊm} in (\ref{ex:GRM-qual-hot-qual}) is viewed as
a qualifier.


Given the arguments put forward, one could analyze the qualifiers as adjectives.
Both are  seen  categorically as nominals  and semantically as properties or
states.  However, there are no lexemes in Chakali  which can be assigned
the category adjective, that is, no lexeme which, in all  linguistic contexts,
can be identified as categorically distinct from nouns and verbs.  For instance,
the lexeme `intelligent' in English is an adjective in all linguistic contexts
and can `never' function as a noun or as  a verb.\footnote{Although
`intelligent' is a noun in a construction like `the
intelligent find cryptic crosswords challenging' (S. Foldvik, p.c.),  it would
not be
controversial to say that `intelligent'  undergoes a zero-derivation, i.e.
adjective $>$ noun.  As mentioned in footnote \ref{ft:GRM-fre-eng-deri},
distinctions between categories in English and French (and other Indo-European
languages) are often not formally signaled. }  There are no such lexemes in
Chakali. Qualifiers are
either derived
linguistic entities or idiomatic
expressions. More than one procedure is attested to construct qualifiers. In
(\ref{ex:GRM-qual-types}),   some types of qualifiers are provided.

\begin{exe}
\ex\label{ex:GRM-qual-types}
 \begin{xlist}
 \ex\label{ex:GRM-qual-t0} àbúmmò `black'  
     \ex\label{ex:GRM-qual-t1} àpʊ́lápʊ́lá `pointed, sharp'
  \ex\label{ex:GRM-qual-t2}  wɪ̀ɛ́zímíí  `wise' 

    % \ex\label{ex:GRM-qual-t-3} síínʊ̀màtɪ́ɪ́nà `wild'

      %  \ex\label{ex:GRM-qual}
 \end{xlist}
\end{exe}


The expression {\S bummo} in (\ref{ex:GRM-qual-t0}) is a nominal lexeme. When
it functions as a qualifier within a noun phrase,  the prefix vowel {\S a-} is
suffixed to the nominal stem. Notice that this prefix vowel also occurs on
numerals (see chapter \ref{sec:NUM}). The type of qualifier found in
(\ref{ex:GRM-qual-t1}) is often used to
describe perceived patterns, including color, texture, sound, manner of motion,
e.g. {\S gã́ã́nɪ̀gã́ã́nɪ̀} `cloud state',  {\S adʒìnèdʒìnè}
`yellowish-brown',  {\S tùfútùfú} `smooth and soft'. Reduplication
characterises the form of this type of qualifiers. When a reduplicated qualifier
occurs in attributive function, i.e. following the head noun, it takes the
prefix {\S a-} as well.\footnote{Although the prefix {\F a-} on qualifier tends
to disappear in
normal speech. The prefix {\F a-} is unacceptable in (\ref{ex:GRM-qual-t2}).}
The word in (\ref{ex:GRM-qual-t2}) is segmented as [[[{\sc
theme}-v]-{\sc nmlz}]-{\sc cl.4}]. The verbal stem {\S zɪm} `know'   sees  its
theme argument incorporated, i.e.  {\S wɪɛ-zɪm} `matters-know',  a structure
which is in turn nominalized by what I called in section \ref{sec:GRM-der-agent}
event-nominalization.  The qualifiers in
(\ref{ex:GRM-qual-types}) are presented 
in corresponding syntactic positions in (\ref{ex:GRM-qual-types-bar}). 

%  The prefix {\S a-} is also found with verbal state
% qualifiers.


\begin{exe}
 
 \ex\label{ex:GRM-qual-types-bar}
 \begin{xlist}

 \ex   {\I  [X àbúmmò]_{NP} dʊ̀à dé}  `The black X  is
there'
  \ex   {\I  [X àpʊ́lápʊ́lá]_{NP} dʊ̀à dé } `The
sharp  X is there'
\ex    {\I  [X wɪ̀ɛ́zímíí]_{NP}  dʊ̀à dé } `The wise X is
there'
  \end{xlist}
\end{exe}



There are  limitations
on the number of qualifiers allowed within a noun phrase. Noun phrases with 
more than three qualifiers are often rejected by language consultants in
elicitation sessions.  The
language simply employs other strategies to stack properties. In fact noun
phrases with two qualifiers are rarely found in the texts
collected. 

The language has phrasal expressions which correspond to  monomorphemic
adjectives in some other languages. These expressions have the characteristic of
being metaphorical; their lexemic denotations may be seen as secondary, and
phrasal  denotations as non-compositional. For instance, a speaker must say {\S
ʊ kpaɣa bambii}, {\it lit.}`he has heart', if he/she wishes to express `he is
brave'. The word `brave' cannot be translated to {\S bambii}, since its primary
meaning is `heart',  but to {\S kpaɣa bambii}  `to be brave'. Another way of
expressing `brave'  is {\S bambii-tɪɪna}, {\it lit.} `owner of heart'. Other
examples  are {\S síí-nʊ̀mà-tɪ́ɪ́nà}, {\it lit.} `eye-hot-owner', `wild,
violent  person'   and {\S síí-tɪ̄ɪ̄nà}, {\it lit.} `eye-owner', `stingy,
greedy person'. These expressions are more frequently used as nouns in the
complement position of the identificational construction, such as in {\S ʊ̀
jáá sísɪ̀àmàtɪ̀ɪ̀ná}, {\it lit.} she is eye-red-owner, `she is serious'.
As mentioned in section \ref{sec:GRM-idiom},  it is often hard to establish
whether an expression is idiomatic when only one of the its components
is used in a non-literal sense.


%\begin{table}[h]
 %\begin{tabular}{lp{4cm}}
%Color & (see section \ref{})\\
%Value & good, bad\\
 %Age & new, old, years old\\
 %Human propensity& mental state, physical state, behavior\\
 %Physical property& sense, consistency, texture, temperature, 
%edibility, sustantiality configuration\\
%Quantity&\\
 %\end{tabular}
%\caption{(frawley
%p.463) \label{tab:GRM-mod}}
%\end{table}

% Examples of 
% physical properties encoding
% 
% Sense
% Consistency
% Temperature
% Edibility
% Substantiality
% Configuration
% 






\subsection{Quantifiers}
\label{sec:GRM-quantifier}

%complex quantifier
Quantifiers are expressions denoting quantities. They refer to the size of the
referent ensemble. The words {\S muŋ} `all',   {\S banɪɛ} `some' and {\S tama}
`few, some' constitute the  monomorphemic quantifiers. The  former can be
expanded with a  nominal prefix. For instance, in {\S ba-muŋ} `{\sc hum}-all'
and {\S wɪ-muŋ} `{\sc abst}-all',  the prefixes identify the semantic class of
the entities which the expressions quantify (see section \ref{sec:classifier}). 
The form of the quantifier {\S banɪɛ} `some'  is  invariable: *{\S anɪɛ}, *{\S
abanɪɛ} and *{\S babanɪɛ} are unacceptable words.  The same can be said for the
word {\S tama}
`few', which stays unchanged even  when it  modifies  nouns of different
semantic classes.  Another word treated as quantifier is {\S máŋá} `only' as
in {\S a nɪhããŋ maŋa kaa waa} `Only the woman is coming'.  The  lexeme {\S
kan}
`abound' is semantically verbal but turns into a quantifier when {\S kɪŋ-}  is
prefixed to it, i.e.  {\S  kɪŋkan} `much' or `many'.  Other evidence for its
verbal status  is the utterance {\S à kánã́ʊ̃́} `they are many' compared to
{\S à jáá tàmá} `they are few'.  In the former, {\S kan} is the
main verb
of an intransitive perfective clause, while in the latter, {\S tama} is the
complement of the verb {\S jaa} in an identificational construction  (section
\ref{sec:GRM-ident-cl}). Apart from
{\S  kɪŋkan} `many',  other plurimorphemic (or complex) quantifiers are based on
the suffixation the morpheme {\S  -lɛɪ} `not'. The expression {\S
wɪ-muŋ-lɛɪ}, {\it lit.} {\sc abst}-all-not, as well as {\S kɪŋ-muŋ-lɛɪ},  {\it
lit.} {\sc conc}-all-not,  both correspond to the English word `nothing'. 
%may be the word maŋa `only'?
%The word {\S kɔta}`is a measure term from English quarter.

The meaning `a few' can be conveyed by  the word {\S aŋmɛna} `how
much/many', which was introduced in section \ref{sec:GRM-interg-pro} as an
interrogative
word. Example  
(\ref{ex:GRM-quant-int-only}) suggest that the word {\S aŋmɛna} can also be
used in a non-interrogative way,  co-occurring here with {\S maŋa} `only',  in
which case it is interpreted  as `amount' or `a certain number'.


\begin{exe}
 \ex\label{ex:GRM-quant-int-only}
\gll áŋmɛ̀nà máŋá tʃájɛ̀ɛ̀\\
   amount only remain.{\pfv}\\
\glt `Only a few are left.'
\end{exe}

Another way to express `(a) few'  is to duplicate the numeral {\S dɪgɪɪ} `one',
e.g.
{\S dɪgɪɪ-dɪgɪɪ ra} `There are just a few of them'.  The  examples in 
(\ref{ex:GRM-quant-mean}) show that the numeral {\S dɪgɪɪ} `one' can
participate in the denotations of both total and partial quantities. 

\begin{exe}

 \ex\label{ex:GRM-quant-mean}
 \begin{xlist}

 \ex  {\I muŋ} `all' ({\it total collective})
 \ex  {\I  dɪgɪɪ muŋ} `each' ({\it total distributive})
 \ex  {\I  dɪgɪɪ dɪgɪɪ} `some, few' ({\it partial distributive})
  \end{xlist}
\end{exe}

The word {\S galɪŋga} `waist' or `middle'  can also carry quantification. In
(\ref{ex:GRM-most}),  {\S galɪŋga} is equivalent to {\S bakaŋ} (< {\I bar-kaŋ},
{\it lit.} part-abound),  and means `most'.

\begin{exe}
 \ex\label{ex:GRM-most}
\gll   a kpããma  galɪŋga/bakaŋ  tʃajɛɛ a lau nɪ\\
{\art} yam.{\pl} most remain.{\pfv} {\art} farm.hut {\postp}\\
\glt  `Most of the yams remain/are left in the farm hut.'
\end{exe}

The word {\S gba} `too' is treated as a quantifier and restricted to appear
after the subject, e.g. (\ref{ex:GRM-too-out-1})-(\ref{ex:GRM-too-out-4}). In
(\ref{ex:GRM-too-pos}), the speaker  considers himself/herself  as part of a
previously established set of individuals who beat their respective child. The
quantifier is additive such that  the denotation of the subject constituent is
added to this previously established set.  In (\ref{ex:GRM-too-neg}), it is
shown that negating the quantified expression results in an interpretation where
the speaker asserts that he/she is not a member of the set of individuals who
beat their child. Since generally there is only one `in focus' constituent in a
clause and that negation and focus cannot co-occur (see sections
\ref{sec:GRM-foc-neg} and  \ref{sc:GRM-focus}), example (\ref{ex:GRM-too})
suggests that {\S gba} is not a focus particle.


\begin{exe}
 \ex\label{ex:GRM-too}
\begin{xlist}
 \ex\label{ex:GRM-too-pos}
\gll ŋ̩̀ gbà máŋá m̩̀ bìè rē \\
{\sc 1.sg} {\quant} beat {\sc 1.sg.poss}  child {\foc}\\
\glt  `I beat my child too.' ({\it lit.} I too/as well/also beat my
child)

 \ex\label{ex:GRM-too-neg}
\gll ŋ̩̀ gbà lɛ̀ɪ́ máŋá  m̩̀ bìé  \\
{\sc 1.sg} {\quant}  {\neg} beat {\sc 1.sg.poss}  child \\
\glt  `I do not beat my child.' ({\it lit.} It is not the case that I also
beat my child)


 \ex\label{ex:GRM-too-out-1}   \textasteriskcentered  gba m̩  maŋa a bie re
\ex \textasteriskcentered  m̩ maŋa gba a bie re
 \ex \textasteriskcentered  m̩ maŋa  a bie gba re
 \ex\label{ex:GRM-too-out-4} \textasteriskcentered  m̩ maŋa  a bie  re gba
\end{xlist}
\end{exe}

\subsection{Demonstratives}
\label{sec:GRM-demons}


Unlike the pronominal demonstrative which acts as a noun phrase, 
a demonstrative
within the noun phrase modifies the head noun. The demonstratives in the noun
phrase are identical to the demonstrative pronouns introduced in section
\ref{sec:GRM-demons-pro}, i.e. ({\it sg.}/{\it pl.}) {\S haŋ}/{\S hama}.  



\begin{exe}
   \ex\label{ex:GRM-dem-sg}{\it Priest talking to the shrine, holding a kola
nut above it}

\gll  ma laa [kapʊsɪɛ haŋ]_{NP} ka ja mɔsɛ tɪɛ wɪɪ tɪŋ ba ta buure\\
{\sc 2.pl} take kola.nut {\sc dem} {\sc conn} {\sc 1.pl} plead give matter {\sc
art} {\sc 3.pl.b} {\sc  egr} want\\
\glt   `Take this kola nut, we implore  you to give them what they desire.'

\end{exe}

Demonstrative  modifiers are mostly used in spatial deixis, but they do not
encode a proximal/distal distinction. Further, when a speaker uses {\S haŋ}  in
a non-spatial context, he/she tends to ignore the plural form (see example
(\ref{ex:GRM-dem-num}) below). In example (\ref{ex:GRM-dem-sg-non-spatial}), the
 demonstrative is placed before the quantifier,  which is not its canonical
position, as will be  shown in the summary examples in section
\ref{sec:GRM-NP-sum}.\footnote{The plural form of {\F tɔʊ} `village' in Katua is
{\F tɔsɪ}. In the lect of Katua, the noun classes resemble the noun classes of
the Pasaale dialect, especially the lect of the villages  Kuluŋ and Yaala.} 



\begin{exe}
   \ex\label{ex:GRM-dem-sg-non-spatial}

\gll  dɪ ʊ nʊ̃ʊ̃  dɪ [tʃakalɪ tɔsɪ haŋ muŋ]_{NP}, dɪ biisa jaa nɪhɪɛ̃, banɪɛ̃ 
ka bɪ ŋma dɪ sɔɣla jaa nɪhɪɛ̃ \\
{\sc comp} {\sc 3.sg}  hear {\sc comp} Chakali villages {\sc dem} {\sc quant}
{\sc comp} Biisa {\sc ident} old some {\sc egr} {\sc itr}  say {\sc comp}
Sogla {\sc
ident} old \\
 \glt `He hears that of all  Chakali
settlements, some say that Biisa (Bisikan) is the oldest,  some
also say Sogola is the oldest.' ({\it Katua, 28/03/08, Jeo Jebuni})

\end{exe}


%How does one makes the difference then? but   we notice that by adding the
%article {\S a} one can capture the meaning of the  proximal/distal
% distinction. 

The examples in (\ref{ex:GRM-dem}) show that the typical position of  the
demonstrative is after the head noun and before the postposition, after the
numeral,  but before the article {\S
tɪŋ}. 


\begin{exe}
  \ex\label{ex:GRM-dem} 
 \begin{xlist}
 
  \ex\label{ex:GRM-dem-n-postp} 
 \gll [tʃʊɔsa haŋ]_{NP} nɪ n̩ di kʊʊ ra\\
 morning {\sc dem} {\sc postp} {\sc 1.sg} eat t.z. {\sc foc} \\
 \glt  `This morning I ate T.Z.'

   \ex\label{ex:GRM-dem-num} 
 \gll [nara balɪɛ haŋ]_{NP} na sɛwa a mʊr\\
person two {\sc dem}  {\sc foc} write {\sc art} story\\
\glt `THESE TWO MEN wrote the story.' 

   \ex\label{ex:GRM-dem-art}
 \gll laa [musa zɪmɪ̃ɪ̃ haŋ tɪŋ]_{NP}\\
 collect Musa fowl  {\sc dem} {\sc art}\\
 \glt `Collect  Musah's  fowl'  


 \end{xlist}
\end{exe}


% \subsection{}
% \label{sec:GRM-}
%gba = too


\subsection{Focus and negation}
\label{sec:GRM-foc-neg}

When the focus is on a noun phrase, the free-standing particle {\S ra} appears
to the right of the noun phrase (see section \ref{sec:focus-forms} for the
various forms the focus particle can take). The particle {\S lɛɪ} `not'  negates
a noun phrase, but it is part of the word in the case of a complex quantifier
(see sections \ref{sec:GRM-quantifier}  and  \ref{sec:classifier}). Focus and
negation particles cannot co-occur together in a single noun phrase.  

\begin{exe}
 \ex\label{ex:GRM-foc-neg} 
 
 \gll [a diebise hama]_{NP} lɛɪ, [hama]_{NP} ra\\
  {\sc art} cats {\sc dem.pl} {\sc neg}  {\sc dem.pl} {\sc foc}\\
 \glt   `Not these cats, THESE CATS.'
\end{exe}

In  (\ref{ex:GRM-foc-neg}), {\S lɛɪ} `not' negates the noun phrase {\S a diebise
hama} and {\S ra} puts the focus on the demonstrative pronoun {\S hama},
referring to a different set of cats.  Both focus and negation particles can be
thought as having scope over the noun phrases, functioning as discourse
(pragmatic) particles. 


\subsection{Coordination of nominals}
\label{sec:GRM-coord-nom}

\subsubsection{Conjunction of nominals}
\label{sec:GRM-conjunc-nom}



The coordination of nominals is accomplished by means of the conjunction
particle {\S
anɪ} ({\it gl.} {\sc conn}).  The vowels of the connective are heavily
centralized and the initial vowel is often dropped in fast speech.
The particle can be weakened to [nə], or simply [n̩], when the preceding and
following phonological material is vocalic.  A coordination of two indefinite
noun phrases is displayed in (\ref{ex:GRM-coor-ani}). 

\begin{exe}
 \ex\label{ex:GRM-coor-ani} 
 
 \gll vaa anɪ diebie ka vala \\
dog {\sc conn} cat {\sc  egr} walk\\
 \glt  `A dog and a cat are walking.'
\end{exe}

 The coordination of a sequence of
 more than two nouns is given in (\ref{ex:GRM-coor-sequen}). It is possible to
repeat the 
connective {\S anɪ}, but a pause between the items in a
sequence is more
frequently found. 

\begin{exe}
 \ex\label{ex:GRM-coor-sequen} 
 
 \gll  bʊ̃ʊ̃ŋ, vaa anɪ diebie ka vala \\
 goat,  dog {\sc conn} cat {\sc  egr} walk\\
 \glt   `A goat, a dog and a cat are walking.'
\end{exe}

When a sequence of  two  modified nouns are conjoined, the head of the second
noun phrase may be omitted if it refers to the same kind of entity as
the first head noun. This is shown in (\ref{ex:GRM-coor-sequen-kind}).


\begin{exe}
 \ex\label{ex:GRM-coor-sequen-kind}
 
 \gll n̩ kpaɣa takta zen ne anɪ (takta) abummo  \\
{\sc 1.sg} have shirt large  {\sc foc} {\sc conn} (shirt)  black \\
 \glt   `I got a large shirt and a black shirt.'
\end{exe}

If the conjoined noun phrase is definite, the article {\S tɪŋ}
follows both conjuncts. This is shown in (\ref{ex:GRM-qual-conj}) where the
connective appears between two qualifiers.

\begin{exe}
 \ex\label{ex:GRM-qual-conj}
 
 \gll   a kor abummo anɪ apʊmma tɪŋ\\
{\sc art}  bench black {\sc conn} white {\sc art} \\
 \glt   `the black and white chair'
\end{exe}


%1.I got large and black shirts (one set with two properties). 'n kpaga takta
%zeng ani abulonso lo. Note: abummo is singular and abulonso is plural. So since
%it's black shirts, it should be takta bulonso.

%2.I got large shirts and black shirts (two sets of more than one shirt) 'n
%kpaga(kpagi) takta zenie ani abulunso lo. Note that the singular of large is
%'zeng' and the plural is 'zenie' so in the above case, large shirts will be
%'takta(takdi) zenie. Those in the brackets are the Motigu tone.

%3.I got a large shirt and a black shirt (two sets of one shirt) 'n kpaga(kpagi)
%takta(takdi) zeng ani abummo(abummaa) lo(la).

%\clearpage
When the weak personal pronouns are conjoined there are limitations on the order
in which they can appear. The disallowed sequences seem to be caused by two
constraints. First, consultants usually disapproved   the sequences where a
singular pronoun is placed after a plural one. Examples are provided in
(\ref{ex:GRM-conj-const-1}).




 \begin{minipage}[h]{12cm}
\begin{exe}
\ex\label{ex:GRM-conj-const-1}
\begin{xlist}

\begin{multicols}{2}

\ex\label{ex:GRM-conj-const-1-g}{\it Acceptable}\\
1.sg {\sc conn} 2.pl $>$ /n̩ anɪ ma/ \\
`I and you ({\it pl})'\\
1.sg {\sc conn} 3.pl.{\sc g}a  $>$   /n̩ anɪ a/ \\
`I and they ({\it hum-})'\\
3.sg  {\sc conn} 2.pl $>$ /ʊ anɪ ma/ \\
 `she and you ({\it pl})'\\
3.sg {\sc conn} 3.pl.{\sc g}b $>$   /ʊ anɪ ba/\\
`she and they ({\it hum+})'


\ex\label{ex:GRM-conj-const-1-ng}{\it Unacceptable}\\
2.pl {\sc conn} 1.sg  $>$ */ma  anɪ n̩/\\
\\
3.pl.{\sc g}a   {\sc conn} 1.sg  $>$  */a anɪ n̩/\\
\\
2.pl  {\sc conn} 3.sg $>$ */ma anɪ ʊ/\\
\\ 
3.pl.{\sc g}b  {\sc conn} 3.sg $>$ */ba anɪ ʊ/\\
\\
\end{multicols}
\end{xlist}
\end{exe}
 \end{minipage}
\vspace*{15pt}


Secondly, the first person pronoun {\S n̩} cannot be found after the
conjunction, irrespective of the pronoun preceding it. The reason may be a
constraint on the syllabification of two successive nasals.  In
(\ref{ex:GRM-coor-sequen-nasal}), it is shown that the vowels of the conjunction
{\S anɪ} either  drops or assimilates the quality of the following vowel. In
addition, a segment  {\S n} is inserted between the conjunction and the
following pronoun. 








\begin{exe}
 \ex\label{ex:GRM-coor-sequen-nasal}
/ʊ anɪ ʊ/  3.sg.  {\sc conn} 3.sg. $>$ [ʊ̀nʊ́nʊ̀]  `she and she' \\
/ʊ anɪ ɪ/  3.sg.  {\sc conn} 2.sg. $>$ [ʊ̀nɪ́nɪ̀] `she and you'\\
/n̩ anɪ ʊ/  1.sg.  {\sc conn} 3.sg. $>$ [ǹ̩nʊ́nʊ̀] `I and she'\\
/n̩ anɪ ɪ/  1.sg.  {\sc conn} 2.sg. $>$ [ǹ̩nɪ́nɪ̀]  `I and you' \\
/ɪ anɪ n̩/    2.sg.  {\sc conn} 1.sg. $>$ *[ɪn(V)nn̩] \\
\end{exe}



 If the first person pronoun {\S
n̩} were to follow the conjunction, there would be  (i) no vowel quality to
assimilate, and (ii) three successive homorganic nasals, i.e. one from the
conjunction, one inserted and one from the first person pronoun, which
would give  rise to a sequence {\S n(V)nn̩}. 

As shown in table \ref{tab:GRM-conj-pron}, these problems do not arise when the
strong pronouns ({\sc st}) are used. 


\begin{table}[htb!]
 \caption[Conjunction of pronouns]{Conjunction of pronouns;  weak
pronoun ({\sc wk}) and   strong pronoun ({\sc st}) \label{tab:GRM-conj-pron}}

  \centering
  \begin{Itabular}{l|llll}
\Hline 
 & 3.sg. \& 3.sg. & 3.sg. \& 2.sg. & 3.sg. \& 1.sg. &
2.sg. \&1.sg.\\ \hline

{\sc wk conn wk} &
ʊnʊnʊ & ʊnʊnɪ & \textasteriskcentered & \textasteriskcentered\\

{\sc wk conn wk} &
ʊnʊnʊ & ɪnʊnʊ &  n̩nʊnʊ &  n̩nɪnɪ\\

{\sc wk conn st} &
ʊnɪwa & ʊnɪhɪŋ &  {\M ʊnɪmɪŋ} &  {\M ɪnɪmɪŋ} \\

{\sc st conn wk} & 
wanʊnʊ & hɪnnʊnʊ & mɪnnʊnʊ & mɪnnɪnɪ\\

{\sc st conn st} &
wanɪwa & wanɪhɪŋ &  {\M wanɪmɪŋ}  & mɪnnɪhɪŋ\\
\Hline
    
  \end{Itabular}
 
\end{table}


\paragraph{Apposition}
\label{sec:GRM-np-apposition}

There is another conjunction-type of nominal coordination. The noun phrase {\S ʊ
ɲɪna kuoru} `her father chief'  in (\ref{ex:GRM-coor-appo}) is treated as two
noun phrases in apposition. In this case, apposition is represented as [[ʊ
ɲɪna]_{NP}[kuoru]_{NP}]_{NP}.
% in which  the definite noun phrase precedes the
%indefinite one.


\begin{exe}
 \ex\label{ex:GRM-coor-appo} 
 
 \gll kùórù bìnɪ̀hã́ã̀ŋ ŋmá tɪ̀ɛ̀ [ʊ̀ ɲɪ́ná kùórù]_{NP} dɪ́ à
báàl párá 	à kùó pétùù  (...)\\
chief young.girl say  give {\sc 3.sg.poss} father 	chief that 
{\art} man  	farm {\art}		farm 	finish.{\foc}  (...)	\\

 \glt  `The chief's daughter told her father that the young
man had finished weeding the farm (...)' (CB 014)
\end{exe}




\subsubsection{Disjunction of nomimals}
\label{sec:GRM-disjunct-nom}

In a disjunctive coordination, the language indicates a
contrast or a choice by means of a high tone and long {\S káá},    equivalent
to 
English `or'. 
The connective {\S káá}  is  placed between  two disjuncts. This is
shown in (\ref{ex:GRM-disjct}). This  connective
 should not be confused with the three conjunctions used to connect verb
phrases and clauses, i.e. {\S aka}, {\S ka} and {\S a} (see section
\ref{GRM-clause-coord}).   

%It can also be used at the end
%of a yes/no interrogative clause (see
%section \ref{}).


\begin{exe}
\ex\label{ex:GRM-disjct}
\begin{xlist}

\ex\label{ex:GRM-disjct-1}

\gll ɪ buure ti re kaa kɔfɪ\\
        {\sc 2.sg} want tea {\sc foc} {\sc conn} coffee\\
\glt  `Do you want tea or coffee?' 
 

\ex\label{ex:GRM-disjct-2}

\gll ɪ buure ti re kaa kɔfɪ ra ɪ dɪ buure\\
        {\sc 2.sg} want tea {\sc foc} {\sc conn} coffee {\sc foc} {\sc 2.sg}
{\sc ipfv} want \\
\glt  `Do you want tea or do you want coffee?' 
 

\end{xlist}
\end{exe}

Example (\ref{ex:GRM-dij-vp4.5}) shows that the same
particle may also occur between
 temporal adverbs. 




\begin{exe}
\ex\label{ex:GRM-dij-vp4.5}

\gll ɪ kaa tʊma tɪɛ a kuoru ro zaaŋ kaa tʃɪa\\
        {\sc 2.sg} {\sc fut} work give {\sc art} chief  {\sc foc} today or
tomorrow\\
\glt  `Will you work for the chief today or tomorrow?' 
\end{exe}



\subsection{Two types of agreement}
\label{sec:GRM-agrrement}

Agreement is a phenomenon which operates
across word boundaries: it is a relation between a controller and a
target in a given syntactic domain. In \cite{Corb04, Corb06} 
  agreement is defined as follow: (i) the element which determines the
  agreement is the controller, (ii) the element whose form is determined by
  agreement is the target and (iii) the syntactic environment in which
  agreement occurs is the domain. Agreement features refer to the information
which is shared in an agreement domain. Finally there may be conditions on
  agreement, that  is, there is a particular type of agreement provided certain
  other conditions apply. Chakali has two types of agreement based on animacy.
They are presented in the two subsequent sections. 

\subsubsection{The gender system}
\label{sec:GRM-gender}

% The glosses for the gender values in the examples are indicated
% with {\it {\sc g}a} for residuals and with {\it {\sc g}b} for human. 

Gender is identified as the grammatical encoding of an agreement class.  
Chakali has four domains
in which agreement in gender can be observed; antecedent-anaphor,
possessive-noun, numeral-noun and  quantifier-noun.
The values shared reflect the humanness property of the referent,
dichotomizing the lexicon of nominals into a set of lexemes $a$ (i.e.
human-) and a set $b$ (i.e.  human+), thus {\sc gender} $a$ or
$b$.  It is usually accepted that ``(g)ender is not restricted to
sex-based classifications (`male/female'); other semantic
possibilities include `animate', `small', `insect', `non-flesh food'
and so on'' \citep[293]{Corb00}. Therefore, treating humanness as gender
is not controversial.

In Chakali the values for the feature {\sc gender} are presented in
table \ref{tab:genders}. A description that specifies the lexemes in those
terms will properly capture agreement in the language.


\begin{table}[htb!]
  \centering
  \caption{Gender in Chakali}
\label{tab:genders}


\subfloat[][Criteria for gender]{
 \begin{tabular}{llll}
\Hline 
{\sc gender} & && Criteria \\ \hline
\textit{a} &&& \textit{residuals}\\
\textit{b} &&& things that are categorized as human\\    
\Hline  \end{tabular}
}

\subfloat[][Gender in weak  and strong third-person
pronouns]{
  \begin{tabular}{lll}
\Hline
Pronoun & {\sc wk}  & {\sc st} \\
Grammatical function  &    {\sc s|o}  &  {\sc s} \\ \hline

{\it 3.sg.}  & {\S ʊ} & {\S wa}\\
{\it  3.pl.a} & {\S a} & {\S awa} \\
{\it  3.pl.b} &  {\S ba} &  {\S bawa} \\
    
\Hline
  \end{tabular}
}

\subfloat[][Agreement prefix forms]{
 \begin{tabular}{lcc}
\Hline
&\textsc{-hum}=\textsc{g}\textit{a}&\textsc{+hum}=\textsc{g}\textit{b}\\
\hline
{\it sg.}&{\S a-} &{\S a-}\\
{\it pl.}&{\S a-} &{\S ba-}\\
\Hline
 \end{tabular}
}


\end{table}



  In addition to
the gender values proposed in table \ref{tab:genders}(a),  a condition
constrains
the controller to be plural to observe the humanness distinction in
agreement. As table  \ref{tab:genders}(b)  and   \ref{tab:genders}(c) show, 
the personal pronouns in the language do not distinguish humanness in
 the singular but only in the plural.\footnote{\cite{Brin07c}  argues
that the situation  violates universal
  37 (and perhaps 45) of \cite{Gree63}: ``A language never has more
  gender categories in non-singular numbers than in the singular''.
  Although very rare, four languages, i.e. Fur (Sudan:  Nilo-Saharan, Fur), 
Kiowa (Oklahoma, USA: Kiowa Tanoan, Kiowa-Towa, Kiowa),  Sinhala (Sri
Lanka: Indo-European, Indo-Iranian,
  Indo-Aryan, Sinhalese-Maldivian) and Dagaare are known to display a pronominal
system resembling that of Chakali. The
information was extracted from Dik Bakker's typological database
(http://www.lotschool.nl/Research/ltrc/agreement.htm).  Vagla (vag),
Deg (mzw), Tampulma (tpm), Safaliba (saf), Hanga (hag) and Waali (wlx), to my
knowledge, can also be considered languages which violate Greenberg's
universal 37.} 

The boundary separating human from non-human is subject to conceptual
flexibility. In story telling non-human characters are ``humanized'', sometimes
called personification, as (\ref{ex:antanaH+}) exemplifies: animals talk, are
capable of thoughts and feelings, and can plan to go to funerals. If one
compares the non-human referents in example (\ref{ex:antanaH+}) and
(\ref{ex:domquantH-}), the former reflects personification, while the latter
does
not.


\begin{exe}
  \ex\label{ex:antanaH+}{\it Domain: antecedent-anaphor}\\
\gll   vaa  mãã  sʊwa   ʊ   ŋma   dɪ   ʊ  tʃɛna  ŋmeliŋmɪ̃ʊ̃   dɪ   
\textbf{ba}  kaalɪ  ʊ mãã     luho \\
       {dog.{\sc sg}} {mother.{\sc sg}} {die} {he} {said} {\sc comp} {his}
{friend} {bird's name} {\sc comp} {{\sc 3pl.g}{\it b}} {go} {his}
{mother} {funeral} \\
\glt `The Dog's mother died. Dog asks his friend Bird to accompany him to his
mother's funeral.'  (lit: (...) that \textbf{they} should go to his mother's
funeral.) 
\end{exe}





In (\ref{ex:domquant}) the quantifier {\S muŋ} `all'
agrees in gender with the nouns {\S nɪbaala} `men' and
{\S bɔlasa} `elephants'.  The form {\S amuŋ} is
used with non-human, irrespective of the number value, and for human if
the referent is unique. The form  {\S bamuŋ} can only  appear in such a phrase
if the referent is human and the number of the referent is greater than one. In
this example a contrast is being made between
human-reference and animal-reference to show that it is not animacy in general 
but
humanness
which presents an opposition in the language.


\begin{exe}
  \ex\label{ex:domquant}{\it Domain: Quantifier + Noun}\\
\begin{xlist}

\ex\label{ex:domquantH+}

\gll   nɪ-baal-a     \textbf{ba}-muŋ \\
       {person({\sc g}{\it b})-male-{\sc pl}} {\sc g}{\it b}-{\sc all} \\
\glt `all men' \\

\ex\label{ex:domquantH-}

\gll   bɔla-sa  \textbf{a}-muŋ\\
        {elephant({\sc g}{\it a})-{\sc pl}}  {\sc g}{\it a}-{\sc all}\\
\glt `all elephants' \\

\end{xlist}
\end{exe}

In section \ref{secːGRM-poss-pro}, it was shown   that the possessive pronouns
have the same forms as
the corresponding weak
pronouns.  In
(\ref{ex:domposs}),  the target pronouns agree with the covert
controller, which is the possessor of the possessive kinship relation.
The nouns referring to goat and human mothers, trigger
{\sc g(ender)}{\it a} and {\sc g(ender)}{\it b}
respectively. In cases where the possessor is covert the proper
assignment of humanness is dependent on the humanness of the possessed
argument (i.e. `their child' is ambiguous in Chakali unless one can 
retrieve the relevant semantic  information of the possessed entity).

\begin{exe}
  \ex\label{ex:domposs}{\it Domain: Possessive (possessor) + Noun}\\
\begin{xlist}

\ex\label{ex:dompossH+}

\gll (mããma.muŋ.na)   \textbf{ba}   bi-se\\
       mother({\sc g}{\it b}).all.{\sc foc}  {\sc poss.3pl.g}{\it b}
{child-{\sc pl}} \\
\glt `their children' (possessor = human mothers) 

\ex\label{ex:dompossH-}

\gll (mããma.muŋ.na)   \textbf{a}   bʊ̃ʊ̃n-a   \\
      mother({\sc g}{\it a}).all.{\sc foc}  {\sc poss.3pl.g}{\it a} {goat-{\sc
pl}} \\
\glt `their goats' (possessor = goat mothers) 




\end{xlist}
\end{exe}
%\hfill{(Tampulma)}
 

%%%%%%%%%%HERE 05-10-09
Example (\ref{ex:domnum}) displays agreement between the numeral
{\S à-náásɛ̀} `four' and the nouns  {\S bʊ̃́ʊ̃̀nà}
({\sc cl.3}) `goats',  {\S tàátá} ({\sc cl.7})
`languages',  {\S vííné} ({\sc cl.5}) `cooking
pots' and  {\S bìsé} ({\sc cl.1}) `children'. The  numerals that agree
in gender with the noun they
modify are  {\S á-lɪ̀ɛ̀} `two',
 {\S à-tòrò} `three',  {\S à-náásɛ̀}
`four',  {\S à-ɲɔ̃́} `five',
 {\S à-lòrò} `six' and  {\S á-lʊ̀pɛ̀}
`seven'. Here again, animate (other than human), concrete (inanimate) and
abstract entities on the one hand, and human on the other hand do not
trigger the same agreement pattern ({\sc anim} in (\ref{ex:domnumHA}),
 {\sc abst} in  (\ref{ex:domabst}),   {\sc conc} in (\ref{ex:domnumI})  vs. 
{\sc hum} in  (\ref{ex:domnumH+})). Clearly, as shown below, noun
class membership is not reflected in agreement ({\S tàátá} ({\sc cl.7})
`languages' triggers {\sc g}a in (\ref{ex:domabst}) and 
 {\S bìsé} ({\sc cl.1}) `children' triggers {\sc g}b in (\ref{ex:domnumH+})).


\begin{exe}
  \ex\label{ex:domnum}{\it Domain: Numeral + Noun}\\
\begin{xlist}

\ex\label{ex:domnumHA}

\gll  {n̩}  {kpaga}  {bʊ̃ʊ̃-na}  {\textbf{a}-naasɛ}  \\
       {\sc 1sg}  {have}  {goat({\sc g}{\it a})-{\sc pl}}  {{\sc 3pl.g}{\it
a}-four} \\
\glt `I have four goats.' \\

\ex\label{ex:domabst}

\gll   {n̩}  {ŋma}  {taa-ta}  {\textbf{a}-naasɛ}  \\
        {\sc 1sg}  {speak}  {language({\sc g}{\it a})-{\sc pl}}   {{\sc
3pl.g}{\it a}-four} \\
\glt `I speak four languages.' \\



\ex\label{ex:domnumI}

\gll  {n̩}  {kpaga}  {vii-ne}   {\textbf{a}-naasɛ}  \\
        {\sc 1sg}  {have}  {cooking.pot({\sc g}{\it a})-{\sc pl}}   {{\sc
3pl.g}{\it a}-four} \\
\glt `I have four cooking pots.' \\



\ex\label{ex:domnumH+}

\gll  {n̩}  {kpaga}  {bi-se}  {\textbf{ba}-naasɛ}   \\
        {\sc 1sg}  {have}  {child({\sc g}{\it b})-{\sc pl}}   {{\sc 3pl.g}{\it
b}-four} \\
\glt `I have four children.' \\


\end{xlist}
\end{exe}



Example (\ref{ex:all}) shows that in a coordination construction
involving the conjunction form {\S (a)nɪ},  the targets display
consistently {\sc g}{\it b} when one of the conjuncts is
human-denoting.  In (\ref{ex:alla}) the noun
phrase {\S a} {\S baal} `the man' and the noun phrase
 {\S ʊ  kakumuso} `his donkeys' unite to form the noun
phrase acting as controller.  The noun phrase  {\S a
 baal nɪ ʊ kakumuso} `the man and his
donkeys' triggers {\sc g}{\it b} on targets.  Consequently, the
form of the subject pronoun, the quantifier, the possessive pronoun
and the numeral must expose  {\S ba} ({\sc 3.pl.}{\it b}).
The rule in (\ref{ex:rule}) constrains coordinate noun phrases to
trigger {\sc g}{\it b} if any of the conjuncts is specified as
{\sc g}{\it b}. No test has been applied to verify whether the
alignment of the conjunct noun phrases affects gender
resolution.






\begin{exe}
  \ex\label{ex:all}{\it Domain: Coordinate structure with {\S nɪ}}\\
\begin{xlist}

\ex\label{ex:alla}

\gll  [{a}  {baal}   {nɪ}  {ʊ}  {kakumu-so}]_{NP}  {vala}  {kaalɪ}  {tamale}
{ra}
\\
      {\sc art} {man} {\sc conn}  {\sc 3.sg.poss} {donkey-{\sc pl}} {walk}  {go}
{Tamale} {\foc} \\
\glt `The man and his donkeys walked to TAMALE' \\

\ex\label{ex:Tamanaphor}

\gll  \textbf{ba}  {kʊ̃ʊ̃wãʊ̃}  \\
       {{\sc 3pl}.{\sc g}{\it b}} tire.{\pfv}.{\foc}\\
\glt `They are tired' \\

\ex\label{ex:Tamquant}

\gll   \textbf{ba}-muŋ  {nããsa} {tʃɔgaʊ}  \\
       {3.{\sc pl}.{\sc g}{\it b}-all} {feet.{\sc pl}} spoil.{\pfv}.{\foc}\\
\glt `All  feet were hurting' \\

\ex\label{ex:Tamposs}

\gll     \textbf{ba}  {nããsa}  {tʃɔgaʊ}  \\
        {{\sc 3.pl.poss.g}{\it b}}  {feet.{\sc pl}}  spoil.{\pfv}.{\foc}\\
\glt `Their feet were hurting them' \\

\ex\label{ex:Tamnum}

\gll   \textbf{ba}  {jaa}    \textbf{ba}-ɲɔ̃  \\
        {{\sc 3.pl.g}{\it b}} {{\sc ident}}
{3.{\sc pl}.{\sc g}{\it
b}-five} \\
\glt `They were five altogether' \\


\ex\label{ex:rule}

{\sc resolution rule}: When unlike gender values are conjoined
(i.e. {\sc gender} {\it  a} and {\sc gender} {\it b}), the
coordinate noun phrase determines {\sc gender} {\it b} (i.e.
{\sc g}{\it a} + {\sc g}{\it a} = {\sc g}{\it a},
{\sc g}{\it a} + {\sc g}{\it b} = {\sc g}{\it b},
{\sc g}{\it b} + {\sc g}{\it a} = {\sc g}{\it b} and
{\sc g}{\it b} + {\sc g}{\it b} = {\sc g}{\it b}).



\end{xlist}
\end{exe}

Examples (\ref{ex:antanaH+}) to (\ref{ex:all}) demonstrate how one can analyse
the humaness distinction as gender. The comparison between humans, animals,
concrete inanimate entities and abstract entities uncovers the sort of animacy
encoded in the language. The next section  presents a phenomenon which
shows
some similarity to gender agreement.



\subsubsection{The classifier system}
\label{sec:classifier}

%get rid of dummy substantive>> classifier
 While there is abundant  literature describing Niger-Congo nominal
classifications and
agreement systems, the grammatical phenomenon  described in
this section  has not received much attention.  Consider the examples in
(\ref{ex:agr1}):      


\begin{exe}
  \ex\label{ex:agr1}
\begin{xlist}


\ex\label{ex:agrA}

\gll 
     {dʒɛtɪ} {kɪn}-{bɔm} {na} \\
	{lion.{\sc sg}} {\sc anim}-{dangerous.{\sc sg}} {\sc foc} \\
\glt  `A lion is DANGEROUS' (generic reading) 

 {\S dʒɛ̀tɪ̀ kɪ́mbɔ́n  ná}\\

\ex\label{ex:agrB}
\gll
      {dʒɛtɪsa} {kɪn}-{bɔma} {ra} \\
	{lion.{\sc pl}}  {\sc  conc; anim}-{dangerous.{\sc pl}} {\sc foc} \\
\glt  `The lions are DANGEROUS' (individual reading) \\
 {\S  dʒɛ̀tɪ̀sá kɪ̀mbɔ́má  rá} \\


\ex\label{ex:agrD}
\gll
{m̩} {bɪɛrɪ-sa} {nɪ}-{bɔma} {ra} \\
	{{\sc poss.1.sg}} {brother.{\sc pl}} {\sc hum}-{dangerous.{\sc pl}} {\sc
foc} \\
\glt  `My brothers are DANGEROUS' \\
 {\S  m̩̀ bɪ̀ɛ̀rəsá  nɪ̀bɔ́má  rá} \\

\ex\label{ex:agrE}
\gll
{ba} {jaa} {nɪ}-{bɔma} {ra} \\
	{{\sc  3pl.g}{\it b}} {\sc ident} {\sc hum}-{dangerous.{\sc pl}} {\sc
foc}  \\
\glt  `They are DANGEROUS' (human participants)  \\
{\S  bà  jáá  nɪ̀bɔ́má   rá} \\


\ex\label{ex:agrF}
\gll
{a} {jaa} {kɪn}-{bɔma} {ra} \\
	{{\sc  3pl.g}{\it a}} {\sc ident}  {\sc  conc; anim}-{dangerous.{\sc
pl}}
{\sc foc} \\
\glt  `They are DANGEROUS' (non-human, non-abstract participants) \\
 {\S  à   jáá   kɪ̀mbɔ́má  rá}\\


\ex\label{ex:agrG}
\gll 
{zaɪɪ} {wɪ}-{bɔm} {na} \\
	{fly.{\sc nmlz}} {\sc abst}-{dangerous.{\sc sg}} {\sc foc} \\
\glt  `Flying is dangerous'  \\
{\S záɪ́ɪ́   wɪ̀bɔ́n ná} \\



\ex\label{ex:agrH}
\gll
     {a} {tʃigisii} {wɪ}-{bɔma} {ra}    \\
	{\sc art} {turn.{\sc nmlz.pv}} {\sc abst}-{dangerous.{\sc pl}} {\sc foc}
\\
\glt  `The turnings  are DANGEROUS' (repetitively turning clay bowls for
drying) \\
 {\S  à tʃígísíí wɪ̀bɔ́má rá} 
\end{xlist}
\end{exe}

The sentences in (\ref{ex:agr1})  are made of two successive noun phrases. The
referent of the first
noun phrase is an entity or a process while the second noun phrase is
semantically headed by a state predicate denoting a property.  Although
speakers prefer the presence of   the  verb {\S jaa} between the two
noun phrases, its  absence is acceptable and does not change the meaning of the
sentence. In these  identificational constructions,  the comment identifies the
topic as having a certain property, i.e. being bad, dangerous or risky. The 
focus marker follows the second noun
phrase, hence  $[$NP1 NP2 ra$]$  means `NP1 is NP2' in which salience or novelty
of information comes from NP2. 


The form of  {\S /bɔm/}   `bad' is determined
by the number value of the first noun phrase. Irrespective of the animacy
encoded in the referent, a  singular noun phrase triggers
the form {\S [bɔŋ]} while a plural triggers {\S [bɔma]} (i.e. {\sc cl.3}).  The
number
agreement is illustrated in (\ref{ex:agrA}) and
(\ref{ex:agrB}).\footnote{Notice that the nominalized verbal lexemes in
(\ref{ex:agrG}) and (\ref{ex:agrH}) each triggers a different form for {\F
/bɔm/}. The  form  {\F tʃigisii}  `turning'  is analyzed as a nominalized
pluractional
verb (see section \ref{sec:GRM-PluralVerb}).}  

Properties do not appear as  
freestanding words in
identificational constructions. To say `the lion is dangerous', the grammar
has to combine the predicate with a dummy substantive, i.e. {\it lit.}  `lion is
{\it
thing}-dangerous',  where {\it thing} stands for the slots where animacy is
encoded. This is represented in (\ref{ex:frame}).  


\begin{exe}
\ex\label{ex:frame}
 $[[$ {\it thing}_{animacy}-property$]_{NP}$ {\sc foc}$]_{NP}$
\end{exe}


In  (\ref{ex:agr1}) there are three dummy substantives:   {\S  nɪ-}, {\S  wɪ-}
and  {\S  kɪn}-.  Each of them has a fully fledged noun counterpart; it can be
pluralized, precede a demonstrative, etc. Those forms are 
{\S 
kɪn}/{\S  kɪna} ({\sc cl.3})  `thing',  {\S  nar}/{\S  nara} ({\sc cl.3})
`person' and {\S 
 wɪɪ}/{\S  wɪɛ} ({\sc cl.4}) `matter, palaver, problem, etc.'.  


\begin{table}[htb!]

  \caption{Classifiers and Nouns   \label{tab:nounclassifier}}
  \centering
  \begin{Gtabular}[h]{ll|lll}
    \hline 
 Classifier   & Animacy & Noun class  & Sing. & Plur.\\
\hline  \hline
   {nɪ-} &  $[${\sc hum}$]$ & Class 3 & nár &  nárá\\ 
 {wɪ-} &  $[${\sc abst}$]$ & Class 4 &    wɪ́ɪ́ &   wɪ́ɛ́ \\
  {kɪn}-  &  $[${\sc conc; anim}$]$ & Class 3&    kɪ̀n &   kɪ̀nà\\
 \hline 
  \end{Gtabular}
\end{table}

 %(see {\it forme radicale} in
%\citet[506]{Mane64})

In table \ref{tab:nounclassifier}, the three possible
distinctions are provided. Also, a dummy substantive is now labelled a {\it
classifier}.  That is because the construction
concordance between the form of
the  classifier and the semantic information encoded in the head of the
first noun phrase reflects one major analytical criterion for  
classifier systems \citep{Dixo86, Corb91, Grin00}. It is clear that the
phenomenon
under investigation shows certain similarities with  
classifier systems, and perhaps could be on a grammaticalization path towards
 being one. Since there are form and sense compatibilities between the
inflecting noun pairs 
and the forms of the expressions preceding the qualitative predicate,  a common 
radical form for each is identified: {\S kɪn}-
{\sc [conc; anim] } `thing, non-human, non-abstract',  {\S nɪ-} {\sc
 [hum] } `person, human being'  and  {\S wɪ-} {\sc  [abst] } `non-concrete,
non-person' are the three classifiers in Chakali. 


All the sentences in (\ref{ex:agr1}) are ungrammatical without a classifier. The
three classifiers  combine with {\S bɔŋ}/{\S bɔma}  to  make  proper
constituents for an identificational construction. They  provide
animacy information, and their forms are determined by the sense properties
relevant
to animacy  encoded in the head of the first noun phrase.  The complex unit made
out of  a property and a licensing classifier is schematically presented
in (\ref{ex:frame2}).  


\begin{exe}
\ex\label{ex:frame2}\textit{Identificational Construction with a classifier}\\

 $[ [^{topic} \ \ {\it head}_{x}  ]_{NP} (jaa)[[^{comment}${\sc
clf}_{x}-property$]_{NP}$ {\sc foc}$]_{NP}]_{S}$
\end{exe}



The structural setting  is the result of a combination of grammatical
constraints which specify that: (i) a property in predicative function cannot
stand on its own, (ii) in predicative function,  a property must be joined with
a classifier, (iii) the merging of the classifier and the property forms a
proper syntactic constituent for an identificational construction, and (iv) the
form of the classifier is dependent on the animacy encoded in the argument of a
qualitative predicate. 

Finally,  classifiers are also found in the formation of the words
meaning  `something' and `nothing'. Consider the examples in 
(\ref{ex:something}) and (\ref{ex:nothing}):


\begin{multicols}{2}
\begin{exe}
  \ex\label{ex:something}
\begin{xlist}
\ex\label{ex:somethingH}
\gll {nɪ-muŋ-lɛɪ}\\
 {\sc hum}-all-not\\
\glt `no one'\\
\ex\label{ex:somethingC}
\gll  {wɪ-muŋ-lɛɪ}\\
 {\sc abst}-all-not\\
\glt `nothing'\\
\ex\label{ex:somethingA}
\gll  {kɪn-muŋ-lɛɪ}\\
 {\sc  conc; anim}-all-not\\
\glt `nothing'\\
\end{xlist}
\end{exe}

\begin{exe}
  \ex\label{ex:nothing}
\begin{xlist}
\ex\label{ex:somethingH}
\gll {nɪ-dɪgɪɪ}\\
 {\sc hum}-one\\
\glt `someone'\\
\ex\label{ex:somethingC}
\gll  {wɪ-dɪgɪɪ}\\
 {\sc abst}-one\\
\glt `something'\\
\ex\label{ex:somethingA}
\gll {kɪn-dɪgɪɪ}\\
 {\sc conc; anim}-one\\
\glt `something'\\
\end{xlist}
\end{exe}
\end{multicols}

As with the role of classifiers in identificational constructions, here again
the classifiers narrow down the tracking of a  referent when one of those
quantifiers is used. The grammar of Chakali arranges the four animacies into
three categories, i.e.  {\sc abst}, {\sc conc; anim} and {\sc hum}.  A
distinction is also made in English between {\sc hum} (i.e. someone, no one) and
 {\sc anim; conc; abst} (i.e. something, nothing), however English does not have
a distinction which captures  specifically abstract entities.

\subsection{Summary}
\label{sec:GRM-NP-sum}

The term nominal in the present context was argued to represent two separate
notions. The first is  conceptual. Nominal stems denote classes of entities
whereas verbal stems denote events. The second notion is  formal. A nominal stem
was opposed to  a verbal stem in noun formation.  As a syntactic unit,  the
nominal  constitutes an obligatory support to the main predicate and was
presented above in  various forms:   as a pro-form, a single noun, or 
 noun phrases
consisting of a noun with a qualifier(s), an article(s), a demonstrative,  among
others.

To summarize, table \ref{tab:npstruc} lists acceptable
noun
phrases. Certain orders are
favored, but a strict linear order, especially among the qualifiers, needs 
further investigation.   Notice that each
noun phrase in (\ref{ex:GRM-np-list}), except for the weak personal pronoun in
(\ref{ex:GRM-pro}) may or may not be in focus and may or may not be definite
(i.e. accompanied by the article {\S tɪŋ}). Also,  the column
{\sc head} in table \ref{tab:npstruc} is not only represented in the
examples by a noun;  example (\ref{ex:GRM-dhq}) is headed by a demonstrative
pronoun. Needless to say, this list of possible distributions of nominal
elements
within the noun phrase is not exhaustive. Again, caution should be taken since
the examples in (\ref{ex:GRM-np-list}), particularly those towards the end of
the list, are the result of controlled elicitation. The order of appearance in 
table  \ref{tab:npstruc} may be   interpreted  as  an approximation of the 
frequency of each kind of noun phrase.

% -Dakubu
% You can do a better job of generalizing.  It looks as though the order of most
% elements is quite regular and only the qualifiers vary.  Even those could
% probably be subclassified according to how many places they can occur


%I got a large black shirt 'n kpaga takta bummo zeng ne'

%n kpagi takta zeng ne ani a bummo
%I got a large black shirt 'n kpaga takta bummo zeng ne'
 %I got a large shirt and a black shirt 'n kpagi takta zeng ne ani a bummo'



%(order frawley p.482)

\begin{table}[htp]
\caption{Noun phrase members and linear order \label{tab:npstruc}}
  \centering
%\begin{small}

\begin{Itabular}{p{1cm}p{.6cm}p{.6cm}p{.6cm}p{.6cm}p{.7cm}p{.6cm}p{.6cm}
p{.6cm}p{.9cm}p{.6cm}}
%\begin{Itabular}{cccccccccccc}
    \Hline


\textsc{art/poss} & \textsc{head}&\textsc{qual}&\textsc{qual}&\textsc{num}
 &
\textsc{quant} &  \textsc{dem} & \textsc{quant} & \textsc{art} &
\textsc{foc/neg} &  ex. \\[1ex] \hline
&$\surd$&&&&&&&&& \ref{ex:GRM-pro} \\
&$\surd$&&&&&&&&($\surd$)& \ref{ex:GRM-h} \\
$\surd$&$\surd$&&&&&&&&($\surd$)& \ref{ex:GRM-ah}\\
$\surd$&$\surd$&&&&&&&$\surd$&($\surd$)& \ref{ex:GRM-aha}\\
$\surd$&$\surd$&&&&&&&&($\surd$)& \ref{ex:GRM-ph} \\
$\surd$&$\surd$&&&&&&&$\surd$&($\surd$)& \ref{ex:GRM-pha} \\
&$\surd$&&&&&$\surd$&&&($\surd$)& \ref{ex:GRM-dhq} \\

&$\surd$&&&&&$\surd$&$\surd$&&($\surd$)& \ref{ex:GRM-hdq} \\

&$\surd$&&&&&$\surd$&&&($\surd$)& \ref{ex:GRM-hd} \\

&$\surd$&&&&$\surd$&&&&($\surd$)& \ref{ex:GRM-hq-all} \\
&$\surd$&&&&$\surd$&&&&($\surd$)& \ref{ex:GRM-hq-many} \\
&$\surd$&&&$\surd$&&&&&($\surd$)& \ref{ex:GRM-hn} \\
$\surd$&$\surd$&$\surd$&&$\surd$&&&&&($\surd$)& \ref{ex:GRM-ahqln} \\
$\surd$&$\surd$&$\surd$&&$\surd$&$\surd$&&&&($\surd$)& \ref{ex:GRM-ahqlnd} \\
$\surd$&$\surd$&$\surd$&$\surd$&$\surd$&&&&&($\surd$)& \ref{ex:GRM-ahqlqln} \\

$\surd$&$\surd$&$\surd$&&&$\surd$&&&&($\surd$)& \ref{ex:GRM-ahqlq} \\
$\surd$&$\surd$&$\surd$&$\surd$&&$\surd$&&&&($\surd$)& \ref{ex:GRM-ahqlqlq} \\
$\surd$&$\surd$&$\surd$&$\surd$&&&$\surd$&&&($\surd$)& \ref{ex:GRM-ahqlqlqd} \\
$\surd$&$\surd$&$\surd$&&$\surd$&$\surd$&&&&($\surd$)& \ref{ex:GRM-phqlnq} \\
%&&&&&&&&&&& \ref{} \\
%&&&&&&&&&&& \ref{} \\
\Hline
  \end{Itabular}


%\end{small}
\end{table}

\begin{exe}
  \ex\label{ex:GRM-np-list} 
 \begin{xlist}
 
  \ex\label{ex:GRM-pro} 
ɪ   `you' \\ {\sc head}

  \ex\label{ex:GRM-h} 
hããŋ   `woman' \\ 
{\sc head} 

  \ex\label{ex:GRM-ah} 
a hããŋ  `the woman'  \\  
{\sc art1} {\sc head} 

  \ex\label{ex:GRM-aha}
a hããŋ tɪŋ   `the woman'   \\  
{\sc art1} {\sc head} {\sc art2}

  \ex\label{ex:GRM-ph} 
ʊ hããŋ  `his woman'  \\ 
 {\sc poss}  {\sc head}

  \ex\label{ex:GRM-pha} 
ʊ hããŋ tɪŋ  `his woman' \\
 {\sc poss} {\sc head} {\sc art2}  

 \ex\label{ex:GRM-dhq}
hama muŋ  `all these'  \\  
{\sc head} {\sc quant}  

 \ex\label{ex:GRM-hdq}
 nɪhããna hama muŋ  `all these women'  \\ 
  {\sc head} {\sc dem} {\sc quant}  

  \ex\label{ex:GRM-hd} 
hããŋ haŋ  `this woman'  \\ 
  {\sc head} {\sc dem}  

  \ex\label{ex:GRM-hq-all} 
 nɪhããna muŋ   `all women'  \\  
{\sc head} {\sc quant}  

\ex\label{ex:GRM-hq-many} 
 nɪhããkana   `many women' \\   
{\sc head-quant}  

  \ex\label{ex:GRM-hn} 
nara balɪɛ   `three person' \\  
  {\sc head} {\sc num}  

  \ex\label{ex:GRM-ahqln} 
a nɪhããna pɔlɛɛ balɪɛ  `the two fat women'  \\ 
 {\sc art1} {\sc head} {\sc qual} {\sc num}  

  \ex\label{ex:GRM-ahqlnd} 
a nɪhããna balɪɛ hama  `these two women' \\ 
 {\sc art1} {\sc head} {\sc num} {\sc dem}  

  \ex\label{ex:GRM-ahqlqln}
 a nɪhããna ɲʊlʊma pɔlɛɛ balɪɛ  `the two fat blind women'  \\
  {\sc art1} {\sc head}  {\sc qual} {\sc qual}  {\sc num}  

\ex\label{ex:GRM-ahqlq} 
 a nɪhããna pɔlɛɛ kana   `many fat women'  \\  
{\sc art1} {\sc head} {\sc qual} {\sc quant}  

 \ex\label{ex:GRM-ahqlqlq} 
 a nɪhããna pɔlɛɛ ɲʊlʊma kana   `many fat blind women'   \\ 
{\sc art1} {\sc head} {\sc qual} {\sc qual} {\sc quant}  

  \ex\label{ex:GRM-ahqlqlqd} 
a nɪhããna pɔlɛɛ ɲʊlʊma kana hama   `these many fat blind women'  \\ 
{\sc art1} {\sc head} {\sc qual} {\sc qual} {\sc quant}  {\sc dem}

 \ex\label{ex:GRM-phqlnq} 
m̩ para awire atoro banɪɛ `some of my three good hoes' \\ 
{\sc 1.sg.poss} {\sc head} {\sc qual} {\sc num} {\sc quant}

%\ex\label{ex:GRM-}
 \end{xlist}
\end{exe}



\section{Verbals}
\label{sec:GRM-verbals}


It was  shown in section 
 \ref{sec:GRM-interr-clause} that a clause consists minimally of  a simple
predicate, one or two arguments and an
optional adjunct.  The structure introduced in (\ref{ex:GRM-clause-frame}) is
repeated below:


\begin{exe}
\exp{ex:GRM-clause-frame}
 {\sc s|a}  $+$ {\sc p} $\pm$ {\sc o} $\pm$ {\sc adj} 
\end{exe}

In this section,  the morphological  and distributional characteristics of
the predicate {\sc p} are discussed.  First, any word or phrase which can take
the place of  the predicate {\sc p} in  (\ref{ex:GRM-clause-frame}) is
identified as `verbal'.  Secondly, the term `verbal'  can refer to a
semantic notion at the lexeme level. The language is analyzed as exhibiting two
types of verbal lexeme: the {\it stative} lexeme and the {\it active} lexeme.  
It was shown in section \ref{sec:GRM-der-agent} that
both types of verbal lexeme take part in nominalization processes. The verbal
stem in (\ref{ex:verb-VP})  must be instantiated with a
verbal lexeme. Thirdly, the
term `verbal' can refer to the whole of the verbal constituent, including the
verbal modifiers.


\begin{exe}
\ex\label{ex:verb-VP}
 
\Tree[-1]{   &\Kq{VG}\Bq{dl}\Bq{dr} &&  \\
\Kq{\it preverb} &&\Kq{\it verb}\Bq{dl}\Bq{dr}&\\
 & \Kq[-3]{\it verbal stem} &&\Kq{\it suffx}}

\vspace*{4ex}

[[{\it preverb}]_{EVG} [{\it stem}]-[{\it suffix}]]_{VG}


\end{exe} 

                 
The verbal group (VG) illustrated in (\ref{ex:verb-VP})
consists of linguistic slots which encode   various aspects of an event  which
may be realized in an utterance. A free standing verb is the minimal requirement
to satisfy the role of a predicative expression. The verbal modifiers, which
are called preverbs here,  are grammatical items which specify the event
according to various  semantic distinctions. They precede the  verb(s) and take
part in the {\it expanded verbal group} (EVG). The expanded verbal group
identifies  a domain which excludes the main verb, so a {\it verbal group}
without preverbs would  be equivalent to a verb or a series of verbs (see SVC in
section \ref{sec:GRM-multi-verb-clause}). The term and notion is inspired from
analyses of the verbal system of Gã \citep{Daku70, Daku08, Daku08b,
Hell10}.\footnote{Although the extended verbal complex in \citet[48]{Hell10}
includes the verb, the decision to exclude the verb(s) from the  expanded verbal
group has no implications for the present exposition. The syntax/semantics of a
SVC is obviously not captured by the simple representation given in
(\ref{ex:verb-VP}) \cite[see][]{Daku08b}. A {\it verbal group} is unlike the
(traditional) verb phrase in that it does not include its internal argument,
i.e. direct object. I am aware of the obvious need to unify the descriptions of
the nominal constituent and the verbal constituent.} While a verbal stem
provides the core meaning of the predication,  a suffix may supply information
on  aspect, whether or not the verbal constituent is in focus and/or the index
of participant(s) (i.e. {\sc o}-clitic).  Despite there being little focus on
tone and intonation, attention on the tonal melody of the verbal constituent is
necessary since this also affects the interpretation of the event. These
characteristics are presented below in a brief overview of the verbal system. 

% 
% one melodies affecting not
% only elements of the verbal constituent but elements immediately preceding or
% following it, and (iii) affixes
%without its participant(s) and other peripheral expressions.



\subsection{Verbal lexeme}
\label{sec:GRM-verb-lexeme}


\subsubsection{Syllable structure and tonal melody}
\label{sec:GRM-verb-syll-und-tone}

 Apart from [dʒ], all other segments  are attested in
verbs.\footnote{The current lexicon contains 388 verbs  (02/10/10 version).} The
majority consist of the open syllable types CV and CVV. The  common
syllable sequences found among the verbs are CV, CVV, CVCV, CVCCV,
CVVCV and CVCVCV.  Monosyllabic verbs make up approximately 21\% of the verbs in
current database, bisyllabic 69\% and trisyllabic  10\%. Apart  from [dʒ], all
segments are attested in onset position word initially, but only {\S m, t, s, n,
r, l, g, ŋ} and {\S w} are found in onset position word-medially in bisyllabic
verbs, and only {\S  m, t, s, n,  l} and {\S g} are found  in onset position
word-medially in trisyllabic verbs.   All trisyllabic, CVVCV  and CVCCV verbs
have one of the front vowels (\{e, ɛ, i, ɪ \}) in the nucleus of their last
syllable.  The data suggests that {\sc atr}-harmony is operative, but not   {\sc
ro}-harmony, in these three environments, e.g. {\S fùólè} `whistle'. 
There
is no restriction on vowel quality for the monosyllabic or bisyllabic verbs
and both harmonies are operative.

% \footnote{Only three verbs have a consonant in final position; {\S zááŋ}
% `greet',   {\S
% fʊ́g} `light' and {\S  kàn} `abundant'.  The latter two are stative verbs. It
% is believed that the search for verbs' underlying form may reveal far more
% verbs with  a consonant in final position. }

Little evidence is available on lexical tonal melodies on verbs lexeme. Two
minimal pairs are identified:  one is {\S télé} `lean on' and {\S tèlè}
`reach', but this may turn out to be a kind of  `aspectual switch' (from stative
to
process, or vice-versa), similar to the category switch discussed in section
\ref{sec:GRM-der-cat-switch}, and the other is {\S pɔ̀} `protect' and  {\S pɔ́}
`plant'.  Still, the high (H) and low (L) register tones are assumed. Table
\ref{tab:GRM-verb-tone-melody} presents some verbs which are classified based on
their syllable structures and  tonal melodies.  Despite the various attested
melodies, low tone CV verbs,   falling tone CVV verbs,   rising CVCV verbs
are marginal in the database. High tone CVV and more than one tone CVCVCV are
not attested.



%\clearpage
\begin{table}[htb]
\renewcommand{\arraystretch}{0.8}
\centering
\caption{Tonal melodies on verbs  \label{tab:GRM-verb-tone-melody}}
        
\begin{Itabular}{llll}
\Hline
Syllable type &  Tonal melody  & Example & Gloss\\ [1ex] \hline

CV 		&  H		&  pó	&	divide	\\
		&  H		& pɔ́		&plant	\\
		&  H		&pú		&cover	\\
		&  H		&pʊ́		&spit	\\
	        &  L 	&pɔ̀		&protect	  \\[0.5ex]
\hline

CVV 	&  L		& pàà 	& 	take\\
		& L		&  tùù	 &	go down	\\
	       & LH 	 &tìé		&	chew	\\
		& LH 	& tʃìí	&	forbid	\\
  		&HL 	& tʊ́ɔ̀  	&	deny\\
		&HL	&   ʔṹũ̀	 &	bury\\[0.5ex] 
\hline

CVCV&  L		&		bìlè	&     put	\\
 	  &  L		&		hàlà	&	fry\\
 	  &  LH		&		mɪ̀ná	 &    attach	\\
 	  &  LH		&		zɪ̀má	&    know	\\
 	  &  HL		&		dʊ́mà	&	bite		\\
 	  &  HL		&		ŋmɛ́nà	&	cut\\
	  &  HH		&		hẽ́sí	&	announce	\\
 	  &  HH		&		jóló	&	pour	\\[0.5ex]
 \hline

CVCCV 	& HH & bóntí	&	divide\\
		& HH &tóŋlí	&	squat \\
		&LL &sùmmè	&	beg \\
		&LL & zèŋsì	&	limp\\[0.5ex] 
 \hline

 CVVCV & HH.H&kpã́ã́nɪ́	&	hunt \\
		&HH.H& wáásɪ́		& reach boiling point	 \\
		&HH.L& vʊ́ʊ́rɪ̀	&	decide \\
		&HH.L& fíísè	&	wipe anus \\
		&LH.L&fùólè	&	whistle \\
		&LH.L&kʊ̀ɔ́rɪ̀	&	make \\
		&LL.L&bùòlì	&	sing\\
		&LL.L& fɪ̀ɪ̀lɛ̀	&	peek\\[0.5ex] 
 \hline

CVCVCV 	& LLL  &làgàmɛ̀	&	gather\\
			& LLL   &dʊ̀gʊ̀nɪ̀	&	chase\\
			& HHH  &ɲúŋísé	&	lose sight of\\		
			& HHH  & vílímí	&	whirl\\[0.5ex]

\Hline
\end{Itabular}      
\end{table} 
 




\subsubsection{Verbal state and verbal process lexemes}
\label{sec:GRM-verb-stative-active}


A verbal lexeme denotes an event, as opposed to an entity. A general distinction
between stative and non-stative events  is made: {\it verbal state} (stative
event) and {\it verbal process} (active event) 
lexemes are assumed. A verbal state lexeme can be identificational,
existential, possessive,  qualitative, quantitative, cognitive or  locative, and
refers more or less to a state or condition which is static, as opposed to
dynamic. The `copula' verbs {\S jaa} and {\S dʊa} (and its allolexe {\S tuwo})
are treated as subtypes of verbal stative lexemes since they are the only verbal
lexemes which cannot function as a main verb in  a perfective intransitive
construction (see section \ref{sec:GRM-verb-perf-intran}). Their meaning and
distribution was introduced in two sections concerned with the identificational
construction (section \ref{sec:GRM-ident-cl}) and existential construction
(section \ref{sec:GRM-loc-cl}).  The possessive 
{\S kpaga} `have'  is treated as  a verbal state lexeme as well (see possessive
clause in section   \ref{sec:GRM-poss-cl}).  A qualitative verbal state lexeme
establishes a relation between an entity and a quality. Examples are given in
(\ref{ex:GRM-v-stat-qual}).

\begin{exe}
\ex\label{ex:GRM-v-stat-qual}{\it Qualitative verbal state lexeme}\\
%{\it Descriptive:}\\
 {\I boro}  `short'  $>$ {\I à dáá bóróó} `The tree is short'\\
{\I goro}    `curved'  $>$ {\I à dáá góróó} `The wood is curved'\\
{\I jɔɣɔsɪ}    `soft'   $>$ {\I   à bìé bàtɔ́ŋ jɔ́gɔ́sɪ́ʊ̀}  `The baby's
skin is soft'
\end{exe}

Similarly, a quantitative verbal state lexeme  establishes a relation between an
entity and a quantity. Yet, in (\ref{ex:GRM-v-stat-quant}), the subject of   {\S
maase} is the impersonal pronoun {\S a} which refers to a situation and not an
individual. The verb {\S hɪɛ̃}  `age' or `old'  is a quantitative verbal state
lexeme in the sense that it can measure objective maturity between
two individuals, i.e. {\S mɪŋ hɪɛ̃-ɪ}, {\it lit.} {\sc 1.sg.st} age-{\sc
2.sg.wk}, `I am older than you'. 


\begin{exe}
\ex\label{ex:GRM-v-stat-quant}{\it Quantitative verbal state lexeme}\\
%{\it Descriptive:}\\
 {\I kana}  `abundant'  $>$ {\I ba kanãʊ̃} `They are plenty (people)' \\
{\I maase} `enough'  $>$   {\I a maasejo keŋ} `It is o.k. like that'\\
{\I hɪɛ̃} `age' $>$ {\I mɪŋ hɪɛ̃ɪ̃} `I am older than you'
\end{exe}

Cognitive verbs such as {\S liise} `think',  {\S kʊ̃ʊ̃} `wonder, 
{\S kisi} `wish',    {\S tʃii} `hate', etc.  are also treated as verbal state
lexemes. Section \ref{sec:SPA-post-verb} will  cover another
type  of verbal state lexeme,  the locative verbs. 

Verbal process lexemes denote non-stative events. They are often partitioned
along the
(lexical) aspectual distinctions of  \cite{Vend57}, i.e. activities, 
achievements, accomplishments. Such verbal categories did not formally
emerge,\footnote{Although I am not able to identify these particular
distinctions, section \ref{sec:GRM-verb-suffix} suggests that there is a system
of verbal derivation which needs to be uncovered.}
nor did I specifically look for them, therefore I am not in a position to
categorize the verbal process lexemes at this point in the research (see
\citet[51]{Bonv88} for a thorough
description of a Grusi verbal system).   Thus, verbs which express that the
participant(s) is actively doing something, undergoes a process, performs an
action, etc. all fall within the  set of verbal process
lexemes. 



\subsubsection{Complex verb}
\label{sec:GRM-complex-verb}

A complex verb is  composed of more than one verbal lexeme. For
instance, when {\S laa} `take' and {\S di}
`eat' are brought together in a SVC (section \ref{sec:GRM-multi-verb-clause}),
they denote a taking and eating event, but
the same sequence can be interpreted  differently. The difference between a
SVC  and what I call a complex verb is that the latter is strictly
collocational and non-compositional. Also, unlike complex stem nouns but like
SVCs, the elements which compose a complex verb
must not necessarily be contiguous.  The sequences  {\S laa}+{\S di}
is glossed  `believe', and {\S laa}+{\S dʊ}, {\it lit.} take put,  `adopt', as
the examples in  (\ref{ex:cpx-verb-laa-di}) show.

\begin{exe}
\ex\label{ex:cpx-verb-laa-di}
\begin{xlist}
\ex
 \gll ǹ̩ láá kùósò díù \\
{\sc 1.sg} take G.  eat.{\foc}  \\
\glt `I believe in God.'

\ex
 \gll  ǹ̩ láá bìé dʊ̀ʊ̀ \\
{\sc 1.sg} take child put.{\foc}  \\
\glt `I adopted a child.'

\end{xlist}
\end{exe}

Other examples are {\S zɪma sii}, {\it lit.} know raise, `understand',  {\S
kpa ta}, {\it lit.}  take abandon, `drop' or `stop', {\S gɪla zɪma}, {\it lit.}
allow know, `prove', among others. 



\subsubsection{Verb forms}
\label{sec:GRM-verb-word}

The base form of a verb is identified as the segmental sequence which  would
appear in a positive imperative clause (section
\ref{sec:GRM-imper-clause}).\footnote{The verbs I elicited in that position
display high tone, but I do not have an exhaustive list so I prefer only to be
specific in equating the verb's base form with its segmental sequence.}
The form also corresponds  with verb elicited in isolation, which in
(\ref{ex:GRM-base-form}) is called the  citation form of the verb. 

\begin{exe}
\ex\label{ex:GRM-base-form}{\it base form = citation form = positive
imperative form}
\begin{xlist}
mara $>$ (dɪ) márá  `Attach!' \\
kpa  $>$  (dɪ)  kpá  `Take!'\\
tele  $>$  (dɪ)  télé  `Lean again!'\\
kpe $>$   (dɪ)  kpé  `Crack and remove (the seed from the shell)!'

\end{xlist}
\end{exe}




The inflectional system of Chakali verbs has  few verb
forms and has a closer
resemblance to neighbor Oti-Volta languages than, for instance,  a
`conservative' Grusi language like Kasem \cite[51]{Bonv88}.\footnote{Dagbani is
described as a language where the ``inflectional system  for verbs is relatively
poor''  \cite[96]{Olaw99}. It has an imperfective suffix {\F -di}
\cite[97]{Olaw99} and  an imperative suffix {\F -ma}/{\F mi} \cite[101]{Olaw99}.
\citet[81]{Bodo97} writes that Dagaare has four verb forms: a dictionary
form, a perfective aspectual form, a perfective intransitive aspectual form and
an imperfective aspectual form. Also for Dagaare, \cite{Saan03}  talks about
four forms: perfective A and B, and Imperfective A  and B.}  Besides the
derivational suffixes (section \ref{sec:GRM-deri-suff}), the verb in Chakali is
limited to two
inflectional suffixes and one assertive suffix:  (i) one signals negation in the
negative imperative clause (i.e.  {\S  kpʊ́} `Kill',  {\S tɪ́  kpʊ́ɪ̀} `Don't
kill'),  (ii) another attaches to some verb stems in the perfective intransitive
only, and (iii)  the other signals assertion and puts the verbal constituent in
focus. Since the negative imperative clause has already been presented in
section
\ref{sec:GRM-imper-clause}, the perfective and imperfective intransitive
constructions are discussed next.  The former may contain both the perfective
suffix and the assertive suffix simultaneously, while the latter  displays the
base form of the  verb, with or without the assertive suffix.

\paragraph{Perfective intransitive construction}
\label{sec:GRM-verb-perf-intran}

As its name suggests, a perfective intransitive construction lacks a grammatical
object and implies the anteriority of an event (i.e. past),  the end  or the
reaching point of an action.  In the case of verbal state,
the  perfective  implies that the given state has been reached, or 
that the entity in subject position   satisfies the property encoded in
the verbal state lexeme. In (\ref{ex:GRM-intperfc-frame}),  two suffixes are
attached on  one  ({\sc +atr}, CV) verbal process stem and one  ({\sc
+atr}, CVCV) verbal state stem (see section \ref{sec:nasalization-verb-suffix}
for the general phonotactics involved).\footnote{The presence of schwa
({\F ə}) in a CVCəCV surface form as in (\ref{ex:GRM-intperfc-frame-state})  is
predicted by a rule operating on weak syllables (section \ref{sec:epenthesis}).}


\begin{exe}
\ex\label{ex:GRM-intperfc-frame}{\it Perfective intransitive construction}
\begin{xlist}

\ex\label{ex:GRM-intperfc-frame-process}{{\it  Verbal process:} {\sc s}  $+$
{\sc p} }
\gll àfíá díjóò\\
A. {di-j[{\sc +mid, -hi, -ro}]-[{\sc +hi,+ro}]}\\

\glt `Afia ate.'

\ex afia wa dije `Afia did not eat'

\ex\label{ex:GRM-intperfc-frame-state}{{\it  Verbal state:} {\sc s}  $+$ {\sc p}
}
\gll à dáá télə́jóò\\
{\art} daa  {tele-j[{\sc +mid, -hi, -ro}]-[{\sc +hi,+ro}]}\\
\glt `The stick leans'

\ex a daa wa teləje `The stick doesn't lean.'
\end{xlist}
\end{exe}

The first suffix to attach is the perfective suffix, i.e. -j[{\sc +mid, -hi,
-ro}] or simply /jE/. Although it appears on every (positive and
negative) stem in (\ref{ex:GRM-intperfc-frame}),  it does not surface on all
verb stems. The
information in table \ref{tab:GRM-perf-suff} partly predicts whether or not a
stem will surface with a suffix, and if it does, which form this suffix will
have.


\begin{table}[htb]
 \centering
\caption{Perfective intransitive suffixes
\label{tab:GRM-perf-suff}}
\begin{Itabular}{p{2cm}p{2cm}p{2cm}}
\Hline
Suffix /-jE/ & Suffix /-wA/ & No suffix  \\[1ex]
\hline

CV &  CVV & 1-CVCV \\
 2-CVCV & & \\ 

 \Hline
\end{Itabular}
\end{table} 

Table \ref{tab:GRM-perf-suff} shows that, in a perfective intransitive
construction, a CV stem must
be suffixed with {\S -jE} and  a CVV verb with {\S -wa}. The examples in
(\ref{ex:GRM-jE-wA}) are negative in order to prevent the assertive
suffix from appearing (see section \ref{sc:GRM-focus} on why negation and the
assertive suffix cannot co-occur).


\begin{exe}
\ex\label{ex:GRM-jE-wA}
\begin{xlist}

\ex{\it CV}\\
po	 $>$  àfíá wá    pójè		`Afia didn't divide'	\\
pɔ		 $>$ àfíá wá   pɔ́jɛ̀	 `Afia didn't  plant'\\
pu		 $>$ àfíá wá  pújè	 `Afia didn't  cover'	\\
pʊ		 $>$ àfíá wá  pʊ́jɛ̀	 `Afia didn't  spit'	\\
kpe		 $>$ àfíá wá  kpéjè	 `Afia didn't  crack and
remove'\\
kpa		 $>$ àfíá wá  kpájɛ̀	 `Afia didn't  take'	

\ex{\it CVV}\\
tuu $>$ àfíá wá  tūūwō   `Afia didn't  go down'\\
tie $>$  àfíá wá   tīēwō `Afia didn't chew'\\
sii  $>$  àfíá wá  sīīwō   `Afia didn't  raise'\\
jʊʊ   $>$  àfíá wá  jʊ̄ʊ̄wā  `Afia didn't  marry'\\
tɪɛ $>$  àfíá wá tɪ̄ɛ̄wā  `Afia didn't  give'\\
wɪɪ $>$  àfíá wá  wɪ̄ɪ̄wā  `Afia is not  ill'
  	
\end{xlist}
\end{exe}

The surface form of the perfective suffix which attaches to CV stems (i.e. {\S
-je}/{\S
-jɛ}) is predicted by the {\sc atr}-harmony rule of section
\ref{sec:vowel-harmony}. Notice that  {\sc ro}-harmony does not operate
in that domain. The CVV stems display  harmony between the stem
vowel(s) and the suffix vowel which is easily captured by a variable feature
alpha notation, as shown in rule (\ref{PHO-rule-perf-wa}).


\begin{Rule}\label{PHO-rule-perf-wa}{Prediction  for perfective intransitive 
-/wA/ suffix}\\
If the vowel of a CVV stem is
{\sc +atr},
the vowel of the suffix is {\sc +ro}, and if the vowel of a CVV stem is {\sc
-atr}, the vowel of the suffix is {\sc -ro}.\\
-/wA/ $>$  $\alpha${\sc ro}_{suffix}  /  $\alpha${\sc atr}_{stem} \_     
\end{Rule}

Rule  \ref{PHO-rule-perf-wa} assumes that the segment [{\S o}] is the
[{\sc +ro, +atr}]-counterpart of [{\S a}]. Notice also that I perceived the
same tonal melody in all clauses, raising doubts  on the tonal melodies of the
`citation forms' offered in table \ref{tab:GRM-verb-tone-melody}.

Predicting the set (i.e. either set 1-CVCV or  set 2-CVCV) into which a CVCV
stem
will fall  has proven unsuccessful. Provisionally,  I suggest that a CVCV
stem must be stored with such an information. One piece of evidence
supporting this claim comes from
the minimal pair {\S tèlè} `reach' and  {\S télé} `lean against':  the
former displays 2-CVCV (i.e. tele-jE),  whereas the latter displays 1-CVCV
(i.e. tele-\O).  However,  a CVCV stem with round vowels is less likely to
behave like a 1-CVCV stem, yet {\S pumo} `hatch' is a counter-example, i.e.
{\S a zal wa puməje} `the fowl didn't hatch'. The CVCCV, CVVCV and CVCVCV stems
have  not been investigated, but {\S kaalɪ} `go', a common  CVVCV verb, takes
the
/-jE/ suffix.  


\paragraph{Imperfective intransitive construction}
\label{sec:GRM-verb-perf-intran}

The imperfective  conveys the unfolding of an event, and it is often used to
describe an event taking place at the moment of speech. In addition, the
behavior of the egressive marker {\S ka} (section \ref{sec:GRM-EVC-egr-ingr})
suggest that the imperfective may be interpreted as a progressive event,  or
one which will happen in the future.  The imperfective is indicated by the base
form of a verb. As in the perfective intransitive, the assertive suffix may be
found attached to the verb stem. 


\begin{exe}
\ex\label{ex:GRM-assert-suff}
[[{\it verb stem}]-[{\sc +hi,+ro}]]_{verb \ in \ focus}
\end{exe}

Again, the constraints licensing the combination of the verb stem and the vowel
features  shown in (\ref{ex:GRM-assert-suff})   are (i) none of the other
constituents in the clause are in focus, (ii) the clause does not include
negative polarity items, and (iii) the clause is intransitive, that is, there is
no grammatical object. 

% , as opposed to an
% event perceived as bounded (i.e. perfective) or a hypothetical event (i.e.
% imperative)



\begin{exe}
\ex\label{ex:GRM-pos-neg-take}
%\begin{multicols}{2}
\begin{xlist}
\ex{\it Positive}

 ʊ̀ kàá kpáʊ̀   `She will take'\\
   ʊ̀ʊ̀ kpáʊ́ 	 `She  is taking/takes'

\begin{xlist}
\ex\label{ex:GRM-ipfv-out-nfoc}
\textasteriskcentered  a baal la  kpaʊ ({\sc art} man {\sc foc} take.{\sc foc})
`The MAN is taking'
\ex\label{ex:GRM-ipfv-out-stpro}
 \textasteriskcentered  wa  kpaʊ ({\sc 3.sg.st} take.{\sc foc})  `HE is
taking'
\ex\label{ex:GRM-ipfv-out-obj}
\textasteriskcentered  a baal   kpaʊ a kisie ({\sc art} man take.{\sc foc} {\sc
art} knife) `The man is taking the knife'
\end{xlist}

\ex\label{ex:GRM-ipfv-out-neg}{\it Negative}\\
 ʊ̀ wàá kpá   `She will not take'\\
   ʊ̀ʊ̀   wàà	 kpá `She  is not taking/does not take'

\end{xlist}
%\end{multicols}
\end{exe}

In (\ref{ex:GRM-pos-neg-take}), the forms of the verb in the
intransitive imperfective take the assertive suffix to signal that the verbal
constituent is in focus, as opposed to the nominal argument. It cannot appear
when the subject is in focus (\ref{ex:GRM-ipfv-out-nfoc}) or when the strong
pronoun is used as subject (\ref{ex:GRM-ipfv-out-stpro}), when a grammatical
object follows the verb  (\ref{ex:GRM-ipfv-out-obj}), or when the negation
preverb {\S waa} is present  (\ref{ex:GRM-ipfv-out-neg}).



\paragraph{Intransitive vs. transitive}
\label{sec:GRM-trans-intran}

Intransitive  and transitive clauses are shown in  
(\ref{ex:GRM-clause-core-intrans}) and (\ref{ex:GRM-clause-core-trans})
respectively.  



\begin{exe}
\ex\label{ex:GRM-clause-core}
\begin{xlist}
 \ex\label{ex:GRM-clause-core-intrans}{\it Intranstive clause}
\glll Kala dijoo \\
       {\sc s} {\sc p}\\
 Kala eat.{\sc pfv.foc} \\
\glt  `Kala ATE.' 
\ex\label{ex:GRM-clause-core-trans}{\it Transtive clause}
\glll Kala di sɪɪmaa ra\\
        {\sc s} {\sc p}  {\sc o} {} \\
Kala eat.{\sc pfv}  food {\sc foc}\\
\glt  `Kala ate FOOD' 
\end{xlist}
\end{exe}


Many verbs can occur in either  intransitive or transitive clauses. In
(\ref{ex:GRM-clause-ambitran}) the subject of the intransitve ({\sc s})
corresponds to the subject of the transitive ({\sc a}), and the same verb is
found with and without an object ({\sc o}).


\begin{exe}
\ex\label{ex:GRM-clause-ambitran}
\begin{xlist}

\ex\label{ex:vp26.14.}
\glll ʊ̀ʊ̀ búólùù \\
{\sc s} {\sc p}\\
       {\psg} sing.{\ipfv.\foc} \\
\glt  `He is singing.' 

\ex\label{ex:vp26.15.}
\glll  ʊ̀ʊ̀ búólù bùòl lò \\
{\sc a} {\sc p} {\sc o} {}\\
       {\psg}  sing.{\ipfv} song {\foc}    \\
\glt  `He is singing a song.' 

\end{xlist}
\end{exe}


It is possible to promote a prototypical theme argument to the subject position.
However,  informants have difficulty with some nominals in the subject
position of
intransitive clauses.   The topic needs further investigation, although it is
certainly related to a semantic anomaly.  The data in
(\ref{ex:GRM-intran-theme-subj}), where the  prototypical {\sc o}(bject) is in
 {\sc a}-position, illustrates the problem. In order to concentrate on the `goat
beating'- and `tree climbing'-activities and turn the two clauses
(\ref{ex:GRM-int-th-su-out-1}) and (\ref{ex:GRM-int-th-su-out-2}) into
acceptable utterances, the optimal solution is to use the
impersonal pronoun {\S ba} in subject position, e.g. \textasteriskcentered  {\S
bʊ̃ʊ̃ŋ   kaa maŋãʊ̃}
$>$  {\S ba kaa maŋa a bʊ̃ʊ̃ŋ (na)} `the goat is being
beaten'  (see impersonal
pronoun in
section \ref{sec:GRM-impers-pro}).



\begin{exe}
\ex\label{ex:GRM-intran-theme-subj}
\begin{xlist}
\ex
a bʊɔ kaa hireu  `the hole is being dug'
\ex\label{ex:GRM-int-th-su-out-1}
\textasteriskcentered a bʊ̃ʊ̃ŋ   kaa maŋãʊ̃  `the goat is being beaten'
\ex\label{ex:GRM-int-th-su-out-2}
\textasteriskcentered a daa kaa zɪnãʊ̃  `the tree is being climbed'

\end{xlist}
\end{exe}


Given that  the inflectional system of the verb is rather poor, and that the 
perfective
and assertive suffixes occur only in intransitive clauses,  how does one
encode a basic contrast like the one between a transitive perfective and
transitive imperfective? The paired examples in (\ref{ex:tra-pfv}) and
(\ref{ex:tra-ipfv})  illustrate 
 relevant contrasts.


  \begin{minipage}[h]{12cm}
\begin{multicols}{2}
\begin{exe}
  \ex\label{ex:tra-pfv}{\it Transitive perfective}
\begin{xlist}
  \ex\label{ex:tra-pfv-eat}
ǹ̩ dí kʊ̄ʊ̄ rā\\
 `I ate T.Z..' 
 \ex\label{ex:tra-pfv-plant}
ǹ̩ pɔ́ dāā rā\\
`I planted a TREE.'
 \ex\label{ex:tra-pfv-cover}
ǹ̩ tʃígé vīī rē\\
`I covered a POT.' 
 \ex\label{ex:tra-pfv-tie}
ǹ̩ lómó bʊ̃́ʊ̃́ŋ ná\\
`I tied a GOAT.' 
 \ex\label{ex:tra-pfv-carry}
m̩̀ mɔ́ná díŋ nē\\
`I carried  FIRE.' 
\end{xlist}
\end{exe}

\begin{exe}
  \ex\label{ex:tra-ipfv}{\it Transitive imperfective}
\begin{xlist}
 \ex\label{ex:tra-ipfv-eat}
ǹ̩ dí kʊ́ʊ́ rá\\
`I am eating T.Z..' 
 \ex\label{ex:tra-ipfv-plant}
m̩̀ pɔ́ dáá rá\\
`I am planting a TREE.' 
 \ex\label{ex:tra-ipfv-cover}
ǹ̩ tʃígè vīī ré\\
`I am covering  a POT.' 

 \ex\label{ex:tra-ipfv-tie}
ǹ̩ lómò bʊ̃̄ʊ̃̄ŋ ná\\
`I am tying  a GOAT.' 
 \ex\label{ex:tra-ipfv-carry}
m̩̀ mɔ́nà dīŋ né\\
`I am carrying  FIRE.' 
\end{xlist}
\end{exe}
\end{multicols}
 \end{minipage}
\vspace*{15pt}

% \begin{exe}
% \ex\label{ex:GRM-imperf-cons}
% \begin{xlist}
% 
% 
% \end{xlist}
% \end{exe}

Each pair in the verbal frames of  (\ref{ex:tra-pfv}) and (\ref{ex:tra-ipfv})
presents fairly regular patterns:  the high tone {\it versus} the falling tone
on the CVCV verbs is one instance. Another is the systematic change of the tonal
melodies on the grammatical objects in the two CV-verb cases. The data suggest
that
it is the tonal melody, and not exclusively the one associated with the verb,
which supports aspectual function in the language.
Thus, when the verb is followed by an argument, both perfective and the
imperfective are expressed with the base form of the verb.  However,  the tonal
melody alone  can determine whether a phonological string is to be understood as
a bounded event which occurred in the past or an unbounded event unfolding at
the moment of speech.



Tonal melody is crucial in the following examples as well. The examples in
(\ref{GRM-pfv-inter}) are three polar questions (see section
\ref{sec:GRM-interr-polar}), one perfective (i.e. past event) and two
imperfective (i.e. present and future events). The two first have the
same segmental content, and the last contains the egressive preverb {\S kaa}
with a rising tone indicating the future tense.  In order to signal a polar
question, each has  an extra-low tone and is slightly lengthened at the end of
the utterance. 

\begin{exe}
\ex\label{GRM-pfv-inter}
\begin{xlist}

\ex\label{GRM-pfv-inter-pfv}
\glll {\T } {\T   } {\T  } {\T  } {\T }\\
 ɪ   teŋesi  a  namɪã  raa \\
          {\sc 2.sg} {cut.{\sc pv}} {\sc art} {meat} {\sc foc}\\
\glt `Did you cut the meat (into pieces)?'\\



\ex\label{GRM-pfv-inter-impf}

\glll {\T } {\T   } {\T  } {\T  } {\T }\\
ɪ   teŋesi  a  namɪã  raa \\
          {\sc 2.sg} {cut.{\sc pv}} {\sc art} {meat} {\sc foc}\\
\glt `Are you cutting the meat (into pieces)?'\\


\ex\label{GRM-pfv-inter-impf-fut}

\glll {\T } {\T  } {\T   } {\T  } {\T  } {\T }\\
ɪ  kaa teŋesi  a  namɪã  raa \\
          {\sc 2.sg} {\sc ipfv.fut} {cut.{\sc pv}} {\sc art} {meat} {\sc foc}\\
\glt  `Will you (be) cut(ting) the meat (into pieces)?'\\

 \end{xlist}
\end{exe}

The only distinction perceived between (\ref{GRM-pfv-inter-pfv})  and
(\ref{GRM-pfv-inter-impf}) is a pitch difference on the third syllable of the
verb. The tonal melody associated with the verb in 
(\ref{GRM-pfv-inter-impf-fut}) is the
same as the one in (\ref{GRM-pfv-inter-impf}).







\paragraph{Ex-situ subject imperfective particle}
\label{sec:GRM-ipfv-part}

One topic-marking strategy is to prepose a non-subject constituent to the
beginning of the clause.  In  (\ref{ex:GRM-foc-top}),  the focus particle may or
may not
appear after the non-subjectival topic. Notice that one effect of this 
topic-marking strategy is that the particle {\S dɪ} appears between the subject
and
the verb when the non-subject constituent is preposed and when the clause is
used to describe what is happening at the moment of speech. This particular
relation between  assertion, aspect and linear order is a phenomenon
which is still obscure.



\begin{exe}
\ex\label{ex:GRM-foc-top}
\begin{xlist}
 \ex\label{ex:GRM-foc-top-chew-pres.prog}{\it Imperfective}
\gll  sɪ́gá (ra)  ʊ̀ dɪ̀  tíè   \\
 bean  ({\foc}) {3\sg} {\ipfv} chew\\
\glt `It is BEANS he is chewing'


 \ex\label{ex:GRM-foc-top-chew-pres.prog}{\it Perfective}
\gll  sɪ́gá (ra) ʊ̀   tìè     \\
 bean  ({\foc}) {3\sg}  chew \\
\glt `It is BEANS he chewed'

 \ex\label{ex:GRM-foc-top-go-pres.prog}{\it Imperfective}
\gll   wáá (ra) ʊ̀ dɪ̀  káálɪ̀   \\
Wa    ({\foc}) {3\sg} {\ipfv} go\\
\glt `It is to WA that he is going'


 \ex\label{ex:GRM-foc-top-go-pres.prog}{\it Perfective}
\gll   wáá (ra)  ʊ̀ kààlɪ̀    \\
Wa   ({\foc}) {3\sg}  go\\
\glt `It is to WA that he went'

\end{xlist}
\end{exe}

The position of {\S dɪ} in  (\ref{ex:GRM-foc-top-chew-pres.prog}) and 
 (\ref{ex:GRM-foc-top-go-pres.prog}), that is between the subject and the verb,
is generally occupied by linguistic items called  {\it preverbs},  to which the
discussion turns next.  Provisionally, the particle {\S dɪ} may be treated as a
preverb constrained to occur with  a preposed  non-subject constituent and
an imperfective aspect.\footnote{I do not treat topicalization in this work,
although the left-dislocation strategy in (\ref{ex:GRM-foc-top}) is the only
one I know to exist.}

\subsection{Preverb particles}
\label{sec:GRM-precerv}

Preverb particles are grammatical morphemes which encode various event-related
meanings. They are part of the verbal domain  called the expanded verbal group
(EVG), which  consist of  one or more preverbs.  These grammatical morphemes are
not verbs, in the sense that they do not contribute to SVCs as verbs do,  but as
`auxiliaries'. Still,  some of the preverbs may historically derive from verbs, 
and  some others may synchronically function as verbs.  Examples of the latter
are the egressive particle {\S ka} and ingressive particle {\S wa},  which are
discussed first. 


Nevertheless, given the data available,  it would not be incorrect to analyze
some of the preverbs  either as additional SVC verbs or as adverbs. Despite
these analytical options, the domain which follows the subject and precedes the
main verb(s) is generally accessible  only to a limited set of linguistic items.
For that reason, the expanded verbal group is a domain whose members are
identified as preverbs. However, we will see  that a preverb differs from a verb
in that it exposes functional categories,  cannot inflect for the perfective or
assertive suffix,  and never takes  a complement, such as a grammatical object,
or cannot be modified by  an adjunct. But again,  a first verb in a
SVC
and a preverb are categories which can be hard to distinguish. Structurally and
functionally, many of them may be analysed as grammaticalized serial verbs.
These characteristics are not special to Chakali; similar, but not identical,
behavior is described for Gã and Gurene \citep{Daku07b, Daku08}.



\subsubsection{Egressive and ingressive particles}
\label{sec:GRM-EVC-egr-ingr}


The egressive particle {\S ka(a)} ({\it gl.} {\sc egr})   `movement away from
the
deictic centre'  and   the ingressive
particle {\S wa(a)} ({\it gl.} {\sc ingr})  `movement towards
the deictic centre' are  assumed to derive from the  verbs
{\S kaalɪ} `go' and  {\S waa} `come'.\footnote{A discussion on some aspects of
grammaticalization of  `come' and `go' can be read in  \cite{Bour92}. In the
literature, egressive  is also known as  {\it itive} (i.e. away from the
speakers,  `thither')  and  ingressive  is  known as {\it ventive} (i.e. towards
the speakers,   `hither'). }  Table
\ref{tab:deict-pre-verb} shows that  {\S kaalɪ} `go' and {\S waa} `come',  like 
other verbs, change forms (and are acceptable) in these paradigms,  but {\S
ka(a)} `go' is  not.


\begin{table}[h]
\centering
\caption{Deictic verbs and preverbs \label{tab:deict-pre-verb}}

\begin{Itabular}{lllll}
\Hline
Verb & $\sigma$  & Aspect & Positive & Negative\\[1ex] \hline


{\I waa} `come' & CV 	& {\sc pfv} 	&  ʊ waawaʊ   & ʊ wa waawa\\
					&&& `she came' & `she didn't come'\\

 		  & 	& {\sc ipfv} 	&  ʊʊ waaʊ  & ʊ wa waa\\
					&&& `she is coming' & `she is not
coming'\\[1ex] \hline




{\I kaalɪ} `go' & CVVCV 	& {\sc pfv} 	&  ʊ kaalɪjʊ   & ʊ wa kaalɪjɛ\\
					&&& `she went' & `she didn't go'\\

 		  & 	& {\sc ipfv} 	&  ʊʊ kaalʊʊ  & ʊ wa kaalɪ\\
					&&& `she is going' & `she is not
going'\\[1ex] \hline


{\I ka} `go' & CV 	& {\sc pfv} 	&  *ʊ kaʊ   & *ʊ wa kajɛ\\
				

 		  & 	& {\sc ipfv} 	&  *ʊ kaʊ  & *ʊ wa ka\\
				
\Hline

 

\end{Itabular}         
\end{table}


If the verbs {\S kaalɪ} `go' and  {\S waa} `come'
occur in a SVC,  they surface as {\S ka} and {\S wa} respectively. In
(\ref{GRM-prev}),  both verbs take part in  a two-verb SVC in which they are
first in the sequence.


\begin{exe}
\ex\label{GRM-prev}
\begin{xlist}

\ex\label{GRM-prev-SVC-ka}
\glll gbɪ̃̀ã́         	bààŋ       	té    	kà         	sáŋá 	à   
píé  {(...)} \\
monkey  	 quickly 	early 	go 		sit  	{\art}
yam.mound.pl   {(...)} \\
{} [[{\it pv} {\it pv}]_{EVG} {\it v} {\it v}]_{VP} {} {}
{(...)}\\
\glt `Monkey quickly went and sat on the (eighth) yam mound (...)'  (LB 012)

\ex\label{GRM-prev-SVC-wa}
\glll    ŋmɛ́ŋtɛ́l   làà nʊ̀ã̀  nɪ́    ká  ŋmá dɪ́    ʊ̀  wá  
ɲʊ̃̀ã̀ nɪ́ɪ́ \\
spider collect mouth {\postp}  {\conn} say {\comp}  
{\sc 3.sg}  come   drink water\\
 {} {} {} {} {}  {} {} {} {\it v} {\it v} {} \\
\glt  `(Monkey went to spider's farm to greet him.)  Spider accepted
(the
greetings) and (Spider) asked him (Monkey) to come and drink water.'  (LB 011)

 \end{xlist}
\end{exe}


Because they derive from deictic verbs (historically or synchronically),  the
preverbs  indicate non-spatial `event movement'  to or from a deictic centre.
This phenomenon is not uncommon cross-linguistically. \citet[62]{Nico07}
maintains  that when a movement verb becomes a tense marker, it may be reduced
to a verbal affix and its meaning can develop ``into meaning relating temporal
relations between events and reference times''. In Chakali, the  preverb {\S
kaa} contributes   temporal information to an expression. Consider in
(\ref{exe:GRM-crack-remove-attach}) the distribution and contribution of  {\S
kaa} to  the clauses headed by the verbs {\S kpe} `crack a shell and remove a
seed from it' (c\&r) and {\S mara} `attach'.\footnote{In Gurene (Oti-Volta), it
is the ingressive particle which has a similar role. The ingressive  is 
commonly used before the verb, and can, among other things,  express future
tense \citep[see][59]{Daku07b}.}



\begin{exe}
\ex\label{exe:GRM-crack-remove-attach}
%\begin{multicols}{2}
\begin{xlist}
\ex

 ʊ̀ kàá kpéù   `She will c\&r'\\
   ʊ̀ʊ̀ kpéú   	 `She  is c-ing\&r-ing/c-s\&r-s'\\
   ʊ̀ kpéjòò   `She   c-ed\&r-ed'\\
   kpé  		 `C\&r!'
\ex
 ʊ̀ kàá māràʊ̀   `She will attach'\\
   ʊ̀ʊ̀ máràʊ̀   	 `She  is attaching/attaches'\\
  ʊ̀ márɪ̄jʊ̀    `She   attached'\\
   márá		 `Attach'
\end{xlist}
%\end{multicols}
\end{exe}


The preverb particle {\S kaa} can also be used to express that an event is
ongoing at the moment of speech, which I call the present 
progressive.   However,  when it is used to describe what is happening
now, {\S kaa} can only appear when the subject is not a pronoun and its tone
melody differs from that of the future tense. These contrasts are given in
(\ref{exe:GRM-kaa-attach}).

\begin{exe}
\ex\label{exe:GRM-kaa-attach}
 ʊ̀ kàá márāʊ̀   `She will attach'\\
   ʊ̀ʊ̀ márāʊ̀   	 `She  is attaching'\\
wʊ̀sá kàá márāʊ̀   `Wusa will attach'\\
wʊ̀sá káá márāʊ̀   `Wusa is attaching'\\
\textasteriskcentered  wʊsa   maraʊ  	  `Wusa is
attaching'
\end{exe}

Paradigm  (\ref{exe:GRM-kaa-attach}) shows that when the preverb particle {\S
kaa} appears with a rising tonal melody it  expresses the future tense, but  in
order to convey that a situation is ongoing at the time of speech (i.e. present
progressive), the preverb particle {\S kaa} has a high tone. Thus, it is the
tonal melody on {\S kaa} which distinguishes between the future and the present
progressive (both treated as imperfective),  plus the fact that pronouns cannot
co-occur with the preverb particle {\S kaa} in the present progressive. 



Although little evidence is available, the preverb {\S wa} may also be used to
express a sort of hypothetical  mood.  In  (\ref{ex:GRM-prev-wa-hypo}), the
preverb {\S wa} should be seen as contributing a supposition, or a hypothetical
circumstance where
someone would be found calling the number 8.


\begin{exe}
\ex\label{ex:GRM-prev-wa-hypo}

\gll ŋmɛ́ŋtɛ́l   ŋmá    dɪ́,    kɔ̀sánáɔ̃̀, 	tɔ́ʊ́tɪ̀ɪ̀nà  ŋmá 
dɪ́,  	námùŋ   wá    jɪ̀rà ŋmɛ́ŋtɛ́l sɔ́ŋ,  	bá  kpágʊ́ʊ̀    wà 
bà kpʊ́\\
spider     say   {\comp}   buffalo  	land.owner say {\comp}  anyone   
{\ingr}   call eight     name {\sc 3.pl.hum+}    catch.{\sc 3.sg} {\foc}
 {\sc 3.pl.hum+}  kill\\
\glt `Spider told Buffalo that landowner said anyone who calls the number 8
should be brought to him to be killed.' (LB 009)
\end{exe}



Finally, the example in (\ref{ex:GRM-verb-ta})  intends to show that some
elders of Ducie and
Gurumbele use  {\S ta}  instead of {\S ka(a)},  as a variant of the
preverb.\footnote{I gathered that  (i)   {\F ta} is not a different
preverb (Gurene is said to have  a preverb {\F ta}  signifying intentional
action, (M.E.K. Dakubu, p.c.)),  and  (ii)   {\F ta} can be heard in  Ducie and
Gurumbele from people of the oldest generation, but somebody suggested to me
that {\F ta} is the common form in Motigu (Mba Zien, p.c.).  This distinction
is an issue in need of further research. } 
 

\begin{exe}
   \ex\label{ex:GRM-verb-ta}{\it Priest talking to the shrine, holding a kola
nut above it}

\gll  ma laa kapʊsɪɛ haŋ ka ja mɔsɛ tɪɛ wɪɪ tɪŋ ba \underline{ta} buure\\
{\sc 2.pl} take kola.nut {\sc dem} {\sc conn} {\sc 1.pl} plead give matter {\sc
art} {\sc 3.pl.}b {\sc  egr} want\\
\glt   `Take this kola nut, we implore  you to give them what they desire.'

\end{exe}


Unfortunately, since the relation between tense, aspect and tonal melody is not
well-understood at this stage of research, the  egressive {\S ka}   and the
ingressive  {\S wa} are  broadly glossed as {\egr} and {\ingr} respectively, but
can also be associated with composite glosses such as {\ipfv .\fut} or  {\ipfv
.\pres}  in cases where a distinction is clear.




\subsubsection{Negation preverb}
\label{sec:GRM-verb-neg}
%check negative concord with nobodu, no one, all, nothing


%  Their
% lengths may vary depending on the speech rate, but  they are always long
% in 

There are three different particles of negation in the language:  the forms {\S
lɛɪ} and {\S tɪ}   were discussed in section   \ref{sec:GRM-foc-neg} and
\ref{sec:GRM-imper-clause} respectively.  The negative preverb particle {\S
wa(a)} precedes the verb and is used in the verbal group (in non-imperative
mood). The same form is found in both  main and dependent clauses. Notice that a
tonal quality on the negation particle and following verb  distinguishes
between the present
progressive and  the future,  as the preverb {\S kaa} does.  Consider 
paradigm (\ref{ex:GRM-neg-pres-fut}). 

\begin{exe}
\ex\label{ex:GRM-neg-pres-fut}
\begin{xlist}
\ex
\gll ʊ̀  wàá pɛ̀ \\
   {\sc 3.sg}  {\neg} add\\
\glt  `She will not add.'

 \ex 
\gll  ʊ̀ʊ̀ wàà pɛ́\\
     {\sc 3.sg} {\neg} add \\
\glt  `She is not adding.'


 \ex 
\gll  ʊ̀ wà pɛ́jɛ̀\\
     {\sc 3.sg} {\neg} add \\
\glt  `She didn't  add.'
\end{xlist}
\end{exe}


When the negation particle {\S wa(a)} and a quantifier appear in the same clause
the quantifier is  in the positive. This is shown in (\ref{ex:neg-quant-any}).

\begin{exe}
\ex\label{ex:neg-quant-any}
 \begin{xlist}
  
\ex\label{ex:neg-quant-any-1}
\gll namuŋ wa na-ŋ \\
 {\clf}.all {\neg} see-{1.\sg}\\
\glt  `Nobody saw me.' ({\it lit.} everyone not see me) 

\ex\label{ex:neg-quant-any-2}
\gll  n̩ wa na namuŋ  \\
  {1.\sg}  {\neg}   see  {\clf}.all\\
\glt  `I did not see anyone.' ({\it lit.} I not see everyone) 
\end{xlist}
\end{exe}


The negative preverb is often hard to distinguish from the verb  {\S
waa} `come',  even though the former always precedes the latter. Length (CV or
CVV) is especially hard to differentiate in normal speech. Examples
(\ref{ex:GRM-neg-come}) suggest that the tonal melody and length  indicate
meaning differences.


\begin{exe}
\ex\label{ex:GRM-neg-come}
 \begin{xlist}
  
\ex\label{}
\gll ʊ̀ wà wáá dì\\
{\sc 3.sg} {\neg} come eat\\
\glt `She did not come to eat.'
\ex\label{}
\gll ʊ̀ wàá wàà dí\\
{\sc 3.sg} {\neg} come eat \\
\glt `She will not come to eat.'
\end{xlist}
\end{exe}

Assertion and negation seem to avoid one another and constrain the grammar  in
the following way:  {\it If a clause is negated,  none of its constituents can
be in focus.} In section \ref{sec:GRM-personal-pronouns},  it was shown that (i)
negation cannot co-occur with the strong pronouns, and (ii) negation cannot
co-occur with an argument of the predicate in focus, i.e. with {\S ra} or one of
its variants having scope over the noun phrase. The third non-occurrence of
negation concerns  the assertive form of the verb (see section
\ref{sc:GRM-focus}).  Consider the forms of the verb {\S mara} `attach' in the
two paradigms in (\ref{ex:GRM-verb-neg-foc}).\footnote{The particle {\F la} in
Dagaare has similar constraints. \citet[94]{Bodo97} calls it a {\it factitve}
particle.}

\begin{exe}
   \ex\label{ex:GRM-verb-neg-foc}
\begin{xlist}
   \ex\label{ex:GRM-verb-neg-foc-pos}{\it Positive}

 ʊ̀ kàá māràʊ̀   `She will attach'\\
  ʊ̀ʊ̀ máràʊ̀     	 `She  is attaching/attaches'\\
  ʊ̀ márɪ̄jʊ̀       `She   attached'\\
  \ex\label{ex:GRM-verb-neg-foc-neg}{\it Negative}

 ʊ̀ wàá màrà  `She will not attach'\\
   ʊ̀ʊ̀ wàà márá  	 `She  is  not attaching/does not attach'\\
  ʊ̀ wà márɪ̄jɛ̀   `She   did not attach'\\

\end{xlist}
\end{exe}

The paradigms in (\ref{ex:GRM-verb-neg-foc})  suggest
that the negation particle and the assertive suffix are in complementary
distribution. 





\subsubsection{Tense preverbs}
\label{sec:GRM-verb-neg}


\paragraph{fɪ}

The preverb {\S fɪ}   is identified with two different but interrelated
meanings.  First, the preverb {\S fɪ}  ({\it gl.} {\sc pst}) is a neutral
past tense particle (i.e.  as opposed to {\S dɪ} in  section
\ref{sec:GRM-preverb-three-int-tense} which is specific), and the event referred
to in the past can no longer be in effect in the present.

\begin{exe} 
\ex\label{ex-preverb-fi-neut}
\begin{xlist}
\ex
\gll ʊ̀ jáá  ǹ̩ títʃà rà \\
  {\sc 3.sg} {\ident}    {\sc 3.sg.poss}  teacher {\foc}  \\
\glt  `He is my TEACHER.' 

\ex
\gll   ʊ̀  fɪ̀ jáá  ǹ̩ títʃà rà\\
        {\sc 3.sg} {\pst} {\ident}    {\sc 3.sg.poss}  teacher {\foc}  \\  
\glt  `He was my TEACHER.' 

\end{xlist}
\end{exe} 

 Secondly, the preverb {\S fɪ}   ({\it gl.} {\sc mod}) can have  deontic
meaning.  In (\ref{ex-preverb-fi-deonc}),  its presence still conveys  past
tense, but in addition it expresses that the situation did not really occur, yet
it was objectively supposed to occur or subjectively expected to occur or
awaited. The lengthening of the preverb {\S fɪ} in the positive  is not
accounted for, but I suspect it  signals the imperfective. Compare the first two
sentences in (\ref{ex-preverb-fi-deonc}) with the last two  which convey the
neutral past. 


\begin{exe} 
\ex\label{ex-preverb-fi-deonc}
\begin{xlist}
\ex\label{ex-preverb-fi-deonc-pos}
\gll ʊ̀ fɪ́ɪ́ jàà  ǹ̩ títʃà rà \\
  {\sc 3.sg}  {\mod}  {\ident}    {\sc 3.sg.poss}  teacher {\foc}  \\
\glt  `He should have been my TEACHER.' 

\ex
\gll    ʊ̀ fɪ̀ wáá jàà  ǹ̩ títʃà \\
        {\sc 3.sg} {\mod} {\neg} {\ident}    {\sc 3.sg.poss}  teacher   \\  
\glt   `He should not have been my teacher.'  


\ex
 ʊ̀  fɪ̀ jáá  ǹ̩ títʃà rà `He was my TEACHER.'
\ex
 ʊ̀  fɪ̀ wà jáá  ǹ̩ títʃà `He was not my teacher.'
\end{xlist}
\end{exe} 

The positive sentence in (\ref{ex-preverb-fi-deonc-pos}) can receive  a
translation along these lines:  In a desirable possible world, he was my
teacher, but it is not what happened in
the real world. 

\begin{exe} 
\ex\label{ex-preverb-fi-pure-deonc}
\begin{xlist}
\ex\label{ex:GRM-vp11.2}
\gll m̩̀ mɪ̀bʊ̀à fɪ́  bɪ́rgɪ̀ \\
    {\sc 1.sg.poss} life    {\mod}  delay    \\
\glt  `May I live long!' 

\ex\label{ex:GRM-vp11.3}
\gll tɪ́ɛ́ m̩̀ mɪ̀bʊ̀à bɪ́rgɪ̀ \\
      give {\sc 1.sg.poss} life delay  \\
\glt  `Let me live long!' 
\end{xlist}
\end{exe} 

Finally, the preverb {\S fɪ}  in (\ref{ex-preverb-fi-pure-deonc}) still conveys
 deontic modality, where the speaker prays or asks permission for a 
situation. Notice, however,  that it cannot refer to a past event. The two
sentences
in (\ref{ex-preverb-fi-pure-deonc}) have a corresponding meaning. Example
(\ref{ex:GRM-vp11.3}) is framed in an imperative clause (see optative in section
\ref{sec:GRM-imper-clause}). 


\paragraph{Preverb three-interval tense}
\label{sec:GRM-preverb-three-int-tense}

Chakali encodes  in  preverbs  a type of
time categorization  known as three-interval tense  \citep[366]{Fraw92}. It is
possible to express that an event occurred specifically yesterday, as opposed to
earlier today and the day before yesterday, i.e. {\it hesternal tense}
({\it gl.} {\sc hest}), or specifically tomorrow, as opposed to later today and
the
day after tomorrow, i.e. {\it
crastinal tense}  ({\it gl.}  {\sc cras}). 

The hesternal tense particle {\S dɪ}/{\S
de} ({\it gl.} {\sc hest})  refers to the day
preceding the speech time.  It has the time adverbial counterpart  {\S dɪare
(tɪŋ)} `yesterday'.  In (\ref{ex:vp2.11.a}) the adverbial phrase {\S dɪare tɪn}
`yesterday' is optional,  and  when it is used it must be expressed at the end
or
at the beginning of
the clause.


\begin{exe} 
\ex\label{ex:vp2.11.a} 
\gll {(dɪare tɪn)} ʊ nɪ ʊ tʃɛna dɪ waawa  {(dɪare tɪn)}\\
{(yesterday)}    {\sc 3.sg} {\sc conn} {\sc 3.sg.poss} friend
{\sc hest}  come.{\sc pfv} {(yesterday)} \\ 
\glt  `He arrived with his friend yesterday.'
 \end{exe}


The crastinal tense preverb {\S tʃɪ} ({\it gl.} {\cras})  in  (\ref{ex:vp4.5})
functions as future particle,  but is limited to the day following the event
time.
In that sentence the event time referred to follows  the utterance
time by one day.  The time
adverbial counterpart  of {\S tʃɪ} is {\S  tʃɪa} `tomorrow'. As
for the hesternal tense and the corresponding adverbial,  the  adverbial may or
may not co-occur with the crastinal tense particle. 


\begin{exe} 
\ex\label{ex:vp4.5} {\it Will you work for the chief today or tomorrow?}
\gll  n̩ tʃɪ ka tʊma tɪɛʊ ra, zaaŋ,  n̩ kaa hɪɛ̃sʊ \\
    {\sc 1.sg} {\sc cras}  go  work give.{\sc 3.sg} {\sc foc},
today,   {\sc 1.sg}  {\sc egr} rest.{\sc foc} \\
\glt  `I shall work for
him tomorrow, today,  I shall rest.' 
 \end{exe}


The hesternal tense particle {\S dɪ} is homophonous with the ({\it ex-situ
subject}) imperfective particle  {\S dɪ} discussed in section
\ref{sec:GRM-ipfv-part}.  In addition, the question arises as to whether the
crastinal tense  is inherently future, and if so, whether or not it can
co-occur with the egressive preverb discussed in section
\ref{sec:GRM-EVC-egr-ingr}. Consider their distribution and meaning in the
examples given in (\ref{ex:GRM-prev-dist}).

\begin{exe} 
\ex\label{ex:GRM-prev-dist}
\begin{xlist}
\ex\label{ex:GRM-prev-dist-chew-presprog}{\it Present progressive}
\gll  sɪ́gá (ra)  ʊ̀ dɪ̀  tíè   \\
 bean  ({\foc}) {3\sg} {\ipfv} chew\\
\glt `It is BEANS he is chewing'

 \ex\label{ex:GRM-prev-dist-chew-past}{\it Past}
\gll  sɪ́gá (ra) ʊ̀   tìè     \\
 bean  ({\foc}) {3\sg}  chew \\
\glt `It is BEANS he chewed'


 \ex\label{ex:GRM-prev-dist-chew-past}{\it Hesternal past}
\gll  sɪ́gá (ra) ʊ̀ dɪ́    tìè     \\
 bean  ({\foc}) {3\sg} {\hest}  chew \\
\glt `It is BEANS he chewed yesterday'


 \ex\label{ex:GRM-prev-dist-chew-past-pro}{\it Hesternal past progressive}
\gll  sɪ́gá (ra) ʊ̀ dɪ́ɪ́    tìè     \\
 bean  ({\foc}) {3\sg} {\hest}  chew \\
\glt `It is BEANS he was chewing yesterday'

 \ex\label{ex:GRM-prev-dist-chew-futprog}{\it Future (progressive)}
\gll  sɪ́gá (ra) ʊ̀  kàá   tíè     \\
 bean  ({\foc}) {3\sg} {\fut}  chew \\
\glt `It is BEANS he will be chewing / will chew'

 \ex\label{ex:GRM-foc-top-chew-crasfutprog}{\it Crastinal future (progressive)}
\gll  sɪ́gá (ra) ʊ̀ tʃɪ́  kàá   tíè     \\
 bean  ({\foc}) {3\sg} {\cras} {\fut}   chew \\
\glt `It is BEANS he will be chewing / will chew tomorrow '

\end{xlist}
\end{exe} 

A specific tonal melody associated with  the sequence {\S dɪ tie} can express
either a present progressive (\ref{ex:GRM-prev-dist-chew-presprog}) or a
hesternal past (\ref{ex:GRM-prev-dist-chew-past}). Lengthening the hesternal
past particle allows one to express the tense associated with the particle, in
addition to indicating  progressive (\ref{ex:GRM-prev-dist-chew-past-pro}). This
strategy seems to correspond semantically  to the syntactically anomalous
*{\S dɪ dɪ},  {\it lit.} {\sc hest} {\sc ipfv}.  The example in
(\ref{ex:GRM-foc-top-chew-crasfutprog}) shows that the crastinal tense particle
and the egressive particle signaling  future  can co-occur.  Inserting the
imperfective particle {\S  dɪ} between the egressive particle and
the verb in  (\ref{ex:GRM-prev-dist-chew-futprog}) and
(\ref{ex:GRM-foc-top-chew-crasfutprog}) is  unacceptable. It is unclear whether
these two
examples must be interpreted as progressive or not.  



\paragraph{te}
\label{sec:GRM-preverb-te}

Lacking a corresponding verb to capture its meaning, the verb {\S te} is glossed
with
the English adverb `early'. Even though  it is attested as main verb,  {\S te}
can  function  as a preverb and it is indeed more common to find it in that
function.  It contributes a manner, one in which the
event is carried out before the expected or usual time.  It cannot be used
otherwise as an adverb, i.e. in a canonical adjunct position (section
\ref{sec:GRM-adjuncts}).  The main verb {\S te}
and the preverb {\S te} are shown respectively in  (\ref{ex:GRM-prev-early})
and  (\ref{GRM-prev-SVC-ka}), which are repeated below.


\begin{exe}
 \ex\label{ex:GRM-prev-early}
\gll  ɪ̀ téjòò\\
     {\sc 2.sg} early.{\foc}   \\
\glt  `You are early.'
\end{exe} 

\begin{exe}
\exp{GRM-prev-SVC-ka}
\glll gbɪ̃̀ã́         	bààŋ       	té    	kà         	sáŋá 	à   
píé  {(...)}\\
monkey  	quickly 	early 	go 		sit  	{\art}
yam.mound.{\pl}   {(...)} \\
{} {\it pv} {\it pv}  {\it v} {\it v} {}  {} {} \\
\glt `Monkey quickly went and sat on the (eighth) yam mound (...)'  (LB 012)
\end{exe} 



\paragraph{zɪ}
\label{sec:GRM-preverb-after-then}

The preverb {\S zɪ} is marginal in the corpus.\footnote{There is another
similar particle, {\F ze}  ({\it gl.} {\sc exp}),  which I still do not
understand: (i) it occurs after the noun phrase, and  (ii) its meaning
corresponds to
 `expected (by both the speaker and the
hearer, or only by the speaker)'. It informs that the referent of
the noun phrase was anticipated before the utterance time (or relative time) by
the speaker and hearer (or only the speaker).  Consider  the following
example:

\begin{exe}
\sn[]{ 
 \gll ba ze  waawaʊ \\
{\sc 3.pl.b} {\sc  exp} come.{\sc pfv}\\
`They (the expected people) have come.'}
\end{exe}
} There is no corresponding
verb in the language.   It is used when an event is seen in
succession to another, as (\ref{ex:GRM-prev-zi-1}) shows. However,  as
(\ref{ex:GRM-prev-zi-2}) illustrates,  the preceding event may be presupposed, 
so  it is not necessarily uttered.


\begin{exe}
\ex
\begin{xlist}
\ex\label{ex:GRM-prev-zi-1}  {\it A father is giving a sequence of tasks to
his son}

 \glll tʊma  a  zɪɛ̃  mʊã  ka  ka  tʊma  kuo   aka   zɪ ka  tʊma a  gar  \\
  {work} \textsc{art}    {wall} {before}  \textsc{conn} {\sc egr}    {work}
{farm} 
\textsc{conn} {after}  {go} {work} \textsc{art} {cattle.fence}\\  
{} {} {} {} {}  {} {} {} {} {\it pv} {\it v} {\it v} {} {}\\
\glt  `First repair the wall, then go and farm, then repair the cattle fence.'

 \ex\label{ex:GRM-prev-zi-2}
  \glll {(kaalɪ dɪa)} zɪ́ kààlɪ̀ kùó\\
{go  house} then go farm\\
{} {\it pv} {\it v} {}\\
 \glt `(Go to the house and) Then go to the farm.'

\end{xlist}
\end{exe}


\subsubsection{Miscellaneous elements in the EVC}
\label{sec:GRM-preverb-misc}


In this section, the words  {\S baaŋ} `must',   {\S bɪ}
`again',  {\S bra} `return',   {\S ja} `do',  and {\S ha} `yet'  are described
as preverbs.  A basic overview of their meanings and
distributions is given.  

\paragraph{baaŋ}
\label{sec:GRM-preverb-baang}

 The preverb  {\S baaŋ}  ({\it gl.}
{\sc mod})  is primarily modal and is  translated with 
`must', `immediately', `quickly'  or `just'.  First,   as the examples in
(\ref{sec:GRM-prev-bg-must}) show,  {\S baaŋ} conveys an obligation and the
notion of temporality is secondary. 


\begin{exe}
\ex\label{sec:GRM-prev-bg-must}
\begin{xlist}
          
\ex\label{ex:GRM-7.17}
\gll  kuoru ŋma dɪ n̩ ka baaŋ bɔ bʊ̃ʊ̃na  fi re \\
 chief say {\comp} {\sc 1.sg} {\egr} {\mod}   pay  goat.{\pl} ten {\sc foc} \\
\glt  `The chief says that I must pay him ten goats.' 

\ex\label{ex:GRM-14.3}
\gll  ɪ ka baaŋ jaʊ ra\\
{\sc 2.sg} {\egr} {\mod} do.{\sc 3.sg} {\foc}\\
\glt  `You must do it.'

\end{xlist}
 \end{exe}
 

Secondly, the preverb  {\S baaŋ} can express an  abrupt or
swift   manner.

\begin{exe}
\ex\label{sec:GRM-prev-bg-time}
\glll   {(...)} a kpa ʊ neŋ a saga ʊ nɪ dɪ ʊ baaŋ te bɛrɛgɪ dʊ̃ʊ̃\\
     {(...})  {\conn}  take {\sc 3.sg.poss} arm {\conn} {be.on} {\sc
3.sg}  {\postp} {\conn} {\sc 3.sg} {\mod} {early} turn.into python \\
{} {} {} {} {} {} {} {} {} {} {} {\it pv} {\it pv} {\it v} {} \\

\glt  `(...) then put his hand on her  and quickly turned into
a python.' 
%(Pyhton story 025)
\end{exe}
      
Finally, the preverb  {\S baaŋ} may act as a discourse particle used mainly to
emphasize or intensify the action carried out, reminiscent of  the use of 
`just' in
some English registers.  It is often translated in text as `immediately',
`suddenly', `then',  or simply `just'. Some excerpts from a folk tale are
given in
(\ref{ex:GRM-prev-bg-excerpt}).


%example from python story

%  
 \begin{exe}
\ex\label{ex:GRM-prev-bg-excerpt}
\begin{xlist}
\ex
\gll kawa baaŋ tarɪ keeeeŋ \\
pumpkin just creep {\advm}\\
\glt `A pumpkin just crept like that ...' 

\ex
\gll ʊ baaŋ tɪŋaʊ \\
{\sc 3.sg} just follow.{\sc 3.sg}\\
 \glt `She just followed it ...'

\ex
\gll ʊ baaŋ jɪraʊ \\
{\sc 3.sg} just call.{\sc 3.sg}\\
 \glt `She then called her ...'

\ex
\gll dɪ mãã tɪŋ baaŋ ŋma nɪŋ mmmm\\
{\comp} mother {\art} just say {\advm} mmmm\\
\glt `That the mother just said like ``mmmm'' ...'

\ex
\gll diŋ baaŋ jaa tʊl\\
fire  just {\ident} flame\\
\glt `The fire suddenly became flame.'
\end{xlist}
 \end{exe}
% 



%desiderative mood ŋma



\paragraph{bɪ}
\label{sec:GRM-preverb-iteration}

The examples in (\ref{ex:GRM-prev-bi}) show that  the preverb particle {\S bɪ}
expresses iteration, but also the single repetition of an event, and follows the
negation particle. 

% bɪ kuor ŋma
%  repeat
%  bɪ pɪlɪ
% start again
% start
\begin{exe} 

\ex\label{ex:GRM-prev-bi}
\begin{xlist}
\ex\label{ex:vp33.2.}
\gll ʊ bɪ kʊɔrɛ sãã ʊ dɪa ra \\
 {\sc 1.sg}     {\itr} make build {\sc 3.sg.poss} house {\foc}    \\
\glt  `He rebuilt his hut' 


\ex\label{ex:GRM-vp10.4}
\gll a bitʃelii bɪ siiu\\
 {\art}  child.fall   {\itr} raise.{\foc}    \\
\glt  `The fallen child gets up again.' 



\ex\label{ex:vp10.4.}
\gll ʊ wa bɪ tuwo \\
       {3.\sg} {\neg} {\itr} be.at\\
\glt  `She is not here again.' 
\end{xlist}
\end{exe} 


Unlike other preverbs,  {\S bɪ} may also appear within noun phrases to express
frequency time. This is shown in (\ref{ex:GRM-vp19.2.}) (see also section
\ref{sec:NUM-repet}).



\begin{exe} 
\ex\label{ex:GRM-vp19.2.}
\gll  n̩ ja  kaalɪ ʊ pe re tʃɔpɪsɪ bɪ-muŋ \\
{\sc 1.sg} {\hab} go {\sc 3.sg.poss} end {\foc}  day.break {\itr}-all\\
\glt  `I do visit him every day.' 

\end{exe} 



 \paragraph{bra}
\label{sec:GRM-preverb-return}

The preverb {\S bra} has a corresponding verb with the same form. It is
primarily a motion verb which conveys a change of direction. The examples 
in (\ref{ex:GRM-verb-bra}) present the verb {\S bra} in imperative clauses
separated by the connectives {\S a} and {\S aka}.


\begin{exe}
\ex\label{ex:GRM-verb-bra}
\begin{xlist}
\ex
\gll brà à káálɪ̀\\
return {\conn} go\\
\glt `Go back.'

\ex
\gll brà àká tʃáʊ̀\\
return {\conn} leave.{\sc 3.sg}\\
\glt `Return and leave him.'
\end{xlist}
\end{exe}


When {\S bra} functions as a preverb, it loosely keeps its sense of motion and
conveys in addition a sort of repetition. It differs from the morpheme {\S bɪ}
introduced in
 section \ref{sec:GRM-preverb-iteration} because it does not mean that an
action is
done
repeatedly.  Instead, the preverb {\S bra} is associated with actions done `once
more', `over again',  or `anew'.


\begin{exe}
\ex\label{ex:vp33.1.}
\gll ʊ bra tʊma a tʊma tɪŋ ka wa wire keŋ \\
 {\sc 3.sg}  {again}  {work} {\art} {work}   {\art} {\egr} {\neg} well {\advm}\\
\glt  `He redid the work that was 
 badly done.'
\end{exe}





\paragraph{ja}
\label{sec:GRM-preverb-hab}

The preverb {\S ja} ({\it gl.} {\sc hab})  indicates habitual aspect. It may be
argued that the
particle derives from the  verb {\S ja} `do'. Example
(\ref{ex:GRM-prev-hab-do}) shows that the preverb {\S ja} and the verb {\S ja}
can
co-occur. 


\begin{exe}
\ex\label{ex:GRM-prev-hab}
\begin{xlist}

\ex\label{ex:GRM-prev-hab-do}
\gll tʃɔpɪsɪ bɪ-muŋ ʊ ja jaʊ \\
 day.break {\itr}-all {\sc 3.sg} {\hab} do.{\sc 3.sg}\\
\glt `He does it every day.'

\ex
\gll taŋu ja tie ger re\\
T.  {\hab} chew lizard {\foc}\\
\glt `Tangu do eat lizard.'

\ex
\gll jʊʊ nɪ dʊɔŋ ja waaʊ\\
rainy.season {\postp} rain  {\hab} come.{\foc} \\
\glt `During the rainy season, it rains.'
\end{xlist}
\end{exe}





\paragraph{ha}
\label{sec:GRM-preverb-yet}

The morpheme {\S ha} ({\it gl.} {\sc mod}) is similar in meaning to the English
morpheme `yet'. It is used when an event is or was anticipated and a speaker
considers or considered probable the occurence of the event. As  for the English
`yet', it is frequently found in negative polarity. In such cases the morpheme
{\S ha} indicates that the event is expected to happen and the negative marker
{\S wa} indicates that the event has not unfolded or happened at the referred
time. In the cases where {\S ha} is found in a positive polarity,  it  conveys 
a continuative aspect, similar to English `still',  as in 
(\ref{ex:vp32.24}). The
morpheme  {\S ha} is circumscribed to the expanded verbal group, although I
translate as `yet'  another expression in (\ref{ex:yet-conn}) which
functions as connective. The expression {\S haalɪ} is not frequent in the data
available. 

\begin{exe}
\ex
\begin{xlist}
 

\ex\label{ex:vp32.24}
\gll ʊ ha diu \\
     {3.\sg}  {\mod} eat.\foc  \\
\glt  `He is still eating.' 


\ex\label{ex:vp20.3.2.}
\gll ʊ ha wa dije \\
 {3.\sg}  {\mod} {\neg} eat.{\pfv}   \\
\glt  `He has not eaten yet.'


\ex\label{ex:vp21.2.1.}
\gll ba ɲine ʊ gɛrɛga ra aka ʊ ha wɪɪ \\
 {\sc 3.pl.hum+} look {\sc 3.sg.poss} sickness {\foc} {\conn}  {\sc 3.sg}
{\mod} ill \\
\glt  `He has been cared for to no avail; he is still ill.' 


\ex\label{ex:vp20.1.1.}
\gll ʊ ha wa waa baaŋ muŋ \\
       {3.\sg} {\mod}  {\neg} come {\dem} {\quant}\\
\glt  `He does not come here (ever).' 


\ex\label{ex:vp20.3.1.}
\gll ʊ̀ há wà wááwá \\
       {3.\sg} {\mod}   {\neg} come.{\pfv} \\
\glt  `He has not come yet.' 


\ex\label{ex:yet-conn}
\gll ʊ jireʊ saŋa muŋ, haalɪ ʊ ha wa waawa \\
       {3.\sg}  call.{\sc 3.sg} time all {\conn}   {\sc 3.sg} {\sc mod}  {\neg}
come.{\pfv} \\
\glt  `He called her long time ago, yet she has not
come.' 
\end{xlist}
\end{exe}



\paragraph{tu and zɪn}
\label{sec:GRM-preverb-up-down} 

The verbs {\S tuu} and {\S zɪna} are motion
expressions making reference to two opposite paths. When they are used as main
predicate, as in example (\ref{ex:GRM-verb-up-down}),  they denote `go down' and
`go up' and  surface as {\S
tuu} and {\S zɪna} respectively.  The interpretation of one  consultant suggest
that  {\S tuu} and {\S
zɪn} in 
(\ref{ex:GRM-preverb-up-down}) are not preverbs, but first verbs in SVCs.

\begin{exe}
\ex\label{ex:GRM-verb-up-down}
\begin{xlist} 

\ex
\gll n̩ zɪna sal la m̩ paa tʃuono\\
{\sc 1.sg} go.up flat.roof {\foc} {\sc 1.sg} take.{\pv} shea.nut.seed.{\pl}\\
\glt  `I go up on the roof to collect my shea nut seeds.'

\ex
\gll n̩ tuu dɪa ra\\
{\sc 1.sg} go.down house {\foc}\\
\glt I went down to the house.'
\end{xlist}
\end{exe}



\begin{exe}
\ex\label{ex:GRM-preverb-up-down}
\begin{xlist} 

\ex\label{ex:GRM-preverb-up}
\gll zɪn tʃɔ  dɪ kaalɪ  \\
      {go.up} run {\conn} go  \\
\glt  `Go up,  run and leave'  (*Run upwardly and go)

\ex\label{ex:GRM-preverb-down}
\gll tu tʃɔ  dɪ kaalɪ\\
      {go.down} run {\conn} go \\
\glt  `Go down, run and leave'  (*Run downwardly and go)
\end{xlist}
\end{exe}


% 6
%  ́
% The directional particles he (‘itive’, related to the homophonous verb meaning
% ‘go’ (departure from
%  ́
% deictic center or indexically determined location)) and va (‘ventive’, related
% to the homophonous verb meaning
% ‘come’ (arrival at deictic center or indexically determined location)) belong
%to % the class of preverbs of Ewe.
% These are forms that mark functional categories such as aspect, modality, and
% voice on verbs. Preverbs differ
% from verbs in that they do not head VPs, do not inflect for habitual aspect,
%and % do not take NP or PP
% complements (cf. Ameka 1991, 2005a,b, Ansre 1966).


% The particle  {\S ja} is
% polyfunctional:  when it precedes a main verb it  means  either `do'   to
% emphasize the event or conveys an habitual reading, or as, in the present
%case,
% it links two noun phrases. The latter case is glossed in example
%(\ref{ex:agrE})
% and (\ref{ex:agrF}) as {\sc ident}. 


% --Dakubu
% I wonder whether what you call IPFV is an egressive particle? such a particle
% derived from 'go' is quite common.  If it is incompletive / progressive this
% might have to do with the tone pattern


%lenghten preverf fii dii ...



%   Aorist & Imperfective & Perfective 
% Positive
% Negative



\subsection{Verbal suffixes}
\label{sec:GRM-verb-suffix}


In presenting the verb forms in section \ref{sec:GRM-verb-word}, two suffixes
were introduced: the perfective intransitive suffix and the assertive suffix. It
was shown that the perfective intransitive suffix surfaces either as {\S -jE},
{\S -wA} or {-\O} depending on  the verb stem.  The assertive suffix
appears  in the imperfective and perfective  intransitive construction if  (i)
none of the other constituents in the clause are in focus, (ii) the clause does
not include negative polarity items, and (iii) the clause is intransitive, that
is, there is no grammatical object. Also,  as mentioned in section
\ref{sec:GRM-imper-clause},  the suffix {\S -ɪ}/{\S -i} appears in the negative
imperative. 

In this section,  the incorporated object pronoun  ({\sc
o}-clitic), the pluractional  suffix, and  other derivative suffixes whose
functions are not yet understood are introduced.


\subsubsection{Incorporated object pronoun}
\label{sec:GRM-morph-opro}


The object pronoun  is represented as being incorporated into the verb,  and 
together they form a phonological word (e.g.  {\S wʊ̀sá tɪ́ɛ́ń nā} < {\S
wʊsa tɪɛ-n̩ na}  `Wusa gave-{\sc 1.sg} {\sc foc}').  For that reason I refer to
this incorporated object pronoun as the {\sc o}-clitic. Given the constraints
governing the appearance of the perfective intransitive suffix and the assertive
suffix, it is obvious that the {\sc o}-clitic cannot coexist with any of them.
Recall that the  weak subject pronoun and object  pronoun are identical (see
section \ref{sec:GRM-personal-pronouns}).


\begin{table}[!htb]
\centering
\caption{Incorporated object pronouns on  CV(V) stems\label{tab:object-clitic}}

\subfloat[tɪɛ `give']{
\begin{Itabular}{ll}
 wʊsa tɪɛ-n̩ na & `Wusa gave ME'\\
 wʊsa tɪɛ-ɪ ra & `Wusa gave YOU'\\
 wʊsa tɪɛ-ʊ ra &  `Wusa gave HER'\\
 wʊsa tɪɛ-ja ra &  `Wusa gave US'\\
 wʊsa tɪɛ-ma ra & `Wusa gave YOU'  \\
 wʊsa tɪɛ-a ra & `Wusa gave THEM'  \\
 wʊsa tɪɛ-ba ra &  `Wusa gave THEM'  \\
\end{Itabular} 
}
\quad
\subfloat[tie `cheat']{
\begin{Itabular}{ll}
 wʊsa tie-n̩ ne & `Wusa cheated ME'\\
 wʊsa tie-i re & `Wusa cheated YOU'\\
 wʊsa tie-u ro &  `Wusa cheated HER'\\
 wʊsa tie-ja ra &  `Wusa cheated US'\\
 wʊsa tie-ma ra & `Wusa cheated YOU' \\
 wʊsa tie-a ra & `Wusa cheated THEM'\\
 wʊsa tie-ba ra &  `Wusa cheated THEM'\\
\end{Itabular} 
}
\quad
\subfloat[tie `cheat']{
\begin{Itabular}{ll}
 wʊsa tie-je re &  `Wusa cheated US'\\
 wʊsa tie-me re & `Wusa cheated YOU' \\
 wʊsa tie-e re & `Wusa cheated THEM'\\
 wʊsa tie-be re &  `Wusa cheated THEM'\\
\end{Itabular} 
}
\quad
\subfloat[po `divide']{
\begin{Itabular}{ll}
 wʊsa po-je re &  `Wusa divided US'\\
 wʊsa po-mo ro & `Wusa divided YOU' \\
 wʊsa po-a ra & `Wusa divided THEM'\\
 wʊsa po-be re &  `Wusa divided THEM'\\
\end{Itabular} 
}
\end{table}

Table \ref{tab:object-clitic} shows that the {\sc atr}-harmony 
operates in the domain produced by the {\sc
o}-clitic merging with a CV or CVV stem, but may or may not affect the
plural pronouns, as tables \ref{tab:object-clitic}(b) and 
\ref{tab:object-clitic}(c) display. The form of the focus particle is determined
by
the preceding material (i.e. the phonological word  verb+{\sc
o}-clitic) and the harmony rules introduced in
section
\ref{sec:focus-forms}.  The irregularities in table \ref{tab:object-clitic}(d)
are not accounted for.  I did perceive rounding throughout in conversations
(i.e.  {\S wʊsa poma ra} $>$ {\S wʊsa pomo ro} `Wusa divided you.{\sc pl}'), but
I was unable to get a consultant to produce it in an elicitation session. Table
\ref{tab:object-clitic}(d) should be seen as displaying various renditions,
i.e. with and without {\sc atr-}harmony or {\sc ro-}harmony.


A CVCV stem differs from a CV or CVV stem by exhibiting vowel apocope and/or 
vowel
coalescence.  Table \ref{tab:object-clitic-CVCV} provides paradigms for {\S
kpaga} `catch' and {\S goro} `(go in) circle'. 



\begin{table}[!htb]
\centering
\caption{Incorporated object pronouns on  CVCV stems
\label{tab:object-clitic-CVCV}}

\subfloat[kpaga `catch']{
\begin{Itabular}{ll}
 wʊsa kpaɣn̩ na & `Wusa caught ME'\\
 wʊsa kpaɣɪɪ ra & `Wusa caught YOU'\\
 wʊsa kpaɣʊʊ ra &  `Wusa caught HER'\\
 wʊsa kpaɣəja wa &  `Wusa caught US'\\
 wʊsa kpaɣəma wa & `Wusa caught YOU' \\
 wʊsa kpaɣaa wa & `Wusa caught THEM'\\
 wʊsa kpaɣəba wa &  `Wusa caught THEM'\\
\end{Itabular} 
}
\quad
\subfloat[goro `(go in) circle']{
\begin{Itabular}{ll}
wʊsa gorn̩ no & `Wusa circled ME'\\
 wʊsa gorii re & `Wusa circled YOU'\\
 wʊsa goruu ro &  `Wusa circled HER'\\
 wʊsa gorəja wa &  `Wusa circled US'\\
 wʊsa gorəma wa & `Wusa circled YOU'\\
 wʊsa goraa wa & `Wusa circled THEM'\\
 wʊsa gorəba wa &  `Wusa circled THEM'\\
\end{Itabular} 
}
\end{table}
The schwas ({\I [ə]}) in {\S kpaɣəja} and  {\S gorəja} are perceived as fronted,
and the ones in {\S kpaɣəma} and {\S gorəma}  as rounded. Although this is
certainly due to the following consonant, they are so weak that they can only be
heard when they are carefully pronounced. The focus particle {\S wa} is a
variant of {\S ra}. Consultants  agree that these forms are in free variation,
yet the {\S wa} form coexists only with  the plural in the paradigms elicited.
Nonetheless, such paradigm elicitations are particularly subject to
unnaturalness.\footnote{I personally believe that the alteration is
determined by some kind of sandhi, not number. As to why {\F wa} appears only in
the plural, a scenario may be that (i) first, I install a routine by starting
with `ME' and ending with `THEM', (ii) in the process of eliciting, the passage
from third singular to first plural triggers  a different verb shape, i.e.
CVCVV/CVCN  to CVCVCV, and (iii)  although formally identical to the verb forms
of the singular, the reason why {\F wa} follows the third plural non-human could
be explained by psychological habituation.}

\subsubsection{Pluractional suffixes}
\label{sec:GRM-PluralVerb}


A pluractional verb is defined as a verb which can (i) express the repetition of
an event,  (ii)   subcategorize for a plural object and/or  plural subject,
and/or  (iii)  be marked by the pluractional suffix {\S
-sI}, a derivative suffix whose  vowel quality is always high and
front
and  {\sc atr} value determined by the stem vowel(s).\footnote{An exposition of
the
`plural verbs' in Vagla can be found in \cite{Blen03}. \citet[viii]{Daku07}
calls a similar morpheme `iterative' (i.e. Gurene {\F -sɛ}).  Among the West
African
languages, it is the pluractional verbs in Hausa which have received most
attention \citep[see][]{Jose08}.}  According to (i) above, the iterativeness may
affect the interpretation of the number of participants of an event. Consider
the contrasts between the 
sentences in (\ref{ex:GRM-pv-cut}), where none of the arguments are in the
plural (i.e. contra (ii)).


\begin{exe}
\ex\label{ex:GRM-pv-cut}
  \begin{xlist}
    \ex\label{GRM-pv-cutsg}
\gll   n̩  teŋe  namɪã  ra  \\
       {\sc 1.sg} {cut} {meat} {\sc foc}\\
\glt `I cut a piece of meat (i.e.  made a cut in the flesh or cut into two
pieces).'

\ex\label{GRM-pv-cutpl}
\gll    n̩    teŋe-si  a  namɪã  ra \\
          {\sc 1.sg} {cut-{\sc pv}} {\sc art} {meat} {\sc foc}\\
\glt `I cut the meat into pieces.'

 \end{xlist}
\end{exe}

In  (\ref{GRM-pv-cutpl}),  the formal distinction on the verb `cut',  compared
to (\ref{GRM-pv-cutsg}),  causes  the event to be interpreted as one which
involves the repetition of the `same'  sub-event.  The word {\S namɪã} `meat'
is allowed in both the contexts of (\ref{GRM-pv-cutsg}) and
(\ref{GRM-pv-cutpl}), although one may argue that the word {\S namɪã} is
inherently
plural but grammatically singular,  and that the word is appropriate in both
contexts. Despite the fact that  `meat' has indeed a plural form, i.e. {\S
nansa}, it is probably the mass term denotation of {\S namɪã} which 
makes (\ref{GRM-pv-cutpl}) acceptable.

 In (\ref{GRM-pv-turn}), however,  the grammatical object of a
pluractional verb {\S tʃigesi} `turn iteratively' or `put on face
down iteratively'  must refer to individuated entities. 

\begin{exe}
\ex\label{GRM-pv-turn}
  \begin{xlist}
    \ex\label{GRM-pv-turnsg}
\gll   n̩  tʃige  a  hɛna  ra  \\
        {\sc 1.sg} {turn} {\sc art} {bowl.\sg} {\sc foc}\\
\glt `I turn (upside down) the bowl.'

 \ex\label{GRM-pv-turnpl1}
\gll   n̩  tʃige-si  a  hɛnsa  ra   \\
         {\sc 1.sg}   {turn-{\sc pv}} {\sc art} {bowl.\pl} {\sc foc}\\
\glt `I turn (upside down) the bowls (one after the other).'


 \ex\label{GRM-pv-turnpl2}
\gll {\textasteriskcentered}  n̩  tʃige-si   a  hɛna  ra \\
       {}  {\sc 1.sg}    {turn-{\sc pv}} {\sc art}  {bowl.\sg}  {\sc foc}\\
\glt `I turn (upside down in a repetitional fashion) the bowl.'

 \end{xlist}
\end{exe}

Comparing  (\ref{GRM-pv-turnsg}) and (\ref{GRM-pv-turnpl2}) with 
(\ref{GRM-pv-turnpl1}),   the pluractional verb cannot coexist with a singular
noun as grammatical object due to the fact that  the `turning' event cannot be
conceived as affecting the same object in a repetitive fashion. However, in
(\ref{GRM-pv-beat}) the `beating' can affect  one or several
individuals. 


\begin{exe}
\ex\label{GRM-pv-beat}
  \begin{xlist}
    \ex\label{GRM-pv-beat.sg}
\gll   n̩   tugo  a bie  re  \\
            {\sc 1.sg}  {beat} {\sc art} {child.\sg} {\sc foc}\\
\glt `I beat the child.'

\ex\label{GRM-pv-beat.pl1}
\gll   n̩    tugo-si  a bise  re   \\
         {\sc 1.sg}   {beat-{\sc pv}} {\sc art} {child.\pl} {\sc foc}\\
\glt ` I beat the children.'


\ex\label{GRM-pv-beat.pl2}
\gll    n̩    tugo-si  a  bie  re   \\
         {\sc 1.sg}    {beat-{\sc pv}} {\sc art}  {child.\sg} {\sc foc} \\
\glt `I beat the child (more than once, over a short period of time).'


 \end{xlist}
\end{exe}

Whereas  (\ref{GRM-pv-beat.pl2})
has a possible interpretation, two language consultants
could not assign a meaning to (\ref{GRM-pv-catchout}) below. 




\begin{exe}
\ex\label{GRM-pv-catch}
  \begin{xlist}
    \ex\label{GRM-pv-catchsg}
\gll    ŋ̩  kpaga  a  zal  la  \\
         {\sc 1.sg}   {caught} {\sc art} {chicken.\sg} {\sc foc}\\
\glt `I caught a chicken.'


 \ex\label{GRM-pv-catchpl1}
\gll    ŋ̩    kpaga-sɪ  a  zalɪɛ ra  \\
       {\sc 1.sg} {caught-{\sc pv}} {\sc art} {chicken.\pl} {\sc foc}\\
\glt `I caught chickens (i.e. in repeated actions).'


 \ex\label{GRM-pv-catchpl2}
\gll   ŋ̩     kpaga  a  zalɪɛ ra   \\
       {\sc 1.sg}  {caught} {\sc art} {chicken.\pl} {\sc foc}\\
\glt `I caught chickens (i.e. in one move).'


 \ex\label{GRM-pv-catchout}
\gll *     ŋ̩  kpaga-sɪ  a  zal  la  \\
     {}   {\sc 1.sg}  {caught-{\sc pv}} {\sc art} {chicken.\sg} {\sc foc}\\
\glt `I caught a chicken (i.e. after unsuccessful attempts until finally
succeeding with
one particular chicken).'

 \end{xlist}
\end{exe}

A pluractional verb usually denotes an action, but not a state. Therefore, in
(\ref{GRM-pv-catch}), the sense of {\S kpaga}_{1} is related to `catch', and not
to the  possessive sense of the verbal state lexeme   {\S kpaga}_{2}
`have'.\footnote{Though I like to treat {\F dʊasɪ} as a counterexample.  The
pluractional verb {\F dʊasɪ} `be in a row'  may be  derived from the existential
predicate {\F dʊa} `be on/at/in'.  In section \ref{sec:SPA-leaning-v}, it is
shown that the verbs {\F tele} `lean'   and {\F telege} `lean' are determined by
the number value ({\it sg.}/{\it  pl.})  of the subject.  If more examples like
these  arise, {\it pluractional} would then loose its literal signification.}
Examples of pluractional verbs are {\S dʊma} `bite' $>$ {\S dʊnsɪ}  `bite
iteratively', {\S jaga} `hit' $>$ {\S jagasɪ} `hit iteratively', {\S tʃige}
`cover' $>$ {\S tʃigesi}  `cover iteratively' and {\S teŋe} `cut' $>$ {\S
teŋesi}  `cut iteratively',  among others.  Beside the suffix {\S -sI}, {\S
-gE} may also turn a verbal process lexeme into a pluractional verb, e.g.   {\S
tɔtɪ} `pluck' $>$ {\S  tɔrəgɛ} `pluck iteratively' and  {\S keti} `break'  $>$
{\S kerigi} `break iteratively'.



Finally, a pluractional verb must not necessarily display the
suffixation pattern
described above. This is confirmed by the pair {\S kpa}/{\S paa} `take'  in
(\ref{ex:GRM-kpa-paa}).

\begin{exe}
    \ex\label{ex:GRM-kpa-paa}
  \begin{xlist}
    \ex\label{ex:GRM-kpa}
\gll ka kpa zal haŋ ta\\
go take.{\sc pl} fowl.{\sg} {\dem} let.free\\
\glt `Go and take this fowl away.'
      \ex\label{ex:GRM-paa}
\gll ka paa zalɪɛ hama ta\\
go take.{\sc pl} fowl.{\pl} {\dem}.{\pl}  let.free\\
\glt `Go and take these fowls away.'
 \end{xlist}
\end{exe}



\subsubsection{Possible derivational suffixes}
\label{sec:GRM-deri-suff}

 \citet[37]{Daku09} identifies some derivational suffixes in
Gurene, but writes that their signification is hard to establish. Similarly,
\citet[69]{Bonv88} writes that making out the identity, productivity and
functioning of verbal derivations in Kasem is not easy. However,
their descriptions indicate that  derivational suffixes mainly encode aspectual
distinctions.

As mentioned in section \ref{sec:GRM-verb-syll-und-tone}, about 90\% of the
verbs are either monosyllabic or bisyllabic, and  only the consonants {\S m,
t, s, n,  l} and {\S g} are found  in onset position word-medially in
trisyllabic verbs. This situation could suggest that 10\% of the verbs in the
current lexicon are the product of verbal derivation, and that the consonants
found  in onset position word-medially in trisyllabic verbs are part of
derivational suffixes. However, apart from the pluractional suffix discussed in
the previous section,  it is impossible at this stage of the research to
establish a systematic mapping between the third syllable of a trisyllabic verb
and a meaning. Table \ref{tab:GRM-der-suff}
presents  some indications that {\S m, l} and {\S g}, i.e. CVCV\{m, l, g\}V,
are involved in some kinds of derivation, although the glosses (and part of
speech categories) assigned to them clearly indicate that the next step would be
to determine their exact meaning (and category).\footnote{The verb pair {\F
go} `round'  and {\F goro}  `(go in) circle'  is  manifestly a derivation as
well, i.e.
CV $>$ CV-rV.} 



\begin{table}[!htb]
\centering
\caption{Possible derivational suffixes\label{tab:GRM-der-suff}}

 
\begin{Itabular}{lllll}
\Hline

 &&&{\S -gV}&\\

wʊra {(v)}& `dismantle' & $>$ & wʊrɪgɪ {(v)}& `collapse'\\
tara  {(v)}& `support' & $>$ &taragɛ {(v)}& `pull' \\
%bɪla {(v)}& `turn repetitively' & $>$ & bɪlgɪ {(v)}& `clean' \\
bra {(v)}& `return' & $>$ & bɛrɛgɪ  {(v)}& `change direction'\\[1ex]\hline

&&&{\S -mV} &\\
ɲagɛɛ  {(v)} & `sour' &$>$ & ɲagamɪ  {(v)}& `ferment' \\
vil {(n)} &`well' & $>$ &vilimi {(v)} & `whirl' \\
 mɪla {(v)} & `turn' & $>$ &mɪlɪmɪ {(v)}& `turn'\\[1ex]\hline

&&&{\S -lV}&\\
 kaga {(v)}& `choke'& $>$ & kagalɛ {(v)} & `lie across' \\
 \Hline
\end{Itabular}
\end{table}




\subsection{Adjunct types}
\label{sec:GRM-adjuncts}


The adjunct constituent  ({\sc adj}) in (\ref{ex:GRM-clause-frame}), repeated
below,  may
consist of  a single word or a  syntactic constituent. Although in section
\ref{sec:GRM-interr-clause}  it was originally positioned at the end of the
sequence,  some adjuncts can  be used clause initially before the
subject. 

\begin{exe}
\exp{ex:GRM-clause-frame}
 {\sc adj}  $\pm$ {\sc s|a}  $+$ {\sc p} $\pm$ {\sc o} $\pm$ {\sc adj} 
\end{exe}

Reference to space, manner and time are the main denotations of these
peripheral arguments. 

% 
% They are glossed {\sc advl} (i.e. adverb locative), {\sc advm}  (i.e.
% adverb manner) and {\sc advt}  (i.e. adverb time) respectively. Examples are
% provided below.


\subsubsection{Oblique object phrase}
\label{sec:GRM-obl-phrase}

The oblique object phrase is an element of a clause whose  semantics is
characterised by an  affected or effected object, although realized by a
postpositional phrase.  In section \ref{sec:SPA-postp},  it is claimed that the
postposition {\S nɪ} (i) identifies an oblique object phrase, (ii) conveys that
the oblique object phrase contains the ground object in localization, and
(iii) follows its complement. 

While localization is
the main function of  {\S nɪ}, the postposition can also be found when there
is no  reference to space. For instance, in section \ref{sec:GRM-manner-adv}, I
discuss the connective {\S denɪ} (i.e. {\advl}+{\postp}), whose role in
discourse is to signal a temporal transition, not a spatial one.  The
examples in (\ref{ex:GRM-obl-obj-no-spa}) illustrate some of the non-spatial
uses
of the oblique object phrase headed by {\S nɪ}.


\begin{exe}
\ex\label{ex:GRM-obl-obj-no-spa}
\begin{xlist}
\ex
\gll ʊ ɲʊ̃ã  [laɣalaɣa nɪ]  \\
      {\psg} drink {\advm} {\postp}    \\
\glt  `He drinks quickly.' 

\ex
\gll baaŋ ɪ fɪ ka tesi [tʃʊɔsa tɪn nɪ]\\
 {\q} {\sc 2.pl} {\pst} {\egr} crush  morning {\art} {\postp}    \\
\glt  `What were you crushing this morning?' 

\ex\label{ex:GRM-obl-obj-no-spa-foc}
\gll a kuoru ŋma dɪ ʊ baaŋ ka sii [n̩ nɪ] re\\
{\art} chief say {\comp} {\sc 3sg.poss} temper {\egr} raise {\sc 1.sg} {\postp}
{\foc}\\
\glt  `The chief told me that he was very angry with me.' 

\end{xlist}
\end{exe}


\subsubsection{Phrasal adverbs}
\label{sec:GRM-obl-phrase}

In section \ref{sec:GRM-compar-ct}, the dubitative construction was identified 
with the phrase {\S a bɔnɪɛ̃ nɪ} `perhaps'  opening the clause. There are other
constructions in which temporal, locative, manner or tense-aspect-mood meaning 
is signaled by the presence of a phrasal adverb clause initially. Two examples
are given in (\ref{ex:GRM-phra-adv}),  but since most of the adverbs convey
temporal, locative and manner signification, other examples are presented in
the subsequent sections. In (\ref{ex:GRM-phra-adv-time}), the phrase {\S tama
finii} is not inherently temporal, but must be interpreted as such in the given
context. 


\begin{exe}
\ex\label{ex:GRM-phra-adv}
\begin{xlist}

\ex\label{ex:GRM-phra-adv-time}{\it Temporal}
\gll [tama finii] ʊ fɪ sʊwa\\
few little {\sc 3.sg} {\mod} die\\
\glt `A little longer and she would  have died.'


\ex\label{ex:GRM-phra-adv-}{\it Evidential}
\gll [wɪdɪɪŋ na] dɪ ʊ naʊ ra\\
truth {\foc} {\comp} {\sc 3.sg} see.{\sc 3.sg} {\foc} \\
\glt  `It is certain that he saw him.


\end{xlist}
\end{exe}

\subsubsection{Locative adverbs}
\label{sec:GRM-deic-adv}


A speaker-subjective,  two-way contrast  exists to locate entities in space. The
deictic locative adverb {\S baŋ} designates the location of the speaker, while 
the deictic adverb {\S de} designates  where the
speaker is not located. They represent what is known as the `proximal' and
`distal' 
dimensions of  place deixis. In (\ref{ex:deic-adv-prox}) and
(\ref{ex:deic-adv-dist}),  they are translated as `here' and 'there'
respectively, and glossed {\sc advl}, standing for `locative adverb'. In these
two examples  the postposition {\S nɪ} is optional.  The locative
adverbs cannot occur clause initially, as  (\ref{ex:deic-adv-prox-out})  and
  (\ref{ex:deic-adv-dist-out}) show. 


\begin{exe}
\ex\label{ex:vp}
\begin{xlist}

\ex\label{ex:deic-adv-prox}
\gll wa ban (nɪ)\\
     come {\advl} {\postp} \\
\glt  `Come here'
\ex\label{ex:deic-adv-prox-out}
\textasteriskcentered baŋ wa 

 \ex\label{ex:deic-adv-dist}
\gll ʊ  dʊa de (nɪ) \\
       {\psg}  be.at  {\advl}  {\postp}\\
\glt  `He is there'

\ex\label{ex:deic-adv-dist-out}
\textasteriskcentered de ʊ  dʊa 
\end{xlist}
\end{exe}



\subsubsection{Temporal adverbs}
\label{sec:GRM-manner-adv}

A temporal adverb  ({\it gl.} {\sc advt}) is an expression which typically 
indicates when  an event occurs. In section
\ref{sec:GRM-preverb-three-int-tense}, the three-interval tense system was
introduced. It was shown that the temporal adverbs {\S  dɪare} `yesterday' and
{\S tʃɪa} `tomorrow'  have preverbs counterpart. The  temporal
adverb  {\S zaaŋ} (or {\S zalaŋ}) expresses `today',  and   {\S tɔmʊsʊ} can
express either  `the day before yesterday' or  `the day after tomorrow',   yet 
neither {\S zaaŋ} and {\S tɔmʊsʊ}   have a corresponding
preverb.\footnote{The time words {\F  dɪare} `yesterday',  {\F tʃɪa}
`tomorrow'  and  {\F zaaŋ} `today', which  typically function as peripheral
arguments and can be disjunctively connected (e.g. (\ref{ex:GRM-dij-vp4.5})),  
may be better treated as nominals. This is a matter of further research.}


\begin{exe}
\ex\label{ex:GRM-adj-temp-adv}
\begin{xlist}
\ex\label{ex:GRM-adj-temp-adv-thatday}

\gll awʊzʊʊrɪ n̩ wa tuwo nɪ  \\
{\advt} {\sc 1.sg} {\neg} {be.at} {\postp}\\
\glt `That day I wasn't there.'


\ex\label{ex:GRM-adj-temp-adv-LB5}

\gll àwʊ̀zʊ́ʊ́rɪ̀  dɪ́gɪ́ɪ́    	     kɔ̀sánàɔ̃̀ 	vàlà 	 (...)\\
{\advt}  one           buffalo  	walked (...) \\
\glt `One day a buffalo walked (by and greeted the spider).' (LB 005)

\ex\label{ex:GRM-adj-temp-adv-CB17}
\gll [dénɪ̀],         [sáŋà      dɪ́gɪ́ɪ́]    	   à   
hã́ã̀ŋ jà 	pàà  	à   	báàl    	   zòmò  (...) \\
{\advt}    time      one      	   {\art}	wife   	{\hab} 	take.{\pl} 
{\art} husband insult.{\pl} (...)\\
\glt `[During their life, it happened] on one occasion that the woman
did insult  the man (...)' .  (CB 017)

\ex\label{ex:GRM-adj-temp-adv-everyday}
\gll  n̩ ja kaalɪ ʊ pe re [{tʃʊɔsɪm pɪsa} bɪ muŋ]\\
 {\sc 1.sg}    {\hab} go {\sc 3.sg} end {\foc} day.break {\itr} all\\
\glt `I visit him every day.'

\ex\label{ex:GRM-adj-temp-adv-nownow}
\gll [laɣalaɣa han nɪ] n̩ kʊtɪ a ʔãã peti\\
{\advt} {\dem} {\postp} {1.\sg} {skin} {\art} bushbuck  finish\\
\glt `I  just finished skinning the bushbuck.'

\end{xlist}
\end{exe}


Some expressions tagged as temporal adverbs are treated as complex, though
opaque, expressions. For instance,  {\S awʊzʊʊrɪ} is translated as  `that
day' in (\ref{ex:GRM-adj-temp-adv}), but the forms {\S wʊsa} `sun' and
{\S zʊʊ} `enter'  are perceptible. The phrase {\S laɣalaɣa han nɪ} in
(\ref{ex:GRM-adj-temp-adv-nownow}) literally
means `now.now this on' ({\advt} {\dem} {\postp}), but `only a moment
ago'  is a better translation.  Similarly, {\S denɪ} is analysed as a
temporal adverb, but usually functions as a connective. It is made from  the
locative adverb {\S de} and the potsposition {\S nɪ}, and is translated to
English as `thereupon', `after that', `at that point', or simply `then'. It is
mainly used at the beginning of a sentence to signal a transition  between the
preceding  and the following situations;
(\ref{ex:GRM-adj-temp-thereupon}) suggests a transition of the resultative type.
The appendix contains other examples of {\S denɪ}, e.g.  CB (008, 017, 019)
and 
LB (006, 016).


\begin{exe}
\ex\label{ex:GRM-adj-temp-thereupon}
\gll dénɪ̀      ré           ʊ̀             	hã́ã̀ŋ 	tɪ̀ŋ 
ŋmá 	dɪ́   	ààí (...) \\
 {\advt} 	{\foc}     {\sc 3.sg.poss}	wife   	{\art}	say 	{\comp} 
no    (...)\\
\glt [The man said: `Don't cry, if you tell your father that I drove the tsetse
flies away,  weeded the farm and took you as a wife, I will also tell your
father you are freeing yourself in bed.']  `Then the wife said: `No, (I won't
say
anything to my father'.)' (CB 036)
\end{exe}



\subsubsection{Manner adverbs}
\label{sec:GRM-manner-adv}

A manner adverb  ({\it gl.} {\sc advm}) describes the way the event denoted by
the verb(s) is carried out. The examples in (\ref{ex:GRM-adj-mann}) illustrate
the meaning and distribution of some manner  adverbs.


\begin{exe}
\ex\label{ex:GRM-adj-mann}
\begin{xlist}

\ex\label{ex:GRM-adj-mann-carefully}
\gll dɪ sãã bʊɛ̃ɪbʊɛ̃ɪ \\
{\comp} drive {\advm}\\
\glt `Drive carefully.'

\ex\label{ex:GRM-adj-mann-slowly}
\gll dɪ ŋma bʊɛ̃ɪbʊɛ̃ɪ\\
{\comp} talk {\advm}\\
\glt `Talk slowly.'

\ex\label{ex:GRM-adj-mann-lighly}
\gll ʊ tʃɔjɛ kaalɪ fɛlfɛl\\
 {\sc 3.sg} run.{\pfv} go {\advm} \\
\glt `She ran away lightly (manner of movement, as a light weight
entity).'

\ex\label{ex:GRM-adj-mann-silently}
\gll  	hã́ã̀ŋ 	  sáŋá 	tʃérím, (...) \\
woman   sit  	{\advm} (...)\\
\glt `The woman sat quietly,  (...)' (CB 032)

\end{xlist}
\end{exe}

It is common for an ideophone to function as a manner adverb (see ideophone in
section \ref{sec:GRM-onoma}). One could argue that  all the manner adverbs in
(\ref{ex:GRM-adj-mann}) are ideophones, i.e. they display reduplicated forms
and {\S tʃerim} is one of a few words which ends with a bilabial nasal. 

The
examples in (\ref{ex:GRM-adj-mann-ideo-adv}) show the repetition of two
expressions; one is an ideophone, i.e. {\S kaŋkalaŋ} `crawl of a snake', and the
other an adverb, i.e.  {\S laɣa} `now'.  The latter is a temporal adverb, but is
treated as a manner adverb when reduplicated, i.e. {\S laɣalaɣa} `quickly'. The
repetition of {\S kaŋkalaŋ} and {\S  laɣalaɣa} conveys that the motion was
({\S kpa} `taken'  and) occurring with great speed.
%inceptive meaning of kpa


\begin{exe}
\ex\label{ex:GRM-adj-mann-ideo-adv}
\begin{xlist}

\ex\label{ex:GRM-adj-mann-ideo}
\gll a baaŋ kpa {kaŋkalaŋ kaŋkalaŋ kaŋkalaŋ}\\
{\conn} just take crawl.rapidly\\
\glt `(She was after the python) but (he) started to crawl away like a shot.'

\ex\label{ex:GRM-adj-mann-adv}
\gll  ka baaŋ kpa laɣalaɣa laɣalaɣa\\
{\conn} just take {\advm} {\advm}\\
\glt `(She) started to (walk) quickly.'

\end{xlist}
\end{exe}


The manner adverb {\S kɪŋkaŋ} `abundantly',  which is composed of the classifier
{\S kɪn} and the verb {\S kana} `abundant',  typically quantifies or intensifies
the
event and always comes after the word encoding the event.  Notice in
(\ref{ex:GRM-adj-mann-alot-v})  and (\ref{ex:GRM-adj-mann-alot-n})   that {\S
kɪŋkaŋ} follows a verb and a nominalized verb respectively. However, in
(\ref{ex:GRM-adj-mann-alot-quant}), {\S kɪŋkaŋ} does not function as a
manner adverb but as a quantifier.



\begin{exe}
\ex\label{ex:GRM-adj-mann-alot}
\begin{xlist}
\ex\label{ex:GRM-adj-mann-alot-v}
\gll gbɪ̃̀ã́         	ɪ̀     	jáárɪ́jɛ́      	kɪ́ŋkàŋ     	nà
(...)\\ 
monkey 	you   	unable.{\pfv} 	  {\advm} {\foc} (...)\\
\glt `Monkey, you are so incompetent, (...).' (LB 016)

\ex\label{ex:GRM-adj-mann-alot-n}

\gll duo tʃʊɔɪ kɪŋkaŋ wa wire\\
asleep lie.{\nmlz} {\advm} {\neg} good\\
\glt `Sleeping too much is not good.'

\ex\label{ex:GRM-adj-mann-alot-quant}
\gll kùórù 	kùò  	tɪ́ŋ 	kà   kpágá kìrìnsá wá  kɪ̀ŋkáŋ̀\\
 chief farm {\art} {\egr} have tsetse.fly.{\sc pl}   {\foc}  {\quant}\\
\glt 	`At the chiefs farm there are many tsetse flies.' (CB 002)

\end{xlist}
\end{exe}






\subsection{Adverbial pro-forms {\it keŋ} and {\it nɪŋ}}
\label{sec:GRM-adv-pro}

The expressions  {\S keŋ} and {\S nɪŋ} are treated as  two (manner) adverb
pro-forms ({\it gl.} {\sc advm}).  They are referred to as pro-forms since they
can substitute for an antecedent,  and  as  adverbs since they must refer back
to a
degree, quality or manner indicated by an event, or  a condition or property by 
an entity. Because they usually refer to an eventual sort/kind specified or
understood, they are provisionally categorized as manner adverbs.

The adverbs   {\S keŋ} and {\S nɪŋ} are very frequent in the
language and bring
to mind the Norwegian word {\S sånn} or the French phrase {\S comme \c{c}a}. 
English `this way' or `like this/that', as in `He did it this way',  more
or less corresponds to
the meaning of   {\S keŋ} and {\S nɪŋ}, which (\ref{ex:GRM-adv-pro-keng-ning})
illustrates.


\begin{exe}
\ex\label{ex:GRM-adv-pro-keng-ning}
 \begin{xlist}
 
 \ex\label{ex:GRM-adv-pro-ning}
\gll baaŋ ɲuãsa ka sii baŋ nɪ nɪŋ\\
{\q}  smoke  {\egr} rise {\advl} {\postp} {\advm}\\
\glt `What smoke is rising here like this?'  
%(Python story 059)
  \ex\label{ex:GRM-adv-pro-keng}
 \gll baaŋ ka ja keŋ?\\
  {\sc q} {\sc egr} do  {\advm}\\ 
 \glt `What is doing like that?' (Reaction to a sound coming from inside a pot)
 
 \end{xlist}
\end{exe}   


I translate {\S nɪŋ} as `like this' and {\S keŋ} as `like that'. This is
motivated by the way they  encode a sort of psychological saliency on a
proximal/distal dimension. This distinction needs more evidence than the one I
provide,  but consider the conversation between A and B in
(\ref{ex:GRM-adv-pro-keng-AB}). 


\begin{exe}
\ex\label{ex:GRM-adv-pro-keng-AB}
 \begin{xlist}
 
 \ex\label{ex:GRM-adv-pro-A}
\gll A: nɪn na baaba ŋma\\
 {} {\advm} {\foc} B. say\\
\glt `Is this what Baaba said?'

  \ex\label{ex:GRM-adv-pro-B}
 \gll B: ɛ̃ɛ̃ ken ne ʊ ŋma\\
 {} yes {\advm} {\foc} {\sc 3.sg} say\\
\glt `Yes, that is what he said.'
 
 \end{xlist}
\end{exe}   

Similarly,  the (fictional) discourse excerpt in
(\ref{ex:GRM-adv-kapok}) concerns a father (A) addressing his son (B) on the
topic of  how to ignite kapok fiber. The sentence (\ref{ex:GRM-adv-kapok-A-2})
is accompanied with a demonstration on how to strike a cutlass on a stone.


\begin{exe}
\ex\label{ex:GRM-adv-kapok}
 \begin{xlist}
 
 \ex\label{ex:GRM-adv-kapok-A-1}
\gll A: kpa koŋ a ŋmɛna diŋ\\
{}  take kapok {\conn} ignite fire\\
\glt `Take some kapok and start a fire.'

 \ex\label{ex:GRM-adv-kapok-B}
\gll B:  ɲɪnɪɛ̃ ba ja ka ŋmɛna\\
{} {\q} {\sc 3.pl} do {\egr} ignite\\
\glt `How does one ignite.' 

 \ex\label{ex:GRM-adv-kapok-A-2}
\gll  A: ŋmɛna nɪŋ\\
{} ignite {\advm}\\
\glt `Ignite like this.'

 \ex\label{ex:GRM-adv-kapok-A-3}
 \gll  A: tʃɪa dɪ tʃɪ waawa ŋmɛna keŋ\\
{} tomorrow {\conn} {\cras} come.{\pfv} ignite {\advm}\\
\glt `Tomorrow when you come, ignite like that.'
 \end{xlist}
\end{exe}  
 
In the context of (\ref{ex:GRM-adv-kapok}), at the farm the next day, the boy
(B) would tell a colleague: {\S ken ne ba ja ŋmɛna},  {\it lit} like.that they
do ignite, `that is how one ignites'. 

In (\ref{ex:GRM-ning-prop-2}), {\S nɪŋ} refers to the condition of  the room,
which is not a manner  but a property of the room. 


\begin{exe}
\ex\label{ex:GRM-ning-prop-2}
 \gll nɪŋ lɛɪ ʊ dɪa haŋ ja dʊ\\
 {\advm} {\sc neg}  {\sc 3.sg.poss} house {\sc dem} do be\\
\glt  `This is not how his room used to be.'
%(Python story 078)

\end{exe}

In addition, {\S keŋ} and {\S nɪŋ} can function as  discourse particles, whose
meanings resemble   English `like' in some registers \citep{Muff02}. In
(\ref{ex:GRM-keng-like}), {\S keŋ} is considered superfluous since it does not
contribute to the manner of  motion or the state of the
participant.\footnote{Something identical to the translation of
(\ref{ex:GRM-keng-like}) may be heard in  Ghanaian Pidgin English spoken in Wa.
This suggest that Waali and/or Dagaare has one or more similar adverbial
pro-form(s).} 

\begin{exe}
 \ex\label{ex:GRM-keng-like} 
 \gll n̩ kaalʊʊ keŋ \\
 {\sc 1.sg} go.{\ipfv .\foc}  {\sc advm}\\
 \glt `I am leaving like that'
\end{exe}

Also, depending on the intonation associated with it, and whether or not  the
focus
particle  is  present, {\S keŋ} and {\S nɪŋ} can function as
interjections used to convey comprehension or surprise. So a phrase like {\S
kén nȅȅ} could be roughly translated as `Is that so?', {\S kén nè}   has a
similar function to the English  tag-question `Isn't it?', but {\S kéēèŋ} or
{\S kén né} could be translated as `yes, that is it'. 

Finally, \cite{Mcgi99} presents  {\S nyɛ} and {\S ɛɛ} (variant {\S gɛɛ}) as
demonstrative pronouns in Pasaale, which can also modify an entire clause. The
former
corresponds to `this' and the latter to `that'. At this point, it is a matter of
comparing the two languages and the terminology employed.  Nonetheless, in the
majority of the examples provided by \cite{Mcgi99}, Chakali {\S keŋ} and {\S
nɪŋ} seem to have the same function. 


\subsection{Focus}
\label{sc:GRM-focus}

Since the notion of focus has been discussed separately in connection with
nominals and verbals, this section offers a basic overview of what has been
stated.  \citet[326]{Dik97} writes that   ``the focal information in a
linguistic expression is
that
information which is relatively the most important or salient in the given
communicative setting''.  In Chakali, we saw  two ways in which a
speaker can integrate focal information, and both of them put `in focus' a
constituent.\footnote{The  terminology employed in the literature is probably
the result
of  complex and still obscure phenomena. For instance, for the
post-verbal particle {\F lá} in Dagaare, \cite{Bodo97} uses the term
`factitive' and `affirmative' particle interchangeably, \cite{Daku05} uses
`(broad- and narrow-)  focus' and glosses it either as {\sc aff} or {\sc foc},
and
\cite{Saan03} uses post-verbal particle and glosses it as {\sc aff}. In-depth
accounts of focus in Grusi languages can be found in \cite{Blas90, Mcgi99}.
 Anne Schwarz has worked extensively on the topic in some Gur and Kwa
languages \citep{Schw10}.}   The first
encodes focal information in a particle which  always
 follows a nominal, i.e. {\S ra} and variants. Its  phonological shape is
determined by the
preceding phonological material (see sections \ref{sec:focus-forms} and 
\ref{sec:GRM-foc-neg}). The second, which was called the assertive suffix, takes
the form of vowel features which
are suffixed onto the verb  (see sections \ref{sec:GRM-verb-perf-intran} and 
\ref{sec:GRM-verb-suffix}). It was claimed that  the assertive suffix surfaces
only if (i) none of the other constituents in the
clause are in focus, (ii) the clause does not include negative polarity items,
and (iii) the clause is intransitive.
The second criterion (ii) is applicable to the particle {\S ra} as well: thus
focal
information can only exist in affirmative clauses, negation automatically
prevents information from being in focus.\footnote{\citet[94]{Bodo97} writes
(for
Dagaare) that
``[the factitive particle {\F lá}] is in complementary distribution with the
negative polarity particles, as one would expect of an affirming particle".}  In
 (\ref{ex:GRM-focus}),  the
examples illustrate  how the  focal information is
encoded when the object (\ref{ex:GRM-focus-obj}), the subject
(\ref{ex:GRM-focus-subj}) and the predicate  (\ref{ex:GRM-focus-pred}) are
considered the most important piece of information. 


\begin{exe}
\ex\label{ex:GRM-focus}
\begin{xlist}
 \ex\label{ex:GRM-focus-obj}{\it Focus on object: What has the man chewed?}
\gll   à báál tíè sɪ́gà rá\\
      {\art} man chew bean {\foc} \\
\glt `The man chewed BEANS'

\ex\label{ex:GRM-focus-subj}{\it Focus on subject: Who has chewed the beans?}
\gll   à báál lá   tíè sɪ́gà   \\
       {\art} man {\foc} chew bean    \\
\glt `The MAN chewed beans'

\ex\label{ex:GRM-focus-pred}{\it Focus on predicate: What happened?}
\gll    à báàl tíéwó  \\
   {\art} man chew.{\pfv .\foc}    \\
\glt `The man CHEWED'

\end{xlist}
\end{exe}

The focus particle does not differentiate between  grammatical functions and
is not obligatory. In fact, focus is quite rare in narratives.  

Interestingly, if the postposition {\S nɪ} occurs between the focus particle and
the preceding nominal, one would expect  the focus particle to surface in its
default form, i.e. {\S ra},  since the required adjacency is no longer satisfied
(see  section  \ref{sec:focus-forms}). 


\begin{exe}
\ex\label{ex:GRM-focus-form}
\begin{xlist}
\ex\label{ex:GRM-foc-form-X}
\TExt{\Txt{$\alpha$}\COn{rr}\ \  & \Txt{\I nɪ} & \ \  \TXT{\foc}}
 
\ex\label{ex:GRM-foc-form-1}
\gll  a maŋkɪsɪ ɲuu nɪ ro \\
       {\art} {match} {\reln} {\postp} {\foc}\\
\glt `on the top of the matchbox'

\ex\label{ex:GRM-foc-form-2}
\gll  a  pul nɪ ro \\
       {\sc art} {river} {\sc postp}  {\sc foc}\\
\glt `on/at the river'
\end{xlist}
\end{exe}

However, on several occasions, the postposition becomes `transparent' and 
vowel-harmony can still operate (i.e. though not the consonantal one). The
phenomenon is shown in (\ref{ex:GRM-focus-form}).\footnote{A more extreme case
is found in example (\ref{ex:GRM-obl-obj-no-spa-foc}).}


\section{Miscellaneous}
\label{sec:GRM-mis-lin-phen}


\subsection{Linguistic taboos}
\label{sec:GRM-ling-taboo}

A linguistic taboo is defined here as the avoidance of
certain words on certain occasions due to  misfortune associated with those
words. 
These circumstances depend on belief; they can be widespread or marginal. The
avoidance of certain words may depend on the time of the day or action carried
out. For instance, not only  is sweeping  not allowed when someone eats, but
uttering the word {\S tʃãã} `broom' is also forbidden. Also, mentioning
certain animal names is excluded as they may either be tabooed by someone
present or attract their attention, i.e the animal may believe that it is being
called. The strategy is to substitute a word with another. The examples in
(\ref{ex:GRM-taboo-synonyms})  are called {\it taboo synonyms}; the word on the
left of the arrow is the word avoided and the one on the right is its
substitute(s). The consumption taboos discussed in section
\ref{sec:SOC-religion} are intimately related to taboo synonyms.


\begin{exe}
\ex\label{ex:GRM-taboo-synonyms}{\it Taboo synonyms}

 {\I bɔ̀là} $\leftrightarrow$ {\I selzeŋ}/{\I neŋtɪɪna} `elephant'\\
{\I dʒɛ̀tɪ̀} $\leftrightarrow$ {\I ɲuzeŋtɪɪna} `lion'\\
{\I bʊ́ɔ̀mànɪ́ɪ̀} $\leftrightarrow$ {\I ɲuwietɪɪna}/{\I nebietɪɪna} `leopard'\\
{\I váà} $\leftrightarrow$ {\I nʊãtɪɪna}/{\I nʊãzɪmɪɪtɪɪna} `dog'\\
{\I kɔ́ŋ} $\leftrightarrow$ {\I nɪɪtɪɪna} `cobra' \\
{\I gbɪ̃̀ã́} $\leftrightarrow$ {\I neŋgaltɪɪna} `monkey'\\
{\I hèlé} $\leftrightarrow$ {\I muŋzɪŋtɪɪna} `type of squirrel'\\
{\I dʊ̃̀ʊ̃̀wìé} $\leftrightarrow$ {\I mábíéwāátèlèpúsíŋ} `type of
snake'\\
{\I tébíǹ} $\leftrightarrow$ {\I batʃogo}/{\I sankara} `night'\\
{\I ɲʊ́lʊ́ŋ} $\leftrightarrow$ {\I ɲúbíríŋtɪ́ɪ́nà} `blind'\\
{\I tʃã́ã́} $\leftrightarrow$ {\I kɪmpɪɪgɪɪ} `broom'\\
{\I búmmò} $\leftrightarrow$ {\I doŋ} `black'\\

\end{exe}

The substitutes to the right or the arrows are usually complex stem nouns. Many
of them use
 the stem {\S tɪɪna} `owner of': {\S neŋtɪɪna}, {\it lit.} arm-owner.of,
`elephant',  {\S muŋzɪŋtɪɪna}, {\it lit.} tail-large-owner.of,  `type of
squirrel', {\S mábíéwāátèlèpúsíŋ},  {\it lit.}
sibling-not-reach-meat-me `type of snake', etc. 



\subsection{Ideophone and interjection}
\label{sec:GRM-onoma}

 In Chakali, ideophones typically suggest the description of an abstract
property
or the manner in which an event unfolds.  The majority of ideophones are
categorized as qualifiers (section \ref{sec:GRM-qualifier}) or manner adverbs
(section \ref{sec:GRM-adjuncts}), and are usually
difficult to translate. A large set of color expressions can be treated as
 ideophonic expressions (see section \ref{sec:termvari}).  In
(\ref{ex:GRM-ideo}) two examples of ideophone are provided.
                                             
%    the sounds that
%are produced show the concepts that they express

\begin{exe}
\ex\label{ex:GRM-ideo}
\begin{xlist}
 \ex\label{ex:GRM-ideo-advm}
\gll koŋ baaŋ sii dɪa nɪ pə̀pə̀pə̀\\
kapok just rise house {\postp} {\advm}\\
  \glt `The kapok fiber was burning in the house at an increasing rate.'
%\hfill{Python story (line 135)}

 \ex\label{ex:GRM-ideo-qual}
\gll {} aka baaŋ jaa wə̀rwə̀rwə̀rwə̀r\\
(...) {\conn} just {\ident} {\qual}\\
  \glt `and (the eyes of the Python) are glittery.' 
%\hfill{Python story (line 132)}

\end{xlist}
\end{exe}


Apart from {\S pəpəpə} `increase in intensity' and   {\S wərwərwərwər}
`glittery', other
examples are {\S fɔbɔp} `move briskly', {\S kaŋkalaŋ kaŋkalaŋ kaŋkalaŋ} `rapid
crawl of a snake', {\S krrrr} `sound of running', 
 and {\S pã̀ã̀} `sound of an
eruption caused by lighting a fire', among others.

An onomatopoeia is a type of ideophone which not only suggests the concept  it
expresses with sound, but imitates  the actual sound of an entity or event.
Examples of onomatopoeia are {\S púpù} `motorbike', {\S tʃétʃé} `bicycle',
{\S tʃɔkɔ̃ɪ̃ tʃɔkɔ̃ɪ̃} `sound
of a guinea fowl' and {\S gbàgbá}  `duck'\footnote{The word for `duck' is
probably 
borrowed from Waali.
This bird was probably introduced recently and was hard to find in the
villages
visited.}  and {\S kpókòkpókòkpókò} `sound of knocking on a clay pot'.


Reduplication of one or two syllables is the general structural shape of
ideophones and onomatopoeias. Another characteristic of ideophones and
onomatopoeias  is their
tendency to violate the general
phonological structure of the language.  For instance,  {\S fɔbɔp} `move
briskly'  is the only expression I found ending with a bilabial stop. 

Similarly, a strategy to convey an amplified meaning or the idea of
continuity is to lengthen the sound of an existing word. Consider 
(\ref{ex:GRM-lenght}) below.

\begin{exe}
 \ex\label{ex:GRM-lenght}
   \gll  kawaa sii tarɪ keeeeeeeŋ, aka dʊa  ba dɪanʊã nɪ\\
pumpkin rise {creep} {\advm} {\conn} {be.at} {\sc 3.sg.poss} door {\postp}\\
\glt `The pumpkin crept, crept, crept and crept up to their
door mat.'
%\hfill{Python story (line 56)}
\end{exe}

In (\ref{ex:GRM-lenght}) the adverbial pro-form {\S keŋ} (see section
\ref{sec:GRM-adv-pro}) is stretched to simulate the extention in time of the
event, i.e. the pumpkin grew until it reached the door.\footnote{An equivalent
meaning may be expressed in Ghanaian Pidgin English with the adverbial
expression  {\F ãããã}, as in {\F I worked ãããã, until night time.}}


\subsection{Formulaic language}
\label{sec:GRM-greet}


This section introduces some pieces of formulaic language, which is defined as
conventionalized and idiomatic words or phrases. It usually include greetings,
idioms, proverbs, collocations, etc. \citep[see][]{Wray05}. First, common
interjections are included in table \ref{tab:GRM-interj},\footnote{The etymology
of {\F
ʔàmé} has not been confirmed and {\F gáfrà} is ultimately Hausa. The word
{\F ʃɪ́ã̀ã̀} is equivalent to the
function  associated with the action of {\F tʃuuse} in Chakali ({\F tʃʊʊrɪ} in
Dagaare, {\F tʃʊʊhɛ} in Waali, `puf' or `paf'  in Ghanaian English ($<$
English `pout')), which
is a
fricative sound produced by a non-pulmonic, velarized ingressive airstream
mechanism, articulated with the lower lip and the upper front teeth while the
lips are protruded. Thanks to Egil Albertsen for helping me to describe the
sound.} then some greetings and idioms are presented. Needless to say, since
they are conventionalized and idiomatic, the translation formulaic language is
always a rough  approximation.



\begin{table}[!htb]
\centering
\caption{Selected interjections \label{tab:GRM-interj}}

\begin{Itabular}{lp{8cm}}
\Hline
Interjection & Gloss\\[1ex] \hline
ʔàɪ́	&	no	\\
ʔɛ̃ɛ̃		&yes	\\
gáfrà	&	excuse \\
tóù	&	o.k.	\\
ʔàmé	 & so be it  ({\G etym.} Amen)	\\
ʔóí	&	indicates surprise\\
fíó	& totally not	\\
ʔánsà	&	welcome	\\
hĩ́ĩ̀ĩ́	&	expressing disappreciation of an action
carried out by someone else\\
ʔàwó	&	reply to greetings, a sign of appraisal of the interlocutor's
concerns\\
 ʔábà & indicates new and unexpected information\\
ʃɪ́ã̀ã̀ & insult when uttered after someone's remark or simply intended at
someone\\ 
\Hline
\end{Itabular}
\end{table}


\subsubsection{Greetings}
\label{sec:GRM-greet}

Crucial and obligatory prior to any communicative exchange, greetings trigger
both attention and respect. When meeting with elders, one should  squat  or bend
forward hands-on-knees  while greeting. Clan names can be used in greetings,
e.g. {\S ɪ́tʃà} `respect to you and to your clan'. In table
\ref{tab:greetings},  I provide typical greeting lines with some responses.



\begin{table}[!htb]
\centering
\caption{Greetings\label{tab:greetings}}

\begin{Itabular}{llp{7cm}}
\Hline
Time & Speaker A & Following by either speaker A or B\\ \hline

Morning  & ánsùmōō  & ɪ̀ sìwȍȍ `you stood?', ɪ̄ dɪ̀ tʃʊ́àwʊ̏ʊ̏ `and your
lying?', ɪ̀ bàtʃʊ̀àlɪ́ɪ̀ wīrȍȍ `you sleeping place was good?' \\[1ex]

Afternoon   & ántèrēē & ɪ́ wɪ́sɪ́ tèlȅȅ  `has the sun reached you?' ɪ́
dɪ̄à
`and your house?'  ɪ̄ bìsé mūŋ `and all your children?'\\[1ex]
  

Evening & ɪ́ dʊ̀ànāā &   ɪ́  dʊ̄ɔ̄n tèlȅȅ  `your evening has reached', 
ɪ́ kùó `and your farm?'\\
\Hline
\end{Itabular} 
\end{table}



The second singular pronoun {\S ɪ} is replaced by the  second singular plural 
{\S ma}, i.e.  {\S ánsùmōō} $\leftrightarrow$ {\S māānsùmōō},
when there is more than one adreesee or when there is  a single person but the
greetings
are intended to the entire house/family: thus  the distinction {\S ɪ}/{\S ma}
does
not correspond to the politeness function of French {\S tu}/{\S vous}. Chakali
morning and afternoon greetings resemble those of Waali and other languages of
the area.
The response to various greetings such as {\S ɪ́ dɪ̄à} `and your house?',  {\S
ʔánsà} `welcome, thanks' and many others is the multifunctional expression {\S
ʔàwó},  which is, among other things, a sign of appraisal of the
interlocutor's
concerns. The same expression is found in Gonja, but it may have different
functions. I was told that the more extensive the greetings, the more
respect one shows the addressee.  For instance, the elders do not
appreciate the tendency of
the youths to morning-greet as {\S ã̄sūmō}, but prefer something like {\S
áánsùùmōōō}. 
 



\subsubsection{Idioms}
\label{sec:GRM-idiom}

An idiom is a  composite expression which does not convey the literal  meaning
 of the composition  of its parts. Common among many African languages is a
strategy by which  abstract nominals are expressed in compound forms. These
compounds are made of stems whose meanings are disassociated from their ordinary
usage. Some examples have already been provided in section
\ref{sec:GRM-qualifier}. 
In Chakali, words identifying mental states and habits/behaviors are
often idiomatic, e.g. {\S síínʊ̀màtɪ́ɪ́nà} ({síí-nʊ̀mà-tɪ́ɪ́nà}, {\it
lit.} eye-hot-owner), `wild' or {\S nʊ̀ã̀pʊ̀mmá} ({nʊã-pʊmma}, {\it lit.} 
mouth-white), `unreserved'. Even though the expression {\S síínʊ̀màtɪ́ɪ́nà}
is made out of three lexical
roots, it is a `sealed' expression and is associated with the manner in which a
person behaves, i.e. a wild person. The sequence {\S ja
nʊã dɪgɪmaŋa} in (\ref{ex:GRM-idiom-mouth}), {\it lit.} do-mouth-one,  is also
treated as an idiomatic expression.


\begin{exe}
 \ex\label{ex:GRM-idiom-mouth}
   \gll   ba ja nʊã dɪgɪmaŋa a summe dɔŋa\\
{\sc 3.pl} do mouth one {\conn} help {\recp} \\
\glt `They should agree and help each other.'

\end{exe}

Needless to say, it is often difficult to  distinguish between an idiomatic
expression and  an expression in which only one of the  components is use in a
 non-literal sense.
% eat-patience
% However, it is often difficult to  distinguish between an idiomatic
% expression from an expression in which one of its component is use in a
% non-literal sense. For instance, one could treat {\S di kaɲɪtɪ},  {\it lit.}
% eat-patience,  as an idiom, but the verb {\S di} 
% Other idiomatic expressions are 
% {\S}, {\it lit.} , `'
% {\S}, {\it lit.} , `'
% {\S}, {\it lit.} , `'
% {\S}, {\it lit.} , `'



\subsubsection{Clicks}
\label{sec:GRM-greet}

\citet[151]{Nade89} writes that clicks\footnote{A click may be roughly defined
as  the release of a pocket of air enclosed between two points of contact in
the mouth. The air is rarefied by a sucking action of the tongue
\cite[see][]{Lade93}.}  may be  heard in the Gur-speaking area to  mean `yes',
`I'm listening' or `Oh, dear!'.  This also occurs in the villages where I
stayed, but
I noticed that one click usually means `yes', `I understand' or `I agree',
whereas two clicks mean the opposite. The click is either palatal or velar and
is produced with the lips closed.





