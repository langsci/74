
\chapter*{Appendix B}
\label{chap:TXT-text}
\pagestyle{plain}


\section*{TOC file}
\label{chap:TXT-TOC}

The contribution of the work  is thus the bulk of material  supplied to
prospective linguists working on Chakali or Southwestern
Grusi  languages: phonological and
lexical data,  datasets in the form of  paradigms and structured data, and  
narrative texts.    As a result,  including Chakali into Southwestern
Grusi
comparative, historical or lexicography work is finally practicable. The
knowledge provided on the language allows for a better understanding of the 
archived audio and video data (see table \ref{}),  and consequently  their
analysis can be carried out in a less random manner. 