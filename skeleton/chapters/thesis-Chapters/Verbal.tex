
\section{Verbals}
\label{sec:GRM-verbals}


It was  shown in section 
 \ref{sec:GRM-interr-clause} that a clause consists minimally of  a simple
predicate, one or two arguments and an
optional adjunct.  The structure introduced in (\ref{ex:GRM-clause-frame}) is
repeated below:


\begin{exe}
\exp{ex:GRM-clause-frame}
 {\sc s|a}  $+$ {\sc p} $\pm$ {\sc o} $\pm$ {\sc adj} 
\end{exe}

In this section,  the morphological  and distributional characteristics of
the predicate {\sc p} are discussed.  First, any word or phrase which can take
the place of  the predicate {\sc p} in  (\ref{ex:GRM-clause-frame}) is
identified as `verbal'.  Secondly, the term `verbal'  can refer to a
semantic notion at the lexeme level. The language is analyzed as exhibiting two
types of verbal lexeme: the {\it stative} lexeme and the {\it active} lexeme.  
It was shown in section \ref{sec:GRM-der-agent} that
both types of verbal lexeme take part in nominalization processes. The verbal
stem in (\ref{ex:verb-VP})  must be instantiated with a
verbal lexeme. Thirdly, the
term `verbal' can refer to the whole of the verbal constituent, including the
verbal modifiers.


\begin{exe}
\ex\label{ex:verb-VP}
 
\Tree[-1]{   &\Kq{VG}\Bq{dl}\Bq{dr} &&  \\
\Kq{\it preverb} &&\Kq{\it verb}\Bq{dl}\Bq{dr}&\\
 & \Kq[-3]{\it verbal stem} &&\Kq{\it suffx}}

\vspace*{4ex}

[[{\it preverb}]_{EVG} [{\it stem}]-[{\it suffix}]]_{VG}


\end{exe} 

                 The verbal group (VG) illustrated in (\ref{ex:verb-VP})
consists of linguistic slots which encode   various aspects of an event  which
may be realized in an utterance. A free standing verb is the minimal requirement
to satisfy the role of a predicative expression. Those verbal modifiers, which
are called preverbs here,  are grammatical items which specify the event
according to various  semantic distinctions. They precede the  verb(s) and take
part in the {\it expanded verbal group} (EVG). The expanded verbal group
identifies  a domain which excludes the main verb, so a {\it verbal group}
without preverbs would  be equivalent to a verb or a series of verbs (see SVC in
section \ref{sec:GRM-multi-verb-clause}). The term and notion is inspired from
analyses of the verbal system of Gã \citep{Daku70, Daku08, Daku08b,
Hell10}.\footnote{Although the extended verbal complex in \citet[48]{Hell10}
includes the verb, the decision to exclude the verb(s) from the  expanded verbal
group has no implications for the present exposition. The syntax/semantics of a
SVC is obviously not captured by the simple representation given in
(\ref{ex:verb-VP}) \cite[see][]{Daku08b}. A {\it verbal group} is unlike the
(traditional) verb phrase in that it does not include its internal argument,
i.e. direct object. I am aware of the obvious need to unify the descriptions of
the nominal constituent and the verbal constituent.} While a verbal stem
provides the core meaning of the predication,  a suffix may supply information
on  aspect, whether or not the verbal constituent is in focus and/or the index
of participant(s) (i.e. {\sc o}-clitic).  Despite there being little focus on
tone and intonation, attention on the tonal melody of the verbal constituent is
necessary since this also affects the interpretation of the event. These
characteristics are presented below in a brief overview of the verbal system. 

% 
% one melodies affecting not
% only elements of the verbal constituent but elements immediately preceding or
% following it, and (iii) affixes
%without its participant(s) and other peripheral expressions.



\subsection{Verbal lexeme}
\label{sec:GRM-verb-lexeme}


\subsubsection{Syllable structure and tonal melody}
\label{sec:GRM-verb-syll-und-tone}

 Apart from [dʒ], all other segments  are attested in
verbs.\footnote{The current lexicon contains 388 verbs  (02/10/10 version).} The
majority consist of the open syllable types CV and CVV. The  common
syllable sequences found among the verbs are CV, CVV, CVCV, CVCCV,
CVVCV and CVCVCV.  Monosyllabic verbs make up approximately 21\% of the verbs in
current database, bisyllabic 69\% and trisyllabic  10\%. Apart  from [dʒ], all
segments are attested in onset position word initially, but only {\S m, t, s, n,
r, l, g, ŋ} and {\S w} are found in onset position word-medially in bisyllabic
verbs, and only {\S  m, t, s, n,  l} and {\S g} are found  in onset position
word-medially in trisyllabic verbs.   All trisyllabic, CVVCV  and CVCCV verbs
have one of the front vowels (\{e, ɛ, i, ɪ \}) in the nucleus of their last
syllable.  The data suggests that {\sc atr}-harmony is operative, but not   {\sc
ro}-harmony, in these three environments, e.g. {\S fùólè} `whistle'. 
There
is no restriction on vowel quality for the monosyllabic or bisyllabic verbs
and both harmonies are operative.

% \footnote{Only three verbs have a consonant in final position; {\S zááŋ}
% `greet',   {\S
% fʊ́g} `light' and {\S  kàn} `abundant'.  The latter two are stative verbs. It
% is believed that the search for verbs' underlying form may reveal far more
% verbs with  a consonant in final position. }

Little evidence is available on lexical tonal melodies on verbs lexeme. Two
minimal pairs are identified:  one is {\S télé} `lean on' and {\S tèlè}
`reach', but this may turn out to be a kind of `aspectual switch' (from stative
to
process, or vice-versa), similar to the category switch discussed in section
\ref{sec:GRM-der-cat-switch}, and the other is {\S pɔ̀} `protect' and  {\S pɔ́}
`plant'.  Still, the high (H) and low (L) register tones are assumed. Table
\ref{tab:GRM-verb-tone-melody} presents some verbs which are classified based on
their syllable structures and  tonal melodies.  Despite the various attested
melodies, low tone CV verbs,   falling tone CVV verbs,   rising CVCV verbs
are marginal in the database. High tone CVV and more than one tone CVCVCV are
not attested.



%\clearpage
\begin{table}[htb]
\renewcommand{\arraystretch}{0.8}
\centering
\caption{Tonal melodies on verbs  \label{tab:GRM-verb-tone-melody}}
        
\begin{Itabular}{llll}
\Hline
Syllable type &  Tonal melody  & Example & Gloss\\ [1ex] \hline

CV 		&  H		&  pó	&	divide	\\
		&  H		& pɔ́		&plant	\\
		&  H		&pú		&cover	\\
		&  H		&pʊ́		&spit	\\
	        &  L 	&pɔ̀		&protect	  \\[0.5ex]
\hline

CVV 	&  L		& pàà 	& 	take\\
		& L		&  tùù	 &	go down	\\
	       & LH 	 &tìé		&	chew	\\
		& LH 	& tʃìí	&	forbid	\\
  		&HL 	& tʊ́ɔ̀  	&	deny\\
		&HL	&   ʔṹũ̀	 &	bury\\[0.5ex] 
\hline

CVCV&  L		&		bìlè	&     put	\\
 	  &  L		&		hàlà	&	fry\\
 	  &  LH		&		mɪ̀ná	 &    attach	\\
 	  &  LH		&		zɪ̀má	&    know	\\
 	  &  HL		&		dʊ́mà	&	bite		\\
 	  &  HL		&		ŋmɛ́nà	&	cut\\
	  &  HH		&		hẽ́sí	&	announce	\\
 	  &  HH		&		jóló	&	pour	\\[0.5ex]
 \hline

CVCCV 	& HH & bóntí	&	divide\\
		& HH &tóŋlí	&	squat \\
		&LL &sùmmè	&	beg \\
		&LL & zèŋsì	&	limp\\[0.5ex] 
 \hline

 CVVCV & HH.H&kpã́ã́nɪ́	&	hunt \\
		&HH.H& wáásɪ́		& reach boiling point	 \\
		&HH.L& vʊ́ʊ́rɪ̀	&	decide \\
		&HH.L& fíísè	&	wipe anus \\
		&LH.L&fùólè	&	whistle \\
		&LH.L&kʊ̀ɔ́rɪ̀	&	make \\
		&LL.L&bùòlì	&	sing\\
		&LL.L& fɪ̀ɪ̀lɛ̀	&	peek\\[0.5ex] 
 \hline

CVCVCV 	& LLL  &làgàmɛ̀	&	gather\\
			& LLL   &dʊ̀gʊ̀nɪ̀	&	chase\\
			& HHH  &ɲúŋísé	&	lose sight of\\		
			& HHH  & vílímí	&	whirl\\[0.5ex]

\Hline
\end{Itabular}      
\end{table} 
 




\subsubsection{Verbal state and verbal process lexemes}
\label{sec:GRM-verb-stative-active}


A verbal lexeme denotes an event, as opposed to an entity. A general distinction
between stative and non-stative events  is made: {\it verbal state} (stative
event) and {\it verbal process} (active event) 
lexemes are assumed. A verbal state lexeme can be identificational,
existential, possessive,  qualitative, quantitative, cognitive or  locative, and
refers more or less to a state or condition which is static, as opposed to
dynamic. The `copula' verbs {\S jaa} and {\S dʊa} (and its allolexe {\S tuwo})
are treated as subtypes of verbal stative lexemes since they are the only verbal
lexemes which cannot function as a main verb in  a perfective intransitive
construction (see section \ref{sec:GRM-verb-perf-intran}). Their meaning and
distribution was introduced in two sections concerned with the identificational
construction (section \ref{sec:GRM-ident-cl}) and existential construction
(section \ref{sec:GRM-loc-cl}).  The possessive 
{\S kpaga} `have'  is treated as  a verbal state lexeme as well (see possessive
clause in section   \ref{sec:GRM-poss-cl}).  A qualitative verbal state lexeme
establishes a relation between an entity and a quality. Examples are given in
(\ref{ex:GRM-v-stat-qual}).

\begin{exe}
\ex\label{ex:GRM-v-stat-qual}{\it Qualitative verbal state lexeme}\\
%{\it Descriptive:}\\
 {\I boro}  `short'  $>$ {\I à dáá bóróó} `The tree is short'\\
{\I goro}    `curved'  $>$ {\I à dáá góróó} `The wood is curved'\\
{\I jɔɣəsɪ}    `soft'   $>$ {\I   à bìé bàtɔ́ŋ jɔ́gɔ́sɪ́ʊ̀}  `The baby's
skin is soft'
\end{exe}

Similarly, a quantitative verbal state lexeme  establishes a relation between an
entity and a quantity. Yet, in (\ref{ex:GRM-v-stat-quant}), the subject of   {\S
maase} is the impersonal pronoun {\S a} which refers to a situation and not an
individual. The verb {\S hɪɛ̃}  `age' or `old'  is a quantitative verbal state
lexeme in the sense that it can measure objective maturity between
two individuals, i.e. {\S mɪŋ hɪɛ̃-ɪ}, {\it lit.} {\sc 1.sg.st} age-{\sc
2.sg.wk}, `I am older than you'. 


\begin{exe}
\ex\label{ex:GRM-v-stat-quant}{\it Quantitative verbal state lexeme}\\
%{\it Descriptive:}\\
 {\I kana}  `abundant'  $>$ {\I ba kanãʊ̃} `They are plenty (people)' \\
{\I maase} `enough'  $>$   {\I a maasejo keŋ} `It is o.k. like that'\\
{\I hɪɛ̃} `age' $>$ {\I mɪŋ hɪɛ̃ɪ̃} `I am older than you'
\end{exe}

Cognitive verbs such as {\S liise} `think',  {\S kʊ̃ʊ̃} `wonder, 
{\S kisi} `wish',    {\S tʃii} `hate', etc.  are also treated as verbal state
lexemes. Section (\ref{sec:SPA-post-verb}) will  cover another
type  of verbal state lexeme,  the locative verbs. 

Verbal process lexemes denote non-stative events. They are often partitioned
along the
(lexical) aspectual distinctions of  \cite{Vend57}, i.e. activities, 
achievements, accomplishments. Such verbal categories did not formally
emerge,\footnote{Although I am not able to identify these particular
distinctions, section \ref{sec:GRM-verb-suffix} suggests that there is a system
of verbal derivation which needs to be uncovered.}
nor did I specifically look for them, therefore I am not in a position to
categorize the verbal process lexemes at this point in the research (see
\citet[51]{Bonv88} for a thorough
description of a Grusi verbal system).   Thus, verbs which express that the
participant(s) is actively doing something, undergoes a process, performs an
action, etc. all fall within the  set of verbal process
lexemes. 



\subsubsection{Complex verb}
\label{sec:GRM-complex-verb}

A complex verb is  composed of more than one verbal lexeme. For
instance, when {\S laa} `take' and {\S di}
`eat' are brought together in a SVC (section \ref{sec:GRM-multi-verb-clause}),
they denote a taking and eating event, but
the same sequence can be interpreted  differently. The difference between a
SVC  and what I call a complex verb is that the latter is strictly
collocational and non-compositional. Also, unlike complex stem nouns but like
SVCs, the elements which compose a complex verb
must not necessarily be contiguous.  The sequences  {\S laa}+{\S di}
is glossed  `believe', and {\S laa}+{\S dʊ}, {\it lit.} take put,  `adopt', as
the examples in  (\ref{ex:cpx-verb-laa-di}) show.

\begin{exe}
\ex\label{ex:cpx-verb-laa-di}
\begin{xlist}
\ex
 \gll ǹ̩ láá kùósò díù \\
{\sc 1.sg} take G.  eat.{\foc}  \\
\glt `I believe in God.'

\ex
 \gll  ǹ̩ láá bìé dʊ̀ʊ̀ \\
{\sc 1.sg} take child put.{\foc}  \\
\glt `I adopted a child.'

\end{xlist}
\end{exe}

Other examples are {\S zɪma sii}, {\it lit.} know raise, `understand',  {\S
kpa ta}, {\it lit.}  take abandon, `drop' or `stop', {\S gɪla zɪma}, {\it lit.}
allow know, `prove', among others. 



\subsubsection{Verb forms}
\label{sec:GRM-verb-word}

The base form of a verb is identified as the segmental sequence which  would
appear in a positive imperative clause (section
\ref{sec:GRM-imper-clause}).\footnote{The verbs I elicited in that position
display high tone, but I do not have an exhaustive list so I prefer only to be
specific in equating the verb's base form with its segmental sequence.}
The form also corresponds  with verb elicited in isolation, which in
(\ref{ex:GRM-base-form}) is called the  citation form of the verb. 

\begin{exe}
\ex\label{ex:GRM-base-form}{\it base form = citation form = positive
imperative form}
\begin{xlist}
mara $>$ (dɪ) márá  `Attach!' \\
kpa  $>$  (dɪ)  kpá  `Take!'\\
tele  $>$  (dɪ)  télé  `Lean againt!'\\
kpe $>$   (dɪ)  kpé  `Crack and remove (the seed from the shell)!'

\end{xlist}
\end{exe}




The inflectional system of Chakali verbs has  few verb
forms and has a closer
resemblance to neighbor Oti-Volta languages than, for instance,  a
`conservative' Grusi language like Kasem \cite[51]{Bonv88}.\footnote{Dagbani is
described as a language where the ``inflectional system  for verbs is relatively
poor''  \cite[96]{Olaw99}. It has an imperfective suffix {\F -di}
\cite[97]{Olaw99} and  an imperative suffix {\F -ma}/{\F mi} \cite[101]{Olaw99}.
\citet[81]{Bodo97} writes that Dagaare has four verb forms: a dictionary
form, a perfective aspectual form, a perfective intransitive aspectual form and
an imperfective aspectual form. Also for Dagaare, \cite{Saan03}  talks about
four forms: perfective A and B, and Imperfective A  and B.}  Besides the
derivational suffixes (section \ref{sec:GRM-deri-suff}), the verb in Chakali is
limited to two
inflectional suffixes and one assertive suffix:  (i) one signals negation in the
negative imperative clause (i.e.  {\S  kpʊ́} `Kill',  {\S tɪ́  kpʊ́ɪ̀} `Don't
kill'),  (ii) another attaches to some verb stems in the perfective intransitive
only, and (iii)  the other signals assertion and puts the verbal constituent in
focus. Since the negative imperative clause has already been presented in
section
\ref{sec:GRM-imper-clause}, the perfective and imperfective intransitive
constructions are discussed next.  The former may contain both the perfective
suffix and the assertive suffix simultaneously, while the latter  displays the
base form of the  verb, with or without the assertive suffix.

\paragraph{Perfective intransitive construction}
\label{sec:GRM-verb-perf-intran}

As its name suggests, a perfective intransitive construction lacks a grammatical
object and implies the anteriority of an event (i.e. past),  the end  or the
reaching point of an action.  In the case of verbal state,
the  perfective  implies that the given state has been reached, or 
that the entity in subject position   satisfies the property encoded in
the verbal state lexeme. In (\ref{ex:GRM-intperfc-frame}),  two suffixes are
attached on  one  ({\sc +atr}, CV) verbal process stem and one  ({\sc
+atr}, CVCV) verbal state stem (see section \ref{sec:nasalization-verb-suffix}
for the general phonotactics involved).\footnote{The presence of schwa
({\F ə}) in a CVCəCV surface form as in (\ref{ex:GRM-intperfc-frame-state})  is
predicted by a rule operating on weak syllables (section \ref{sec:epenthesis}).}


\begin{exe}
\ex\label{ex:GRM-intperfc-frame}{\it Perfective intransitive construction}
\begin{xlist}

\ex\label{ex:GRM-intperfc-frame-process}{{\it  Verbal process:} {\sc s}  $+$
{\sc p} }
\gll àfíá díjóò\\
A. {di-j[{\sc +mid, -hi, -ro}]-[{\sc +hi,+ro}]}\\

\glt `Afia ate.'

\ex afia wa dije `Afia did not eat'

\ex\label{ex:GRM-intperfc-frame-state}{{\it  Verbal state:} {\sc s}  $+$ {\sc p}
}
\gll à dáá télə́jóò\\
{\art} daa  {tele-j[{\sc +mid, -hi, -ro}]-[{\sc +hi,+ro}]}\\
\glt `The stick leans'

\ex a daa wa teləje `The stick doesn't lean.'
\end{xlist}
\end{exe}

The first suffix to attach is the perfective suffix, i.e. -j[{\sc +mid, -hi,
-ro}] or simply /jE/. Although it appears on every (positive and
negative) stem in (\ref{ex:GRM-intperfc-frame}),  it does not surface on all
verb stems. The
information in table \ref{tab:GRM-perf-suff} partly predicts whether or not a
stem will surface with a suffix, and if it does, which form this suffix will
have.


\begin{table}[htb]
 \centering
\caption{Perfective intransitive suffixes
\label{tab:GRM-perf-suff}}
\begin{Itabular}{p{2cm}p{2cm}p{2cm}}
\Hline
Suffix /-jE/ & Suffix /-wA/ & No suffix  \\[1ex]
\hline

CV &  CVV & 1-CVCV \\
 2-CVCV & & \\ 

 \Hline
\end{Itabular}
\end{table} 

Table \ref{tab:GRM-perf-suff} shows that, in a perfective intransitive
construction, a CV stem must
be suffixed with {\S -jE} and  a CVV verb with {\S -wa}. The examples in
(\ref{ex:GRM-jE-wA}) are negative in order to prevent the assertive
suffix from appearing (see section \ref{sc:GRM-focus} on why negation and the
assertive suffix cannot co-occur).


\begin{exe}
\ex\label{ex:GRM-jE-wA}
\begin{xlist}

\ex{\it CV}\\
po	 $>$  àfíá wá    pójè		`Afia didn't divide'	\\
pɔ		 $>$ àfíá wá   pɔ́jɛ̀	 `Afia didn't  plant'\\
pu		 $>$ àfíá wá  pújè	 `Afia didn't  cover'	\\
pʊ		 $>$ àfíá wá  pʊ́jɛ̀	 `Afia didn't  spit'	\\
kpe		 $>$ àfíá wá  kpéjè	 `Afia didn't  crack and
remove'\\
kpa		 $>$ àfíá wá  kpájɛ̀	 `Afia didn't  take'	

\ex{\it CVV}\\

tuu $>$ àfíá wá  tūūwō   `Afia didn't  go down'\\
tie $>$  àfíá wá   tīēwō `Afia didn't chew'\\
sii  $>$  àfíá wá  sīīwō   `Afia didn't  raise'\\
jʊʊ   $>$  àfíá wá  jʊ̄ʊ̄wā  `Afia didn't  marry'\\
tɪɛ $>$  àfíá wá tɪ̄ɛ̄wā  `Afia didn't  give'\\
wɪɪ $>$  àfíá wá  wɪ̄ɪ̄wā  `Afia is not  ill'
  	
\end{xlist}
\end{exe}

The surface form of the perfective suffix which attaches to CV stems (i.e. {\S
-je}/{\S
-jɛ}) is predicted by the {\sc atr}-harmony rule of section
\ref{sec:vowel-harmony}. Notice that  {\sc ro}-harmony does not operate
in that domain. The CVV stems display  harmony between the stem
vowel(s) and the suffix vowel which is easily captured by a variable feature
alpha notation, as shown in rule (\ref{PHO-rule-perf-wa}).


\begin{Rule}\label{PHO-rule-perf-wa}{Prediction  for perfective intransitive 
-/wA/ suffix}\\
If the vowel of a CVV stem is
{\sc +atr},
the vowel of the suffix is {\sc +ro}, and if the vowel of a CVV stem is {\sc
-atr}, the vowel of the suffix is {\sc -ro}.\\
-/wA/ $>$  $\alpha${\sc ro}_{suffix}  /  $\alpha${\sc atr}_{stem} \_     
\end{Rule}

The rule in (\ref{PHO-rule-perf-wa}) assumes that the segment [{\S o}] is the
[{\sc +ro, +atr}]-counterpart of [{\S a}]. Notice also that I perceived the
same tonal melody in all clauses, raising doubts  on the tonal melodies of the
`citation forms' offered in table \ref{tab:GRM-verb-tone-melody}.

Predicting the set (i.e. either set 1-CVCV or  set 2-CVCV) into which a CVCV
stem
will fall  has proven unsuccessful. Provisionally,  I suggest that a CVCV
stem must be stored with such an information. One piece of evidence
supporting this claim comes from
the minimal pair {\S tèlè} `reach' and  {\S télé} `lean against':  the
former displays 2-CVCV (i.e. tele-jE),  whereas the latter displays 1-CVCV
(i.e. tele-\O).  However,  a CVCV stem with round vowels is less likely to
behave like a 1-CVCV stem, yet {\S pumo} `hatch' is a counter-example, i.e.
{\S a zal wa puməje} `the fowl didn't hatch'. The CVCCV, CVVCV and CVCVCV stems
have  not been investigated, but {\S kaalɪ} `go', a common  CVVCV verb, takes
the
/-jE/ suffix.  


\paragraph{Imperfective intransitive construction}
\label{sec:GRM-verb-perf-intran}

The imperfective  conveys the unfolding of an event, and it is often used to
describe an event taking place at the moment of speech. In addition, the
behavior of the egressive marker {\S ka} (section \ref{sec:GRM-EVC-egr-ingr})
suggest that the imperfective may be interpreted as an progressive event,  or
one which will happen in the future.  The imperfective is indicated by the base
form of a verb. As in the perfective intransitive, the assertive suffix may be
found attached to the verb stem. 


\begin{exe}
\ex\label{ex:GRM-assert-suff}
[[{\it verb stem}]-[{\sc +hi,+ro}]]_{verb \ in \ focus}
\end{exe}

Again, the constraints licensing the combination of the verb stem and the vowel
features  shown in (\ref{ex:GRM-assert-suff})   are (i) none of the other
constituents in the clause are in focus, (ii) the clause does not include
negative polarity items, and (iii) the clause is intransitive, that is, there is
no grammatical object. 

% , as opposed to an
% event perceived as bounded (i.e. perfective) or a hypothetical event (i.e.
% imperative)



\begin{exe}
\ex\label{ex:GRM-pos-neg-take}
%\begin{multicols}{2}
\begin{xlist}
\ex{\it Positive}

 ʊ̀ kàá kpáʊ̀   `She will take'\\
   ʊ̀ʊ̀ kpáʊ́ 	 `She  is taking/takes'

\begin{xlist}
\ex\label{ex:GRM-ipfv-out-nfoc}
\textasteriskcentered  a baal la  kpaʊ ({\sc art} man {\sc foc} take.{\sc foc})
`The MAN is taking'
\ex\label{ex:GRM-ipfv-out-stpro}
 \textasteriskcentered  wa  kpaʊ ({\sc 3.sg.st} take.{\sc foc})  `HE is
taking'
\ex\label{ex:GRM-ipfv-out-obj}
\textasteriskcentered  a baal   kpaʊ a kisie ({\sc art} man take.{\sc foc} {\sc
art} knife) `The man is taking the knife'
\end{xlist}

\ex\label{ex:GRM-ipfv-out-neg}{\it Negative}\\
 ʊ̀ wàá kpá   `She will not take'\\
   ʊ̀ʊ̀   wàà	 kpá `She  is not taking/does not take'

\end{xlist}
%\end{multicols}
\end{exe}

In (\ref{ex:GRM-pos-neg-take}), the forms of the verb in the
intransitive imperfective take the assertive suffix to signal that the verbal
constituent is in focus, as opposed to the nominal argument. It cannot appear
when the subject is in focus (\ref{ex:GRM-ipfv-out-nfoc}) or when the strong
pronoun is used as subject (\ref{ex:GRM-ipfv-out-stpro}), when a grammatical
object follows the verb  (\ref{ex:GRM-ipfv-out-obj}), or when the negation
preverb {\S waa} is present  (\ref{ex:GRM-ipfv-out-neg}).



\paragraph{Intransitive vs. transitive}
\label{sec:GRM-trans-intran}

Intransitive  and transitive clauses are shown in  
(\ref{ex:GRM-clause-core-intrans}) and (\ref{ex:GRM-clause-core-trans})
respectively.  



\begin{exe}
\ex\label{ex:GRM-clause-core}
\begin{xlist}
 \ex\label{ex:GRM-clause-core-intrans}{\it Intranstive clause}
\glll Kala dijoo \\
       {\sc s} {\sc p}\\
 Kala eat.{\sc pfv.foc} \\
\glt  `Kala ATE.' 
\ex\label{ex:GRM-clause-core-trans}{\it Transtive clause}
\glll Kala di sɪɪmaa ra\\
        {\sc s} {\sc p}  {\sc o} {} \\
Kala eat.{\sc pfv}  food {\sc foc}\\
\glt  `Kala ate FOOD' 
\end{xlist}
\end{exe}


Many verbs can occur in either  intransitive or transitive clauses. In
(\ref{ex:GRM-clause-ambitran}) the subject of the intransitve ({\sc s})
corresponds to the subject of the transitive ({\sc a}), and the same verb is
found with and without an object ({\sc o}).


\begin{exe}
\ex\label{ex:GRM-clause-ambitran}
\begin{xlist}

\ex\label{ex:vp26.14.}
\glll ʊ̀ʊ̀ búólùù \\
{\sc s} {\sc p}\\
       {\psg} sing.{\ipfv.\foc} \\
\glt  `He is singing.' 

\ex\label{ex:vp26.15.}
\glll  ʊ̀ʊ̀ búólù bùòl lò \\
{\sc a} {\sc p} {\sc o} {}\\
       {\psg}  sing.{\ipfv} song {\foc}    \\
\glt  `He is singing a song.' 

\end{xlist}
\end{exe}


It is possible to promote a prototypical theme argument to the subject position.
However,  informants have difficulty with some nominals in the subject
position of
intransitive clauses.   The topic needs further investigation, although it is
certainly related to a semantic anomaly.  The data in
(\ref{ex:GRM-intran-theme-subj}), where the  prototypical {\sc o}(bject) is in
 {\sc a}-position, illustrates the problem. In order to concentrate on the `goat
beating'- and `tree climbing'-activities and turn the two clauses
(\ref{ex:GRM-int-th-su-out-1}) and (\ref{ex:GRM-int-th-su-out-2}) into
acceptable utterances, the optimal solution is to use the
impersonal pronoun {\S ba} in subject position, e.g. \textasteriskcentered  {\S
bʊ̃ʊ̃ŋ   kaa maŋãʊ̃}
$>$  {\S ba kaa maŋa a bʊ̃ʊ̃ŋ (na)} `the goat is being
beaten'  (see impersonal
pronoun in
section \ref{sec:GRM-impers-pro}).



\begin{exe}
\ex\label{ex:GRM-intran-theme-subj}
\begin{xlist}
\ex
a bʊɔ kaa hireu  `the hole is being dug'
\ex\label{ex:GRM-int-th-su-out-1}
\textasteriskcentered a bʊ̃ʊ̃ŋ   kaa maŋãʊ̃  `the goat is being beaten'
\ex\label{ex:GRM-int-th-su-out-2}
\textasteriskcentered a daa kaa zɪnãʊ̃  `the tree is being climbed'

\end{xlist}
\end{exe}


Given that  the inflectional system of the verb is rather poor, and that the 
perfective
and assertive suffixes occur only in intransitive clauses,  how does one
encode a basic contrast like the one between a transitive perfective and
transitive imperfective? The paired examples in (\ref{ex:tra-pfv}) and
(\ref{ex:tra-ipfv})  illustrate 
 relevant contrasts.


  \begin{minipage}[h]{12cm}
\begin{multicols}{2}
\begin{exe}
  \ex\label{ex:tra-pfv}{\it Transitive perfective}
\begin{xlist}
  \ex\label{ex:tra-pfv-eat}
ǹ̩ dí kʊ̄ʊ̄ rā\\
 `I ate T.Z..' 
 \ex\label{ex:tra-pfv-plant}
ǹ̩ pɔ́ dāā rā\\
`I planted a TREE.'
 \ex\label{ex:tra-pfv-cover}
ǹ̩ tʃígé vīī rē\\
`I covered a POT.' 
 \ex\label{ex:tra-pfv-tie}
ǹ̩ lómó bʊ̃́ʊ̃́ŋ ná\\
`I tied a GOAT.' 
 \ex\label{ex:tra-pfv-carry}
m̩̀ mɔ́ná díŋ nē\\
`I carried  FIRE.' 
\end{xlist}
\end{exe}

\begin{exe}
  \ex\label{ex:tra-ipfv}{\it Transitive imperfective}
\begin{xlist}
 \ex\label{ex:tra-ipfv-eat}
ǹ̩ dí kʊ́ʊ́ rá\\
`I am eating T.Z..' 
 \ex\label{ex:tra-ipfv-plant}
m̩̀ pɔ́ dáá rá\\
`I am planting a TREE.' 
 \ex\label{ex:tra-ipfv-cover}
ǹ̩ tʃígè vīī ré\\
`I am covering  a POT.' 

 \ex\label{ex:tra-ipfv-tie}
ǹ̩ lómò bʊ̃̄ʊ̃̄ŋ ná\\
`I am tying  a GOAT.' 
 \ex\label{ex:tra-ipfv-carry}
m̩̀ mɔ́nà dīŋ né\\
`I am carrying  FIRE.' 
\end{xlist}
\end{exe}
\end{multicols}
 \end{minipage}
\vspace*{15pt}

% \begin{exe}
% \ex\label{ex:GRM-imperf-cons}
% \begin{xlist}
% 
% 
% \end{xlist}
% \end{exe}

Each pair in the verbal frames of  (\ref{ex:tra-pfv}) and (\ref{ex:tra-ipfv})
presents fairly regular patterns:  the high tone {\it versus} the falling tone
on the CVCV verbs is one instance. Another is the systematic change of the tonal
melodies on the grammatical objects in the two CV-verb cases. The data suggest
that
it is the tonal melody, and not exclusively the one associated with the verb,
which supports aspectual function in the language.
Thus, when the verb is followed by an argument, both perfective and the
imperfective are expressed with the base form of the verb.  However,  the tonal
melody alone  can determine whether a phonological string is to be understood as
a bounded event which occurred in the past or an unbounded event unfolding at
the moment of speech.



Tonal melody is crucial in the following examples as well. The examples in
(\ref{GRM-pfv-inter}) are three polar questions (see section
\ref{sec:GRM-interr-polar}), one perfective (i.e. past event) and two
imperfective (i.e. present and future events). The two first have the
same segmental content, and the last contains the egressive preverb {\S kaa}
with a rising tone indicating the future tense.  In order to signal a polar
question, each has  an extra-low tone and is slightly lengthened at the end of
the utterance. 

\begin{exe}
\ex\label{GRM-pfv-inter}
\begin{xlist}

\ex\label{GRM-pfv-inter-pfv}
\glll {\T } {\T   } {\T  } {\T  } {\T }\\
 ɪ   teŋesi  a  namɪã  raa \\
          {\sc 2.sg} {cut.{\sc pv}} {\sc art} {meat} {\sc foc}\\
\glt `Did you cut the meat (into pieces)?'\\



\ex\label{GRM-pfv-inter-impf}

\glll {\T } {\T   } {\T  } {\T  } {\T }\\
ɪ   teŋesi  a  namɪã  raa \\
          {\sc 2.sg} {cut.{\sc pv}} {\sc art} {meat} {\sc foc}\\
\glt `Are you cutting the meat (into pieces)?'\\


\ex\label{GRM-pfv-inter-impf-fut}

\glll {\T } {\T  } {\T   } {\T  } {\T  } {\T }\\
ɪ  kaa teŋesi  a  namɪã  raa \\
          {\sc 2.sg} {\sc ipfv.fut} {cut.{\sc pv}} {\sc art} {meat} {\sc foc}\\
\glt  `Will you (be) cut(ting) the meat (into pieces)?'\\

 \end{xlist}
\end{exe}

The only distinction perceived between (\ref{GRM-pfv-inter-pfv})  and
(\ref{GRM-pfv-inter-impf}) is a pitch difference on the third syllable of the
verb. The tonal melody associated with the verb in 
(\ref{GRM-pfv-inter-impf-fut}) is the
same as the one in (\ref{GRM-pfv-inter-impf}).







\paragraph{Ex-situ subject imperfective particle}
\label{sec:GRM-ipfv-part}

One topic-marking strategy is to prepose a non-subject constituent to the
beginning of the clause.  In  (\ref{ex:GRM-foc-top}),  the focus particle may or
may not
appear after the non-subjectival topic. Notice that one effect of this 
topic-marking strategy is that the particle {\S dɪ} appears between the subject
and
the verb when the non-subject constituent is preposed and when the clause is
used to describe what is happening at the moment of speech. This particular
relation between  assertion, aspect and linear order is a phenomenon
which is still obscure.



\begin{exe}
\ex\label{ex:GRM-foc-top}
\begin{xlist}
 \ex\label{ex:GRM-foc-top-chew-pres.prog}{\it Imperfective}
\gll  sɪ́gá (ra)  ʊ̀ dɪ̀  tíè   \\
 bean  ({\foc}) {3\sg} {\ipfv} chew\\
\glt `It is BEANS he is chewing'


 \ex\label{ex:GRM-foc-top-chew-pres.prog}{\it Perfective}
\gll  sɪ́gá (ra) ʊ̀   tìè     \\
 bean  ({\foc}) {3\sg}  chew \\
\glt `It is BEANS he chewed'

 \ex\label{ex:GRM-foc-top-go-pres.prog}{\it Imperfective}
\gll   wáá (ra) ʊ̀ dɪ̀  káálɪ̀   \\
Wa    ({\foc}) {3\sg} {\ipfv} go\\
\glt `It is to WA that he is going'


 \ex\label{ex:GRM-foc-top-go-pres.prog}{\it Perfective}
\gll   wáá (ra)  ʊ̀ kààlɪ̀    \\
Wa   ({\foc}) {3\sg}  go\\
\glt `It is to WA that he went'

\end{xlist}
\end{exe}

The position of {\S dɪ} in  (\ref{ex:GRM-foc-top-chew-pres.prog}) and 
 (\ref{ex:GRM-foc-top-go-pres.prog}), that is between the subject and the verb,
is generally occupied by linguistic items called  {\it preverbs},  to which the
discussion turns next.  Provisionally, the particle {\S dɪ} may be treated as a
preverb constrained to occur with  a preposed  non-subject constituent and
an imperfective aspect.\footnote{I do not treat topicalization in this work,
although the left-dislocation strategy in (\ref{ex:GRM-foc-top}) is the only
one I know to exist.}

\subsection{Preverb particles}
\label{sec:GRM-precerv}

Preverb particles are grammatical morphemes which encode various event-related
meanings. They are part of the verbal domain  called the expanded verbal group
(EVG), which  consist of  one or more preverbs.  These grammatical morphemes are
not verbs, in the sense that they do not contribute to SVCs as verbs do,  but as
`auxiliaries'. Still,  some of the preverbs may historically derive from verbs, 
and  some others may synchronically function as verbs.  Examples of the latter
are the egressive particle {\S ka} and ingressive particle {\S wa},  which are
discussed first. 


Nevertheless, given the data available,  it would not be incorrect to analyze
some of the preverbs  either as additional SVC verbs or as adverbs. Despite
these analytical options, the domain which follows the subject and precedes the
main verb(s) is generally accessible  only to a limited set of linguistic items.
For that reason, the expanded verbal group is a domain whose members are
identified as preverbs. However, we will see  that a preverb differs from a verb
in that it exposes functional categories,  cannot inflect for the perfective or
assertive suffix,  and never takes  a complement, such as a grammatical object,
or be modified by an oblique or an adjunct. But again,  a first verb in a SVC
and a preverb are categories which are hard to distinguish. Structurally and
functionally, many of them may be analysed as grammaticalized serial verbs.
These characteristics are not special to Chakali; similar, but not identical,
behavior is described for Gã and Gurene \citep{Daku07b, Daku08}.



\subsubsection{Egressive and ingressive particles}
\label{sec:GRM-EVC-egr-ingr}


The egressive particle {\S ka(a)} ({\it gl.} {\sc egr})   `movement away from
the
deictic centre'  and   the ingressive
particle {\S wa(a)} ({\it gl.} {\sc ingr})  `movement towards
the deictic centre' are  assumed to derive from the  verbs
{\S kaalɪ} `go' and  {\S waa} `come'.\footnote{A discussion on some aspects of
grammaticalization of  `come' and `go' can be read in  \cite{Bour92}. In the
literature, egressive  is also known as  {\it itive} (i.e. away from the
speakers,  `thither')  and  ingressive  is  known as {\it ventive} (i.e. towards
the speakers,   `hither'). }  Table
\ref{tab:deict-pre-verb} shows that  {\S kaalɪ} `go' and {\S waa} `come',  like 
other verbs, change forms (and are acceptable) in these paradigms,  but {\S
ka(a)} `go' is  not.


\begin{table}[h]
\centering
\caption{Deictic verbs and preverbs \label{tab:deict-pre-verb}}

\begin{Itabular}{lllll}
\Hline
Verb & $\sigma$  & Aspect & Positive & Negative\\[1ex] \hline


{\I waa} `come' & CV 	& {\sc pfv} 	&  ʊ waawaʊ   & ʊ wa waawa\\
					&&& `she came' & `she didn't come'\\

 		  & 	& {\sc ipfv} 	&  ʊʊ waaʊ  & ʊ wa waa\\
					&&& `she is coming' & `she is not
coming'\\[1ex] \hline




{\I kaalɪ} `go' & CVVCV 	& {\sc pfv} 	&  ʊ kaalɪjʊ   & ʊ wa kaalɪjɛ\\
					&&& `she went' & `she didn't go'\\

 		  & 	& {\sc ipfv} 	&  ʊʊ kaalʊʊ  & ʊ wa kaalɪ\\
					&&& `she is going' & `she is not
going'\\[1ex] \hline


{\I ka} `go' & CV 	& {\sc pfv} 	&  *ʊ kaʊ   & *ʊ wa kajɛ\\
				

 		  & 	& {\sc ipfv} 	&  *ʊ kaʊ  & *ʊ waa ka\\
				
\Hline

 

\end{Itabular}         
\end{table}


If the verbs {\S kaalɪ} `go' and  {\S waa} `come'
occur in a SVC,  they surface as {\S ka} and {\S wa} respectively. In
(\ref{GRM-prev}),  both verbs take part in  a two-verb SVC in which they are
first in the sequence.


\begin{exe}
\ex\label{GRM-prev}
\begin{xlist}

\ex\label{GRM-prev-SVC-ka}
\glll gbɪ̃̀ã́         	bààŋ       	té    	kà         	sáŋá 	à   
píé  {(...)} \\
monkey  	 quickly 	early 	go 		sit  	{\art}
yam.mound.pl   {(...)} \\
{} [[{\it pv} {\it pv}]_{EVG} {\it v} {\it v}]_{VP} {} {}
{(...)}\\
\glt `Monkey quickly went and sat on the (eighth) yam mound (...)'  (LB 012)

\ex\label{GRM-prev-SVC-wa}
\glll    ŋmɛ́ŋtɛ́l   làà nʊ̀ã̀  nɪ́    ká  ŋmá dɪ́    ʊ̀  wá  
ɲʊ̃̀ã̀ nɪ́ɪ́ \\
spider collect mouth {\postp}  {\conn} say {\comp}  
{\sc 3.sg}  come   drink water\\
 {} {} {} {} {}  {} {} {} {\it v} {\it v} {} \\
\glt  `(Monkey went to spider's farm to greet him.)  Spider accepted
(the
greetings) and (Spider) asked him (Monkey) to come and drink water.'  (LB 011)

 \end{xlist}
\end{exe}


Because they derive from deictic verbs (historically or synchronically),  the
preverbs  indicate non-spatial `event movement'  to or from a deictic centre.
This phenomenon is not uncommon cross-linguistically. \citet[62]{Nico07}
maintains  that when a movement verb becomes a tense marker, it may be reduced
to a verbal affix and its meaning can develop ``into meaning relating temporal
relations between events and reference times''. In Chakali, the  preverb {\S
kaa} contributes   temporal information to an expression. Consider in
(\ref{exe:GRM-crack-remove-attach}) the distribution and contribution of  {\S
kaa} to  the clauses headed by the verbs {\S kpe} `crack a shell and remove a
seed from it' (c\&r) and {\S mara} `attach'.\footnote{In Gurene (Oti-Volta), it
is the ingressive particle which has a similar role. The ingressive  is 
commonly used before the verb, and can, among other things,  express future
tense \citep[see][59]{Daku07b}.}



\begin{exe}
\ex\label{exe:GRM-crack-remove-attach}
%\begin{multicols}{2}
\begin{xlist}
\ex

 ʊ̀ kàá kpéù   `She will c\&r'\\
   ʊ̀ʊ̀ kpéú   	 `She  is c-ing\&r-ing/c-s\&r-s'\\
   ʊ̀ kpéjòò   `She   c-ed\&r-ed'\\
   kpé  		 `C\&r!'
\ex
 ʊ̀ kàá māràʊ̀   `She will attach'\\
   ʊ̀ʊ̀ máràʊ̀   	 `She  is attaching/attaches'\\
  ʊ̀ márɪ̄jʊ̀    `She   attached'\\
   márá		 `Attach'
\end{xlist}
%\end{multicols}
\end{exe}


The preverb particle {\S kaa} can also be used to express that an event is
ongoing at the moment of speech, which I call the present 
progressive.   However,  when it is used to describe what is happening
now, {\S kaa} can only appear when the subject is not a pronoun and its tone
melody differs from that of the future tense. These contrasts are given in
(\ref{exe:GRM-kaa-attach}).

\begin{exe}
\ex\label{exe:GRM-kaa-attach}
 ʊ̀ kàá márāʊ̀   `She will attach'\\
   ʊ̀ʊ̀ márāʊ̀   	 `She  is attaching'\\
wʊ̀sá kàá márāʊ̀   `Wusa will attach'\\
wʊ̀sá káá márāʊ̀   `Wusa is attaching'\\
\textasteriskcentered  wʊsa   maraʊ  	  `Wusa is
attaching'
\end{exe}

Paradigm  (\ref{exe:GRM-kaa-attach}) shows that when the preverb particle {\S
kaa} appears with a rising tonal melody it  expresses the future tense, but  in
order to convey that a situation is ongoing at the time of speech (i.e. present
progressive), the preverb particle {\S kaa} has a high tone. Thus, it is the
tonal melody on {\S kaa} which distinguishes between the future and the present
progressive (both treated as imperfective),  plus the fact that pronouns cannot
co-occur with the preverb particle {\S kaa} in the present progressive. 



Although little evidence is available, the preverb {\S wa} may also be used to
express a sort of hypothetical  mood.  In  (\ref{ex:GRM-prev-wa-hypo}), the
preverb {\S wa} should be seen as contributing a supposition, or a hypothetical
circumstance where
someone would be found calling the number 8.


\begin{exe}
\ex\label{ex:GRM-prev-wa-hypo}

\gll ŋmɛ́ŋtɛ́l   ŋmá    dɪ́,    kɔ̀sánáɔ̃̀, 	tɔ́ʊ́tɪ̀ɪ̀nà  ŋmá 
dɪ́,  	námùŋ   wá    jɪ̀rà ŋmɛ́ŋtɛ́l sɔ́ŋ,  	bá  kpágʊ́ʊ̀    wà 
bà kpʊ́\\
spider     say   {\comp}   buffalo  	land.owner say {\comp}  anyone   
{\ingr}   call eight     name {\sc 3.pl.hum+}    catch.{\sc 3.sg} {\foc}
 {\sc 3.pl.hum+}  kill\\
\glt `Spider told Buffalo that landowner said anyone who calls the number 8
should be brought to him to be killed.' (LB 009)
\end{exe}



Finally, the example in (\ref{ex:GRM-verb-ta})  intends to show that some
elders of Ducie and
Gurumbele use  {\S ta}  instead of {\S ka(a)},  as a variant of the
preverb.\footnote{I gathered that  (i)   {\F ta} is not a different
preverb (Gurene is said to have  a preverb {\F ta}  signifying intentional
action, (M.E.K. Dakubu, p.c.)),  and  (ii)   {\F ta} can be heard in  Ducie and
Gurumbele from people of the oldest generation, but somebody suggested to me
that {\F ta} is the common form in Motigu (Mba Zien, p.c.).  This distinction
is an issue in need of further research. } 
 

\begin{exe}
   \ex\label{ex:GRM-verb-ta}{\it Priest talking to the shrine, holding a kola
nut above it}

\gll  ma laa kapʊsɪɛ haŋ ka ja mɔsɛ tɪɛ wɪɪ tɪŋ ba \underline{ta} buure\\
{\sc 2.pl} take kola.nut {\sc dem} {\sc conn} {\sc 1.pl} plead give matter {\sc
art} {\sc 3.pl.}b {\sc  egr} want\\
\glt   `Take this kola nut, we implore  you to give them what they desire.'

\end{exe}


Unfortunately, since the relation between tense, aspect and tonal melody is not
well-understood at this stage of research, the  egressive {\S ka}   and the
ingressive  {\S wa} are  broadly glossed as {\egr} and {\ingr} respectively, but
can also be associated with composite glosses such as {\ipfv .\fut} or  {\ipfv
.\pres}  in cases where a distinction is clear.




\subsubsection{Negation preverb}
\label{sec:GRM-verb-neg}
%check negative concord with nobodu, no one, all, nothing


%  Their
% lengths may vary depending on the speech rate, but  they are always long
% in 

There are three different particles of negation in the language:  the forms {\S
lɛɪ} and {\S tɪ}   were discussed in section   \ref{sec:GRM-foc-neg} and
\ref{sec:GRM-imper-clause} respectively.  The negative preverb particle {\S
wa(a)} precedes the verb and is used in the verbal group (in non-imperative
mood). The same form is found in both  main and dependent clauses. Notice that a
tonal quality on the negation particle and following verb  distinguishes
between the present
progressive and  the future,  exactly as the preverb {\S kaa} does.  Consider 
paradigm (\ref{ex:GRM-neg-pres-fut}). 

\begin{exe}
\ex\label{ex:GRM-neg-pres-fut}
\begin{xlist}
\ex
\gll ʊ̀  wàá pɛ̀ \\
   {\sc 3.sg}  {\neg} add\\
\glt  `She will not add.'

 \ex 
\gll  ʊ̀ʊ̀ wàà pɛ́\\
     {\sc 3.sg} {\neg} add \\
\glt  `She is not adding.'


 \ex 
\gll  ʊ̀ wà pɛ́jɛ̀\\
     {\sc 3.sg} {\neg} add \\
\glt  `She didn't  add.'
\end{xlist}
\end{exe}


When the negation particle {\S wa(a)} and a quantifier appear in the same clause
the quantifier is  in the positive. This is shown in (\ref{ex:neg-quant-any}).

\begin{exe}
\ex\label{ex:neg-quant-any}
 \begin{xlist}
  
\ex\label{ex:vp2.6}
\gll namuŋ wa na-ŋ \\
 {\clf}.all {\neg} see-{1.\sg}\\
\glt  `Nobody saw me.' ({\it lit.} everyone not see me) 

\ex\label{ex:vp2.5}
\gll  n̩ wa na namuŋ  \\
  {1.\sg}  {\neg}   see  {\clf}.all\\
\glt  `I did not see anyone.' ({\it lit.} I not see everyone) 
\end{xlist}
\end{exe}


The negative preverb is often hard to distinguish from the verb  {\S
waa} `come',  even though the former always precedes the latter. Length (CV or
CVV) is especially hard to differentiate in normal speech. Examples
(\ref{ex:GRM-neg-come}) suggest that the tonal melody and length  indicate
meaning differences.


\begin{exe}
\ex\label{ex:GRM-neg-come}
 \begin{xlist}
  
\ex\label{}
\gll ʊ̀ wà wáá dì\\
{\sc 3.sg} {\neg} come eat\\
\glt `She did not come to eat.'
\ex\label{}
\gll ʊ̀ wàá wàà dí\\
{\sc 3.sg} {\neg} come eat \\
\glt `She will not come to eat.'
\end{xlist}
\end{exe}

Assertion and negation seem to avoid one another and constrain the grammar  in
the following way:  {\it If a clause is negated,  none of its constituents can
be in focus.} In section \ref{sec:GRM-personal-pronouns},  it was shown that (i)
negation cannot co-occur with the strong pronouns, and (ii) negation cannot
co-occur with an argument of the predicate in focus, i.e. with {\S ra} or one of
its variants having scope over the noun phrase. The third non-occurrence of
negation concerns  the assertive form of the verb (see section
\ref{sc:GRM-focus})  Consider the forms of the verb {\S mara} `attach' in the
two paradigms in (\ref{ex:GRM-verb-neg-foc}).\footnote{The particle {\F la} in
Dagaare has similar constraints. \citet[94]{Bodo97} calls it a {\it factitve}
particle.}

\begin{exe}
   \ex\label{ex:GRM-verb-neg-foc}
\begin{xlist}
   \ex\label{ex:GRM-verb-neg-foc-pos}{\it Positive}

 ʊ̀ kàá māràʊ̀   `She will attach'\\
  ʊ̀ʊ̀ máràʊ̀     	 `She  is attaching/attaches'\\
  ʊ̀ márɪ̄jʊ̀       `She   attached'\\
  \ex\label{ex:GRM-verb-neg-foc-neg}{\it Negative}

 ʊ̀ wàá màrà  `She will not attach'\\
   ʊ̀ʊ̀ wàà márá  	 `She  is  not attaching/does not attach'\\
  ʊ̀ wà márɪ̄jɛ̀   `She   did not attach'\\

\end{xlist}
\end{exe}

The paradigms in (\ref{ex:GRM-verb-neg-foc})  suggest
that the negation particle and any particles of assertion are in complementary
distribution. 





\subsubsection{Tense preverbs}
\label{sec:GRM-verb-neg}


\paragraph{fɪ}

The preverb {\S fɪ}   is identified with two different but interrelated
meanings.  First, the preverb {\S fɪ}  ({\it gl.} {\sc pst}) is a neutral
past tense particle (i.e.  as opposed to {\S dɪ} in  section
\ref{sec:GRM-preverb-three-int-tense} which is specific), and the event referred
to in the past can no longer be in effect in the present.

\begin{exe} 
\ex\label{ex-preverb-fi-neut}
\begin{xlist}
\ex
\gll ʊ̀ jáá  ǹ̩ títʃà rà \\
  {\sc 3.sg} {\ident}    {\sc 3.sg.poss}  teacher {\foc}  \\
\glt  `He is my TEACHER.' 

\ex
\gll   ʊ̀  fɪ̀ jáá  ǹ̩ títʃà rà\\
        {\sc 3.sg} {\pst} {\ident}    {\sc 3.sg.poss}  teacher {\foc}  \\  
\glt  `He was my TEACHER.' 

\end{xlist}
\end{exe} 

 Secondly, the preverb {\S fɪ}   ({\it gl.} {\sc mod}) can have  deontic
meaning.  In (\ref{ex-preverb-fi-deonc}),  its presence still conveys  past
tense, but in addition it expresses that the situation did not really occur, yet
it was objectively supposed to occur or subjectively expected to occur or
awaited. The lengthening of the preverb {\S fɪ} in the positive  is not
accounted for, but I suspect it  signals the imperfective. Compare the first two
sentences in (\ref{ex-preverb-fi-deonc}) with the last two  which convey the
neutral past. 


\begin{exe} 
\ex\label{ex-preverb-fi-deonc}
\begin{xlist}
\ex\label{ex-preverb-fi-deonc-pos}
\gll ʊ̀ fɪ́ɪ́ jàà  ǹ̩ títʃà rà \\
  {\sc 3.sg}  {\mod}  {\ident}    {\sc 3.sg.poss}  teacher {\foc}  \\
\glt  `He should have been my TEACHER.' 

\ex
\gll    ʊ̀ fɪ̀ wáá jàà  ǹ̩ títʃà \\
        {\sc 3.sg} {\mod} {\neg} {\ident}    {\sc 3.sg.poss}  teacher   \\  
\glt   `He should not have been my teacher.'  


\ex
 ʊ̀  fɪ̀ jáá  ǹ̩ títʃà rà `He was my teacher.'
\ex
 ʊ̀  fɪ̀ wà jáá  ǹ̩ títʃà`He was not my teacher.'
\end{xlist}
\end{exe} 

The positive sentence in (\ref{ex-preverb-fi-deonc-pos}) can receive  a
translation along these lines:  In a desirable possible world, he was my
teacher, but it is not what happened in
the real world. 

\begin{exe} 
\ex\label{ex-preverb-fi-pure-deonc}
\begin{xlist}
\ex\label{ex:GRM-vp11.2}
\gll m̩̀ mɪ̀bʊ̀à fɪ́  bɪ́rgɪ̀ \\
    {\sc 1.sg.poss} life    {\mod}  delay    \\
\glt  `May I live long!' 

\ex\label{ex:GRM-vp11.3}
\gll tɪ́ɛ́ m̩̀ mɪ̀bʊ̀à bɪ́rgɪ̀ \\
      give {\sc 1.sg.poss} life delay  \\
\glt  `Let me live long!' 
\end{xlist}
\end{exe} 

Finally, the preverb {\S fɪ}  in (\ref{ex-preverb-fi-pure-deonc}) still conveys
 deontic modality, where the speaker prays or asks permission for a 
situation. Notice, however,  that it cannot refer to a past event. The two
sentences
in (\ref{ex-preverb-fi-pure-deonc}) have a corresponding meaning. Example
(\ref{ex:GRM-vp11.3}) is framed in an imperative clause (see optative in section
\ref{sec:GRM-imper-clause}). 


\paragraph{Preverb three-interval tense}
\label{sec:GRM-preverb-three-int-tense}

Chakali encodes  in  preverbs  a type of
time categorization  known as three-interval tense  \citep[366]{Fraw92}. It is
possible to express that an event occurred specifically yesterday, as opposed to
earlier today and the day before yesterday, i.e. {\it hesternal tense}
({\it gl.} {\sc hest}), or specifically tomorrow, as opposed to later today and
the
day after tomorrow, i.e. {\it
crastinal tense}  ({\it gl.}  {\sc cras}). 

The hesternal tense particle {\S dɪ}/{\S
de} ({\it gl.} {\sc hest})  refers to the day
preceding the speech time.  It has the time adverbial counterpart  {\S dɪare
(tɪŋ)} `yesterday'.  In (\ref{ex:vp2.11.a}) the adverbial phrase {\S dɪare tɪn}
`yesterday' is optional,  and  when it is used it must be expressed at the end
or
at the beginning of
the clause.


\begin{exe} 
\ex\label{ex:vp2.11.a} 
\gll {(dɪare tɪn)} ʊ nɪ ʊ tʃɛna dɪ waawa  {(dɪare tɪn)}\\
{(yesterday)}    {\sc 3.sg} {\sc conn} {\sc 3.sg.poss} friend
{\sc hest}  come.{\sc pfv} {(yesterday)} \\ 
\glt  `He arrived with his friend yesterday.'
 \end{exe}


The crastinal tense preverb {\S tʃɪ} ({\it gl.} {\cras})  in  (\ref{ex:vp4.5})
functions as future particle,  but is limited to the day following the event
time.
In that sentence the event time referred to follows  the utterance
time by one day.  The time
adverbial counterpart  of {\S tʃɪ} is {\S  tʃɪa} `tomorrow'. As
for the hesternal tense and the corresponding adverbial,  the  adverbial may or
may not co-occur with the crastinal tense particle. 


\begin{exe} 
\ex\label{ex:vp4.5} {\it Will you work for the chief today or tomorrow?}
\gll  n̩ tʃɪ ka tʊma tɪɛʊ ra, zaaŋ,  n̩ kaa hɪɛ̃sʊ \\
    {\sc 1.sg} {\sc cras}  go  work give.{\sc 3.sg} {\sc foc},
today,   {\sc 1.sg}  {\sc egr} rest.{\sc foc} \\
\glt  `I shall work for
him tomorrow, today,  I shall rest.' 
 \end{exe}


The hesternal tense particle {\S dɪ} is homophonous with the ({\it ex-situ
subject}) imperfective particle  {\S dɪ} discussed in section
\ref{sec:GRM-ipfv-part}.  In addition, the question arises as to whether the
crastinal tense  is inherently future, and if so, whether or not it can
co-occur with the egressive preverb discussed in section
\ref{sec:GRM-EVC-egr-ingr}. Consider their distribution and meaning in the
examples given in (\ref{ex:GRM-prev-dist}).

\begin{exe} 
\ex\label{ex:GRM-prev-dist}
\begin{xlist}
\ex\label{ex:GRM-prev-dist-chew-presprog}{\it Present progressive}
\gll  sɪ́gá (ra)  ʊ̀ dɪ̀  tíè   \\
 bean  ({\foc}) {3\sg} {\ipfv} chew\\
\glt `It is BEANS he is chewing'

 \ex\label{ex:GRM-prev-dist-chew-past}{\it Past}
\gll  sɪ́gá (ra) ʊ̀   tìè     \\
 bean  ({\foc}) {3\sg}  chew \\
\glt `It is BEANS he chewed'


 \ex\label{ex:GRM-prev-dist-chew-past}{\it Hesternal past}
\gll  sɪ́gá (ra) ʊ̀ dɪ́    tìè     \\
 bean  ({\foc}) {3\sg} {\hest}  chew \\
\glt `It is BEANS he chewed yesterday'


 \ex\label{ex:GRM-prev-dist-chew-past}{\it Hesternal past progressive}
\gll  sɪ́gá (ra) ʊ̀ dɪ́ɪ́    tìè     \\
 bean  ({\foc}) {3\sg} {\hest}  chew \\
\glt `It is BEANS he was chewing yesterday'

 \ex\label{ex:GRM-prev-dist-chew-futprog}{\it Future (progressive)}
\gll  sɪ́gá (ra) ʊ̀  kàá   tíè     \\
 bean  ({\foc}) {3\sg} {\fut}  chew \\
\glt `It is BEANS he will be chewing / will chew'

 \ex\label{ex:GRM-foc-top-chew-crasfutprog}{\it Crastinal future (progressive)}
\gll  sɪ́gá (ra) ʊ̀ tʃɪ́  kàá   tíè     \\
 bean  ({\foc}) {3\sg} {\cras} {\fut}   chew \\
\glt `It is BEANS he will be chewing / will chew tomorrow '

\end{xlist}
\end{exe} 

A specific tonal melody associated with  the sequence {\S dɪ tie} can express
either a present progressive (\ref{ex:GRM-prev-dist-chew-presprog}) or a
hesternal past (\ref{ex:GRM-prev-dist-chew-past}). Lengthening the hesternal
past particle allows one to express the tense associated with the particle, in
addition to indicating  progressive (\ref{ex:GRM-prev-dist-chew-past}). This
strategy seems to correspond semantically  to the syntactically anomalous
*{\S dɪ dɪ},  {\it lit.} {\sc hest} {\sc ipfv}.  The example in
(\ref{ex:GRM-foc-top-chew-crasfutprog}) shows that the crastinal tense particle
and the egressive particle signaling  future  can co-occur. Notice that
inserting the imperfective particle {\S  dɪ} between the egressive particle and
the verb in  (\ref{ex:GRM-prev-dist-chew-futprog}) and
(\ref{ex:GRM-foc-top-chew-crasfutprog}) is  unacceptable. It is unclear whether
these two
examples must be interpreted as progressive or not.  



\paragraph{te}
\label{sec:GRM-preverb-te}

Lacking a corresponding verb to capture its meaning, the verb {\S te} is glossed
with
the English adverb `early'. Even though  it is attested as main verb,  {\S te}
can  function  as a preverb and it is indeed more common to find it in that
function.  It contributes a manner, one in which the
event is carried out before the expected or usual time.  It cannot be used
otherwise as an adverb, i.e. in a canonical adjunct position (section
\ref{sec:GRM-adjuncts}).  The main verb {\S te}
and the preverb {\S te} are shown respectively in  (\ref{ex:GRM-prev-early})
and  (\ref{GRM-prev-SVC-ka}), which are repeated below.


\begin{exe}
 \ex\label{ex:GRM-prev-early}
\gll  ɪ̀ téjòò\\
     {\sc 2.sg} early.{\foc}   \\
\glt  `You are early.'
\end{exe} 

\begin{exe}
\exp{GRM-prev-SVC-ka}
\glll gbɪ̃̀ã́         	bààŋ       	té    	kà         	sáŋá 	à   
píé  {(...)}\\
monkey  	quickly 	early 	go 		sit  	{\art}
yam.mound.{\pl}   {(...)} \\
{} {\it pv} {\it pv}  {\it v} {\it v} {}  {} {} \\
\glt `Monkey quickly went and sat on the (eighth) yam mound (...)'  (LB 012)
\end{exe} 



\paragraph{zɪ}
\label{sec:GRM-preverb-after-then}

The preverb {\S zɪ} is marginal in the corpus.\footnote{There is another
similar particle, {\F ze}  ({\it gl.} {\sc exp}),  which I still do not
understand: (i) it occurs after the noun phrase, and  (ii) its meaning
corresponds to
 `expected (by both the speaker and the
hearer, or only by the speaker)'. It informs that the referent of
the noun phrase was anticipated before the utterance time (or relative time) by
the speaker and hearer (or only the speaker).  Consider  the following
example:

\begin{exe}
\sn[]{ 
 \gll ba ze  waawaʊ \\
{\sc 3.pl.b} {\sc  exp} come.{\sc pfv}\\
`They (the expected people) have come.'}
\end{exe}
} There is no corresponding
verb in the language.   It is used when an event is seen in
succession to another, as (\ref{ex:GRM-prev-zi-1}) shows. However,  as
(\ref{ex:GRM-prev-zi-2}) illustrates,  the preceding event may be presupposed, 
so  it is not necessarily uttered.


\begin{exe}
\ex
\begin{xlist}
\ex\label{ex:GRM-prev-zi-1}  {\it A father is giving a sequence of tasks to
his son}

 \glll tʊma  a  zɪɛ̃  mʊã  ka  ka  tʊma  kuo   aka   zɪ ka  tʊma a  gar  \\
  {work} \textsc{art}    {wall} {before}  \textsc{conn} {\sc egr}    {work}
{farm} 
\textsc{conn} {after}  {go} {work} \textsc{art} {cattle.fence}\\  
{} {} {} {} {}  {} {} {} {} {\it pv} {\it v} {\it v} {} {}\\
\glt  `First repair the wall, then go and farm, then repair the cattle fence.'

 \ex\label{ex:GRM-prev-zi-2}
  \glll {(kaalɪ dɪa)} zɪ́ kààlɪ̀ kùó\\
{go  house} then go farm\\
{} {\it pv} {\it v} {}\\
 \glt `(Go to the house and) Then go to the farm.'

\end{xlist}
\end{exe}


\subsubsection{Miscellaneous elements in the EVC}
\label{sec:GRM-preverb-misc}


In this section, the words  {\S baaŋ} `must',   {\S bɪ}
`again',  {\S bra} `return',   {\S ja} `do',  and {\S ha} `yet'  are described
as preverbs.  A basic overview of their meanings and
distributions is given.  

\paragraph{baaŋ}
\label{sec:GRM-preverb-baang}

 The preverb  {\S baaŋ}  ({\it gl.}
{\sc mod})  is primarily modal and is  translated with 
`must', `immediately', `quickly'  or `just'.  First,   as the examples in
(\ref{sec:GRM-prev-bg-must}) show,  {\S baaŋ} conveys an obligation and the
notion of temporality is secondary. 


\begin{exe}
\ex\label{sec:GRM-prev-bg-must}
\begin{xlist}
          
\ex\label{ex:GRM-7.17}
\gll  kuoru ŋma dɪ n̩ ka baaŋ bɔ bʊ̃ʊ̃na  fi re \\
 chief say {\comp} {\sc 1.sg} {\egr} {\mod}   pay  goat.{\pl} ten {\sc foc} \\
\glt  `The chief says that I must pay him ten goats.' 

\ex\label{ex:GRM-14.3}
\gll  ɪ ka baaŋ jaʊ ra\\
{\sc 2.sg} {\egr} {\mod} do.{\sc 3.sg} {\foc}\\
\glt  `You must do it.'

\end{xlist}
 \end{exe}
 

Secondly, the preverb  {\S baaŋ} can express an  abrupt or
swift   manner.

\begin{exe}
\ex\label{sec:GRM-prev-bg-time}
\glll   {(...)} a kpa ʊ neŋ a saga ʊ nɪ dɪ ʊ baaŋ te bɛrɛgɪ dʊ̃ʊ̃\\
     {(...})  {\conn}  take {\sc 3.sg.poss} arm {\conn} {be.on} {\sc
3.sg}  {\postp} {\conn} {\sc 3.sg} {\mod} {early} turn.into python \\
{} {} {} {} {} {} {} {} {} {} {} {\it pv} {\it pv} {\it v} {} \\

\glt  `(...) then put his hand on her  and quickly turned into
a python.' 
%(Pyhton story 025)
\end{exe}
      
Finally, the preverb  {\S baaŋ} may act as a discourse particle used mainly to
emphasize or intensify the action carried out, reminiscent of  the use of 
`just' in
some English registers.  It is often translated in text as `immediately',
`suddenly', `then',  or simply `just'. Some excerpts from a folk tale are
given in
(\ref{ex:GRM-prev-bg-excerpt}).


%example from python story

%  
 \begin{exe}
\ex\label{ex:GRM-prev-bg-excerpt}
\begin{xlist}
\ex
\gll kawa baaŋ tarɪ keeeeŋ \\
pumpkin just creep {\advm}\\
\glt `A pumpkin just crept like that ...' 

\ex
\gll ʊ baaŋ tɪŋaʊ \\
{\sc 3.sg} just follow.{\sc 3.sg}\\
 \glt `She just followed it ...'

\ex
\gll ʊ baaŋ jɪraʊ \\
{\sc 3.sg} just call.{\sc 3.sg}\\
 \glt `She then called her ...'

\ex
\gll dɪ mãã tɪŋ baaŋ ŋma nɪŋ mmmm\\
{\comp} mother {\art} just say {\advm} mmmm\\
\glt `That the mother just said like ``mmmm'' ...'

\ex
\gll diŋ baaŋ jaa tʊl\\
fire  just {\ident} flame\\
\glt `The fire suddenly became flame.'
\end{xlist}
 \end{exe}
% 



%desiderative mood ŋma



\paragraph{bɪ}
\label{sec:GRM-preverb-iteration}


The examples in (\ref{ex:GRM-prev-bi}) show that  the preverb particle {\S bɪ} 
expresses iteration, but also the single repetition of an event,  and follows
the negation particle. 


% bɪ kuor ŋma
%  repeat
%  bɪ pɪlɪ
% start again
% start
\begin{exe} 

\ex\label{ex:GRM-prev-bi}
\begin{xlist}
\ex\label{ex:vp33.2.}
\gll ʊ bɪ kʊɔrɛ sãã ʊ dɪa ra \\
 {\sc 1.sg}     {\itr} make build {\sc 3.sg.poss} house {\foc}    \\
\glt  `He rebuilt his hut' 


\ex\label{ex:GRM-vp10.4}
\gll a bitʃelii bɪ siiu\\
 {\art}  child.fall   {\itr} raise.{\foc}    \\
\glt  `The fallen child got up again.' 



\ex\label{ex:vp10.4.}
\gll ʊ wa bɪ tuwo \\
       {3.\sg} {\neg} {\itr} be.at.{\neg} \\
\glt  `She is not here again.' 
\end{xlist}
\end{exe} 


Unlike other preverbs,  {\S bɪ} may also
appear within noun phrases. This is shown in (\ref{ex:GRM-vp19.2.})  (see also
section \ref{sec:NUM-repet}).



\begin{exe} 
\ex\label{ex:GRM-vp19.2.}
\gll  n̩ ja  kaalɪ ʊ pe re tʃɔpɪsɪ bɪ-muŋ \\
{\sc 1.sg} {\hab} go {\sc 3.sg.poss} end {\foc}  day.break {\itr}-all\\
\glt  `I do visit him every day.' 

\end{exe} 



 \paragraph{bra}
\label{sec:GRM-preverb-return}

The preverb {\S bra} has a corresponding verb with the same form. It is
primarily a motion verb which conveys a change of direction. The examples 
in (\ref{ex:GRM-verb-bra}) present the verb {\S bra} in imperative clauses
separated by the connectives {\S a} and {\S aka}.


\begin{exe}
\ex\label{ex:GRM-verb-bra}
\begin{xlist}
\ex
\gll brà à káálɪ̀\\
return {\conn} go\\
\glt `Go back.'

\ex
\gll brà àká tʃáʊ̀\\
return {\conn} leave.{\sc 3.sg}\\
\glt `Return and leave him.'
\end{xlist}
\end{exe}


When {\S bra} functions as a preverb, it loosely keeps its sense of motion and
conveys in addition a sort of repetition. It differs from the morpheme {\S bɪ}
introduced in
 section \ref{sec:GRM-preverb-iteration} because it does not mean that an
action is
done
repeatedly.  Instead, the preverb {\S bra} is associated with actions done `once
more', `over again',  or `anew'.


\begin{exe}
\ex\label{ex:vp33.1.}
\gll ʊ bra tʊma a tʊma tɪŋ ka wa wire keŋ \\
 {\sc 3.sg}  {again}  {work} {\art} {work}   {\art} {\egr} {\neg} well {\advm}\\
\glt  `He redid the work that was 
 badly done.'
\end{exe}





\paragraph{ja}
\label{sec:GRM-preverb-hab}

The preverb {\S ja} ({\it gl.} {\sc hab})  indicates habitual aspect. It may be
argued that the
particle derives from the  verb {\S ja} `do'. Example
(\ref{ex:GRM-prev-hab-do}) shows that the preverb {\S ja} and the verb {\S ja}
can
co-occur. 


\begin{exe}
\ex\label{ex:GRM-prev-hab}
\begin{xlist}

\ex\label{ex:GRM-prev-hab-do}
\gll tʃɔpɪsɪ bɪ-muŋ ʊ ja jaʊ \\
 day.break {\itr}-all {\sc 3.sg} {\hab} do.{\sc 3.sg}\\
\glt `He does it every day.'

\ex
\gll taŋu ja tie ger re\\
T.  {\hab} chew lizard {\foc}\\
\glt `Tangu do eat lizard.'

\ex
\gll jʊʊ nɪ dʊɔŋ ja waaʊ\\
rainy.season {\postp} rain  {\hab} come.{\foc} \\
\glt `During the rainy season, it rains.'
\end{xlist}
\end{exe}





\paragraph{ha}
\label{sec:GRM-preverb-yet}

The morpheme {\S ha} ({\it gl.} {\sc mod}) is similar in meaning to the English
morpheme `yet'. It is used when an event is or was anticipated and a speaker
considers or considered probable the occurence of the event. As  for the English
`yet', it is frequently found in negative polarity. In such cases the morpheme
{\S ha} indicates that the event is expected to happen and the negative marker
{\S wa} indicates that the event has not unfolded or happened at the referred
time. In the cases where {\S ha} is found in a positive polarity,  it  conveys 
a continuative aspect, similar to English `still',  as in 
(\ref{ex:vp32.24}). The
morpheme  {\S ha} is circumscribed to the expanded verbal group, although I
translate as `yet'  another expression in (\ref{ex:yet-conn}) which
functions as connective. The expression {\S haalɪ} is not frequent in the data
available. 

\begin{exe}
\ex
\begin{xlist}
 

\ex\label{ex:vp32.24}
\gll ʊ ha diu \\
     {3.\sg}  {\mod} eat.\foc  \\
\glt  ` He is still eating. ' 


\ex\label{ex:vp20.3.2.}
\gll ʊ ha wa dije \\
 {3.\sg}  {\mod} {\neg} eat.{\pfv}   \\
\glt  ` He has not eaten yet.'


\ex\label{ex:vp21.2.1.}
\gll ba ɲine ʊ gɛrɛga ra aka ʊ ha wɪɪ \\
 {\sc 3.pl.hum+} look {\sc 3.sg.poss} sickness {\foc} {\conn}  {\sc 3.sg}
{\mod} ill \\
\glt  ` He has been cared for to no avail; he is still ill.' 


\ex\label{ex:vp20.1.1.}
\gll ʊ ha wa waa baaŋ muŋ \\
       {3.\sg} {\mod}  {\neg} come {\dem} {\quant}\\
\glt  `  He does not come here (ever).' 


\ex\label{ex:vp20.3.1.}
\gll ʊ̀ há wà wáwá \\
       {3.\sg} {\mod}   {\neg} come.{\pfv} \\
\glt  `He has not come yet.' 


\ex\label{ex:yet-conn}
\gll ʊ jireʊ saŋa muŋ, haalɪ ʊ ha wa waawa \\
       {3.\sg}  call.{\sc 3.sg} time all {\conn}   {\sc 3.sg} {\sc mod}  {\neg}
come.{\pfv} \\
\glt  `He called her long time ago, yet she has not
come.' 
\end{xlist}
\end{exe}



\paragraph{tu and zɪn}
\label{sec:GRM-preverb-up-down} 

The verbs {\S tuu} and {\S zɪna} are motion
expressions making reference to two opposite paths. When they are used as main
predicate, as in example (\ref{ex:GRM-verb-up-down}),  they denote `go down' and
`go up' and  surface as {\S
tuu} and {\S zɪna} respectively.  The interpretation of one  consultant suggest
that  {\S tuu} and {\S
zɪn} in 
(\ref{ex:GRM-preverb-up-down}) are not preverbs, but first verbs in SVCs.

\begin{exe}
\ex\label{ex:GRM-verb-up-down}
\begin{xlist} 

\ex
\gll n̩ zɪna sal la m̩ paa tʃuono\\
{\sc 1.sg} go.up flat.roof {\foc} {\sc 1.sg} take.{\pv} shea.nut.seed.{\pl}\\
\glt  `I go up on the roof to collect my shea nut seeds.'

\ex
\gll n̩ tuu dɪa ra\\
{\sc 1.sg} go.down house {\foc}\\
\glt I went down to the house.'
\end{xlist}
\end{exe}



\begin{exe}
\ex\label{ex:GRM-preverb-up-down}
\begin{xlist} 

\ex\label{ex:GRM-preverb-up}
\gll zɪn tʃɔ  dɪ kaalɪ  \\
      {go.up} run {\conn} go  \\
\glt  `Go up,  run and leave'  (*Run upwardly and go)

\ex\label{ex:GRM-preverb-down}
\gll tu tʃɔ  dɪ kaalɪ\\
      {go.down} run {\conn} go \\
\glt  `Go down, run and leave'  (*Run downwardly and go)
\end{xlist}
\end{exe}


% 6
%  ́
% The directional particles he (‘itive’, related to the homophonous verb meaning
% ‘go’ (departure from
%  ́
% deictic center or indexically determined location)) and va (‘ventive’, related
% to the homophonous verb meaning
% ‘come’ (arrival at deictic center or indexically determined location)) belong
%to % the class of preverbs of Ewe.
% These are forms that mark functional categories such as aspect, modality, and
% voice on verbs. Preverbs differ
% from verbs in that they do not head VPs, do not inflect for habitual aspect,
%and % do not take NP or PP
% complements (cf. Ameka 1991, 2005a,b, Ansre 1966).


% The particle  {\S ja} is
% polyfunctional:  when it precedes a main verb it  means  either `do'   to
% emphasize the event or conveys an habitual reading, or as, in the present
%case,
% it links two noun phrases. The latter case is glossed in example
%(\ref{ex:agrE})
% and (\ref{ex:agrF}) as {\sc ident}. 


% --Dakubu
% I wonder whether what you call IPFV is an egressive particle? such a particle
% derived from 'go' is quite common.  If it is incompletive / progressive this
% might have to do with the tone pattern


%lenghten preverf fii dii ...



%   Aorist & Imperfective & Perfective 
% Positive
% Negative



\subsection{Verbal suffixes}
\label{sec:GRM-verb-suffix}


In presenting the verb forms in section \ref{sec:GRM-verb-word}, two suffixes
were introduced: the perfective intransitive suffix and the assertive suffix. It
was shown that the perfective intransitive suffix surfaces either as {\S -jE},
{\S -wA} or {\O} depending on  the verb stem.  The assertive suffix
appears  in the imperfective and perfective  intransitive construction if  (i)
none of the other constituents in the clause are in focus, (ii) the clause does
not include negative polarity items, and (iii) the clause is intransitive, that
is, there is no grammatical object. Also,  as mentioned in section
\ref{sec:GRM-imper-clause},  the suffix {\S -ɪ}/{\S -i} appears in the negative
imperative. 

In this section,  the incorporated object pronoun  ({\sc
o}-clitic), the pluractional  suffix, and  other derivative suffixes whose
functions are not yet understood are introduced.


\subsubsection{Incorporated object pronoun}
\label{sec:GRM-morph-opro}


The object pronoun  is
represented as being incorporated into the verb,  and  together they form a
phonological word.  For that reason I refer to this
incorporated object pronoun as the {\sc
o}-clitic. Given the constraints governing the appearance of the perfective
intransitive suffix and the assertive suffix, it is obvious that the {\sc
o}-clitic cannot coexist with any of them. Recall that the  weak subject
pronoun 
and object  pronoun are identical (see section
\ref{sec:GRM-personal-pronouns}).


\begin{table}[!htb]
\centering
\caption{Incorporated object pronouns on  CV(V) stems\label{tab:object-clitic}}

\subfloat[tɪɛ `give']{
\begin{Itabular}{ll}
 wʊsa tɪɛn na & `Wusa gave ME'\\
 wʊsa tɪɛɪ ra & `Wusa gave YOU'\\
 wʊsa tɪɛʊ ra &  `Wusa gave HER'\\
 wʊsa tɪɛja ra &  `Wusa gave US'\\
 wʊsa tɪɛma ra & `Wusa gave YOU'  \\
 wʊsa tɪɛa ra & `Wusa gave THEM'  \\
 wʊsa tɪɛba ra &  `Wusa gave THEM'  \\
\end{Itabular} 
}
\quad
\subfloat[tie `cheat']{
\begin{Itabular}{ll}
 wʊsa tien ne & `Wusa cheated ME'\\
 wʊsa tiei re & `Wusa cheated YOU'\\
 wʊsa tieu ro &  `Wusa cheated HER'\\
 wʊsa tieja ra &  `Wusa cheated US'\\
 wʊsa tiema ra & `Wusa cheated YOU' \\
 wʊsa tiea ra & `Wusa cheated THEM'\\
 wʊsa tieba ra &  `Wusa cheated THEM'\\
\end{Itabular} 
}
\quad
\subfloat[tie `cheat']{
\begin{Itabular}{ll}
 wʊsa tieje re &  `Wusa cheated US'\\
 wʊsa tieme re & `Wusa cheated YOU' \\
 wʊsa tiee re & `Wusa cheated THEM'\\
 wʊsa tiebe re &  `Wusa cheated THEM'\\
\end{Itabular} 
}
\quad
\subfloat[po `divide']{
\begin{Itabular}{ll}
 wʊsa poje re &  `Wusa divided US'\\
 wʊsa pomo ro & `Wusa divided YOU' \\
 wʊsa poa ra & `Wusa divided THEM'\\
 wʊsa pobe re &  `Wusa divided THEM'\\
\end{Itabular} 
}
\end{table}

Table \ref{tab:object-clitic} shows that the {\sc atr}-harmony 
operates in the domain produced by the {\sc
o}-clitic merging with a CV or CVV stem, but may or may not affect the
plural pronouns, as tables \ref{tab:object-clitic}(b) and 
\ref{tab:object-clitic}(c) display. The form of the focus particle is determined
by
the preceding material (i.e. the phonological word  verb+{\sc
o}-clitic) and the harmony rules introduced in
section
\ref{sec:focus-forms}.  The irregularities in table \ref{tab:object-clitic}(d)
are not accounted for.  I did perceive rounding throughout in conversations
(i.e.  {\S wʊsa poma ra} $>$ {\S wʊsa pomo ro} `Wusa divided you {\it pl}'), but
I was unable to get a consultant to produce it in an elicitation session. Table
\ref{tab:object-clitic}(d) should be seen as displaying various renditions,
i.e. with and without {\sc atr-}harmony or {\sc ro-}harmony.


A CVCV stem differs from a CV or CVV stem by exhibiting vowel apocope and/or 
vowel
coalescence.  Table \ref{tab:object-clitic-CVCV} provides paradigms for {\S
kpaga} `catch' and {\S goro} `(go in) circle'. 



\begin{table}[!htb]
\centering
\caption{Incorporated object pronouns on  CVCV stems
\label{tab:object-clitic-CVCV}}

\subfloat[kpaga `catch']{
\begin{Itabular}{ll}
 wʊsa kpaɣn̩ na & `Wusa caught ME'\\
 wʊsa kpaɣɪɪ ra & `Wusa caught YOU'\\
 wʊsa kpaɣʊʊ ra &  `Wusa caught HER'\\
 wʊsa kpaɣəja wa &  `Wusa caught US'\\
 wʊsa kpaɣəma wa & `Wusa caught YOU' \\
 wʊsa kpaɣaa wa & `Wusa caught THEM'\\
 wʊsa kpaɣəba wa &  `Wusa caught THEM'\\
\end{Itabular} 
}
\quad
\subfloat[goro `(go in) circle']{
\begin{Itabular}{ll}
wʊsa gorn̩ no & `Wusa circled ME'\\
 wʊsa gorii re & `Wusa circled YOU'\\
 wʊsa goruu ro &  `Wusa circled HER'\\
 wʊsa gorəja wa &  `Wusa circled US'\\
 wʊsa gorəma wa & `Wusa circled YOU'\\
 wʊsa goraa wa & `Wusa circled THEM'\\
 wʊsa gorəba wa &  `Wusa circled THEM'\\
\end{Itabular} 
}
\end{table}
The schwas ({\I [ə]}) in {\S kpaɣəja} and  {\S gorəja} are perceived as fronted,
and the ones in {\S kpaɣəma} and {\S gorəma}  as rounded. Although this is
certainly due to the following consonant, they are so weak that they can only be
heard when they are carefully pronounced. The focus particle {\S wa} is a
variant of {\S ra}. Consultants  agree that these forms are in free variation,
yet the {\S wa} form coexists only with  the plural in the paradigms elicited.
Nonetheless, such paradigm elicitations are particularly subject to
unnaturalness.\footnote{I personally believe that the alteration is
determined by some kind of sandhi, not number. As to why {\F wa} appears only in
the plural, a scenario may be that (i) first, I install a routine by starting
with `ME' and ending with `THEM', (ii) in the process of eliciting, the passage
from third singular to first plural triggers  a different verb shape, i.e.
CVCVV/CVCN  to CVCVCV, and (iii)  although formally identical to the verb forms
of the singular, the reason why {\F wa} follows the third plural non-human could
be explained by psychological habituation.}

\subsubsection{Pluractional suffixes}
\label{sec:GRM-PluralVerb}


A pluractional verb is defined as a verb which can (i) express the repetition of
an event,  (ii)   subcategorize for a plural object and/or  plural subject,
and/or  (iii)  be marked by the pluractional suffix {\S
-sI}, a derivative suffix whose  vowel quality is always high and
front
and  {\sc atr} value determined by the stem vowel(s).\footnote{An exposition of
the
`plural verbs' in Vagla can be found in \cite{Blen03}. \citet[viii]{Daku07}
calls a similar morpheme `iterative' (i.e. Gurene {\F -sɛ}).  Among the West
African
languages, it is the pluractional verbs in Hausa which have received most
attention \citep[see][]{Jose08}.}  According to (i) above, the iterativeness may
affect the interpretation of the number of participants of an event. Consider
the contrasts between the 
sentences in (\ref{ex:GRM-pv-cut}), where none of the arguments are in the
plural (i.e. contra (ii)).


\begin{exe}
\ex\label{ex:GRM-pv-cut}
  \begin{xlist}
    \ex\label{GRM-pv-cutsg}
\gll   n̩  teŋe  namɪã  ra  \\
       {\sc 1.sg} {cut} {meat} {\sc foc}\\
\glt `I cut a piece of meat (i.e.  made a cut in the flesh or cut into two
pieces).'

\ex\label{GRM-pv-cutpl}
\gll    n̩    teŋesi  a  namɪã  ra \\
          {\sc 1.sg} {cut.{\sc pv}} {\sc art} {meat} {\sc foc}\\
\glt `I cut the meat into pieces.'

 \end{xlist}
\end{exe}

In  (\ref{GRM-pv-cutpl}),  the formal distinction on the verb `cut',  compared
to (\ref{GRM-pv-cutsg}),  causes  the event to be interpreted as one which
involves the repetition of the `same'  sub-event.  The word {\S namɪã} `meat'
is allowed in both the contexts of (\ref{GRM-pv-cutsg}) and
(\ref{GRM-pv-cutpl}), although one may argue that the word {\S namɪã} is
inherently
plural but grammatically singular,  and that the word is appropriate in both
contexts. Despite the fact that  `meat' has indeed a plural form, i.e. {\S
nansa}, it is probably the mass term denotation of {\S namɪã} which 
makes (\ref{GRM-pv-cutpl}) acceptable.

 In (\ref{GRM-pv-turn}), however,  the grammatical object of a
pluractional verb {\S tʃigesi} `turn iteratively' or `put on face
down iteratively'  must refer to individuated entities. 

\begin{exe}
\ex\label{GRM-pv-turn}
  \begin{xlist}
    \ex\label{GRM-pv-turnsg}
\gll   n̩  tʃige  a  hɛna  ra  \\
        {\sc 1.sg} {turn} {\sc art} {bowl.\sg} {\sc foc}\\
\glt `I turn (upside down) the bowl.'

 \ex\label{GRM-pv-turnpl1}
\gll   n̩  tʃigesi  a  hɛnsa  ra   \\
         {\sc 1.sg}   {turn.{\sc pv}} {\sc art} {bowl.\pl} {\sc foc}\\
\glt `I turn (upside down) the bowls (one after the other).'


 \ex\label{GRM-pv-turnpl2}
\gll {\textasteriskcentered}  n̩  tʃigesi   a  hɛna  ra \\
       {}  {\sc 1.sg}    {turn.{\sc pv}} {\sc art}  {bowl.\sg}  {\sc foc}\\
\glt `I turn (upside down in a repetitional fashion) the bowl.'

 \end{xlist}
\end{exe}

Comparing  (\ref{GRM-pv-turnsg}) and (\ref{GRM-pv-turnpl2}) with 
(\ref{GRM-pv-turnpl1}),   the pluractional verb cannot coexist with a singular
noun as grammatical object due to the fact that  the `turning' event cannot be
conceived as affecting the same object in a repetitive fashion. However, in
(\ref{GRM-pv-beat}) the `beating' can affect  one or several
individuals. 


\begin{exe}
\ex\label{GRM-pv-beat}
  \begin{xlist}
    \ex\label{GRM-pv-beat.sg}
\gll   n̩   tugo  a bie  re  \\
            {\sc 1.sg}  {beat} {\sc art} {child.\sg} {\sc foc}\\
\glt `I beat the child.'

\ex\label{GRM-pv-beat.pl1}
\gll   n̩    tugosi  a bise  re   \\
         {\sc 1.sg}   {beat.{\sc pv}} {\sc art} {child.\pl} {\sc foc}\\
\glt ` I beat the children.'


\ex\label{GRM-pv-beat.pl2}
\gll    n̩    tugosi  a  bie  re   \\
         {\sc 1.sg}    {beat.{\sc pv}} {\sc art}  {child.\sg} {\sc foc} \\
\glt `I beat the child (more than once, over a short period of time).'


 \end{xlist}
\end{exe}

Whereas  (\ref{GRM-pv-beat.pl2})
has a possible interpretation, two language consultants
could not assign a meaning to (\ref{GRM-pv-catchout}) below. 




\begin{exe}
\ex\label{GRM-pv-catch}
  \begin{xlist}
    \ex\label{GRM-pv-catchsg}
\gll    ŋ̩  kpaga  a  zal  la  \\
         {\sc 1.sg}   {caught} {\sc art} {chicken.\sg} {\sc foc}\\
\glt `I caught a chicken.'


 \ex\label{GRM-pv-catchpl1}
\gll    ŋ̩    kpagasɪ  a  zalɪɛ ra  \\
       {\sc 1.sg} {caught.{\sc pv}} {\sc art} {chicken.\pl} {\sc foc}\\
\glt `I caught chickens (i.e. in repeated actions).'


 \ex\label{GRM-pv-catchpl2}
\gll   ŋ̩     kpaga  a  zalɪɛ ra   \\
       {\sc 1.sg}  {caught} {\sc art} {chicken.\pl} {\sc foc}\\
\glt `I caught chickens (i.e. in one move).'


 \ex\label{GRM-pv-catchout}
\gll *     ŋ̩  kpagasɪ  a  zal  la  \\
     {}   {\sc 1.sg}  {caught.{\sc pv}} {\sc art} {chicken.\sg} {\sc foc}\\
\glt `I caught a chicken (i.e. after unsuccessful attempts until finally
succeeding with
one particular chicken).'

 \end{xlist}
\end{exe}

A pluractional verb usually denotes an action, but not a state. Therefore, in
(\ref{GRM-pv-catch}), the sense of {\S kpaga}_{1} is related to `catch', and not
to the  possessive sense of the verbal state lexeme   {\S kpaga}_{2}
`have'.\footnote{Though I like to treat {\F dʊasɪ} as a counterexample.  The
pluractional verb {\F dʊasɪ} `be in a row'  may be  derived from the existential
predicate {\F dʊa} `be on/at/in'.  In section \ref{sec:SPA-leaning-v}, it is
shown that the verbs {\F tele} `lean'   and {\F telege} `lean' are determined by
the number value ({\it sg.}/{\it  pl.})  of the subject.  If more examples like
these  arise, {\it pluractional} would then loose its literal signification.}
Examples of pluractional verbs are {\S dʊma} `bite' $>$ {\S dʊnsɪ}  `bite
iteratively', {\S jaga} `hit' $>$ {\S jagasɪ} `hit iteratively', {\S tʃige}
`cover' $>$ {\S tʃigesi}  `cover iteratively' and {\S teŋe} `cut' $>$ {\S
teŋesi}  `cut iteratively',  among others.  Beside the suffix {\S -sI}, {\S
-gE} may also turn a verbal process lexeme into a pluractional verb, e.g.   {\S
tɔtɪ} `pluck' $>$ {\S  tɔrəgɛ} `pluck iteratively' and  {\S keti} `break'  $>$
{\S kerigi} `break iteratively'.



Finally, a pluractional verb must not necessarily display the
suffixation pattern
described above. This is confirmed by the pair {\S kpa}/{\S paa} `take'  in
(\ref{ex:GRM-kpa-paa}).

\begin{exe}
    \ex\label{ex:GRM-kpa-paa}
  \begin{xlist}
    \ex\label{ex:GRM-kpa}
\gll ka kpa zal haŋ ta\\
go take.{\sc pl} fowl.{\sg} {\dem} let.free\\
\glt `Go and take this fowl away.'
      \ex\label{ex:GRM-paa}
\gll ka paa zalɪɛ hama ta\\
go take.{\sc pl} fowl.{\pl} {\dem} let.free\\
\glt `Go and take these fowls away.'
 \end{xlist}
\end{exe}



\subsubsection{Possible derivational suffixes}
\label{sec:GRM-deri-suff}

 \citet[37]{Daku09} identifies some derivational suffixes in
Gurene, but writes that their signification is hard to establish. Similarly,
\citet[69]{Bonv88} writes that making out the identity, productivity and
functioning of verbal derivations in Kasem is not easy. However,
their descriptions indicate that  derivational suffixes mainly encode aspectual
distinctions.

As mentioned in section \ref{sec:GRM-verb-syll-und-tone}, about 90\% of the
verbs are either monosyllabic or bisyllabic, and  only the consonants {\S m,
t, s, n,  l} and {\S g} are found  in onset position word-medially in
trisyllabic verbs. This situation could suggest that 10\% of the verbs in the
current lexicon are the product of verbal derivation, and that the consonants
found  in onset position word-medially in trisyllabic verbs are part of
derivational suffixes. However, apart from the pluractional suffix discussed in
the previous section,  it is impossible at this stage of the research to
establish a systematic mapping between the third syllable of a trisyllabic verb
and a meaning. Table \ref{tab:GRM-der-suff}
presents  some indications that {\S m, l} and {\S g}, i.e. CVCV\{m, l, g\}V,
are involved in some kinds of derivation, although the glosses (and part of
speech categories) assigned to them clearly indicate that the next step would be
to determine their exact meaning (and category).\footnote{The verb pair {\F
go} `round'  and {\F goro}  `(go in) circle'  is  manifestly a derivation as
well, i.e.
CV $>$ CV-rV.} 



\begin{table}[!htb]
\centering
\caption{Possible derivational suffixes\label{tab:GRM-der-suff}}

 
\begin{Itabular}{lllll}
\Hline

 &&&{\S -gV}&\\

wʊra {(v)}& `dismantle' & $>$ & wʊrɪgɪ {(v)}& `collapse'\\
tara  {(v)}& `support' & $>$ &taragɛ {(v)}& `pull' \\
%bɪla {(v)}& `turn repetitively' & $>$ & bɪlgɪ {(v)}& `clean' \\
bra {(v)}& `return' & $>$ & bɛrɛgɪ  {(v)}& `change direction'\\[1ex]\hline

&&&{\S -mV} &\\
ɲagɛɛ  {(v)} & `sour' &$>$ & ɲagamɪ  {(v)}& `ferment' \\
vil {(n)} &`well' & $>$ &vilimi {(v)} & `whirl' \\
 mɪla {(v)} & `turn' & $>$ &mɪlɪmɪ {(v)}& `turn'\\[1ex]\hline

&&&{\S -lV}&\\
 kaga {(v)}& `choke'& $>$ & kagalɛ {(v)} & `lie across' \\
 \Hline
\end{Itabular}
\end{table}




\subsection{Adjunct types}
\label{sec:GRM-adjuncts}


The adjunct constituent  ({\sc adj}) in (\ref{ex:GRM-clause-frame}) below may
consist of  a single word or a  syntactic constituent. Although in section
\ref{sec:GRM-interr-clause}  it was originally positioned at the end of the
sequence,  some adjuncts can  be used clause initially before the
subject. 

\begin{exe}
\exp{ex:GRM-clause-frame}
 {\sc adj}  $\pm$ {\sc s|a}  $+$ {\sc p} $\pm$ {\sc o} $\pm$ {\sc adj} 
\end{exe}

Reference to space, manner and time are the main denotations of these
peripheral arguments. 

% 
% They are glossed {\sc advl} (i.e. adverb locative), {\sc advm}  (i.e.
% adverb manner) and {\sc advt}  (i.e. adverb time) respectively. Examples are
% provided below.


\subsubsection{Oblique object phrase}
\label{sec:GRM-obl-phrase}

The oblique object phrase is an element of a clause whose  semantics is
characterised by an  affected or effected object, although realized by a
postpositional phrase.  In section \ref{sec:SPA-postp},  it is claimed that the
postposition {\S nɪ} (i) identifies an oblique object phrase, (ii) conveys that
the oblique object phrase contains the ground object in localization, and
(iii) follows its complement. 

While localization is
the main function of  {\S nɪ}, the postposition can also be found when there
is no  reference to space. For instance, in section \ref{sec:GRM-manner-adv}, I
discuss the connective {\S denɪ} (i.e. {\advl}+{\postp}), whose role in
discourse is to signal a temporal transition, not a spatial one.  The
examples in (\ref{ex:GRM-obl-obj-no-spa}) illustrate some of the non-spatial
uses
of the oblique object phrase headed by {\S nɪ}.


\begin{exe}
\ex\label{ex:GRM-obl-obj-no-spa}
\begin{xlist}
\ex
\gll ʊ ɲʊ̃ã  [laɣalaɣa nɪ]  \\
      {\psg} drink {\advm} {\postp}    \\
\glt  `He drinks quickly.' 

\ex
\gll baaŋ ɪ fɪ ka tesi [tʃʊɔsa tɪn nɪ]\\
 {\q} {\sc 2.pl} {\pst} {\egr} crush  morning {\art} {\postp}    \\
\glt  `What were you crushing this morning?' 

\ex\label{ex:GRM-obl-obj-no-spa-foc}
\gll a kuoru ŋma dɪ ʊ baaŋ ka sii [n̩ nɪ re]\\
{\art} chief say {\comp} {\sc 3sg.poss} temper {\egr} raise {\sc 1.sg} {\postp}
{\foc}\\
\glt  `The chief told me that he was very angry with me .' 

\end{xlist}
\end{exe}


\subsubsection{Phrasal adverbs}
\label{sec:GRM-obl-phrase}

In section \ref{sec:GRM-compar-ct}, the dubitative construction was identified 
with the phrase {\S a bɔnɪɛ̃ nɪ} `perhaps'  opening the clause. There are other
constructions in which temporal, locative, manner or tense-aspect-mood meaning 
is signaled by the presence of a phrasal adverb clause initially. Two examples
are given in (\ref{ex:GRM-phra-adv}),  but since most of the adverbs convey
temporal, locative and manner signification, other examples are presented in
the subsequent sections. In (\ref{ex:GRM-phra-adv-time}), the phrase {\S tama
finii} is not inherently temporal, but must be interpreted as such in the given
context. 


\begin{exe}
\ex\label{ex:GRM-phra-adv}
\begin{xlist}

\ex\label{ex:GRM-phra-adv-time}{\it Temporal}
\gll [tama finii] ʊ fɪ sʊwa\\
few little {\sc 3.sg} {\mod} die\\
\glt `A little longer and she would  have died.'


\ex\label{ex:GRM-phra-adv-}{\it Evidential}
\gll [wɪdɪɪŋ na] dɪ ʊ naʊ ra\\
truth {\foc} {\comp} {\sc 3.sg} see.{\sc 3.sg} {\foc} \\
\glt  `It is certain that he saw him.


\end{xlist}
\end{exe}

\subsubsection{Locative adverbs}
\label{sec:GRM-deic-adv}


A speaker-subjective,  two-way contrast  exists to locate entities in space. The
deictic locative adverb {\S baŋ} designates the location of speaker, while 
the deictic adverb {\S de} designates  where the
speaker is not located. They represent what is known as the `proximal' and
`distal' 
dimensions of  place deixis. In (\ref{ex:deic-adv-prox}) and
(\ref{ex:deic-adv-dist}),  they are translated as `here' and 'there'
respectively, and glossed {\sc advl}, standing for `locative adverb'. In these
two examples  the postposition {\S nɪ} is optional.  The locative
adverbs cannot occur clause initially, as  (\ref{ex:deic-adv-prox-out})  and
  (\ref{ex:deic-adv-dist-out}) show. 


\begin{exe}
\ex\label{ex:vp}
\begin{xlist}

\ex\label{ex:deic-adv-prox}
\gll wa ban (nɪ)\\
     come {\advl} {\postp} \\
\glt  `Come here'
\ex\label{ex:deic-adv-prox-out}
\textasteriskcentered baŋ wa 

 \ex\label{ex:deic-adv-dist}
\gll ʊ  dʊa de (nɪ) \\
       {\psg}  be.at  {\advl}  {\postp}\\
\glt  `He is there'

\ex\label{ex:deic-adv-dist-out}
\textasteriskcentered de ʊ  dʊa 
\end{xlist}
\end{exe}



\subsubsection{Temporal adverbs}
\label{sec:GRM-manner-adv}

A temporal adverb  ({\it gl.} {\sc advt}) is an expression which typically 
indicates when  an event occurs. In section
\ref{sec:GRM-preverb-three-int-tense}, the three-interval tense system was
introduced. It was shown that the temporal adverbs {\S  dɪare} `yesterday' and
{\S tʃɪa} `tomorrow'  have preverbs counterpart. The  temporal
adverb  {\S zaaŋ} (or {\S zalaŋ}) expresses `today',  and   {\S tɔmʊsʊ} can
express either  `the day before yesterday' or  `the day after tomorrow',   yet 
neither {\S zaaŋ} and {\S tɔmʊsʊ}   have a corresponding preverb. 


\begin{exe}
\ex\label{ex:GRM-adj-temp-adv}
\begin{xlist}
\ex\label{ex:GRM-adj-temp-adv-thatday}

\gll awʊzʊʊrɪ n̩ wa tuwo nɪ  \\
{\advt} {\sc 1.sg} {\neg} {be.at} {\postp}\\
\glt `That day I wasn't there.'


\ex\label{ex:GRM-adj-temp-adv-LB5}

\gll àwʊ̀zʊ́ʊ́rɪ̀  dɪ́gɪ́ɪ́    	     kɔ̀sánàɔ̃̀ 	vàlà 	 (...)\\
{\advt}  one           buffalo  	walked (...) \\
\glt `One day a buffalo walked (by and greeted the spider).' (LB 005)

\ex\label{ex:GRM-adj-temp-adv-CB17}
\gll [dénɪ̀],         [sáŋà      dɪ́gɪ́ɪ́]    	   à   
hã́ã̀ŋ jà 	pàà  	à   	báàl    	   zòmò  (...) \\
{\advt}    time      one      	   {\art}	wife   	{\hab} 	take.{\pl} 
{\art} husband insult.{\pl} (...)\\
\glt `[During their life, it happened] on one occasion that the woman
did insult  the man (...)' .  (CB 017)

\ex\label{ex:GRM-adj-temp-adv-everyday}
\gll  n̩ ja kaalɪ ʊ pe re [{tʃʊɔsɪm pɪsa} bɪ muŋ]\\
 {\sc 1.sg}    {\hab} go {\sc 3.sg} end {\foc} day.break {\itr} all\\
\glt `I visit him every day.'

\ex\label{ex:GRM-adj-temp-adv-nownow}
\gll [laɣalaɣa han nɪ] n̩ kʊtɪ a ʔãã peti\\
{\advt} {\dem} {\postp} {1.\sg} {skin} {\art} bushbuck  finish\\
\glt `I  just finished skinning the bushbuck.'

\end{xlist}
\end{exe}


Some expressions tagged as temporal adverbs are treated as complex, though
opaque, expressions. For instance,  {\S awʊzʊʊrɪ} is translated as  `that
day' in (\ref{ex:GRM-adj-temp-adv}), but the forms {\S wʊsa} `sun' and
{\S zʊʊ} `enter'  are perceptible. The phrase {\S laɣalaɣa han nɪ} in
(\ref{ex:GRM-adj-temp-adv-nownow}) literally
means `now.now this on' ({\advt} {\dem} {\postp}), but `only a moment
ago'  is a better translation.  Similarly, {\S denɪ} is analysed as a
temporal adverb, but usually functions as a connective. It is made from  the
locative adverb {\S de} and the potsposition {\S nɪ}, and is translated to
English as `thereupon', `after that', `at that point', or simply `then'. It is
mainly used at the beginning of a sentence to signal a transition  between the
preceding  and the following situations;
(\ref{ex:GRM-adj-temp-thereupon}) suggests a transition of the resultative type.
The appendix contains other examples of {\S denɪ}, e.g.  CB (008, 017, 019)
and 
LB (006, 016).


\begin{exe}
\ex\label{ex:GRM-adj-temp-thereupon}
\gll dénɪ̀      ré           ʊ̀             	hã́ã̀ŋ 	tɪ̀ŋ 
ŋmá 	dɪ́   	ààí (...) \\
 {\advt} 	{\foc}     {\sc 3.sg.poss}	wife   	{\art}	say 	{\comp} 
no    (...)\\
\glt [The man said: `Don't cry, if you tell your father that I drove the tsetse
flies away,  weeded the farm and took you as a wife, I will also tell your
father you are freeing yourself in bed.']  `Then the wife said: `No, (I won't
say
anything to my father'.)' (CB 036)
\end{exe}



\subsubsection{Manner adverbs}
\label{sec:GRM-manner-adv}

A manner adverb  ({\it gl.} {\sc advm}) describes the way the event denoted by
the verb(s) is carried out. The examples in (\ref{ex:GRM-adj-mann}) illustrate
the meaning and distribution of some manner  adverbs.


\begin{exe}
\ex\label{ex:GRM-adj-mann}
\begin{xlist}

\ex\label{ex:GRM-adj-mann-carefully}
\gll dɪ sãã bʊɛ̃ɪbʊɛ̃ɪ \\
{\comp} drive {\advm}\\
\glt `Drive carefully.'

\ex\label{ex:GRM-adj-mann-slowly}
\gll dɪ ŋma bʊɛ̃ɪbʊɛ̃ɪ\\
{\comp} talk {\advm}\\
\glt `Talk slowly.'

\ex\label{ex:GRM-adj-mann-lighly}
\gll ʊ tʃɔjɛ kaalɪ fɛlfɛl\\
 {\sc 3.sg} run.{\pfv} go {\advm} \\
\glt `She ran away lightly (manner of movement, as a light weight
entity).'

\ex\label{ex:GRM-adj-mann-silently}
\gll  	hã́ã̀ŋ 	  sáŋá 	tʃérím, (...) \\
woman   sit  	{\advm} (...)\\
\glt `The woman sat quietly,  (...)' (CB 032)

\end{xlist}
\end{exe}

It is common for an ideophone to function as a manner adverb (see ideophone in
section \ref{sec:GRM-onoma}). One could argue that  all the manner adverbs in
(\ref{ex:GRM-adj-mann}) are ideophones, i.e. they display reduplicated forms
and {\S tʃerim} is one of a few words which ends with a bilabial nasal. 

The
examples in (\ref{ex:GRM-adj-mann-ideo-adv}) show the repetition of two
expressions; one is an ideophone, i.e. {\S kaŋkalaŋ} `crawl of a snake', and the
other an adverb, i.e.  {\S laɣa} `now'.  The latter is a temporal adverb, but is
treated as a manner adverb when reduplicated, i.e. {\S laɣalaɣa} `quickly'. The
repetition of {\S kaŋkalaŋ} and {\S  laɣalaɣa} conveys that the motion was
(`taken' {\S kpa} and) occurring with great speed.
%inceptive meaning of kpa


\begin{exe}
\ex\label{ex:GRM-adj-mann-ideo-adv}
\begin{xlist}

\ex\label{ex:GRM-adj-mann-ideo}
\gll a baaŋ kpa {kaŋkalaŋ kaŋkalaŋ kaŋkalaŋ}\\
{\conn} just take crawl.rapidly\\
\glt `(She was after the python) but (he) started to crawl away like a shot.'

\ex\label{ex:GRM-adj-mann-adv}
\gll  ka baaŋ kpa laɣalaɣa laɣalaɣa\\
{\conn} just take {\advm} {\advm}\\
\glt `(She) started to (walk) quickly.'

\end{xlist}
\end{exe}


The manner adverb {\S kɪŋkaŋ} `abundantly',  which is composed of the classifier
{\S kɪn} and the verb {\S kana} `abundant',  typically quantifies or intensifies
the
event and always comes after the word encoding the event.  Notice in
(\ref{ex:GRM-adj-mann-alot-v})  and (\ref{ex:GRM-adj-mann-alot-n})   that {\S
kɪŋkaŋ} follows a verb and a nominalized verb respectively. However, in
(\ref{ex:GRM-adj-mann-alot-quant}), {\S kɪŋkaŋ} does not function as a
manner adverb but as a quantifier.



\begin{exe}
\ex\label{ex:GRM-adj-mann-alot}
\begin{xlist}
\ex\label{ex:GRM-adj-mann-alot-v}
\gll gbɪ̃̀ã́         	ɪ̀     	jáárɪ́jɛ́      	kɪ́ŋkàŋ     	nà
(...)\\ 
monkey 	you   	unable.{\pfv} 	  {\advm} {\foc} (...)\\
\glt `Monkey, you are so incompetent, (...).' (LB 016)

\ex\label{ex:GRM-adj-mann-alot-n}

\gll duo tʃʊɔɪ kɪŋkaŋ wa wire\\
asleep lie.{\nmlz} {\advm} {\neg} good\\
\glt `Sleeping too much is not good.'

\ex\label{ex:GRM-adj-mann-alot-quant}
\gll kùórù 	kùò  	tɪ́ŋ 	kà   kpágá kìrìnsá wá  kɪ̀ŋkáŋ̀\\
 chief farm {\art} {\egr} have tsetse.fly.{\sc pl}   {\foc}  {\quant}\\
\glt 	`At the chiefs farm there are many tsetse flies.' (CB 002)

\end{xlist}
\end{exe}






\subsection{Adverbial pro-forms {\it keŋ} and {\it nɪŋ}}
\label{sec:GRM-adv-pro}

The expressions  {\S keŋ} and {\S nɪŋ} are treated as  two (manner) adverb
pro-forms ({\it gl.} {\sc advm}).  They are referred to as pro-forms since they
can substitute for an antecedent,  and  as  adverbs since they must refer back
to a
degree, quality or manner indicated by an event, or  a condition or property by 
an entity. Because they usually refer to the `sort/kind specified or
understood', they are provisionally categorized as manner adverbs.

The adverbs   {\S keŋ} and {\S nɪŋ} are very frequent in the
language and bring
to mind the Norwegian word {\S sånn} or the French phrase {\S comme \c{c}a}. 
English `this way' or `like this/that', as in `He did it this way',  more
or less corresponds to
the meaning of   {\S keŋ} and {\S nɪŋ}, which (\ref{ex:GRM-adv-pro-keng-ning})
illustrates.


\begin{exe}
\ex\label{ex:GRM-adv-pro-keng-ning}
 \begin{xlist}
 
 \ex\label{ex:GRM-adv-pro-ning}
\gll baaŋ ɲuãsa ka sii baŋ nɪ nɪŋ\\
{\q}  smoke  {\egr} rise {\advl} {\postp} {\advm}\\
\glt `What smoke is rising here like this?'  
%(Python story 059)
  \ex\label{ex:GRM-adv-pro-keng}
 \gll baaŋ ka ja keŋ?\\
  {\sc q} {\sc egr} do  {\advm}\\ 
 \glt `What is doing like that?' (Reaction to a sound coming from inside a pot)
 
 \end{xlist}
\end{exe}   


I translate {\S nɪŋ} as `like this' and {\S keŋ} as `like that'. This is
motivated by the way they  encode a sort of psychological saliency on a
proximal/distal dimension. This distinction needs more evidence than the one I
provide,  but consider the conversation between A and B in
(\ref{ex:GRM-adv-pro-keng-AB}). 


\begin{exe}
\ex\label{ex:GRM-adv-pro-keng-AB}
 \begin{xlist}
 
 \ex\label{ex:GRM-adv-pro-A}
\gll A: nɪn na baaba ŋma\\
 {} {\advm} {\foc} B. say\\
\glt `Is this what Baaba said?'

  \ex\label{ex:GRM-adv-pro-B}
 \gll B: ɛ̃ɛ̃ ken ne ʊ ŋma\\
 {} yes {\advm} {\foc} {\sc 3.sg} say\\
\glt `Yes, that is what he said.'
 
 \end{xlist}
\end{exe}   

Similarly,  the (fictional) discourse excerpt in
(\ref{ex:GRM-adv-kapok}) concerns a father (A) addressing his son (B) on the
topic of  how to ignite kapok fiber. The sentence (\ref{ex:GRM-adv-kapok-A-2})
is accompanied with a demonstration on how to strike a cutlass on a stone.


\begin{exe}
\ex\label{ex:GRM-adv-kapok}
 \begin{xlist}
 
 \ex\label{ex:GRM-adv-kapok-A-1}
\gll A: kpa koŋ a ŋmɛna diŋ\\
{}  take kapok {\conn} ignite fire\\
\glt `Take some kapok and start a fire.'

 \ex\label{ex:GRM-adv-kapok-B}
\gll B:  ɲɪnɪɛ̃ ba ja ka ŋmɛna\\
{} {\q} {\sc 3.pl} do {\egr} ignite\\
\glt `How does one ignite.' 

 \ex\label{ex:GRM-adv-kapok-A-2}
\gll  A: ŋmɛna nɪŋ\\
{} ignite {\advm}\\
\glt `Ignite like this.'

 \ex\label{ex:GRM-adv-kapok-A-3}
 \gll  A: tʃɪa dɪ tʃɪ waawa ŋmɛna keŋ\\
{} tomorrow {\conn} {\cras} come.{\pfv} ignite {\advm}\\
\glt `Tomorrow when you come, ignite like that.'
 \end{xlist}
\end{exe}  
 
In the context of (\ref{ex:GRM-adv-kapok}), at the farm the next day, the boy
(B) would tell a colleague: {\S ken ne ba ja ŋmɛna},  {\it lit} like.that they
do ignite, `that is how one ignites'. 

In (\ref{ex:GRM-ning-prop-2}), {\S nɪŋ} refers to the condition of  the room,
which is not a manner  but a property of the room. 


\begin{exe}
\ex\label{ex:GRM-ning-prop-2}
 \gll nɪŋ lɛɪ ʊ dɪa haŋ ja dʊ\\
 {\advm} {\sc neg}  {\sc 3.sg.poss} house {\sc dem} do be\\
\glt  `This is not how his room used to be.'
%(Python story 078)

\end{exe}

In addition, {\S keŋ} and {\S nɪŋ} can function as  discourse particles, whose
meanings resemble   English `like' in some registers \citep{Muff02}. In
(\ref{ex:GRM-keng-like}), {\S keŋ} is considered superfluous since it does not
contribute to the manner of  motion or the state of the
participant.\footnote{Something identical to the translation of
(\ref{ex:GRM-keng-like}) may be heard in  Ghanaian Pidgin English spoken in Wa.
This suggest that Waali and/or Dagaare has one or more similar adverbial
pro-forms.} 

\begin{exe}
 \ex\label{ex:GRM-keng-like} 
 \gll n̩ kaalʊʊ keŋ \\
 {\sc 1.sg} go.{\ipfv .\foc}  {\sc advm}\\
 \glt `I am leaving like that'
\end{exe}

Also, depending on the intonation associated with it, and whether or not  the
focus
particle  is  present, {\S keŋ} and {\S nɪŋ} can function as
interjections used to convey comprehension or surprise. So a phrase like {\S
kén nȅȅ} could be roughly translated as `Is that so?', {\S kén nè}   has a
similar function to the English  tag-question `Isn't it?', but {\S kéēèŋ} or
{\S kén né} could be translated as `yes, that is it'. 

Finally, \cite{Mcgi99} presents  {\S nyɛ} and {\S ɛɛ} (variant {\S gɛɛ}) as
demonstrative pronouns in Pasaale, which can also modify an entire clause. The
former
corresponds to `this' and the latter to `that'. At this point, it is a matter of
comparing the two languages and the terminology employed.  Nonetheless, in the
majority of the examples provided by \cite{Mcgi99}, Chakali {\S keŋ} and {\S
nɪŋ} seem to have the same function. 


\subsection{Focus}
\label{sc:GRM-focus}

Since the notion of focus has been discussed separately in connection with
nominals and verbals, this section offers a basic overview of what has been
stated.  \citet[326]{Dik97} writes that   ``the focal information in a
linguistic expression is
that
information which is relatively the most important or salient in the given
communicative setting''.  In Chakali, we saw  two ways in which a
speaker can integrate focal information, and both of them put `in focus' a
constituent.\footnote{The  terminology employed is probably the result
of  complex and still obscure phenomena. For instance, for the
post-verbal particle {\F lá} in Dagaare, \cite{Bodo97} uses the term
`factitive' and `affirmative' particle interchangeably, \cite{Daku05} uses
`(broad- and narrow-)  focus' and glosses it either as {\sc aff} or {\sc foc},
and
\cite{Saan03} uses post-verbal particle and glosses it as {\sc aff}. In-depth
accounts of focus in Grusi languages can be found in \cite{Blas90, Mcgi99}.
 Anne Schwarz has worked extensively on the topic in some Gur and Kwa
languages \citep{Schw10}.}   The first
encodes focal information in a particle which is always
postponed to a nominal, i.e. {\S ra} and variants. Its  phonological shape is
determined by the
preceding phonological material (see sections \ref{sec:focus-forms} and 
\ref{sec:GRM-foc-neg}). The second, which was called the assertive suffix, takes
the form of vowel features which
are suffixed onto the verb  (see sections \ref{sec:GRM-verb-perf-intran} and 
\ref{sec:GRM-verb-suffix}). It was claimed that  the assertive suffix surfaces
only if (i) none of the other constituents in the
clause are in focus, (ii) the clause does not include negative polarity items,
and (iii) the clause is intransitive.
The second criterion (ii) is applicable to the particle {\S ra} as well: thus
focal
information can only exist in affirmative clauses, negation automatically
prevents information from being in focus.\footnote{\citet[94]{Bodo97} writes
(for
Dagaare) that
``[the factitive particle {\F lá}] is in complementary distribution with the
negative polarity particles, as one would expect of an affirming particle".}  In
 (\ref{ex:GRM-focus}),  the
examples illustrate  how the  focal information is
encoded when the object (\ref{ex:GRM-focus-obj}), the subject
(\ref{ex:GRM-focus-subj}) and the predicate  (\ref{ex:GRM-focus-pred}) are
considered the most important piece of information. 


\begin{exe}
\ex\label{ex:GRM-focus}
\begin{xlist}
 \ex\label{ex:GRM-focus-obj}{\it Focus on object: What has the man chewed?}
\gll   à báál tíè sɪ́gà rá\\
      {\art} man chew bean {\foc} \\
\glt `The man chewed BEANS'

\ex\label{ex:GRM-focus-subj}{\it Focus on subject: Who has chewed the beans?}
\gll   à báál lá   tíè sɪ́gà   \\
       {\art} man {\foc} chew bean    \\
\glt `The MAN chewed beans'

\ex\label{ex:GRM-focus-pred}{\it Focus on predicate: What happened?}
\gll    à báàl tíéwó  \\
   {\art} man chew.{\pfv .\foc}    \\
\glt `The man CHEWED'

\end{xlist}
\end{exe}

The focus particle does not differentiate between  grammatical functions and
is not obligatory. In fact, focus is quite rare in narratives.  

Interestingly, if the postposition {\S nɪ} occurs between the focus particle and
the preceding nominal, one would expect  the focus particle to surface in its
default form, i.e. {\S ra},  since the required adjacency is no longer satisfied
(see  section  \ref{sec:focus-forms}). 


\begin{exe}
\ex\label{ex:GRM-focus-form}
\begin{xlist}
\ex\label{ex:GRM-foc-form-X}
\TExt{\Txt{$\alpha$}\COn{rr}\ \  & \Txt{\I nɪ} & \ \  \TXT{\foc}}
 
\ex\label{ex:GRM-foc-form-1}
\gll  a maŋkɪsɪ ɲuu nɪ ro \\
       {\art} {match} {\reln} {\postp} {\foc}\\
\glt `on the top of the matchbox'

\ex\label{ex:GRM-foc-form-2}
\gll  a  pul nɪ ro \\
       {\sc art} {river} {\sc postp}  {\sc foc}\\
\glt `on/at the river'
\end{xlist}
\end{exe}

However, on several occasions, the postposition becomes `transparent' and 
vowel-harmony can still operate (i.e. though not the consonantal one). The
phenomenon is shown in (\ref{ex:GRM-focus-form}).\footnote{A more extreme case
is found in example (\ref{ex:GRM-obl-obj-no-spa-foc}).}


\section{Miscellaneous linguistic phenomena}
\label{sec:GRM-mis-lin-phen}


\subsection{Linguistic taboos}
\label{sec:GRM-ling-taboo}

A linguistic taboo is defined here as the avoidance of
certain words on certain occasions due to  misfortune associated with those
words. 
These circumstances depend on belief; they can be widespread or marginal. The
avoidance of certain words may depend on the time of the day or action carried
out. For instance, not only  is sweeping  not allowed when someone eats, but
uttering the word {\S tʃãã} `broom' is also forbidden. Also, mentioning
certain animal names is excluded as they may either be tabooed by someone
present or attract their attention, i.e the animal may believe that it is being
called. The strategy is to substitute a word with another. The examples in
(\ref{ex:GRM-taboo-synonyms})  are called {\it taboo synonyms}; the word on the
left of the arrow is the word avoided and the one on the right is its
substitute(s). The consumption taboos discussed in section
\ref{sec:SOC-religion} are intimately related to taboo synonyms.


\begin{exe}
\ex\label{ex:GRM-taboo-synonyms}{\it Taboo synonyms}

 {\I bɔ̀là} $\leftrightarrow$ {\I selzeŋ}/{\I neŋtɪɪna} `elephant'\\
{\I dʒɛ̀tɪ̀} $\leftrightarrow$ {\I ɲuzeŋtɪɪna} `lion'\\
{\I bʊ́ɔ̀mànɪ́ɪ̀} $\leftrightarrow$ {\I ɲuwietɪɪna}/{\I nebietɪɪna} `leopard'\\
{\I váà} $\leftrightarrow$ {\I nʊãtɪɪna}/{\I nʊãzɪmɪɪtɪɪna} `dog'\\
{\I kɔ́ŋ} $\leftrightarrow$ {\I nɪɪtɪɪna} `cobra' \\
{\I gbɪ̃̀ã́} $\leftrightarrow$ {\I neŋgaltɪɪna} `monkey'\\
{\I hèlé} $\leftrightarrow$ {\I muŋzɪŋtɪɪna} `type of squirrel'\\
{\I dʊ̃̀ʊ̃̀wìé} $\leftrightarrow$ {\I mábíéwāátèlèpúsíŋ} `type of
snake'\\
{\I tébíǹ} $\leftrightarrow$ {\I batʃogo}/{\I sankara} `night'\\
{\I ɲʊ́lʊ́ŋ} $\leftrightarrow$ {\I ɲúbíríŋtɪ́ɪ́nà} `blind'\\
{\I tʃã́ã́} $\leftrightarrow$ {\I kɪmpɪɪgɪɪ} `broom'\\
{\I búmmò} $\leftrightarrow$ {\I doŋ} `black'\\

\end{exe}

The substitutes to the right are usually complex stem nouns. Many of them use
 the stem {\S tɪɪna} `owner of': {\S neŋtɪɪna}, {\it lit. } arm-owner.of,
`elephant',  {\S muŋzɪŋtɪɪna}, {\it lit. } tail-large-owner.of,  `type of
squirrel', {\S mábíéwāátèlèpúsíŋ},  {\it lit.}
sibling-not-reach-meat-me `type of snake', etc. 



\subsection{Ideophone and interjection}
\label{sec:GRM-onoma}

 In Chakali, ideophones typically suggest the description of an abstract
property
or the manner in which an event unfolds.  The majority of ideophones are
categorized as qualifiers (section \ref{sec:GRM-qualifier}) or manner adverbs
(section \ref{sec:GRM-adjuncts}), and are usually
difficult to translate. A large set of color expressions can be treated as
 ideophonic expressions (see section \ref{sec:termvari}).  In
(\ref{ex:GRM-ideo}) two examples of ideophone are provided.
                                             
%    the sounds that
%are produced show the concepts that they express

\begin{exe}
\ex\label{ex:GRM-ideo}
\begin{xlist}
 \ex\label{ex:GRM-ideo-advm}
\gll koŋ baaŋ sii dɪa nɪ pə̀pə̀pə̀\\
kapok just rise house {\postp} {\advm}\\
  \glt `The kapok fiber was burning in the house at an increasing rate.'
%\hfill{Python story (line 135)}

 \ex\label{ex:GRM-ideo-qual}
\gll {} aka baaŋ jaa wə̀rwə̀rwə̀rwə̀r\\
(...) {\conn} just {\ident} {\qual}\\
  \glt `and (the eyes of the Python) are glittery.' 
%\hfill{Python story (line 132)}

\end{xlist}
\end{exe}


Apart from {\S pəpəpə} `increase in intensity' and   {\S wərwərwərwər}
`glittery', other
examples are {\S fɔbɔp} `move briskly', {\S kaŋkalaŋ kaŋkalaŋ kaŋkalaŋ} `rapid
crawl of a snake', {\S krrrr} `sound of running', 
 and {\S pã̀ã̀} `sound of an
eruption caused by lighting a fire', among others.

An onomatopoeia is a type of ideophone which not only suggests the concept  it
expresses with sound, but imitates  the actual sound of an entity or event.
Examples of onomatopoeia are {\S púpù} `motorbike', {\S tʃétʃé} `bicycle',
{\S tʃɔkɔ̃ɪ̃ tʃɔkɔ̃ɪ̃} `sound
of a guinea fowl' and {\S gbàgbá}  `duck'\footnote{The word for `duck' is
probably 
borrowed from Waali.
This bird was probably introduced recently and was hard to find in the
villages
visited.}  and {\S kpókòkpókòkpókò} `sound of knocking on a clay pot'.


Reduplication of one or two syllables is the general structural shape of
ideophones and onomatopoeias. In chapter \ref{sec:COL-chap} it will be shown
that  reduplication is a common strategy when one characterizes a visual
property. Another characteristic of ideophones and onomatopoeias  is their
tendency to violate the general
phonological structure of the language.  For instance,  {\S fɔbɔp} `move
briskly'  is the only expression I found ending with a bilabial stop. 

Similarly, a strategy to convey an amplified meaning or the idea of
continuity is to lengthen the sound of an existing word. Consider 
(\ref{ex:GRM-lenght}) below.

\begin{exe}
 \ex\label{ex:GRM-lenght}
   \gll  kawaa sii tarɪ keeeeeeeŋ, aka dʊa  ba dɪanʊã nɪ\\
pumpkin rise {creep} {\advm} {\conn} {be.at} {\sc 3.sg.poss} door {\postp}\\
\glt `The pumpkin crept, crept, crept and crept up to their
door mat.'
%\hfill{Python story (line 56)}
\end{exe}

In (\ref{ex:GRM-lenght}) the adverbial pro-form {\S keŋ} (see section
\ref{sec:GRM-adv-pro}) is stretched to simulate the extention in time of the
event, i.e. the pumpkin grew until it reached the door.\footnote{An equivalent
meaning may be expressed in Ghanaian Pidgin English with the adverbial
expression  {\F ãããã}, as in {\F I worked ãããã, until night time.}}


\subsection{Formulaic language}
\label{sec:GRM-greet}


This section introduces some pieces of formulaic language, which is defined as
conventionalized and idiomatic words or phrases. It usually include greetings,
idioms, proverbs, collocations, etc. \citep[see][]{Wray05}. First, common
interjections are included in table \ref{tab:GRM-interj},\footnote{The etymology
of {\F
ʔàmé} has not been confirmed and {\S gáfrà} is ultimately Hausa. The word
{\F ʃɪ́ã̀ã̀} is equivalent to the
function  associated with the action of {\F tʃuuse} in Chakali ({\F tʃʊʊrɪ} in
Dagaare, {\F tʃʊʊhɛ} in Waali, `puf' or `paf'  in Ghanaian English),  which
is a
fricative sound produced by a non-pulmonic, velarized ingressive airstream
mechanism, articulated with the lower lip and the upper front teeth while the
lips are protruded. Thanks to Egil Albertsen for helping me to describe the
sound.} then some greetings and idioms are presented. Needless to say, since
they are conventionalized and idiomatic, the translation formulaic language is
always a rough  approximation.



\begin{table}[!htb]
\centering
\caption{Selected interjections \label{tab:GRM-interj}}


\begin{Itabular}{lp{8cm}}
\Hline
Interjection & Gloss\\[1ex] \hline
ʔàɪ́	&	no	\\
ʔɛ̃ɛ̃		&yes	\\
gáfrà	&	excuse \\
tóù	&	o.k.	\\
ʔàmé	 & so be it  ({\G etym} Amen)	\\
ʔóí	&	indicates surprise\\
fíó	& totally not	\\
ʔánsà	&	welcome	\\
hĩ́ĩ̀ĩ́	&	expressing disappreciation of an action
carried out by someone else\\
ʔàwó	&	reply to greetings, a sign of appraisal of the interlocutor's
concerns\\
 ʔábà & indicates new and unexpected information\\
ʃɪ́ã̀ã̀ & insult when uttered after someone's remark or simply intented at
someone\\ 
\Hline
\end{Itabular} 

\end{table}


\subsubsection{Greetings}
\label{sec:GRM-greet}

Crucial and obligatory prior to any communicative exchange, greetings trigger
both attention and respect. When meeting with elders, one should  squat  or bend
forward hands-on-knees  while greeting. Clan names can be used in greetings,
e.g. {\S ɪ́tʃà} `respect to you and to your clan'. In table
\ref{tab:greetings},  I provide typical greeting lines with some responses.



\begin{table}[!htb]
\centering
\caption{Greetings\label{tab:greetings}}

\begin{Itabular}{llp{7cm}}
\Hline
Time & Speaker A & Following by either speaker A or B\\ \hline

Morning  & ánsùmōō  & ɪ̀ sìwȍȍ `you stood?', ɪ̄ dɪ̀ tʃʊ́àwʊ̏ʊ̏ `and your
lying?', ɪ̀ bàtʃʊ̀àlɪ́ɪ̀ wīrȍȍ `you sleeping place was good?' \\[1ex]

Afternoon   & ántèrēē & ɪ́ wɪ́sɪ́ tèlȅ  `has the sun reached you?' ɪ́
dɪ̄à
`and your house?'  ɪ̄ bìsé mūŋ `and all your children?'\\[1ex]
  

Evening & ɪ́ dʊ̀ànāā &   ɪ́  dʊ̄ɔ̄n tèlȅȅ  `your evening has reached', 
ɪ́ kùó `your farm'\\
\Hline
\end{Itabular} 
\end{table}



The second singular pronoun {\S ɪ} is replaced by the  second singular plural 
{\S ma}, i.e.  {\S ánsùmōō} $\leftrightarrow$ {\S māānsùmōō},
when there is more than one adreesee or when there is  a single person but the
greetings
are intended to the entire house/family: thus  the distinction {\S ɪ}/{\S ma}
does
not correspond to the politeness function of French {\S tu}/{\S vous}. Chakali
morning and afternoon greetings resemble those of Waali and other languages of
the area.
The response to various greetings such as {\S ɪ́ dɪ̄à} `and your house?',  {\S
ánsà} `welcome, thanks' and many others is the multifunctional expression {\S
àwó},  which is, among other things, a sign of appraisal of the interlocutor's
concerns. The same expression is found in Gonja, but it may have different
functions. I was told that the more extensive the greetings, the more
respect one shows the addressee.  For instance, the elders do not
appreciate the tendency of
the youths to morning-greet as {\S ã̄sūmō}, but prefer something like {\S
áánsùùmōōō}. 
 



\subsubsection{Idioms}
\label{sec:GRM-idiom}

An idiom is a  composite expression which does not convey the literal  meaning
 of the composition  of its parts. Common among many African languages is a
strategy by which  abstract nominals are expressed in compound forms. These
compounds are made of stems whose meanings are disassociated from their ordinary
usage. Some examples have already been provided in section
\ref{sec:GRM-qualifier}. 
In Chakali, words indentifying mental states and habits/behaviors are
often idiomatic, e.g. {\S síínʊ̀màtɪ́ɪ́nà} ({ síí-nʊ̀mà-tɪ́ɪ́nà}, {\it
lit.} eye-hot-owner), `wild' or {\S nʊ̀ã̀pʊ̀mmá} ({ nʊã-pʊmma}, {\it lit.} 
mouth-white), `unreserved'. Even though the expression {\S síínʊ̀màtɪ́ɪ́nà}
is made out of three lexical
roots, it is a `sealed' expression and is associated with the manner in which a
person behaves, i.e. a wild person. The sequence {\S ja
nʊã dɪgɪmaŋa} in (\ref{ex:GRM-idiom-mouth}), {\it lit.} do-mouth-one,  is also
treated as an idiomatic expression.


\begin{exe}
 \ex\label{ex:GRM-idiom-mouth}
   \gll   ba ja nʊã dɪgɪmaŋa a summe dɔŋa\\
{\sc 3.pl} do mouth one {\art} help {\recp} \\
\glt `They should agree and help each other.'

\end{exe}

Needless to say, it is often difficult to  distinguish between an idiomatic
expression and  an expression in which only one of the  components is use in a
 non-literal sense.
% eat-patience
% However, it is often difficult to  distinguish between an idiomatic
% expression from an expression in which one of its component is use in a
% non-literal sense. For instance, one could treat {\S di kaɲɪtɪ},  {\it lit.}
% eat-patience,  as an idiom, but the verb {\S di} 
% Other idiomatic expressions are 
% {\S}, {\it lit.} , `'
% {\S}, {\it lit.} , `'
% {\S}, {\it lit.} , `'
% {\S}, {\it lit.} , `'



\subsubsection{Clicks}
\label{sec:GRM-greet}

\citet[151]{Nade89} writes that clicks\footnote{A click may be roughly defined
as  the release of a pocket of air enclosed between two points of contact in
the mouth. The air is rarefied by a sucking action of the tongue
\cite[see][]{Lade93}.}  may be  heard in the Gur-speaking area to  mean `yes',
`I'm listening' or `Oh, dear!'.  This also occurs in the villages where I
stayed, but
I noticed that one click usually means `yes', `I understand' or `I agree',
whereas two clicks mean the opposite. The click is either palatal or velar and
is produced with the lips closed.

